% Options for packages loaded elsewhere
\PassOptionsToPackage{unicode}{hyperref}
\PassOptionsToPackage{hyphens}{url}
%
\documentclass[
  english,
  man, noextraspace]{apa7}
\usepackage{lmodern}
\usepackage{amssymb,amsmath}
\usepackage{ifxetex,ifluatex}
\ifnum 0\ifxetex 1\fi\ifluatex 1\fi=0 % if pdftex
  \usepackage[T1]{fontenc}
  \usepackage[utf8]{inputenc}
  \usepackage{textcomp} % provide euro and other symbols
\else % if luatex or xetex
  \usepackage{unicode-math}
  \defaultfontfeatures{Scale=MatchLowercase}
  \defaultfontfeatures[\rmfamily]{Ligatures=TeX,Scale=1}
\fi
% Use upquote if available, for straight quotes in verbatim environments
\IfFileExists{upquote.sty}{\usepackage{upquote}}{}
\IfFileExists{microtype.sty}{% use microtype if available
  \usepackage[]{microtype}
  \UseMicrotypeSet[protrusion]{basicmath} % disable protrusion for tt fonts
}{}
\makeatletter
\@ifundefined{KOMAClassName}{% if non-KOMA class
  \IfFileExists{parskip.sty}{%
    \usepackage{parskip}
  }{% else
    \setlength{\parindent}{0pt}
    \setlength{\parskip}{6pt plus 2pt minus 1pt}}
}{% if KOMA class
  \KOMAoptions{parskip=half}}
\makeatother
\usepackage{xcolor}
\IfFileExists{xurl.sty}{\usepackage{xurl}}{} % add URL line breaks if available
\IfFileExists{bookmark.sty}{\usepackage{bookmark}}{\usepackage{hyperref}}
\hypersetup{
  pdftitle={The Transition to Grandparenthood and its Impact on the Big Five Personality Traits and Life Satisfaction},
  pdfauthor={Michael D. Krämer1,2, Manon A. van Scheppingen3, William J. Chopik4, \& David Richter1,4},
  pdflang={en-EN},
  pdfkeywords={grandparenthood, Big Five, life satisfaction, development, propensity score matching},
  hidelinks,
  pdfcreator={LaTeX via pandoc}}
\urlstyle{same} % disable monospaced font for URLs
\usepackage{graphicx,grffile}
\makeatletter
\def\maxwidth{\ifdim\Gin@nat@width>\linewidth\linewidth\else\Gin@nat@width\fi}
\def\maxheight{\ifdim\Gin@nat@height>\textheight\textheight\else\Gin@nat@height\fi}
\makeatother
% Scale images if necessary, so that they will not overflow the page
% margins by default, and it is still possible to overwrite the defaults
% using explicit options in \includegraphics[width, height, ...]{}
\setkeys{Gin}{width=\maxwidth,height=\maxheight,keepaspectratio}
% Set default figure placement to htbp
\makeatletter
\def\fps@figure{htbp}
\makeatother
\setlength{\emergencystretch}{3em} % prevent overfull lines
\providecommand{\tightlist}{%
  \setlength{\itemsep}{0pt}\setlength{\parskip}{0pt}}
\setcounter{secnumdepth}{-\maxdimen} % remove section numbering
% Make \paragraph and \subparagraph free-standing
\ifx\paragraph\undefined\else
  \let\oldparagraph\paragraph
  \renewcommand{\paragraph}[1]{\oldparagraph{#1}\mbox{}}
\fi
\ifx\subparagraph\undefined\else
  \let\oldsubparagraph\subparagraph
  \renewcommand{\subparagraph}[1]{\oldsubparagraph{#1}\mbox{}}
\fi
% Manuscript styling
\usepackage{upgreek}
\captionsetup{font=singlespacing,justification=justified}

% Table formatting
\usepackage{longtable}
\usepackage{lscape}
% \usepackage[counterclockwise]{rotating}   % Landscape page setup for large tables
\usepackage{multirow}		% Table styling
\usepackage{tabularx}		% Control Column width
\usepackage[flushleft]{threeparttable}	% Allows for three part tables with a specified notes section
\usepackage{threeparttablex}            % Lets threeparttable work with longtable

% Create new environments so endfloat can handle them
% \newenvironment{ltable}
%   {\begin{landscape}\begin{center}\begin{threeparttable}}
%   {\end{threeparttable}\end{center}\end{landscape}}
\newenvironment{lltable}{\begin{landscape}\begin{center}\begin{ThreePartTable}}{\end{ThreePartTable}\end{center}\end{landscape}}

% Enables adjusting longtable caption width to table width
% Solution found at http://golatex.de/longtable-mit-caption-so-breit-wie-die-tabelle-t15767.html
\makeatletter
\newcommand\LastLTentrywidth{1em}
\newlength\longtablewidth
\setlength{\longtablewidth}{1in}
\newcommand{\getlongtablewidth}{\begingroup \ifcsname LT@\roman{LT@tables}\endcsname \global\longtablewidth=0pt \renewcommand{\LT@entry}[2]{\global\advance\longtablewidth by ##2\relax\gdef\LastLTentrywidth{##2}}\@nameuse{LT@\roman{LT@tables}} \fi \endgroup}

% \setlength{\parindent}{0.5in}
% \setlength{\parskip}{0pt plus 0pt minus 0pt}

% Overwrite redefinition of paragraph and subparagraph by the default LaTeX template
% See https://github.com/crsh/papaja/issues/292
\makeatletter
\renewcommand{\paragraph}{\@startsection{paragraph}{4}{\parindent}%
  {0\baselineskip \@plus 0.2ex \@minus 0.2ex}%
  {-1em}%
  {\normalfont\normalsize\bfseries\itshape\typesectitle}}

\renewcommand{\subparagraph}[1]{\@startsection{subparagraph}{5}{1em}%
  {0\baselineskip \@plus 0.2ex \@minus 0.2ex}%
  {-\z@\relax}%
  {\normalfont\normalsize\itshape\hspace{\parindent}{#1}\textit{\addperi}}{\relax}}
\makeatother

% \usepackage{etoolbox}
\makeatletter
\patchcmd{\HyOrg@maketitle}
  {\section{\normalfont\normalsize\abstractname}}
  {\section*{\normalfont\normalsize\abstractname}}
  {}{\typeout{Failed to patch abstract.}}
\patchcmd{\HyOrg@maketitle}
  {\section{\protect\normalfont{\@title}}}
  {\section*{\protect\normalfont{\@title}}}
  {}{\typeout{Failed to patch title.}}
\makeatother
\shorttitle{Grandparenthood, Big Five, and Life Satisfaction}
\keywords{grandparenthood, Big Five, life satisfaction, development, propensity score matching\newline\indent Word count: abc}
\DeclareDelayedFloatFlavor{ThreePartTable}{table}
\DeclareDelayedFloatFlavor{lltable}{table}
\DeclareDelayedFloatFlavor*{longtable}{table}
\makeatletter
\renewcommand{\efloat@iwrite}[1]{\immediate\expandafter\protected@write\csname efloat@post#1\endcsname{}}
\makeatother
\usepackage{lineno}

\linenumbers
\usepackage{csquotes}
\usepackage{setspace}
\usepackage{amsmath}
\AtBeginEnvironment{tabular}{\singlespacing}
\AtBeginEnvironment{lltable}{\singlespacing}
\AtBeginEnvironment{tablenotes}{\doublespacing}
\captionsetup[table]{font={stretch=1.5}}
\captionsetup[figure]{font={stretch=1.5}}
\ifxetex
  % Load polyglossia as late as possible: uses bidi with RTL langages (e.g. Hebrew, Arabic)
  \usepackage{polyglossia}
  \setmainlanguage[]{english}
\else
  \usepackage[shorthands=off,main=english]{babel}
\fi

\title{The Transition to Grandparenthood and its Impact on the Big Five Personality Traits and Life Satisfaction}
\author{Michael D. Krämer\textsuperscript{1,2}, Manon A. van Scheppingen\textsuperscript{3}, William J. Chopik\textsuperscript{4}, \& David Richter\textsuperscript{1,4}}
\date{}


\note{\clearpage}

\authornote{

\addORCIDlink{Michael D. Krämer}{0000-0002-9883-5676}, Socio-Economic Panel (SOEP), German Institute for Economic Research (DIW Berlin); International Max Planck Research School on the Life Course (LIFE), Max Planck Institute for Human Development\\
Manon A. van Scheppingen, Department of Developmental Psychology, Tilburg School of Social and Behavioral Sciences, Tilburg University\\
William J. Chopik, Department of Psychology, Michigan State University\\
David Richter, Socio-Economic Panel (SOEP), German Institute for Economic Research (DIW Berlin); Survey Research Division, Department of Education and Psychology, Freie Universität Berlin

The authors made the following contributions. Michael D. Krämer: Conceptualization, Data Curation, Formal Analysis, Methodology, Visualization, Writing - Original Draft Preparation, Writing - Review \& Editing; Manon A. van Scheppingen: Methodology, Writing - Review \& Editing; William J. Chopik: Methodology, Writing - Review \& Editing; David Richter: Supervision, Methodology, Writing - Review \& Editing.

Correspondence concerning this article should be addressed to Michael D. Krämer, German Institute for Economic Research, Mohrenstr. 58, 10117 Berlin, Germany. E-mail: \href{mailto:mkraemer@diw.de}{\nolinkurl{mkraemer@diw.de}}

}

\affiliation{\vspace{0.5cm}\textsuperscript{1} German Institute for Economic Research, Germany\\\textsuperscript{2} International Max Planck Research School on the Life Course (LIFE), Germany\\\textsuperscript{3} Tilburg University, Netherlands\\\textsuperscript{4} Michigan State University, USA\\\textsuperscript{5} Freie Universität Berlin, Germany}

\abstract{
abc
}



\begin{document}
\maketitle

Becoming a grandparent is a pivotal life event for many people in midlife or old age (Infurna et al., 2020). At the same time, there is considerable heterogeneity in how intensely grandparents are involved in their grandchildren's lives and care (Meyer \& Kandic, 2017). In the context of an aging demographic, the time that grandparents are alive and in good health during grandparenthood is prolonged compared to previous generations (Leopold \& Skopek, 2015; Margolis \& Wright, 2017). In addition, an increased share of childcare functions are being fulfilled by grandparents (Hayslip et al., 2019; Pilkauskas et al., 2020). Thus, intergenerational relations have received heightened attention from psychological and sociological research in recent years (Bengtson, 2001; Coall \& Hertwig, 2011). With regard to personality development, the transition to grandparenthood has been posited as an important developmental task in old age (Hutteman et al., 2014). However, empirical research into the psychological consequences of becoming a grandparent is sparse. Testing hypotheses derived from neo-socioanalytic theory (Roberts \& Wood, 2006) in a prospective matched control-group design (see Luhmann et al., 2014), we investigate whether the transition to grandparenthood affects the Big Five personality traits and life satisfaction using data from two nationally representative panel studies.

\hypertarget{personality-development-in-middle-adulthood-and-old-age}{%
\subsection{Personality Development in Middle Adulthood and Old Age}\label{personality-development-in-middle-adulthood-and-old-age}}

The life span perspective characterizes aging as a lifelong process of development and adaptation (Baltes et al., 2006). In accordance with this perspective, research has found personality traits to be subject to change throughout the entire life span (Costa et al., 2019; Graham et al., 2020; Specht, 2017; Specht et al., 2014; for recent reviews, see Bleidorn et al., 2021; Roberts \& Yoon, 2021). Although a major portion of personality development takes place in adolescence and emerging adulthood (Bleidorn \& Schwaba, 2017; Schwaba \& Bleidorn, 2018), evidence has accumulated that personality traits also undergo changes in middle and old adulthood (e.g., Allemand et al., 2008; Damian et al., 2019; Kandler et al., 2015; Lucas \& Donnellan, 2011; Mõttus et al., 2012; Mueller et al., 2016; Wagner et al., 2016; for a review, see Specht, 2017).\\
Here, we examine the Big Five personality traits---agreeableness, conscientiousness, extraversion, neuroticism, and openness to experiences---which constitute a broad categorization of universal patterns of thought, affect, and behavior (John et al., 2008). While the policy relevance of the Big Five personality traits has recently been emphasized (Bleidorn et al., 2019)---especially because of their predictive power regarding many important life outcomes (Ozer \& Benet-Martínez, 2005; Roberts et al., 2007; Soto, 2021, 2019), we acknowledge that there are other viable taxonomies of personality (Ashton \& Lee, 2007, 2020) and other levels of breadth and scope that could add valuable insights to personality development in middle adulthood and old age (Mõttus et al., 2017; Mõttus \& Rozgonjuk, 2021).\\
Changes over time in the Big Five occur both in mean trait levels (i.e., mean-level change; Roberts et al., 2006) and in the relative ordering of people to each other on trait dimensions (i.e., rank-order stability; Anusic \& Schimmack, 2016; Roberts \& DelVecchio, 2000). No observed changes in mean trait levels do not necessarily mean that individual trait levels are stable over time, and perfect rank-order stability does not preclude mean-level changes. Mean-level changes in middle adulthood (ca. 30--60 years old; Hutteman et al., 2014) are typically characterized in terms of greater maturity as evidenced by increased agreeableness and conscientiousness, and decreased neuroticism (Damian et al., 2019; Roberts et al., 2006). In old age (ca. 60 years and older; Hutteman et al., 2014), research is generally more sparse but there is some evidence for a reversal of the maturity effect, especially following retirement (sometimes termed \emph{la dolce vita} effect; Asselmann \& Specht, 2021; Marsh et al., 2013; cf.~Schwaba \& Bleidorn, 2019) and at the end of life in ill health (Wagner et al., 2016).\\
In terms of rank-order stability, some prior studies have shown support for an inverted U-shape trajectory (Ardelt, 2000; Lucas \& Donnellan, 2011; Specht et al., 2011; Wortman et al., 2012): Rank-order stability rises until reaching a plateau in midlife, and decreases, again, in old age. However, evidence is mixed whether rank-order stability actually decreases again in old age (see Costa et al., 2019). Nonetheless, the historical view that personality is stable, or \enquote{set like plaster} (Specht, 2017, p. 64) after one reaches adulthood (or leaves emerging adulthood behind; Bleidorn \& Schwaba, 2017) can largely be abandoned (Specht et al., 2014).\\
Theories explaining the mechanisms of personality development in middle adulthood and old age emphasize both genetic influences and life experiences as interdependent sources of stability and change (Specht et al., 2014; Wagner et al., 2020). In a behavior-genetic twin study, Kandler et al.~(2015) found that non-shared environmental factors were the main source of personality plasticity in old age. Here, we conceptualize the transition to grandparenthood as a life experience that offers the adoption of a new social role according to the social investment principle of neo-socioanalytic theory (Lodi-Smith \& Roberts, 2007; Roberts \& Wood, 2006). According to the social investment principle, normative life events or transitions such as entering the work force or becoming a parent lead to personality maturation through the adoption of new social roles (Roberts et al., 2005). These new roles encourage or compel people to act in a more agreeable, conscientious, and emotionally stable (i.e., less neurotic) way, and the experiences in these roles as well as societal expectations towards them are hypothesized to drive long-term personality development (Lodi-Smith \& Roberts, 2007; Wrzus \& Roberts, 2017). Conversely, consistent social roles foster personality stability.\\
The paradoxical theory of personality coherence (Caspi \& Moffitt, 1993) offers another explanation for personality development through role shifts stating that trait change is more likely whenever people transition into unknown environments where pre-existing behavioral responses are no longer appropriate and societal norms or social expectations give clear indications how to behave instead. On the other hand, stability is favored in environments where no clear guidance how to behave is available. Thus, the finding that age-graded, normative life experiences, such as the transition to grandparenthood, drive personality development would also be in line with the paradoxical theory of personality coherence (see Specht et al., 2014). Compared to the transition to parenthood, however, societal expectations on how grandparents should behave (e.g., \enquote{Grandparents should help parents with childcare if needed}) are less clearly defined and strongly dependent on the degree of (possible) grandparental investment (Lodi-Smith \& Roberts, 2007). Thus, societal expectations and role demands might differ depending on how close grandparents live to their children, the quality of the relationship with their children, and other sociodemographic factors that exert conflicting role demands (Bordone et al., 2017; Lumsdaine \& Vermeer, 2015; Silverstein \& Marenco, 2001; cf.~Muller \& Litwin, 2011). In the whole population of first-time grandparents this diversity of role investment might generate pronounced interindividual differences in intraindividual personality change.\\
Empirically, certain life events such as the first romantic relationship (Wagner et al., 2015) or the transition from high school to university or the first job (Asselmann \& Specht, 2021; Lüdtke et al., 2011) have (partly) been found to be accompanied by mean-level increases in line with the social investment principle (for a review, see Bleidorn et al., 2018). However, recent evidence regarding the transition to parenthood failed to empirically support the social investment principle (Asselmann \& Specht, 2020; van Scheppingen et al., 2016). An analysis of monthly trajectories of the Big Five before and after nine major life events only found limited support for the social investment principle: small increases were found in emotional stability following the transition to employment but not for the other traits or for the other life events theoretically linked to social investment (Denissen et al., 2019). Recently, it has also been emphasized that effects of life events on the Big Five personality trends generally tend to be small and need to be properly analyzed using robust, prospective designs, and appropriate control groups (Bleidorn et al., 2018; Luhmann et al., 2014).\\
Overall, much remains unknown regarding the environmental factors underlying personality development in middle adulthood and old age. One indication that age-graded, normative life experiences contribute to change following a period of relative stability in midlife is offered by recent research on retirement (Bleidorn \& Schwaba, 2018; Schwaba \& Bleidorn, 2019). These results were only partly in line with the social investment principle in terms of mean-level changes and displayed substantial individual differences in change trajectories. The authors discuss that as social role \enquote{divestment} (Schwaba \& Bleidorn, 2019, p. 660) retirement functions differently compared to social investment in the classical sense which adds a role. The transition to grandparenthood could represent such an investment into a new role in middle adulthood and old age---given that grandparents have regular contact with their grandchild and actively take part in childcare to some degree (i.e., invest psychologically in the new grandparent role; Lodi-Smith \& Roberts, 2007).

\hypertarget{grandparenthood}{%
\subsection{Grandparenthood}\label{grandparenthood}}

The transition to grandparenthood, that is, the birth of the first grandchild, can be described as a time-discrete life event marking the beginning of one's status as a grandparent (Luhmann et al., 2012). In terms of characteristics of major life events (Luhmann et al., 2020), the transition to grandparenthood stands out in that it is externally caused (by one's own children; see also Arpino, Gumà, et al., 2018; Margolis \& Verdery, 2019), while at the same time being predictable as soon as one's children reveal their pregnancy or family planning. The transition to grandparenthood has been labeled a countertransition due to this lack of direct control over if and when someone has their first grandchild (Hagestad \& Neugarten, 1985; as cited in Arpino, Gumà, et al., 2018). Grandparenthood is also generally positive in valence and emotionally significant---given one maintains a good relationship with their child.\\
Grandparenthood can also be characterized as a developmental task (Hutteman et al., 2014) mostly associated with the period of (early) old age---although considerable variation in the age at the transition to grandparenthood exists both within and between cultures (Leopold \& Skopek, 2015; Skopek \& Leopold, 2017). Still, the period where parents on average experience the birth of their first grandchild coincides with the end of (relative) stability in terms of personality development in midlife (Specht, 2017), where retirement, shifting social roles, and initial cognitive and health declines can be disruptive to life circumstances putting personality development into motion (e.g., Mueller et al., 2016; Stephan et al., 2014). As a developmental task, grandparenthood is expected to be part of a normative sequence of aging that is subject to societal expectations and values differing across cultures and historical time (Baltes et al., 2006; Hutteman et al., 2014).\\
Mastering developmental tasks (i.e., fulfilling roles and expectations to a high degree) is hypothesized to drive personality development towards maturation similarly to propositions by the social investment principle, that is, leading to higher levels of agreeableness and conscientiousness, and lower levels of neuroticism (Roberts et al., 2005; Roberts \& Wood, 2006). In comparison to the transition to parenthood which has been found to be ambivalent in terms of both personality maturation and life satisfaction (Aassve et al., 2021; Johnson \& Rodgers, 2006; Krämer \& Rodgers, 2020; van Scheppingen et al., 2016), Hutteman et al.~(2014) hypothesize that the transition to grandparenthood is generally seen as positive because it (usually) does not impose the stressful demands of daily childcare on grandparents. Grandparental investment in their grandchildren has been discussed as beneficial in terms of the evolutionary, economic, and sociological advantages it provides for the whole intergenerational family structure (Coall et al., 2018; Coall \& Hertwig, 2011).\\
While we could not find prior studies investigating development of the Big Five over the transition to grandparenthood, there is some evidence on changes in life satisfaction over the transition to grandparenthood. In cross-sectional studies, the preponderance of evidence suggests that grandparents who provide grandchild care or have close relationships with their older grandchildren have higher life satisfaction (e.g., Mahne \& Huxhold, 2014; Triadó et al., 2014). There are a few longitudinal studies, albeit they offer conflicting conclusions: Data from the Survey of Health, Ageing and Retirement in Europe (SHARE) showed that the birth of a grandchild was followed by improvements to quality of life and life satisfaction, but only among women (Tanskanen et al., 2019) and only in first-time grandmothers via their daughters (Di Gessa et al., 2019). Several studies emphasized that grandparents actively involved in childcare experienced larger increases in life satisfaction (Arpino, Bordone, et al., 2018; Danielsbacka et al., 2019; Danielsbacka \& Tanskanen, 2016). On the other hand, fixed effects regression models\footnote{Fixed effects regression models exclusively rely on within-person variance (see Brüderl \& Ludwig, 2015; McNeish \& Kelley, 2019).} using SHARE data did not find any effects of first-time grandparenthood on life satisfaction regardless of grandparental investment and only minor decreases of grandmothers' depressive symptoms (Sheppard \& Monden, 2019).\\
In a similar vein, some prospective studies reported beneficial effects of the transition to grandparenthood and of grandparental childcare investment on various health measures, especially in women (Chung \& Park, 2018; Condon et al., 2018; Di Gessa et al., 2016a, 2016b). Again, beneficial effects on self-rated health did not persevere in fixed effects analyses as reported in Ates (2017) who used longitudinal data from the German Aging Survey (DEAS). \\
We are not aware of any study investigating the rank-order stability of traits over the transition to grandparenthood. The occurrence of other life events has been shown to be associated with the rank-order stability of personality and well-being, although only for certain events and traits (e.g., Denissen et al., 2019; Hentschel et al., 2017; Specht et al., 2011).

\hypertarget{current-study}{%
\subsection{Current Study}\label{current-study}}

In the current study, we revisit the development of life satisfaction across the transition to grandparenthood. We extend this research to psychological development in a more general sense by examining the development of Big Five personality traits. Three research questions motivate the current study which is the first to analyze Big Five personality development over the transition to grandparenthood:

\begin{enumerate}
\def\labelenumi{\arabic{enumi}.}
\tightlist
\item
  What are the effects of the transition to grandparenthood on mean-level trajectories of the Big Five traits and life satisfaction?
\item
  How large are interindividual differences in intraindividual change for the Big Five traits and life satisfaction over the transition to grandparenthood?
\item
  How does the transition to grandparenthood affect rank-order stability of the Big Five traits and life satisfaction?
\end{enumerate}

To address these questions, we compare development over the transition to grandparenthood with that of matched participants who do not experience the transition during the study period (Luhmann et al., 2014). This is necessary because pre-existing differences between prospective grandparents and non-grandparents in variables related to the development of the Big Five or life satisfaction introduce confounding bias when estimating the effects of the transition to grandparenthood (VanderWeele et al., 2020). The impact of adjusting (or not adjusting) for pre-existing differences, or background characteristics, has recently been emphasized in the prediction of life outcomes from personality in a mega-analytic framework of ten large panel studies (Beck \& Jackson, 2021). Propensity score matching is one technique to account for confounding bias by equating the groups in their estimated propensity to experience the event in question (Thoemmes \& Kim, 2011). This propensity is calculated from regressing the so-called treatment variable (i.e., the group variable indicating whether someone experienced the event) on covariates related to the likelihood of experiencing the event and to the outcomes. This approach addresses confounding bias by creating balance between the groups in the covariates used to calculate the propensity score (Stuart, 2010).\\
We adopt a prospective design that tests the effects of becoming first-time grandparents separately against two propensity-score-matched control groups: first, a matched control group of parents (but not grandparents) with at least one child in reproductive age, and, second, a matched control group of nonparents. Adopting two control groups allows us to disentangle potential effects attributable to becoming a grandparent from effects attributable to being a parent already, thus addressing selection effects into grandparenthood and confounding more comprehensively than previous research. Thereby, we cover the first two of the three causal pathways to not experiencing grandparenthood pointed out by demographic research (Margolis \& Verdery, 2019): one's own childlessness, childlessness of one's children, and not living long enough to become a grandparent. Our comparative design also controls for average age-related and historical trends in the Big Five traits and life satisfaction (Luhmann et al., 2014), and enables us to report effects of the transition to grandparenthood unconfounded by instrumentation effects, which describe the tendency of reporting lower well-being scores with each repeated measurement (Baird et al., 2010).\footnote{Instrumentation effects caused by repeated assessments have only been described for life satisfaction but we assume similar biases exist for certain Big Five items.}\\
We improve upon previous longitudinal studies utilizing matched control groups (e.g., Anusic et al., 2014a, 2014b; Yap et al., 2012) in that we performed the matching at a specific time point preceding the transition to grandparenthood (at least two years beforehand) and not based on individual survey years. This design choice ensures that the covariates involved in the matching procedure are not already influenced by the event or anticipation of it (Greenland, 2003; Rosenbaum, 1984; VanderWeele, 2019; VanderWeele et al., 2020), thereby reducing the risk of confounding through collider bias (Elwert \& Winship, 2014). Similar approaches in the study of life events have recently been adopted (Balbo \& Arpino, 2016; Krämer \& Rodgers, 2020; van Scheppingen \& Leopold, 2020).\\
Informed by the social investment principle and previous research on personality development in middle adulthood and old age, we preregistered the following hypotheses (prior to data analysis; osf.io/):

\begin{itemize}
\tightlist
\item
  H1a: Following the birth of their first grandchild, grandparents increase in agreeableness and conscientiousness, and decrease in neuroticism compared to the matched control groups of parents (but not grandparents) and nonparents. We do not expect the groups to differ in their trajectories of extraversion and openness to experience.
\item
  H1b: Grandparents' post-transition increases in agreeableness and conscientiousness, and decreases in neuroticism are more pronounced among those who provide substantial grandchild care.
\item
  H1c: Grandmothers increase in life satisfaction following the transition to grandparenthood as compared to the matched control groups but grandfathers do not.
\item
  H2: Individual differences in intraindividual change in the Big Five and life satisfaction are larger in the grandparent group than the control groups.
\item
  H3: Compared to the matched control groups, grandparents' rank-order stability of the Big Five and life satisfaction over the transition to grandparenthood is smaller.
\end{itemize}

Exploratorily, we further probe the moderator performing paid work which could constitute a potential role conflict among grandparents.

\hypertarget{methods}{%
\section{Methods}\label{methods}}

\hypertarget{samples}{%
\subsection{Samples}\label{samples}}

To evaluate these hypotheses, we used data from two population-representative panel studies: the Longitudinal Internet Studies for the Social Sciences (LISS) panel from the Netherlands and the Health and Retirement Study (HRS) from the United States.\\
The LISS panel is a representative sample of the Dutch population initiated in 2008 with data collection still ongoing (Scherpenzeel, 2011; van der Laan, 2009). It is administered by CentERdata (Tilburg University, The Netherlands). Included households are a true probability sample of households drawn from the population register (Scherpenzeel \& Das, 2010). While originally roughly half of invited households consented to participate, refreshment samples were drawn in order to oversample previously underrepresented groups using information about response rates and their association with demographic variables (household type, age, ethnicity; see \url{https://www.lissdata.nl/about-panel/sample-and-recruitment/}). Data collection was carried out online and participants lacking the necessary technical equipment were outfitted with it. We included yearly assessments from 2008 to 2020 from several different modules (see \emph{Measures}) as well as data on basic demographics which was assessed on a monthly rate. For later coding of covariates from these monthly demographic data we used the first available assessment in each year.\\
The HRS is an ongoing longitudinal population-representative study of older adults in the US (Sonnega et al., 2014) administered by the Survey Research Center (University of Michigan, United States). Initiated in 1992 with a first cohort of individuals aged 51-61 and their spouses, the study has since been extended with additional cohorts in the 1990s (see \url{https://hrs.isr.umich.edu/documentation/survey-design/}). In addition to the HRS core interview every two years (in-person or as a telephone survey), the study has since 2006 included a leave-behind questionnaire covering a broad range of psychosocial topics including the Big Five personality traits and life satisfaction. These topics, however, were only administered every four years starting in 2006 for one half of the sample and in 2008 for the other half. We included personality data from 2006 to 2018, all available data for the coding of the transition to grandparenthood from 1996 to 2018, as well as covariate data from 2006 to 2018 including variables drawn from the Imputations File and the Family Data (only available up to 2014).\\
These two panel studies provided the advantage that they contained several waves of personality data as well as information on grandparent status and a broad range of covariates at each wave. While the HRS provided a large sample with a wider age range, the LISS panel was smaller and younger\footnote{The reason for the included grandparents from the LISS panel being younger was that grandparenthood questions were part of the \emph{Work and Schooling} module and---for reasons unknown to us---filtered to participants performing paid work. Thus, older, retired first-time grandparents from the LISS panel could not be identified.} but provided more frequent personality assessments spaced every one to two years. Note that M. van Scheppingen has previously used the LISS panel to analyze correlated changes between life satisfaction and Big Five traits across the lifespan (\url{https://osf.io/3cxuy/}). W. Chopik and M. van Scheppingen have previously used the HRS to analyze Big Five traits and relationship-related constructs (van Scheppingen et al., 2019). W. Chopik has additionally used the HRS to analyze mean-level and rank-order changes in Big Five traits in response to bereavement (Chopik, 2018) and other relationship-related or non-Big Five-related constructs (e.g., optimism; Chopik et al., 2020). These publications do not overlap with the current study in the central focus of grandparenthood.\footnote{Publications using LISS panel data can be searched at \url{https://www.dataarchive.lissdata.nl/publications/}. Publications using HRS data can be searched at \url{https://hrs.isr.umich.edu/publications/biblio/}.} The present study used de-identified archival data in the public domain, and, thus, it was not necessary to obtain ethical approval from an IRB.

\hypertarget{measures}{%
\subsection{Measures}\label{measures}}

\hypertarget{personality}{%
\subsubsection{Personality}\label{personality}}

In the LISS panel, the Big Five personality traits were assessed using the 50-item version of the IPIP Big-Five Inventory scales (Goldberg, 1992). For each Big Five trait, ten 5-point Likert-scale items were answered (1 = \emph{very inaccurate}, 2 = \emph{moderately inaccurate}, 3 = \emph{neither inaccurate nor accurate}, 4 = \emph{moderately accurate}, 5 = \emph{very accurate}). Example items included \enquote{Like order} (conscientiousness), \enquote{Sympathize with others' feelings} (agreeableness), \enquote{Worry about things} (neuroticism), \enquote{Have a vivid imagination} (openness to experience), and \enquote{Start conversations} (extraversion). At each wave, we took a participant's mean of each subscale as their trait score. Internal consistencies at the time of matching, as indicated by McDonald's \(\omega\) (McNeish, 2018), averaged \(\omega =\) 0.83 over all traits ranging from \(\omega =\) 0.77 (conscientiousness in the parent control group) to \(\omega =\) 0.90 (extraversion in the nonparent control group). Other studies have shown measurement invariance for these scales across time and age groups, and convergent validity with the Big Five inventory (BFI-2) (Denissen et al., 2020; Schwaba \& Bleidorn, 2018). The Big Five (and life satisfaction) were contained in the \emph{Personality} module which was administered yearly but with planned missingness in some years for certain cohorts (see Denissen et al., 2019). Thus, there are one to two years between included assessments, given no other sources of missingness.\\
In the HRS, the Midlife Development Inventory (MIDI) scales were administered to measure the Big Five (Lachman \& Weaver, 1997). This instrument was constructed for use in large-scale panel studies of adults and consisted of 26 adjectives (five each for conscientiousness, agreeableness, and extraversion, four for neuroticism, and seven for openness to experience). Participants were asked to rate on a 4-point scale how well each item described them (1 = \emph{a lot}, 2 = \emph{some}, 3 = \emph{a little}, 4 = \emph{not at all}). Example adjectives included \enquote{Organized} (conscientiousness), \enquote{Sympathetic} (agreeableness), \enquote{Worrying} (neuroticism), \enquote{Imaginative} (openness to experience), and \enquote{Talkative} (extraversion). For better comparability with the LISS panel, we reverse scored all items so that higher values corresponded to higher trait levels and, at each wave, took the mean of each subscale as the trait score. Big Five trait scores showed satisfactory internal consistencies at the time of matching which averaged \(\omega =\) 0.75 over all traits ranging from \(\omega =\) 0.68 (conscientiousness in the nonparent control group) to \(\omega =\) 0.81 (agreeableness in the nonparent control group).

\hypertarget{life-satisfaction}{%
\subsubsection{Life Satisfaction}\label{life-satisfaction}}

In both samples, life satisfaction was assessed using the 5-item Satisfaction with Life Scale (SWLS; Diener et al., 1985) which participants answered on a 7-point Likert scale (1 = \emph{strongly disagree}, 2 = \emph{somewhat disagree}, 3 = \emph{slightly disagree}, 4 = \emph{neither agree or disagree}, 5 = \emph{slightly agree}, 6 = \emph{somewhat agree}, 7 = \emph{strongly agree})\footnote{In the LISS panel, the \enquote{somewhat} was omitted and instead of \enquote{or} \enquote{nor} was used.}. An example item was \enquote{I am satisfied with my life}. Internal consistency at the time of matching was \(\omega =\) 0.90 in the LISS panel with the parent control sample (\(\omega =\) 0.88 with the nonparent control sample), and \(\omega =\) 0.91 in the HRS with the parent control sample (\(\omega =\) 0.91 with the nonparent control sample).

\hypertarget{transition-to-grandparenthood}{%
\subsubsection{Transition to Grandparenthood}\label{transition-to-grandparenthood}}

The procedure to obtain information on grandparents' transition to grandparenthood generally followed the same steps in both samples. The items this coding was based on, however, differed slightly: In the LISS panel, participants were asked \enquote{Do you have children and/or grandchildren?} with \enquote{children}, \enquote{grandchildren}, and \enquote{no children or grandchildren} as possible answer categories. This question was part of the \emph{Work and Schooling} module and filtered to participants performing paid work. In the HRS, all participants were asked for the total number of grandchildren: \enquote{Altogether, how many grandchildren do you (or your husband / wife / partner, or your late husband / wife / partner) have? Include as grandchildren any children of your (or your {[}late{]} husband's / wife's / partner's) biological, step- or adopted children}.\footnote{The listing of biological, step-, or adopted children has been added since wave 2006.}\\
In both samples, we tracked grandparenthood status (0 = \emph{no grandchildren}, 1 = \emph{at least one grandchild}) over time. Due to longitudinally inconsistent data in some cases, we included in the grandparent group only participants with exactly one transition from 0 to 1 in this grandparenthood status variable, and no transitions backwards (see Fig. SX). We marked participants who continually indicated that they had no grandchildren as potential members of the control groups.

\hypertarget{moderators}{%
\subsubsection{Moderators}\label{moderators}}

Based on insights from previous research, we tested three variables as potential moderators of the mean-level trajectories of the Big Five and life satisfaction over the transition to grandparenthood: First, we analyzed whether gender acted as a moderator as indicated by research on life satisfaction (see Tanskanen et al., 2019; Di Gessa et al., 2019). We coded a dummy variable indicating female gender (0 = \emph{male}, 1 = \emph{female}).\\
Second, we tested whether performing paid work or not was associated with divergent trajectories of the Big Five and life satisfaction (see Schwaba \& Bleidorn, 2019). Since the LISS subsample of grandparents we identified was based exclusively on participants performing paid work, we performed these analyses only in the HRS subsample. This served two purposes: to test how participants involved in the workforce (even if officially retired) differed from those not working, which might shed light on role conflict and have implications for the social investment mechanisms we described earlier. As a robustness check, these moderation tests also allowed us to assess whether potential differences in the main results between the LISS and HRS samples could be accounted for by including performing paid work as a moderator in analyses of the HRS sample. In other words, perhaps the results in the HRS participants performing paid work are similar to those seen in the LISS sample, which had already been conditioned on this variable through filtering in the questionnaire.\\
Third, we examined how involvement in grandchild care moderated trajectories of the Big Five and life satisfaction in grandparents after the transition to grandparenthood (see Arpino, Bordone, et al., 2018; Danielsbacka et al., 2019; Danielsbacka \& Tanskanen, 2016). We coded a dummy variable (0 = \emph{provided less than 100 hours of grandchild care}, 1 = \emph{provided 100 or more hours of grandchild care}) as a moderator based on the question \enquote{Did you (or your {[}late{]} husband / wife / partner) spend 100 or more hours in total since the last interview / in the last two years taking care of grand- or great grandchildren?}.\footnote{Although dichotomization of a continuous construct (hours of care) is not ideal for moderation analysis (MacCallum et al., 2002), there were too many missing values in the variable assessing hours of care continuously (variables *E063).} This information was only available for grandparents in the HRS; in the LISS panel, too few participants answered follow-up questions on intensity of care to be included in the analyses (\textless50 in the final analysis sample).

\hypertarget{procedure}{%
\subsection{Procedure}\label{procedure}}

Drawing on all available data, three main restrictions defined the final analysis samples of grandparents (see Fig. SX for participant flowcharts): First, we identified participants who indicated having grandchildren for the first time during study participation (see \emph{Measures}; \(N_{LISS} =\) 337; \(N_{HRS} =\) 3272, including HRS waves 1996-2004 before personality assessments were introduced). Second, we restricted the sample to participants with at least one valid personality assessment (valid in the sense that at least one of the six outcomes was non-missing; \(N_{LISS} =\) 335; \(N_{HRS} =\) 1702).\footnote{For the HRS subsample, we also excluded \(N =\) 30 grandparents in a previous step who reported unrealistically high numbers of grandchildren (\(>\) 10) in their first assessment following the transition to grandparenthood.} Third, we included only participants with both a valid personality assessment before and one after the transition to grandparenthood (\(N_{LISS} =\) 253; \(N_{HRS} =\) 859). Lastly, few participants were excluded because of inconsistent or missing information regarding their children\footnote{We opted not to use multiple imputation for these child-related variables such as number of children which defined the control groups and were also later used for computing the propensity scores.} resulting in the final analysis samples of first-time grandparents, \(N_{LISS} =\) 250 (53.60\(\%\) female; age at transition to grandparenthood \(M =\) 57.94, \(SD =\) 4.87) and \(N_{HRS} =\) 846 (54.85\(\%\) female; age at transition to grandparenthood \(M =\) 61.80, \(SD =\) 6.88).\\
To disentangle effects of the transition to grandparenthood from effects of being a parent, we defined two pools of potential control subjects to be involved in the matching procedure: The first pool of potential control subjects comprised parents who had at least one child in reproductive age (defined as \(15 \leq age_{firstborn}\leq65\)) but no grandchildren throughout the observation period (\(N_{LISS} =\) 844 with 3040 longitudinal observations; \(N_{HRS} =\) 1485 with 2703 longitudinal observations). The second pool of potential matches comprised participants who reported being childless throughout the observation period (\(N_{LISS} =\) 1077 with 4337 longitudinal observations; \(N_{HRS} =\) 1340 with 2346 longitudinal observations). The two control groups were, thus, by definition mutually exclusive.\\
In order to match each grandparent with the control participant who was most similar in terms of the included covariates we utilized propensity score matching.

\hypertarget{covariates}{%
\subsubsection{Covariates}\label{covariates}}

For propensity score matching, we used a broad set of covariates (VanderWeele et al., 2020) covering participants' demographics (e.g., education), economic situation (e.g., income), and health (e.g., mobility difficulties). We also included the pre-transition outcome variables as covariates---as recommended in the literature (Cook et al., 2020; Hallberg et al., 2018; Steiner et al., 2010; VanderWeele et al., 2020), as well as the panel wave participation count and assessment year in order to control for instrumentation effects and historical trends (e.g., 2008/2009 financial crisis; Baird et al., 2010; Luhmann et al., 2014). For matching grandparents with the parent control group we additionally included as covariates variables containing information on fertility and family history (e.g., number of children, age of first three children) which were causally related to the timing of the transition to grandparenthood (i.e., entry into treatment; Arpino, Gumà, et al., 2018; Margolis \& Verdery, 2019).\\
Covariate selection has seldom been explicitly discussed in previous longitudinal studies estimating treatment effects of life events (e.g., in matching designs). We see two (in part conflicting) traditions that address covariate selection: First, classical recommendations from psychology argue to include all available variables that are associated with both the treatment assignment process (i.e., selection into treatment) and the outcome (e.g., Steiner et al., 2010; Stuart, 2010). Second, recommendations from a structural causal modeling perspective (see Elwert \& Winship, 2014; Rohrer, 2018) are more cautious aiming to avoid pitfalls such as conditioning on a pre-treatment collider (collider bias) or a mediator (overcontrol bias). Structural causal modeling, however, requires advanced knowledge of the causal structures underlying all involved variables (Pearl, 2009).\\
In selecting covariates, we followed guidelines laid out by VanderWeele et al.~(2019; 2020) which reconcile both views and offer practical guidance\footnote{Practical considerations of covariate selection when using large archival datasets (i.e., with no direct control over data collection) are discussed in VanderWeele et al.~(2020).} when complete knowledge of the underlying causal structures is unknown: These authors propose a \enquote{modified disjunctive cause criterion} (VanderWeele, 2019, p. 218) recommending to select all available covariates which are assumed to be causes of the outcomes, treatment exposure (i.e., the transition to grandparenthood), or both, as well as any proxies for an unmeasured common cause of the outcomes and treatment exposure. To be excluded from this selection are variables assumed to be instrumental variables (i.e., assumed causes of treatment exposure that are unrelated to the outcomes except through the exposure) and collider variables (Elwert \& Winship, 2014). Because all covariates we used for matching were measured at least two years before the birth of the grandchild, we judge the risk of introducing collider bias or overcontrol bias by controlling for these covariates to be relatively small. In addition, as mentioned in the \emph{Introduction}, the event transition to grandparenthood is not planned by or under direct control of grandparents which further reduces the risk of bias introduced by controlling for pre-treatment colliders.\\
An overview of the variables we used to compute the propensity scores for matching can be found in the Supplemental Material (see also Tables \ref{tab:stddiffmeans-balance-liss} \& \ref{tab:stddiffmeans-balance-hrs}). Critically, we also provide justification for each covariate on whether we assume it to be causally related to treatment assignment, the outcomes, or both. We tried to find substantively equivalent covariates in both samples but had to compromise in a few cases (e.g., children's educational level only in HRS vs.~children living at home only in LISS).\\
Estimating propensity scores requires complete covariate data. Therefore, before computing propensity scores, we performed multiple imputations in order to account for missingness in our covariates (Greenland \& Finkle, 1995). Using five imputed data sets computed by classification and regression trees (CART; Burgette \& Reiter, 2010) in the \emph{mice} R package (van Buuren \& Groothuis-Oudshoorn, 2011), we predicted treatment assignment (i.e., the transition to grandparenthood) five times per observation in logistic regressions with a logit link function.\footnote{In these logistic regressions we included all covariates listed above as predictors except for \emph{female} which was later used for exact matching and health-related covariates in LISS-wave 2014 which were not assessed in that wave.} We averaged these five scores per observation to compute the final propensity score to be used for matching (Mitra \& Reiter, 2016). We used imputed data only for propensity score computation and not in later analyses because missing data in the outcome variables due to nonresponse was negligible.

\hypertarget{propensity-score-matching}{%
\subsubsection{Propensity Score Matching}\label{propensity-score-matching}}

Propensity score matching was performed in a grandparent's survey year which preceded the year when the transition was first reported by at least two years (aside from that choosing the smallest available gap between matching and transition). This served the purpose to ensure that the covariates used for matching were not affected by the event itself or its anticipation (i.e., when one's child was already pregnant with their first child; Greenland, 2003; Rosenbaum, 1984; VanderWeele et al., 2020). Propensity score matching was performed using the \emph{MatchIt} R package (Ho et al., 2011) with exact matching on gender combined with Mahalanobis distance matching on the propensity score. In total, four matchings were performed; two per sample (LISS; HRS) and two per control group (parents but not grandparents; nonparents). We matched 1:4 with replacement because of the relatively small pools of available non-grandparent controls. This meant that each grandparent was matched with four control observations in each matching procedure, and that control observations were allowed to be used multiple times for matching (i.e., duplicated in the analysis samples\footnote{In the LISS data, 250 grandparent observations were matched with 1000 control observations (matching with replacement); these control observations corresponded to 523 unique person-year observations stemming from 270 unique participants for the parent control group, and to 464 unique person-year observations stemming from 189 unique participants for the nonparent control group. In the HRS data, 846 grandparent observations were matched with 3384 control observations (matching with replacement); these control observations corresponded to 1393 unique person-year observations stemming from 982 unique participants for the parent control group, and to 1008 unique person-year observations stemming from 704 unique participants for the nonparent control group.}). We did not specify a caliper because our goal was to find matches for all grandparents, and because we achieved satisfactory covariate balance this way.\\
We evaluated the matching procedure in terms of covariate balance and, graphically, in terms of overlap of the distributions of the propensity scores and (non-categorical) covariates (Stuart, 2010). Covariate balance as indicated by the standardized difference in means between the grandparent and the controls after matching was satisfactory (see Tables \ref{tab:stddiffmeans-balance-liss} \& \ref{tab:stddiffmeans-balance-hrs}) lying below 0.25 as recommended in the literature (Stuart, 2010), and below 0.10 with few exceptions (Austin, 2011). Graphically, differences between the distributions of the propensity score and the covariates were also small and indicated no missing overlap (see Fig. SX).\\
After matching, each matched control observation received the same value as their matched grandparent in the \emph{time} variable describing the temporal relation to treatment, and the control subject's other longitudinal observations were centered around this matched observation. Thereby, we coded a counterfactual transition time frame for each control subject. Due to left- and right-censored longitudinal data (i.e., panel entry or attrition), we restricted the final analysis samples to six years before and six years after the transition as shown in Table \ref{tab:piecewise-coding-scheme}. We analyzed unbalanced panel data where not every participant provided all person-year observations. The final LISS analysis samples, thus, contained 250 grandparents with 1368 longitudinal observations, matched with 1000 control subjects with either 5167 (parent control group) or 5340 longitudinal observations (nonparent control group). The final HRS analysis samples contained 846 grandparents with 2262 longitudinal observations, matched with 3384 control subjects with either 8257 (parent control group) or 8167 longitudinal observations (nonparent control group; see Table \ref{tab:piecewise-coding-scheme}. In the HRS, there were a few additional missing values in the outcomes ranging from 18 to 105 longitudinal observations which will be listwise deleted in the respective analyses.

\hypertarget{analytical-strategy}{%
\subsection{Analytical Strategy}\label{analytical-strategy}}

We used R (Version 4.0.4; R Core Team, 2021) and the R-packages \emph{lme4} (Version 1.1.26; Bates et al., 2015), and \emph{lmerTest} (Version 3.1.3; Kuznetsova et al., 2017) for multilevel modeling, as well as \emph{tidyverse} (Wickham et al., 2019) for data wrangling, and \emph{papaja} (Aust \& Barth, 2020) for reproducible manuscript production. Additional modeling details and a list of all software we used is provided in the Supplemental Material. In line with Benjamin et al.~(2018), we set the \(\alpha\)-level for all confirmatory analyses to \(.005\).\\
Our design can be referred to as an interrupted time-series with a \enquote{nonequivalent no-treatment control group} (Shadish et al., 2002, p. 182) where treatment, that is, the transition to grandparenthood, is not deliberately manipulated. First, to analyze mean-level changes, we used linear piecewise regression coefficients in multilevel regression models with person-year observations nested within participants and households (Hoffman, 2015). To model change over time in relation to the birth of the first grandchild, we coded three piecewise regression coefficients: a \emph{before-slope} representing linear change in the years leading up to the transition to grandparenthood, an \emph{after-slope} representing linear change in the years after the transition, and a \emph{shift} coefficient shifting the intercept directly after the transition was first reported, thus representing sudden changes that go beyond changes already modeled by the \emph{after-slope} (see Table \ref{tab:piecewise-coding-scheme} for the coding scheme of these coefficients; Hoffman, 2015). Other studies of personality development have recently adopted similar piecewise growth-curve models (e.g., Bleidorn \& Schwaba, 2018; Krämer \& Rodgers, 2020; Schwaba \& Bleidorn, 2019; van Scheppingen \& Leopold, 2020).\\
All effects of the transition to grandparenthood on the Big Five and life satisfaction were modeled as deviations from patterns in the matched control groups by interacting the three piecewise coefficients with the binary treatment variable (0 = \emph{control}, 1 = \emph{grandparent}). In additional models, we interacted these coefficients with the binary moderator variables resulting in two- or three-way interactions. To test differences in the growth parameters between two groups in cases where these differences were represented by multiple fixed-effects coefficients, we defined linear contrasts using the \emph{linearHypothesis} command from the \emph{car} R package (Fox \& Weisberg, 2019). All models of mean-level changes were estimated using maximum likelihood and included random intercepts but no random slopes of the piecewise regression coefficients. We included the propensity score as a level-2 covariate for a double-robust approach (Austin, 2017).\\
Second, to assess interindividual differences in intraindividual change in the Big Five and life satisfaction we added random slopes to the models assessing mean-level changes (see Denissen et al., 2019 for a similar approach). In other words, we allowed for differences between individuals in their trajectories of change to be modeled, that is, differences in the \emph{before-slope}, \emph{after-slope}, and \emph{shift} coefficients. Because multiple simultaneous random slopes are often not computationally feasible, we added random slopes one at a time and used likelihood ratio test to determine whether the addition of the respective random slope led to a significant improvement in model fit. We plotted distributions of random slopes (for a similar approach, see Denissen et al., 2019; Doré \& Bolger, 2018). To statistically test differences in the random slope variance between the grandparent group and each control group, we respecified the multilevel models as heterogeneous variance models using the \emph{nlme} R package (Pinheiro et al., 2021), which allows for separate random slope variances to be estimated in the grandparent group and the control group within the same model. Model fit of these heterogeneous variance models was compared to the corresponding models with a homogeneous (single) random slope variance via likelihood ratio tests. This was also done separately for the parent and nonparent control groups.\\
Third, to examine rank-order stability in the Big Five and life satisfaction over the transition to grandparenthood, we computed the test-retest correlation of measurements prior to the transition to grandparenthood (at the time of matching) with the first available measurement after the transition. To test the difference in test-retest stability between grandparents and either of the control groups, we then entered the pre-treatment measure as well as the treatment variable (0 = \emph{control}, 1 = \emph{grandparent}) and their interaction into multiple regression models predicting the Big Five and life satisfaction. These interactions test for significant differences in the test-retest stability between those who experienced the transition to grandparenthood and those who did not (for a similar approach, see Denissen et al., 2019; McCrae, 1993).

\hypertarget{results}{%
\section{Results}\label{results}}




\begin{lltable}

\begin{TableNotes}[para]
\normalsize{\textit{Note.} Two models were computed for each of the two samples (LISS panel, HRS): grandparents matched with parent controls (models MXa) and with nonparent controls (models MXb). CI = confidence interval. \(R^2_{M1a} =\) 0.00, \(R^2_{M1b} =\) 0.00, \(R^2_{M2a} =\) 0.00, \(R^2_{M2b} =\) 0.00.}
\end{TableNotes}

\footnotesize{

\begin{longtable}{lrcrrrcrr}\noalign{\getlongtablewidth\global\LTcapwidth=\longtablewidth}
\caption{\label{tab:H1-agree-tab}Fixed Effects of Agreeableness Over the Transition to Grandparenthood.}\\
\toprule
 & \multicolumn{4}{c}{Parent controls} & \multicolumn{4}{c}{Nonparent controls} \\
\cmidrule(r){2-5} \cmidrule(r){6-9}
Parameter & $\hat{\gamma}$ & 95\% CI & $t$ & $p$ & $\hat{\gamma}$ & 95\% CI & $t$ & $p$\\
\midrule
\endfirsthead
\caption*{\normalfont{Table \ref{tab:H1-agree-tab} continued}}\\
\toprule
 & \multicolumn{4}{c}{Parent controls} & \multicolumn{4}{c}{Nonparent controls} \\
\cmidrule(r){2-5} \cmidrule(r){6-9}
Parameter & $\hat{\gamma}$ & 95\% CI & $t$ & $p$ & $\hat{\gamma}$ & 95\% CI & $t$ & $p$\\
\midrule
\endhead
LISS panel (M1a, M1b) &  &  &  &  &  &  &  & \\
\ \ \ Intercept, $\hat{\gamma}_{00}$ \textcolor{white}{L} & 3.86 & [3.80, 3.92] & 130.78 & < .001 & 3.90 & [3.83, 3.97] & 112.95 & < .001\\
\ \ \ Propensity score, $\hat{\gamma}_{02}$ \textcolor{white}{L} & -0.02 & [-0.10, 0.05] & -0.58 & .559 & -0.01 & [-0.08, 0.06] & -0.20 & .838\\
\ \ \ Before-slope, $\hat{\gamma}_{10}$ \textcolor{white}{L} & 0.00 & [-0.01, 0.00] & -0.28 & .779 & -0.01 & [-0.01, 0.00] & -1.81 & .070\\
\ \ \ After-slope, $\hat{\gamma}_{20}$ \textcolor{white}{L} & -0.02 & [-0.02, -0.01] & -6.75 & < .001 & -0.01 & [-0.01, 0.00] & -3.32 & .001\\
\ \ \ Shift, $\hat{\gamma}_{30}$ \textcolor{white}{L} & 0.04 & [0.01, 0.06] & 3.14 & .002 & 0.03 & [0.00, 0.05] & 1.98 & .048\\
\ \ \ Grandparent, $\hat{\gamma}_{01}$ \textcolor{white}{L} & 0.06 & [-0.03, 0.15] & 1.31 & .192 & 0.01 & [-0.08, 0.11] & 0.30 & .768\\
\ \ \ Before-slope * Grandparent, $\hat{\gamma}_{11}$ \textcolor{white}{L} & -0.01 & [-0.02, 0.01] & -1.01 & .312 & 0.00 & [-0.01, 0.01] & -0.26 & .792\\
\ \ \ After-slope * Grandparent, $\hat{\gamma}_{21}$ \textcolor{white}{L} & 0.02 & [0.01, 0.03] & 2.97 & .003 & 0.01 & [0.00, 0.02] & 1.44 & .149\\
\ \ \ Shift * Grandparent, $\hat{\gamma}_{31}$ \textcolor{white}{L} & -0.01 & [-0.07, 0.04] & -0.39 & .700 & 0.00 & [-0.06, 0.06] & 0.08 & .937\\
HRS (M2a, M2b) &  &  &  &  &  &  &  & \\
\ \ \ Intercept, $\hat{\gamma}_{00}$ \textcolor{white}{H} & 3.46 & [3.43, 3.50] & 196.32 & < .001 & 3.48 & [3.44, 3.52] & 166.19 & < .001\\
\ \ \ Propensity score, $\hat{\gamma}_{02}$ \textcolor{white}{H} & 0.08 & [0.02, 0.14] & 2.51 & .012 & 0.05 & [-0.01, 0.11] & 1.51 & .131\\
\ \ \ Before-slope, $\hat{\gamma}_{10}$ \textcolor{white}{H} & 0.01 & [0.00, 0.02] & 1.37 & .169 & -0.01 & [-0.02, 0.00] & -1.33 & .184\\
\ \ \ After-slope, $\hat{\gamma}_{20}$ \textcolor{white}{H} & -0.01 & [-0.02, 0.00] & -2.87 & .004 & -0.02 & [-0.02, -0.01] & -5.16 & < .001\\
\ \ \ Shift, $\hat{\gamma}_{30}$ \textcolor{white}{H} & 0.01 & [-0.01, 0.03] & 0.71 & .476 & 0.04 & [0.02, 0.06] & 4.30 & < .001\\
\ \ \ Grandparent, $\hat{\gamma}_{01}$ \textcolor{white}{H} & 0.02 & [-0.03, 0.08] & 0.88 & .378 & 0.01 & [-0.04, 0.07] & 0.44 & .662\\
\ \ \ Before-slope * Grandparent, $\hat{\gamma}_{11}$ \textcolor{white}{H} & -0.01 & [-0.04, 0.01] & -0.87 & .384 & 0.00 & [-0.02, 0.03] & 0.28 & .781\\
\ \ \ After-slope * Grandparent, $\hat{\gamma}_{21}$ \textcolor{white}{H} & 0.01 & [0.00, 0.03] & 1.71 & .088 & 0.02 & [0.01, 0.04] & 2.78 & .006\\
\ \ \ Shift * Grandparent, $\hat{\gamma}_{31}$ \textcolor{white}{H} & -0.01 & [-0.05, 0.04] & -0.35 & .729 & -0.04 & [-0.09, 0.00] & -1.97 & .049\\
\bottomrule
\addlinespace
\insertTableNotes
\end{longtable}

}

\end{lltable}




\begin{lltable}

\begin{TableNotes}[para]
\normalsize{\textit{Note.} Two models were computed for each of the two samples (LISS panel, HRS): grandparents matched with parent controls (models MXa) and with nonparent controls (models MXb). CI = confidence interval. \(R^2_{M1a} =\) 0.12, \(R^2_{M1b} =\) 0.11, \(R^2_{M2a} =\) 0.08, \(R^2_{M2b} =\) 0.07.}
\end{TableNotes}

\footnotesize{

\begin{longtable}{lrcrrrcrr}\noalign{\getlongtablewidth\global\LTcapwidth=\longtablewidth}
\caption{\label{tab:H1-agree-gender-tab}Fixed Effects of Agreeableness Over the Transition to Grandparenthood Moderated by Gender.}\\
\toprule
 & \multicolumn{4}{c}{Parent controls} & \multicolumn{4}{c}{Nonparent controls} \\
\cmidrule(r){2-5} \cmidrule(r){6-9}
Parameter & $\hat{\gamma}$ & 95\% CI & $t$ & $p$ & $\hat{\gamma}$ & 95\% CI & $t$ & $p$\\
\midrule
\endfirsthead
\caption*{\normalfont{Table \ref{tab:H1-agree-gender-tab} continued}}\\
\toprule
 & \multicolumn{4}{c}{Parent controls} & \multicolumn{4}{c}{Nonparent controls} \\
\cmidrule(r){2-5} \cmidrule(r){6-9}
Parameter & $\hat{\gamma}$ & 95\% CI & $t$ & $p$ & $\hat{\gamma}$ & 95\% CI & $t$ & $p$\\
\midrule
\endhead
LISS panel (M1a, M1b) &  &  &  &  &  &  &  & \\
\ \ \ Intercept, $\hat{\gamma}_{00}$ \textcolor{white}{L} & 3.65 & [3.58, 3.73] & 93.05 & < .001 & 3.66 & [3.57, 3.75] & 79.71 & < .001\\
\ \ \ Propensity score, $\hat{\gamma}_{04}$ \textcolor{white}{L} & -0.01 & [-0.08, 0.06] & -0.24 & .810 & 0.02 & [-0.05, 0.08] & 0.45 & .652\\
\ \ \ Before-slope, $\hat{\gamma}_{10}$ \textcolor{white}{L} & 0.00 & [-0.01, 0.01] & 0.02 & .983 & 0.00 & [-0.01, 0.01] & -0.37 & .714\\
\ \ \ After-slope, $\hat{\gamma}_{20}$ \textcolor{white}{L} & -0.02 & [-0.03, -0.02] & -6.38 & < .001 & -0.01 & [-0.02, 0.00] & -2.49 & .013\\
\ \ \ Shift, $\hat{\gamma}_{30}$ \textcolor{white}{L} & 0.03 & [-0.01, 0.07] & 1.66 & .096 & 0.07 & [0.03, 0.11] & 3.66 & < .001\\
\ \ \ Grandparent, $\hat{\gamma}_{01}$ \textcolor{white}{L} & 0.06 & [-0.06, 0.17] & 0.95 & .340 & 0.04 & [-0.09, 0.16] & 0.59 & .553\\
\ \ \ Female, $\hat{\gamma}_{02}$ \textcolor{white}{L} & 0.38 & [0.28, 0.48] & 7.26 & < .001 & 0.44 & [0.32, 0.56] & 7.11 & < .001\\
\ \ \ Before-slope * Grandparent, $\hat{\gamma}_{11}$ \textcolor{white}{L} & -0.01 & [-0.03, 0.01] & -0.73 & .463 & -0.01 & [-0.02, 0.01] & -0.51 & .614\\
\ \ \ After-slope * Grandparent, $\hat{\gamma}_{21}$ \textcolor{white}{L} & 0.03 & [0.01, 0.04] & 3.43 & .001 & 0.01 & [0.00, 0.03] & 1.64 & .101\\
\ \ \ Shift * Grandparent, $\hat{\gamma}_{31}$ \textcolor{white}{L} & -0.01 & [-0.09, 0.07] & -0.33 & .740 & -0.05 & [-0.14, 0.03] & -1.23 & .218\\
\ \ \ Before-slope * Female, $\hat{\gamma}_{12}$ \textcolor{white}{L} & 0.00 & [-0.01, 0.01] & -0.28 & .782 & -0.01 & [-0.02, 0.00] & -1.14 & .253\\
\ \ \ After-slope * Female, $\hat{\gamma}_{22}$ \textcolor{white}{L} & 0.01 & [0.00, 0.02] & 2.36 & .018 & 0.00 & [-0.01, 0.01] & 0.28 & .781\\
\ \ \ Shift * Female, $\hat{\gamma}_{32}$ \textcolor{white}{L} & 0.02 & [-0.03, 0.07] & 0.62 & .535 & -0.08 & [-0.14, -0.03] & -3.18 & .001\\
\ \ \ Grandparent * Female, $\hat{\gamma}_{03}$ \textcolor{white}{L} & 0.01 & [-0.15, 0.16] & 0.08 & .936 & -0.05 & [-0.22, 0.12] & -0.57 & .566\\
\ \ \ Before-slope * Grandparent * Female, $\hat{\gamma}_{13}$ \textcolor{white}{L} & 0.00 & [-0.02, 0.02] & 0.00 & .997 & 0.00 & [-0.02, 0.03] & 0.36 & .721\\
\ \ \ After-slope * Grandparent * Female, $\hat{\gamma}_{23}$ \textcolor{white}{L} & -0.02 & [-0.04, 0.00] & -1.94 & .052 & -0.01 & [-0.03, 0.01] & -0.94 & .349\\
\ \ \ Shift * Grandparent * Female, $\hat{\gamma}_{33}$ \textcolor{white}{L} & 0.01 & [-0.10, 0.12] & 0.18 & .855 & 0.11 & [-0.01, 0.23] & 1.87 & .061\\
HRS (M2a, M2b) &  &  &  &  &  &  &  & \\
\ \ \ Intercept, $\hat{\gamma}_{00}$ \textcolor{white}{H} & 3.27 & [3.22, 3.32] & 132.92 & < .001 & 3.38 & [3.33, 3.43] & 122.35 & < .001\\
\ \ \ Propensity score, $\hat{\gamma}_{04}$ \textcolor{white}{H} & 0.09 & [0.03, 0.15] & 2.90 & .004 & 0.04 & [-0.03, 0.10] & 1.12 & .261\\
\ \ \ Before-slope, $\hat{\gamma}_{10}$ \textcolor{white}{H} & 0.02 & [0.01, 0.04] & 2.99 & .003 & -0.01 & [-0.02, 0.01] & -1.12 & .262\\
\ \ \ After-slope, $\hat{\gamma}_{20}$ \textcolor{white}{H} & -0.02 & [-0.03, -0.01] & -3.94 & < .001 & -0.02 & [-0.03, -0.01] & -3.43 & .001\\
\ \ \ Shift, $\hat{\gamma}_{30}$ \textcolor{white}{H} & 0.04 & [0.01, 0.07] & 2.76 & .006 & 0.03 & [0.00, 0.06] & 1.68 & .093\\
\ \ \ Grandparent, $\hat{\gamma}_{01}$ \textcolor{white}{H} & 0.08 & [0.00, 0.16] & 2.06 & .040 & -0.01 & [-0.09, 0.08] & -0.16 & .877\\
\ \ \ Female, $\hat{\gamma}_{02}$ \textcolor{white}{H} & 0.33 & [0.27, 0.39] & 10.85 & < .001 & 0.20 & [0.13, 0.26] & 5.76 & < .001\\
\ \ \ Before-slope * Grandparent, $\hat{\gamma}_{11}$ \textcolor{white}{H} & -0.04 & [-0.08, 0.00] & -2.18 & .029 & -0.01 & [-0.04, 0.03] & -0.47 & .640\\
\ \ \ After-slope * Grandparent, $\hat{\gamma}_{21}$ \textcolor{white}{H} & 0.04 & [0.01, 0.06] & 3.00 & .003 & 0.03 & [0.01, 0.05] & 2.85 & .004\\
\ \ \ Shift * Grandparent, $\hat{\gamma}_{31}$ \textcolor{white}{H} & -0.05 & [-0.12, 0.02] & -1.50 & .134 & -0.03 & [-0.10, 0.03] & -1.04 & .298\\
\ \ \ Before-slope * Female, $\hat{\gamma}_{12}$ \textcolor{white}{H} & -0.03 & [-0.05, -0.01] & -2.86 & .004 & 0.00 & [-0.02, 0.02] & 0.38 & .702\\
\ \ \ After-slope * Female, $\hat{\gamma}_{22}$ \textcolor{white}{H} & 0.02 & [0.01, 0.03] & 2.74 & .006 & 0.00 & [-0.01, 0.01] & 0.08 & .937\\
\ \ \ Shift * Female, $\hat{\gamma}_{32}$ \textcolor{white}{H} & -0.06 & [-0.11, -0.02] & -3.06 & .002 & 0.03 & [-0.01, 0.07] & 1.50 & .134\\
\ \ \ Grandparent * Female, $\hat{\gamma}_{03}$ \textcolor{white}{H} & -0.10 & [-0.20, 0.01] & -1.86 & .064 & 0.03 & [-0.07, 0.14] & 0.64 & .521\\
\ \ \ Before-slope * Grandparent * Female, $\hat{\gamma}_{13}$ \textcolor{white}{H} & 0.06 & [0.01, 0.11] & 2.20 & .028 & 0.02 & [-0.03, 0.07] & 0.86 & .392\\
\ \ \ After-slope * Grandparent * Female, $\hat{\gamma}_{23}$ \textcolor{white}{H} & -0.04 & [-0.07, -0.01] & -2.48 & .013 & -0.02 & [-0.05, 0.01] & -1.34 & .180\\
\ \ \ Shift * Grandparent * Female, $\hat{\gamma}_{33}$ \textcolor{white}{H} & 0.08 & [-0.01, 0.17] & 1.73 & .084 & -0.01 & [-0.10, 0.07] & -0.31 & .758\\
\bottomrule
\addlinespace
\insertTableNotes
\end{longtable}

}

\end{lltable}




\begin{lltable}

\begin{TableNotes}[para]
\normalsize{\textit{Note.} Two models were computed for each of the two samples (LISS panel, HRS): grandparents matched with parent controls (models MXa) and with nonparent controls (models MXb). CI = confidence interval. \(R^2_{M1a} =\) 0.00, \(R^2_{M1b} =\) 0.00, \(R^2_{M2a} =\) 0.00, \(R^2_{M2b} =\) 0.02.}
\end{TableNotes}

\footnotesize{

\begin{longtable}{lrcrrrcrr}\noalign{\getlongtablewidth\global\LTcapwidth=\longtablewidth}
\caption{\label{tab:H1-con-tab}Fixed Effects of Conscientiousness Over the Transition to Grandparenthood.}\\
\toprule
 & \multicolumn{4}{c}{Parent controls} & \multicolumn{4}{c}{Nonparent controls} \\
\cmidrule(r){2-5} \cmidrule(r){6-9}
Parameter & $\hat{\gamma}$ & 95\% CI & $t$ & $p$ & $\hat{\gamma}$ & 95\% CI & $t$ & $p$\\
\midrule
\endfirsthead
\caption*{\normalfont{Table \ref{tab:H1-con-tab} continued}}\\
\toprule
 & \multicolumn{4}{c}{Parent controls} & \multicolumn{4}{c}{Nonparent controls} \\
\cmidrule(r){2-5} \cmidrule(r){6-9}
Parameter & $\hat{\gamma}$ & 95\% CI & $t$ & $p$ & $\hat{\gamma}$ & 95\% CI & $t$ & $p$\\
\midrule
\endhead
LISS panel (M1a, M1b) &  &  &  &  &  &  &  & \\
\ \ \ Intercept, $\hat{\gamma}_{00}$ \textcolor{white}{L} & 3.77 & [3.72, 3.83] & 130.16 & < .001 & 3.82 & [3.75, 3.88] & 112.04 & < .001\\
\ \ \ Propensity score, $\hat{\gamma}_{02}$ \textcolor{white}{L} & 0.00 & [-0.08, 0.08] & -0.02 & .986 & 0.01 & [-0.06, 0.08] & 0.24 & .813\\
\ \ \ Before-slope, $\hat{\gamma}_{10}$ \textcolor{white}{L} & 0.00 & [-0.01, 0.00] & -0.84 & .399 & 0.00 & [-0.01, 0.01] & -0.26 & .796\\
\ \ \ After-slope, $\hat{\gamma}_{20}$ \textcolor{white}{L} & -0.02 & [-0.02, -0.01] & -6.17 & < .001 & 0.01 & [0.00, 0.01] & 3.45 & .001\\
\ \ \ Shift, $\hat{\gamma}_{30}$ \textcolor{white}{L} & 0.04 & [0.02, 0.07] & 3.15 & .002 & 0.00 & [-0.03, 0.02] & -0.15 & .881\\
\ \ \ Grandparent, $\hat{\gamma}_{01}$ \textcolor{white}{L} & -0.01 & [-0.10, 0.08] & -0.24 & .811 & -0.06 & [-0.15, 0.04] & -1.22 & .225\\
\ \ \ Before-slope * Grandparent, $\hat{\gamma}_{11}$ \textcolor{white}{L} & 0.00 & [-0.01, 0.02] & 0.78 & .437 & 0.00 & [-0.01, 0.02] & 0.50 & .617\\
\ \ \ After-slope * Grandparent, $\hat{\gamma}_{21}$ \textcolor{white}{L} & 0.02 & [0.00, 0.03] & 2.73 & .006 & -0.01 & [-0.02, 0.00] & -1.62 & .106\\
\ \ \ Shift * Grandparent, $\hat{\gamma}_{31}$ \textcolor{white}{L} & -0.04 & [-0.10, 0.01] & -1.48 & .138 & 0.00 & [-0.06, 0.06] & 0.02 & .986\\
HRS (M2a, M2b) &  &  &  &  &  &  &  & \\
\ \ \ Intercept, $\hat{\gamma}_{00}$ \textcolor{white}{H} & 3.41 & [3.38, 3.44] & 206.26 & < .001 & 3.35 & [3.31, 3.38] & 172.70 & < .001\\
\ \ \ Propensity score, $\hat{\gamma}_{02}$ \textcolor{white}{H} & 0.08 & [0.03, 0.14] & 2.86 & .004 & 0.17 & [0.11, 0.23] & 5.74 & < .001\\
\ \ \ Before-slope, $\hat{\gamma}_{10}$ \textcolor{white}{H} & 0.00 & [-0.01, 0.01] & 0.31 & .754 & 0.00 & [-0.01, 0.01] & 0.72 & .473\\
\ \ \ After-slope, $\hat{\gamma}_{20}$ \textcolor{white}{H} & -0.01 & [-0.02, -0.01] & -4.11 & < .001 & -0.01 & [-0.02, -0.01] & -3.84 & < .001\\
\ \ \ Shift, $\hat{\gamma}_{30}$ \textcolor{white}{H} & 0.02 & [0.00, 0.04] & 1.93 & .053 & 0.00 & [-0.02, 0.02] & 0.01 & .991\\
\ \ \ Grandparent, $\hat{\gamma}_{01}$ \textcolor{white}{H} & 0.02 & [-0.04, 0.07] & 0.60 & .547 & 0.03 & [-0.02, 0.08] & 1.08 & .280\\
\ \ \ Before-slope * Grandparent, $\hat{\gamma}_{11}$ \textcolor{white}{H} & 0.01 & [-0.02, 0.03] & 0.55 & .580 & 0.00 & [-0.02, 0.03] & 0.43 & .664\\
\ \ \ After-slope * Grandparent, $\hat{\gamma}_{21}$ \textcolor{white}{H} & 0.02 & [0.01, 0.04] & 3.06 & .002 & 0.02 & [0.01, 0.04] & 3.01 & .003\\
\ \ \ Shift * Grandparent, $\hat{\gamma}_{31}$ \textcolor{white}{H} & -0.05 & [-0.09, -0.01] & -2.36 & .018 & -0.03 & [-0.07, 0.01] & -1.59 & .111\\
\bottomrule
\addlinespace
\insertTableNotes
\end{longtable}

}

\end{lltable}




\begin{lltable}

\begin{TableNotes}[para]
\normalsize{\textit{Note.} Two models were computed for each of the two samples (LISS panel, HRS): grandparents matched with parent controls (models MXa) and with nonparent controls (models MXb). CI = confidence interval. \(R^2_{M1a} =\) 0.00, \(R^2_{M1b} =\) 0.01, \(R^2_{M2a} =\) 0.02, \(R^2_{M2b} =\) 0.04.}
\end{TableNotes}

\footnotesize{

\begin{longtable}{lrcrrrcrr}\noalign{\getlongtablewidth\global\LTcapwidth=\longtablewidth}
\caption{\label{tab:H1-con-gender-tab}Fixed Effects of Conscientiousness Over the Transition to Grandparenthood Moderated by Gender.}\\
\toprule
 & \multicolumn{4}{c}{Parent controls} & \multicolumn{4}{c}{Nonparent controls} \\
\cmidrule(r){2-5} \cmidrule(r){6-9}
Parameter & $\hat{\gamma}$ & 95\% CI & $t$ & $p$ & $\hat{\gamma}$ & 95\% CI & $t$ & $p$\\
\midrule
\endfirsthead
\caption*{\normalfont{Table \ref{tab:H1-con-gender-tab} continued}}\\
\toprule
 & \multicolumn{4}{c}{Parent controls} & \multicolumn{4}{c}{Nonparent controls} \\
\cmidrule(r){2-5} \cmidrule(r){6-9}
Parameter & $\hat{\gamma}$ & 95\% CI & $t$ & $p$ & $\hat{\gamma}$ & 95\% CI & $t$ & $p$\\
\midrule
\endhead
LISS panel (M1a, M1b) &  &  &  &  &  &  &  & \\
\ \ \ Intercept, $\hat{\gamma}_{00}$ \textcolor{white}{L} & 3.69 & [3.60, 3.77] & 87.26 & < .001 & 3.70 & [3.61, 3.80] & 75.83 & < .001\\
\ \ \ Propensity score, $\hat{\gamma}_{04}$ \textcolor{white}{L} & 0.00 & [-0.08, 0.07] & -0.03 & .975 & 0.01 & [-0.06, 0.08] & 0.34 & .731\\
\ \ \ Before-slope, $\hat{\gamma}_{10}$ \textcolor{white}{L} & 0.00 & [-0.01, 0.01] & 0.64 & .524 & 0.00 & [-0.01, 0.01] & 0.75 & .455\\
\ \ \ After-slope, $\hat{\gamma}_{20}$ \textcolor{white}{L} & -0.01 & [-0.02, -0.01] & -3.43 & .001 & 0.00 & [0.00, 0.01] & 0.71 & .477\\
\ \ \ Shift, $\hat{\gamma}_{30}$ \textcolor{white}{L} & 0.04 & [0.00, 0.08] & 2.16 & .031 & 0.00 & [-0.03, 0.04] & 0.14 & .892\\
\ \ \ Grandparent, $\hat{\gamma}_{01}$ \textcolor{white}{L} & 0.03 & [-0.09, 0.16] & 0.48 & .633 & 0.01 & [-0.13, 0.14] & 0.11 & .909\\
\ \ \ Female, $\hat{\gamma}_{02}$ \textcolor{white}{L} & 0.16 & [0.05, 0.27] & 2.89 & .004 & 0.22 & [0.09, 0.34] & 3.26 & .001\\
\ \ \ Before-slope * Grandparent, $\hat{\gamma}_{11}$ \textcolor{white}{L} & 0.00 & [-0.02, 0.02] & -0.01 & .994 & 0.00 & [-0.02, 0.02] & -0.06 & .953\\
\ \ \ After-slope * Grandparent, $\hat{\gamma}_{21}$ \textcolor{white}{L} & 0.02 & [0.00, 0.04] & 2.53 & .011 & 0.01 & [-0.01, 0.02] & 0.65 & .513\\
\ \ \ Shift * Grandparent, $\hat{\gamma}_{31}$ \textcolor{white}{L} & -0.04 & [-0.13, 0.04] & -1.07 & .286 & -0.01 & [-0.09, 0.08] & -0.14 & .886\\
\ \ \ Before-slope * Female, $\hat{\gamma}_{12}$ \textcolor{white}{L} & -0.01 & [-0.02, 0.00] & -1.62 & .106 & -0.01 & [-0.02, 0.00] & -1.23 & .218\\
\ \ \ After-slope * Female, $\hat{\gamma}_{22}$ \textcolor{white}{L} & -0.01 & [-0.02, 0.00] & -1.11 & .269 & 0.01 & [0.00, 0.02] & 2.38 & .017\\
\ \ \ Shift * Female, $\hat{\gamma}_{32}$ \textcolor{white}{L} & 0.00 & [-0.05, 0.05] & -0.03 & .973 & -0.01 & [-0.06, 0.04] & -0.41 & .683\\
\ \ \ Grandparent * Female, $\hat{\gamma}_{03}$ \textcolor{white}{L} & -0.07 & [-0.24, 0.10] & -0.81 & .416 & -0.12 & [-0.30, 0.06] & -1.30 & .193\\
\ \ \ Before-slope * Grandparent * Female, $\hat{\gamma}_{13}$ \textcolor{white}{L} & 0.01 & [-0.02, 0.03] & 0.61 & .540 & 0.01 & [-0.02, 0.03] & 0.44 & .663\\
\ \ \ After-slope * Grandparent * Female, $\hat{\gamma}_{23}$ \textcolor{white}{L} & -0.01 & [-0.03, 0.01] & -0.85 & .397 & -0.03 & [-0.05, 0.00] & -2.38 & .018\\
\ \ \ Shift * Grandparent * Female, $\hat{\gamma}_{33}$ \textcolor{white}{L} & 0.01 & [-0.10, 0.12] & 0.11 & .912 & 0.02 & [-0.10, 0.13] & 0.28 & .781\\
HRS (M2a, M2b) &  &  &  &  &  &  &  & \\
\ \ \ Intercept, $\hat{\gamma}_{00}$ \textcolor{white}{H} & 3.35 & [3.30, 3.39] & 143.72 & < .001 & 3.26 & [3.21, 3.31] & 124.79 & < .001\\
\ \ \ Propensity score, $\hat{\gamma}_{04}$ \textcolor{white}{H} & 0.09 & [0.03, 0.14] & 3.00 & .003 & 0.17 & [0.11, 0.23] & 5.65 & < .001\\
\ \ \ Before-slope, $\hat{\gamma}_{10}$ \textcolor{white}{H} & 0.01 & [-0.01, 0.02] & 1.19 & .234 & 0.01 & [0.00, 0.03] & 2.08 & .037\\
\ \ \ After-slope, $\hat{\gamma}_{20}$ \textcolor{white}{H} & -0.01 & [-0.02, 0.00] & -2.42 & .016 & 0.00 & [-0.01, 0.01] & -0.10 & .920\\
\ \ \ Shift, $\hat{\gamma}_{30}$ \textcolor{white}{H} & 0.02 & [-0.01, 0.05] & 1.18 & .237 & -0.01 & [-0.04, 0.02] & -0.74 & .462\\
\ \ \ Grandparent, $\hat{\gamma}_{01}$ \textcolor{white}{H} & -0.03 & [-0.10, 0.05] & -0.74 & .461 & 0.01 & [-0.07, 0.09] & 0.28 & .780\\
\ \ \ Female, $\hat{\gamma}_{02}$ \textcolor{white}{H} & 0.11 & [0.05, 0.17] & 3.81 & < .001 & 0.15 & [0.09, 0.22] & 4.67 & < .001\\
\ \ \ Before-slope * Grandparent, $\hat{\gamma}_{11}$ \textcolor{white}{H} & 0.01 & [-0.02, 0.05] & 0.74 & .460 & 0.01 & [-0.03, 0.04] & 0.45 & .651\\
\ \ \ After-slope * Grandparent, $\hat{\gamma}_{21}$ \textcolor{white}{H} & 0.03 & [0.01, 0.05] & 2.64 & .008 & 0.02 & [0.00, 0.04] & 1.71 & .088\\
\ \ \ Shift * Grandparent, $\hat{\gamma}_{31}$ \textcolor{white}{H} & -0.08 & [-0.15, -0.02] & -2.57 & .010 & -0.06 & [-0.12, 0.00] & -1.85 & .064\\
\ \ \ Before-slope * Female, $\hat{\gamma}_{12}$ \textcolor{white}{H} & -0.01 & [-0.03, 0.01] & -1.34 & .180 & -0.02 & [-0.04, 0.00] & -2.16 & .031\\
\ \ \ After-slope * Female, $\hat{\gamma}_{22}$ \textcolor{white}{H} & 0.00 & [-0.02, 0.01] & -0.39 & .695 & -0.02 & [-0.03, -0.01] & -3.05 & .002\\
\ \ \ Shift * Female, $\hat{\gamma}_{32}$ \textcolor{white}{H} & 0.00 & [-0.04, 0.04] & 0.13 & .895 & 0.02 & [-0.02, 0.05] & 0.92 & .356\\
\ \ \ Grandparent * Female, $\hat{\gamma}_{03}$ \textcolor{white}{H} & 0.08 & [-0.02, 0.18] & 1.64 & .101 & 0.03 & [-0.07, 0.13] & 0.62 & .538\\
\ \ \ Before-slope * Grandparent * Female, $\hat{\gamma}_{13}$ \textcolor{white}{H} & -0.01 & [-0.06, 0.03] & -0.47 & .637 & 0.00 & [-0.05, 0.04] & -0.21 & .836\\
\ \ \ After-slope * Grandparent * Female, $\hat{\gamma}_{23}$ \textcolor{white}{H} & -0.01 & [-0.04, 0.02] & -0.79 & .428 & 0.00 & [-0.02, 0.03] & 0.29 & .770\\
\ \ \ Shift * Grandparent * Female, $\hat{\gamma}_{33}$ \textcolor{white}{H} & 0.06 & [-0.03, 0.14] & 1.34 & .181 & 0.05 & [-0.04, 0.13] & 1.11 & .269\\
\bottomrule
\addlinespace
\insertTableNotes
\end{longtable}

}

\end{lltable}




\begin{lltable}

\begin{TableNotes}[para]
\normalsize{\textit{Note.} Two models were computed for each of the two samples (LISS panel, HRS): grandparents matched with parent controls (models MXa) and with nonparent controls (models MXb). CI = confidence interval. \(R^2_{M1a} =\) 0.00, \(R^2_{M1b} =\) 0.00, \(R^2_{M2a} =\) 0.00, \(R^2_{M2b} =\) 0.00.}
\end{TableNotes}

\footnotesize{

\begin{longtable}{lrcrrrcrr}\noalign{\getlongtablewidth\global\LTcapwidth=\longtablewidth}
\caption{\label{tab:H1-extra-tab}Fixed Effects of Extraversion Over the Transition to Grandparenthood.}\\
\toprule
 & \multicolumn{4}{c}{Parent controls} & \multicolumn{4}{c}{Nonparent controls} \\
\cmidrule(r){2-5} \cmidrule(r){6-9}
Parameter & $\hat{\gamma}$ & 95\% CI & $t$ & $p$ & $\hat{\gamma}$ & 95\% CI & $t$ & $p$\\
\midrule
\endfirsthead
\caption*{\normalfont{Table \ref{tab:H1-extra-tab} continued}}\\
\toprule
 & \multicolumn{4}{c}{Parent controls} & \multicolumn{4}{c}{Nonparent controls} \\
\cmidrule(r){2-5} \cmidrule(r){6-9}
Parameter & $\hat{\gamma}$ & 95\% CI & $t$ & $p$ & $\hat{\gamma}$ & 95\% CI & $t$ & $p$\\
\midrule
\endhead
LISS panel (M1a, M1b) &  &  &  &  &  &  &  & \\
\ \ \ Intercept, $\hat{\gamma}_{00}$ \textcolor{white}{L} & 3.25 & [3.18, 3.33] & 87.38 & < .001 & 3.29 & [3.20, 3.39] & 67.72 & < .001\\
\ \ \ Propensity score, $\hat{\gamma}_{02}$ \textcolor{white}{L} & -0.01 & [-0.10, 0.07] & -0.27 & .788 & 0.01 & [-0.07, 0.08] & 0.18 & .860\\
\ \ \ Before-slope, $\hat{\gamma}_{10}$ \textcolor{white}{L} & -0.01 & [-0.01, 0.00] & -1.80 & .071 & 0.00 & [0.00, 0.01] & 0.65 & .515\\
\ \ \ After-slope, $\hat{\gamma}_{20}$ \textcolor{white}{L} & 0.00 & [-0.01, 0.00] & -1.47 & .141 & -0.01 & [-0.02, 0.00] & -3.62 & < .001\\
\ \ \ Shift, $\hat{\gamma}_{30}$ \textcolor{white}{L} & -0.01 & [-0.04, 0.01] & -0.98 & .326 & -0.01 & [-0.03, 0.02] & -0.41 & .683\\
\ \ \ Grandparent, $\hat{\gamma}_{01}$ \textcolor{white}{L} & 0.06 & [-0.05, 0.17] & 1.01 & .311 & 0.01 & [-0.12, 0.14] & 0.19 & .849\\
\ \ \ Before-slope * Grandparent, $\hat{\gamma}_{11}$ \textcolor{white}{L} & 0.00 & [-0.02, 0.01] & -0.36 & .721 & -0.01 & [-0.02, 0.00] & -1.44 & .150\\
\ \ \ After-slope * Grandparent, $\hat{\gamma}_{21}$ \textcolor{white}{L} & 0.00 & [-0.01, 0.02] & 0.55 & .579 & 0.01 & [0.00, 0.02] & 1.45 & .146\\
\ \ \ Shift * Grandparent, $\hat{\gamma}_{31}$ \textcolor{white}{L} & -0.02 & [-0.08, 0.05] & -0.51 & .609 & -0.02 & [-0.08, 0.04] & -0.73 & .467\\
HRS (M2a, M2b) &  &  &  &  &  &  &  & \\
\ \ \ Intercept, $\hat{\gamma}_{00}$ \textcolor{white}{H} & 3.20 & [3.16, 3.24] & 159.82 & < .001 & 3.11 & [3.06, 3.16] & 132.37 & < .001\\
\ \ \ Propensity score, $\hat{\gamma}_{02}$ \textcolor{white}{H} & 0.02 & [-0.05, 0.08] & 0.56 & .577 & 0.05 & [-0.02, 0.12] & 1.45 & .146\\
\ \ \ Before-slope, $\hat{\gamma}_{10}$ \textcolor{white}{H} & 0.00 & [-0.01, 0.01] & -0.52 & .604 & 0.01 & [-0.01, 0.02] & 0.99 & .321\\
\ \ \ After-slope, $\hat{\gamma}_{20}$ \textcolor{white}{H} & 0.00 & [-0.01, 0.01] & -0.64 & .520 & 0.00 & [-0.01, 0.01] & -0.35 & .727\\
\ \ \ Shift, $\hat{\gamma}_{30}$ \textcolor{white}{H} & 0.02 & [0.00, 0.04] & 1.68 & .093 & 0.01 & [-0.01, 0.03] & 1.07 & .286\\
\ \ \ Grandparent, $\hat{\gamma}_{01}$ \textcolor{white}{H} & 0.00 & [-0.06, 0.06] & 0.05 & .957 & 0.07 & [0.01, 0.14] & 2.22 & .026\\
\ \ \ Before-slope * Grandparent, $\hat{\gamma}_{11}$ \textcolor{white}{H} & 0.00 & [-0.02, 0.03] & 0.31 & .757 & 0.00 & [-0.03, 0.02] & -0.35 & .728\\
\ \ \ After-slope * Grandparent, $\hat{\gamma}_{21}$ \textcolor{white}{H} & 0.01 & [0.00, 0.03] & 1.46 & .143 & 0.01 & [0.00, 0.03] & 1.38 & .166\\
\ \ \ Shift * Grandparent, $\hat{\gamma}_{31}$ \textcolor{white}{H} & -0.04 & [-0.09, 0.01] & -1.55 & .121 & -0.03 & [-0.08, 0.02] & -1.31 & .191\\
\bottomrule
\addlinespace
\insertTableNotes
\end{longtable}

}

\end{lltable}




\begin{lltable}

\begin{TableNotes}[para]
\normalsize{\textit{Note.} Two models were computed for each of the two samples (LISS panel, HRS): grandparents matched with parent controls (models MXa) and with nonparent controls (models MXb). CI = confidence interval. \(R^2_{M1a} =\) 0.00, \(R^2_{M1b} =\) 0.00, \(R^2_{M2a} =\) 0.00, \(R^2_{M2b} =\) 0.01.}
\end{TableNotes}

\footnotesize{

\begin{longtable}{lrcrrrcrr}\noalign{\getlongtablewidth\global\LTcapwidth=\longtablewidth}
\caption{\label{tab:H1-extra-gender-tab}Fixed Effects of Extraversion Over the Transition to Grandparenthood Moderated by Gender.}\\
\toprule
 & \multicolumn{4}{c}{Parent controls} & \multicolumn{4}{c}{Nonparent controls} \\
\cmidrule(r){2-5} \cmidrule(r){6-9}
Parameter & $\hat{\gamma}$ & 95\% CI & $t$ & $p$ & $\hat{\gamma}$ & 95\% CI & $t$ & $p$\\
\midrule
\endfirsthead
\caption*{\normalfont{Table \ref{tab:H1-extra-gender-tab} continued}}\\
\toprule
 & \multicolumn{4}{c}{Parent controls} & \multicolumn{4}{c}{Nonparent controls} \\
\cmidrule(r){2-5} \cmidrule(r){6-9}
Parameter & $\hat{\gamma}$ & 95\% CI & $t$ & $p$ & $\hat{\gamma}$ & 95\% CI & $t$ & $p$\\
\midrule
\endhead
LISS panel (M1a, M1b) &  &  &  &  &  &  &  & \\
\ \ \ Intercept, $\hat{\gamma}_{00}$ \textcolor{white}{L} & 3.28 & [3.18, 3.39] & 60.21 & < .001 & 3.22 & [3.08, 3.35] & 46.79 & < .001\\
\ \ \ Propensity score, $\hat{\gamma}_{04}$ \textcolor{white}{L} & -0.01 & [-0.09, 0.08] & -0.15 & .877 & 0.01 & [-0.06, 0.09] & 0.30 & .765\\
\ \ \ Before-slope, $\hat{\gamma}_{10}$ \textcolor{white}{L} & -0.01 & [-0.02, 0.00] & -1.82 & .069 & 0.02 & [0.01, 0.03] & 4.00 & < .001\\
\ \ \ After-slope, $\hat{\gamma}_{20}$ \textcolor{white}{L} & -0.01 & [-0.02, 0.00] & -2.56 & .011 & 0.00 & [-0.01, 0.00] & -1.08 & .280\\
\ \ \ Shift, $\hat{\gamma}_{30}$ \textcolor{white}{L} & -0.04 & [-0.08, 0.01] & -1.68 & .094 & -0.05 & [-0.09, -0.01] & -2.43 & .015\\
\ \ \ Grandparent, $\hat{\gamma}_{01}$ \textcolor{white}{L} & 0.01 & [-0.15, 0.17] & 0.11 & .914 & 0.07 & [-0.11, 0.26] & 0.78 & .435\\
\ \ \ Female, $\hat{\gamma}_{02}$ \textcolor{white}{L} & -0.06 & [-0.20, 0.09] & -0.76 & .448 & 0.13 & [-0.05, 0.31] & 1.45 & .148\\
\ \ \ Before-slope * Grandparent, $\hat{\gamma}_{11}$ \textcolor{white}{L} & 0.00 & [-0.02, 0.02] & 0.13 & .894 & -0.03 & [-0.05, -0.01] & -2.49 & .013\\
\ \ \ After-slope * Grandparent, $\hat{\gamma}_{21}$ \textcolor{white}{L} & 0.01 & [-0.01, 0.03] & 1.19 & .236 & 0.00 & [-0.01, 0.02] & 0.48 & .628\\
\ \ \ Shift * Grandparent, $\hat{\gamma}_{31}$ \textcolor{white}{L} & -0.01 & [-0.10, 0.08] & -0.12 & .905 & 0.01 & [-0.08, 0.10] & 0.22 & .825\\
\ \ \ Before-slope * Female, $\hat{\gamma}_{12}$ \textcolor{white}{L} & 0.01 & [-0.01, 0.02] & 0.84 & .400 & -0.03 & [-0.04, -0.02] & -4.83 & < .001\\
\ \ \ After-slope * Female, $\hat{\gamma}_{22}$ \textcolor{white}{L} & 0.01 & [0.00, 0.02] & 2.11 & .035 & -0.01 & [-0.02, 0.00] & -2.03 & .043\\
\ \ \ Shift * Female, $\hat{\gamma}_{32}$ \textcolor{white}{L} & 0.04 & [-0.02, 0.09] & 1.35 & .176 & 0.08 & [0.03, 0.14] & 2.91 & .004\\
\ \ \ Grandparent * Female, $\hat{\gamma}_{03}$ \textcolor{white}{L} & 0.09 & [-0.13, 0.30] & 0.79 & .429 & -0.11 & [-0.36, 0.13] & -0.90 & .369\\
\ \ \ Before-slope * Grandparent * Female, $\hat{\gamma}_{13}$ \textcolor{white}{L} & -0.01 & [-0.03, 0.02] & -0.50 & .618 & 0.03 & [0.00, 0.06] & 2.09 & .037\\
\ \ \ After-slope * Grandparent * Female, $\hat{\gamma}_{23}$ \textcolor{white}{L} & -0.01 & [-0.04, 0.01] & -1.12 & .262 & 0.01 & [-0.02, 0.03] & 0.71 & .475\\
\ \ \ Shift * Grandparent * Female, $\hat{\gamma}_{33}$ \textcolor{white}{L} & -0.02 & [-0.14, 0.10] & -0.29 & .769 & -0.06 & [-0.18, 0.06] & -0.98 & .328\\
HRS (M2a, M2b) &  &  &  &  &  &  &  & \\
\ \ \ Intercept, $\hat{\gamma}_{00}$ \textcolor{white}{H} & 3.15 & [3.09, 3.21] & 108.70 & < .001 & 3.11 & [3.04, 3.17] & 96.32 & < .001\\
\ \ \ Propensity score, $\hat{\gamma}_{04}$ \textcolor{white}{H} & 0.02 & [-0.04, 0.09] & 0.64 & .520 & 0.05 & [-0.02, 0.12] & 1.31 & .189\\
\ \ \ Before-slope, $\hat{\gamma}_{10}$ \textcolor{white}{H} & 0.01 & [-0.01, 0.02] & 0.70 & .482 & 0.00 & [-0.02, 0.01] & -0.37 & .710\\
\ \ \ After-slope, $\hat{\gamma}_{20}$ \textcolor{white}{H} & 0.01 & [0.00, 0.02] & 2.05 & .040 & 0.00 & [-0.01, 0.01] & 0.51 & .610\\
\ \ \ Shift, $\hat{\gamma}_{30}$ \textcolor{white}{H} & -0.01 & [-0.04, 0.02] & -0.52 & .601 & -0.01 & [-0.04, 0.03] & -0.41 & .683\\
\ \ \ Grandparent, $\hat{\gamma}_{01}$ \textcolor{white}{H} & -0.01 & [-0.10, 0.08] & -0.28 & .782 & 0.02 & [-0.07, 0.12] & 0.44 & .661\\
\ \ \ Female, $\hat{\gamma}_{02}$ \textcolor{white}{H} & 0.08 & [0.01, 0.16] & 2.24 & .025 & 0.01 & [-0.07, 0.09] & 0.31 & .759\\
\ \ \ Before-slope * Grandparent, $\hat{\gamma}_{11}$ \textcolor{white}{H} & -0.02 & [-0.06, 0.02] & -0.85 & .397 & -0.01 & [-0.05, 0.03] & -0.41 & .683\\
\ \ \ After-slope * Grandparent, $\hat{\gamma}_{21}$ \textcolor{white}{H} & 0.00 & [-0.02, 0.03] & 0.35 & .730 & 0.01 & [-0.01, 0.04] & 1.09 & .275\\
\ \ \ Shift * Grandparent, $\hat{\gamma}_{31}$ \textcolor{white}{H} & 0.00 & [-0.08, 0.07] & -0.12 & .905 & -0.01 & [-0.08, 0.06] & -0.19 & .853\\
\ \ \ Before-slope * Female, $\hat{\gamma}_{12}$ \textcolor{white}{H} & -0.02 & [-0.04, 0.01] & -1.44 & .150 & 0.02 & [-0.01, 0.04] & 1.40 & .162\\
\ \ \ After-slope * Female, $\hat{\gamma}_{22}$ \textcolor{white}{H} & -0.03 & [-0.04, -0.01] & -3.28 & .001 & -0.01 & [-0.02, 0.01] & -0.98 & .326\\
\ \ \ Shift * Female, $\hat{\gamma}_{32}$ \textcolor{white}{H} & 0.05 & [0.00, 0.09] & 2.17 & .030 & 0.03 & [-0.01, 0.07] & 1.46 & .145\\
\ \ \ Grandparent * Female, $\hat{\gamma}_{03}$ \textcolor{white}{H} & 0.03 & [-0.09, 0.15] & 0.45 & .649 & 0.10 & [-0.03, 0.22] & 1.50 & .133\\
\ \ \ Before-slope * Grandparent * Female, $\hat{\gamma}_{13}$ \textcolor{white}{H} & 0.04 & [-0.01, 0.09] & 1.42 & .155 & 0.01 & [-0.05, 0.06] & 0.23 & .816\\
\ \ \ After-slope * Grandparent * Female, $\hat{\gamma}_{23}$ \textcolor{white}{H} & 0.01 & [-0.02, 0.05] & 0.79 & .431 & 0.00 & [-0.04, 0.03] & -0.26 & .795\\
\ \ \ Shift * Grandparent * Female, $\hat{\gamma}_{33}$ \textcolor{white}{H} & -0.06 & [-0.16, 0.04] & -1.19 & .234 & -0.04 & [-0.14, 0.05] & -0.88 & .380\\
\bottomrule
\addlinespace
\insertTableNotes
\end{longtable}

}

\end{lltable}




\begin{lltable}

\begin{TableNotes}[para]
\normalsize{\textit{Note.} Two models were computed for each of the two samples (LISS panel, HRS): grandparents matched with parent controls (models MXa) and with nonparent controls (models MXb). CI = confidence interval. \(R^2_{M1a} =\) 0.00, \(R^2_{M1b} =\) 0.00, \(R^2_{M2a} =\) 0.00, \(R^2_{M2b} =\) 0.00.}
\end{TableNotes}

\footnotesize{

\begin{longtable}{lrcrrrcrr}\noalign{\getlongtablewidth\global\LTcapwidth=\longtablewidth}
\caption{\label{tab:H1-neur-tab}Fixed Effects of Neuroticism Over the Transition to Grandparenthood.}\\
\toprule
 & \multicolumn{4}{c}{Parent controls} & \multicolumn{4}{c}{Nonparent controls} \\
\cmidrule(r){2-5} \cmidrule(r){6-9}
Parameter & $\hat{\gamma}$ & 95\% CI & $t$ & $p$ & $\hat{\gamma}$ & 95\% CI & $t$ & $p$\\
\midrule
\endfirsthead
\caption*{\normalfont{Table \ref{tab:H1-neur-tab} continued}}\\
\toprule
 & \multicolumn{4}{c}{Parent controls} & \multicolumn{4}{c}{Nonparent controls} \\
\cmidrule(r){2-5} \cmidrule(r){6-9}
Parameter & $\hat{\gamma}$ & 95\% CI & $t$ & $p$ & $\hat{\gamma}$ & 95\% CI & $t$ & $p$\\
\midrule
\endhead
LISS panel (M1a, M1b) &  &  &  &  &  &  &  & \\
\ \ \ Intercept, $\hat{\gamma}_{00}$ \textcolor{white}{L} & 2.48 & [2.40, 2.56] & 63.09 & < .001 & 2.45 & [2.35, 2.54] & 51.88 & < .001\\
\ \ \ Propensity score, $\hat{\gamma}_{02}$ \textcolor{white}{L} & 0.01 & [-0.09, 0.11] & 0.19 & .852 & 0.00 & [-0.09, 0.09] & 0.04 & .967\\
\ \ \ Before-slope, $\hat{\gamma}_{10}$ \textcolor{white}{L} & 0.00 & [-0.01, 0.01] & -0.56 & .575 & -0.01 & [-0.02, -0.01] & -3.66 & < .001\\
\ \ \ After-slope, $\hat{\gamma}_{20}$ \textcolor{white}{L} & 0.00 & [0.00, 0.01] & 0.94 & .350 & 0.00 & [0.00, 0.01] & 1.31 & .190\\
\ \ \ Shift, $\hat{\gamma}_{30}$ \textcolor{white}{L} & -0.05 & [-0.08, -0.02] & -2.96 & .003 & -0.03 & [-0.06, 0.01] & -1.58 & .115\\
\ \ \ Grandparent, $\hat{\gamma}_{01}$ \textcolor{white}{L} & -0.08 & [-0.20, 0.03] & -1.37 & .170 & -0.04 & [-0.17, 0.08] & -0.67 & .500\\
\ \ \ Before-slope * Grandparent, $\hat{\gamma}_{11}$ \textcolor{white}{L} & 0.00 & [-0.01, 0.02] & 0.43 & .668 & 0.02 & [0.00, 0.03] & 1.83 & .067\\
\ \ \ After-slope * Grandparent, $\hat{\gamma}_{21}$ \textcolor{white}{L} & 0.00 & [-0.02, 0.01] & -0.33 & .744 & 0.00 & [-0.02, 0.01] & -0.48 & .635\\
\ \ \ Shift * Grandparent, $\hat{\gamma}_{31}$ \textcolor{white}{L} & -0.02 & [-0.09, 0.06] & -0.41 & .684 & -0.04 & [-0.12, 0.04] & -1.01 & .312\\
HRS (M2a, M2b) &  &  &  &  &  &  &  & \\
\ \ \ Intercept, $\hat{\gamma}_{00}$ \textcolor{white}{H} & 2.07 & [2.03, 2.12] & 94.16 & < .001 & 2.07 & [2.02, 2.12] & 79.36 & < .001\\
\ \ \ Propensity score, $\hat{\gamma}_{02}$ \textcolor{white}{H} & 0.00 & [-0.07, 0.08] & 0.12 & .902 & 0.15 & [0.07, 0.23] & 3.70 & < .001\\
\ \ \ Before-slope, $\hat{\gamma}_{10}$ \textcolor{white}{H} & -0.01 & [-0.03, 0.00] & -1.91 & .057 & -0.03 & [-0.04, -0.02] & -4.70 & < .001\\
\ \ \ After-slope, $\hat{\gamma}_{20}$ \textcolor{white}{H} & -0.01 & [-0.01, 0.00] & -1.20 & .229 & -0.01 & [-0.02, -0.01] & -3.18 & .001\\
\ \ \ Shift, $\hat{\gamma}_{30}$ \textcolor{white}{H} & 0.01 & [-0.02, 0.03] & 0.42 & .674 & -0.03 & [-0.06, -0.01] & -2.36 & .018\\
\ \ \ Grandparent, $\hat{\gamma}_{01}$ \textcolor{white}{H} & -0.06 & [-0.13, 0.01] & -1.65 & .100 & -0.12 & [-0.19, -0.05] & -3.31 & .001\\
\ \ \ Before-slope * Grandparent, $\hat{\gamma}_{11}$ \textcolor{white}{H} & 0.02 & [-0.01, 0.05] & 1.27 & .203 & 0.04 & [0.01, 0.07] & 2.42 & .016\\
\ \ \ After-slope * Grandparent, $\hat{\gamma}_{21}$ \textcolor{white}{H} & -0.02 & [-0.04, 0.00] & -1.53 & .127 & -0.01 & [-0.03, 0.01] & -0.80 & .424\\
\ \ \ Shift * Grandparent, $\hat{\gamma}_{31}$ \textcolor{white}{H} & -0.06 & [-0.12, 0.00] & -2.11 & .035 & -0.03 & [-0.08, 0.03] & -0.88 & .381\\
\bottomrule
\addlinespace
\insertTableNotes
\end{longtable}

}

\end{lltable}




\begin{lltable}

\begin{TableNotes}[para]
\normalsize{\textit{Note.} Two models were computed for each of the two samples (LISS panel, HRS): grandparents matched with parent controls (models MXa) and with nonparent controls (models MXb). CI = confidence interval. \(R^2_{M1a} =\) 0.01, \(R^2_{M1b} =\) 0.01, \(R^2_{M2a} =\) 0.02, \(R^2_{M2b} =\) 0.01.}
\end{TableNotes}

\footnotesize{

\begin{longtable}{lrcrrrcrr}\noalign{\getlongtablewidth\global\LTcapwidth=\longtablewidth}
\caption{\label{tab:H1-neur-gender-tab}Fixed Effects of Neuroticism Over the Transition to Grandparenthood Moderated by Gender.}\\
\toprule
 & \multicolumn{4}{c}{Parent controls} & \multicolumn{4}{c}{Nonparent controls} \\
\cmidrule(r){2-5} \cmidrule(r){6-9}
Parameter & $\hat{\gamma}$ & 95\% CI & $t$ & $p$ & $\hat{\gamma}$ & 95\% CI & $t$ & $p$\\
\midrule
\endfirsthead
\caption*{\normalfont{Table \ref{tab:H1-neur-gender-tab} continued}}\\
\toprule
 & \multicolumn{4}{c}{Parent controls} & \multicolumn{4}{c}{Nonparent controls} \\
\cmidrule(r){2-5} \cmidrule(r){6-9}
Parameter & $\hat{\gamma}$ & 95\% CI & $t$ & $p$ & $\hat{\gamma}$ & 95\% CI & $t$ & $p$\\
\midrule
\endhead
LISS panel (M1a, M1b) &  &  &  &  &  &  &  & \\
\ \ \ Intercept, $\hat{\gamma}_{00}$ \textcolor{white}{L} & 2.45 & [2.34, 2.56] & 43.45 & < .001 & 2.32 & [2.19, 2.45] & 34.99 & < .001\\
\ \ \ Propensity score, $\hat{\gamma}_{04}$ \textcolor{white}{L} & 0.02 & [-0.09, 0.12] & 0.30 & .767 & 0.02 & [-0.08, 0.11] & 0.33 & .744\\
\ \ \ Before-slope, $\hat{\gamma}_{10}$ \textcolor{white}{L} & -0.01 & [-0.02, 0.00] & -1.89 & .059 & -0.01 & [-0.02, 0.00] & -1.12 & .263\\
\ \ \ After-slope, $\hat{\gamma}_{20}$ \textcolor{white}{L} & 0.01 & [0.00, 0.02] & 2.82 & .005 & 0.01 & [0.00, 0.02] & 2.43 & .015\\
\ \ \ Shift, $\hat{\gamma}_{30}$ \textcolor{white}{L} & -0.06 & [-0.11, -0.01] & -2.24 & .025 & -0.05 & [-0.10, 0.00] & -1.95 & .052\\
\ \ \ Grandparent, $\hat{\gamma}_{01}$ \textcolor{white}{L} & -0.18 & [-0.35, -0.01] & -2.11 & .035 & -0.05 & [-0.23, 0.13] & -0.56 & .574\\
\ \ \ Female, $\hat{\gamma}_{02}$ \textcolor{white}{L} & 0.05 & [-0.09, 0.20] & 0.72 & .474 & 0.22 & [0.05, 0.40] & 2.52 & .012\\
\ \ \ Before-slope * Grandparent, $\hat{\gamma}_{11}$ \textcolor{white}{L} & 0.01 & [-0.01, 0.04] & 0.82 & .413 & 0.01 & [-0.02, 0.03] & 0.46 & .643\\
\ \ \ After-slope * Grandparent, $\hat{\gamma}_{21}$ \textcolor{white}{L} & -0.02 & [-0.04, 0.01] & -1.36 & .173 & -0.01 & [-0.04, 0.01] & -1.15 & .250\\
\ \ \ Shift * Grandparent, $\hat{\gamma}_{31}$ \textcolor{white}{L} & -0.03 & [-0.14, 0.08] & -0.51 & .612 & -0.04 & [-0.15, 0.08] & -0.63 & .529\\
\ \ \ Before-slope * Female, $\hat{\gamma}_{12}$ \textcolor{white}{L} & 0.02 & [0.00, 0.03] & 2.03 & .043 & -0.01 & [-0.03, 0.00] & -1.83 & .067\\
\ \ \ After-slope * Female, $\hat{\gamma}_{22}$ \textcolor{white}{L} & -0.02 & [-0.03, -0.01] & -2.99 & .003 & -0.01 & [-0.03, 0.00] & -2.10 & .036\\
\ \ \ Shift * Female, $\hat{\gamma}_{32}$ \textcolor{white}{L} & 0.01 & [-0.05, 0.08] & 0.39 & .700 & 0.04 & [-0.03, 0.11] & 1.19 & .234\\
\ \ \ Grandparent * Female, $\hat{\gamma}_{03}$ \textcolor{white}{L} & 0.18 & [-0.05, 0.40] & 1.54 & .123 & 0.01 & [-0.24, 0.25] & 0.06 & .951\\
\ \ \ Before-slope * Grandparent * Female, $\hat{\gamma}_{13}$ \textcolor{white}{L} & -0.01 & [-0.05, 0.02] & -0.66 & .508 & 0.02 & [-0.02, 0.05] & 1.08 & .279\\
\ \ \ After-slope * Grandparent * Female, $\hat{\gamma}_{23}$ \textcolor{white}{L} & 0.02 & [-0.01, 0.05] & 1.48 & .138 & 0.02 & [-0.01, 0.05] & 1.08 & .282\\
\ \ \ Shift * Grandparent * Female, $\hat{\gamma}_{33}$ \textcolor{white}{L} & 0.03 & [-0.12, 0.18] & 0.35 & .730 & 0.00 & [-0.16, 0.15] & -0.03 & .975\\
HRS (M2a, M2b) &  &  &  &  &  &  &  & \\
\ \ \ Intercept, $\hat{\gamma}_{00}$ \textcolor{white}{H} & 1.98 & [1.91, 2.04] & 62.75 & < .001 & 2.01 & [1.94, 2.08] & 56.33 & < .001\\
\ \ \ Propensity score, $\hat{\gamma}_{04}$ \textcolor{white}{H} & 0.01 & [-0.07, 0.09] & 0.25 & .801 & 0.15 & [0.07, 0.23] & 3.58 & < .001\\
\ \ \ Before-slope, $\hat{\gamma}_{10}$ \textcolor{white}{H} & -0.02 & [-0.04, 0.00] & -2.11 & .035 & -0.03 & [-0.05, -0.01] & -3.18 & .001\\
\ \ \ After-slope, $\hat{\gamma}_{20}$ \textcolor{white}{H} & -0.02 & [-0.03, 0.00] & -2.40 & .016 & -0.02 & [-0.03, -0.01] & -2.92 & .003\\
\ \ \ Shift, $\hat{\gamma}_{30}$ \textcolor{white}{H} & 0.08 & [0.04, 0.12] & 4.03 & < .001 & 0.00 & [-0.03, 0.04] & 0.21 & .833\\
\ \ \ Grandparent, $\hat{\gamma}_{01}$ \textcolor{white}{H} & -0.06 & [-0.16, 0.04] & -1.11 & .269 & -0.16 & [-0.26, -0.05] & -2.89 & .004\\
\ \ \ Female, $\hat{\gamma}_{02}$ \textcolor{white}{H} & 0.17 & [0.09, 0.25] & 4.25 & < .001 & 0.10 & [0.01, 0.19] & 2.23 & .026\\
\ \ \ Before-slope * Grandparent, $\hat{\gamma}_{11}$ \textcolor{white}{H} & 0.06 & [0.01, 0.10] & 2.26 & .024 & 0.06 & [0.02, 0.11] & 2.72 & .007\\
\ \ \ After-slope * Grandparent, $\hat{\gamma}_{21}$ \textcolor{white}{H} & 0.00 & [-0.03, 0.03] & 0.31 & .756 & 0.01 & [-0.02, 0.04] & 0.48 & .630\\
\ \ \ Shift * Grandparent, $\hat{\gamma}_{31}$ \textcolor{white}{H} & -0.16 & [-0.25, -0.07] & -3.60 & < .001 & -0.08 & [-0.17, 0.00] & -1.89 & .059\\
\ \ \ Before-slope * Female, $\hat{\gamma}_{12}$ \textcolor{white}{H} & 0.01 & [-0.01, 0.04] & 1.04 & .298 & 0.00 & [-0.03, 0.03] & 0.09 & .926\\
\ \ \ After-slope * Female, $\hat{\gamma}_{22}$ \textcolor{white}{H} & 0.02 & [0.00, 0.04] & 2.19 & .029 & 0.01 & [-0.01, 0.03] & 1.15 & .250\\
\ \ \ Shift * Female, $\hat{\gamma}_{32}$ \textcolor{white}{H} & -0.14 & [-0.19, -0.08] & -5.02 & < .001 & -0.06 & [-0.11, -0.01] & -2.33 & .020\\
\ \ \ Grandparent * Female, $\hat{\gamma}_{03}$ \textcolor{white}{H} & 0.00 & [-0.14, 0.13] & -0.01 & .989 & 0.06 & [-0.08, 0.20] & 0.82 & .410\\
\ \ \ Before-slope * Grandparent * Female, $\hat{\gamma}_{13}$ \textcolor{white}{H} & -0.06 & [-0.12, 0.00] & -1.85 & .064 & -0.05 & [-0.11, 0.01] & -1.49 & .137\\
\ \ \ After-slope * Grandparent * Female, $\hat{\gamma}_{23}$ \textcolor{white}{H} & -0.04 & [-0.08, 0.00] & -1.80 & .072 & -0.03 & [-0.07, 0.01] & -1.35 & .176\\
\ \ \ Shift * Grandparent * Female, $\hat{\gamma}_{33}$ \textcolor{white}{H} & 0.17 & [0.06, 0.29] & 2.91 & .004 & 0.10 & [-0.01, 0.21] & 1.71 & .087\\
\bottomrule
\addlinespace
\insertTableNotes
\end{longtable}

}

\end{lltable}




\begin{lltable}

\begin{TableNotes}[para]
\normalsize{\textit{Note.} Two models were computed for each of the two samples (LISS panel, HRS): grandparents matched with parent controls (models MXa) and with nonparent controls (models MXb). CI = confidence interval. \(R^2_{M1a} =\) 0.00, \(R^2_{M1b} =\) 0.01, \(R^2_{M2a} =\) 0.00, \(R^2_{M2b} =\) 0.00.}
\end{TableNotes}

\footnotesize{

\begin{longtable}{lrcrrrcrr}\noalign{\getlongtablewidth\global\LTcapwidth=\longtablewidth}
\caption{\label{tab:H1-open-tab}Fixed Effects of Openness to Experience Over the Transition to Grandparenthood.}\\
\toprule
 & \multicolumn{4}{c}{Parent controls} & \multicolumn{4}{c}{Nonparent controls} \\
\cmidrule(r){2-5} \cmidrule(r){6-9}
Parameter & $\hat{\gamma}$ & 95\% CI & $t$ & $p$ & $\hat{\gamma}$ & 95\% CI & $t$ & $p$\\
\midrule
\endfirsthead
\caption*{\normalfont{Table \ref{tab:H1-open-tab} continued}}\\
\toprule
 & \multicolumn{4}{c}{Parent controls} & \multicolumn{4}{c}{Nonparent controls} \\
\cmidrule(r){2-5} \cmidrule(r){6-9}
Parameter & $\hat{\gamma}$ & 95\% CI & $t$ & $p$ & $\hat{\gamma}$ & 95\% CI & $t$ & $p$\\
\midrule
\endhead
LISS panel (M1a, M1b) &  &  &  &  &  &  &  & \\
\ \ \ Intercept, $\hat{\gamma}_{00}$ \textcolor{white}{L} & 3.48 & [3.42, 3.53] & 118.71 & < .001 & 3.52 & [3.46, 3.59] & 103.85 & < .001\\
\ \ \ Propensity score, $\hat{\gamma}_{02}$ \textcolor{white}{L} & 0.00 & [-0.08, 0.07] & -0.07 & .943 & 0.03 & [-0.03, 0.09] & 1.03 & .304\\
\ \ \ Before-slope, $\hat{\gamma}_{10}$ \textcolor{white}{L} & 0.00 & [-0.01, 0.00] & -1.59 & .113 & 0.00 & [-0.01, 0.00] & -0.68 & .496\\
\ \ \ After-slope, $\hat{\gamma}_{20}$ \textcolor{white}{L} & -0.01 & [-0.01, 0.00] & -2.35 & .019 & 0.00 & [0.00, 0.01] & 1.95 & .051\\
\ \ \ Shift, $\hat{\gamma}_{30}$ \textcolor{white}{L} & 0.02 & [0.00, 0.05] & 1.87 & .061 & 0.00 & [-0.02, 0.02] & 0.00 & .996\\
\ \ \ Grandparent, $\hat{\gamma}_{01}$ \textcolor{white}{L} & 0.01 & [-0.08, 0.09] & 0.16 & .873 & -0.05 & [-0.14, 0.04] & -1.07 & .285\\
\ \ \ Before-slope * Grandparent, $\hat{\gamma}_{11}$ \textcolor{white}{L} & 0.01 & [0.00, 0.02] & 1.23 & .217 & 0.01 & [-0.01, 0.02] & 0.91 & .365\\
\ \ \ After-slope * Grandparent, $\hat{\gamma}_{21}$ \textcolor{white}{L} & 0.00 & [-0.01, 0.01] & 0.11 & .912 & -0.01 & [-0.02, 0.00] & -1.93 & .054\\
\ \ \ Shift * Grandparent, $\hat{\gamma}_{31}$ \textcolor{white}{L} & -0.03 & [-0.08, 0.03] & -1.05 & .295 & -0.01 & [-0.06, 0.04] & -0.23 & .819\\
HRS (M2a, M2b) &  &  &  &  &  &  &  & \\
\ \ \ Intercept, $\hat{\gamma}_{00}$ \textcolor{white}{H} & 3.04 & [3.00, 3.08] & 149.49 & < .001 & 3.01 & [2.96, 3.06] & 129.29 & < .001\\
\ \ \ Propensity score, $\hat{\gamma}_{02}$ \textcolor{white}{H} & 0.03 & [-0.04, 0.09] & 0.82 & .411 & 0.00 & [-0.06, 0.07] & 0.13 & .895\\
\ \ \ Before-slope, $\hat{\gamma}_{10}$ \textcolor{white}{H} & -0.02 & [-0.03, -0.01] & -3.29 & .001 & 0.00 & [-0.01, 0.01] & -0.68 & .495\\
\ \ \ After-slope, $\hat{\gamma}_{20}$ \textcolor{white}{H} & -0.02 & [-0.03, -0.01] & -5.28 & < .001 & -0.02 & [-0.02, -0.01] & -4.83 & < .001\\
\ \ \ Shift, $\hat{\gamma}_{30}$ \textcolor{white}{H} & 0.06 & [0.03, 0.08] & 4.92 & < .001 & 0.03 & [0.01, 0.05] & 3.26 & .001\\
\ \ \ Grandparent, $\hat{\gamma}_{01}$ \textcolor{white}{H} & -0.02 & [-0.08, 0.05] & -0.55 & .582 & 0.02 & [-0.04, 0.09] & 0.75 & .451\\
\ \ \ Before-slope * Grandparent, $\hat{\gamma}_{11}$ \textcolor{white}{H} & 0.02 & [-0.01, 0.04] & 1.36 & .172 & 0.00 & [-0.02, 0.03] & 0.19 & .850\\
\ \ \ After-slope * Grandparent, $\hat{\gamma}_{21}$ \textcolor{white}{H} & 0.02 & [0.00, 0.03] & 2.01 & .044 & 0.01 & [0.00, 0.03] & 1.74 & .083\\
\ \ \ Shift * Grandparent, $\hat{\gamma}_{31}$ \textcolor{white}{H} & -0.07 & [-0.12, -0.02] & -2.86 & .004 & -0.05 & [-0.09, 0.00] & -2.16 & .031\\
\bottomrule
\addlinespace
\insertTableNotes
\end{longtable}

}

\end{lltable}




\begin{lltable}

\begin{TableNotes}[para]
\normalsize{\textit{Note.} Two models were computed for each of the two samples (LISS panel, HRS): grandparents matched with parent controls (models MXa) and with nonparent controls (models MXb). CI = confidence interval. \(R^2_{M1a} =\) 0.01, \(R^2_{M1b} =\) 0.01, \(R^2_{M2a} =\) 0.00, \(R^2_{M2b} =\) 0.00.}
\end{TableNotes}

\footnotesize{

\begin{longtable}{lrcrrrcrr}\noalign{\getlongtablewidth\global\LTcapwidth=\longtablewidth}
\caption{\label{tab:H1-open-gender-tab}Fixed Effects of Openness to Experience Over the Transition to Grandparenthood Moderated by Gender.}\\
\toprule
 & \multicolumn{4}{c}{Parent controls} & \multicolumn{4}{c}{Nonparent controls} \\
\cmidrule(r){2-5} \cmidrule(r){6-9}
Parameter & $\hat{\gamma}$ & 95\% CI & $t$ & $p$ & $\hat{\gamma}$ & 95\% CI & $t$ & $p$\\
\midrule
\endfirsthead
\caption*{\normalfont{Table \ref{tab:H1-open-gender-tab} continued}}\\
\toprule
 & \multicolumn{4}{c}{Parent controls} & \multicolumn{4}{c}{Nonparent controls} \\
\cmidrule(r){2-5} \cmidrule(r){6-9}
Parameter & $\hat{\gamma}$ & 95\% CI & $t$ & $p$ & $\hat{\gamma}$ & 95\% CI & $t$ & $p$\\
\midrule
\endhead
LISS panel (M1a, M1b) &  &  &  &  &  &  &  & \\
\ \ \ Intercept, $\hat{\gamma}_{00}$ \textcolor{white}{L} & 3.47 & [3.39, 3.55] & 81.40 & < .001 & 3.55 & [3.45, 3.64] & 73.11 & < .001\\
\ \ \ Propensity score, $\hat{\gamma}_{04}$ \textcolor{white}{L} & 0.00 & [-0.08, 0.07] & -0.04 & .969 & 0.03 & [-0.03, 0.09] & 0.95 & .340\\
\ \ \ Before-slope, $\hat{\gamma}_{10}$ \textcolor{white}{L} & 0.00 & [-0.01, 0.01] & 0.17 & .864 & 0.01 & [0.00, 0.02] & 2.40 & .017\\
\ \ \ After-slope, $\hat{\gamma}_{20}$ \textcolor{white}{L} & 0.00 & [-0.01, 0.00] & -1.05 & .292 & 0.01 & [0.00, 0.01] & 1.53 & .126\\
\ \ \ Shift, $\hat{\gamma}_{30}$ \textcolor{white}{L} & -0.02 & [-0.05, 0.02] & -0.93 & .353 & -0.01 & [-0.04, 0.02] & -0.64 & .520\\
\ \ \ Grandparent, $\hat{\gamma}_{01}$ \textcolor{white}{L} & 0.11 & [-0.01, 0.24] & 1.78 & .075 & 0.03 & [-0.10, 0.16] & 0.44 & .660\\
\ \ \ Female, $\hat{\gamma}_{02}$ \textcolor{white}{L} & 0.01 & [-0.10, 0.12] & 0.16 & .869 & -0.05 & [-0.17, 0.08] & -0.71 & .478\\
\ \ \ Before-slope * Grandparent, $\hat{\gamma}_{11}$ \textcolor{white}{L} & 0.00 & [-0.02, 0.01] & -0.39 & .693 & -0.01 & [-0.03, 0.00] & -1.43 & .154\\
\ \ \ After-slope * Grandparent, $\hat{\gamma}_{21}$ \textcolor{white}{L} & -0.01 & [-0.02, 0.01] & -0.88 & .380 & -0.02 & [-0.03, 0.00] & -2.16 & .031\\
\ \ \ Shift * Grandparent, $\hat{\gamma}_{31}$ \textcolor{white}{L} & 0.03 & [-0.05, 0.12] & 0.84 & .400 & 0.03 & [-0.05, 0.10] & 0.76 & .450\\
\ \ \ Before-slope * Female, $\hat{\gamma}_{12}$ \textcolor{white}{L} & -0.01 & [-0.02, 0.00] & -1.64 & .101 & -0.02 & [-0.03, -0.01] & -3.90 & < .001\\
\ \ \ After-slope * Female, $\hat{\gamma}_{22}$ \textcolor{white}{L} & 0.00 & [-0.01, 0.01] & -0.78 & .433 & 0.00 & [-0.01, 0.01] & -0.24 & .813\\
\ \ \ Shift * Female, $\hat{\gamma}_{32}$ \textcolor{white}{L} & 0.08 & [0.03, 0.13] & 2.98 & .003 & 0.02 & [-0.03, 0.06] & 0.84 & .401\\
\ \ \ Grandparent * Female, $\hat{\gamma}_{03}$ \textcolor{white}{L} & -0.20 & [-0.37, -0.03] & -2.32 & .021 & -0.15 & [-0.33, 0.03] & -1.62 & .105\\
\ \ \ Before-slope * Grandparent * Female, $\hat{\gamma}_{13}$ \textcolor{white}{L} & 0.02 & [0.00, 0.05] & 1.70 & .089 & 0.03 & [0.01, 0.06] & 2.85 & .004\\
\ \ \ After-slope * Grandparent * Female, $\hat{\gamma}_{23}$ \textcolor{white}{L} & 0.01 & [-0.01, 0.04] & 1.29 & .197 & 0.01 & [-0.01, 0.03] & 1.13 & .259\\
\ \ \ Shift * Grandparent * Female, $\hat{\gamma}_{33}$ \textcolor{white}{L} & -0.12 & [-0.23, -0.01] & -2.11 & .035 & -0.06 & [-0.16, 0.04] & -1.24 & .216\\
HRS (M2a, M2b) &  &  &  &  &  &  &  & \\
\ \ \ Intercept, $\hat{\gamma}_{00}$ \textcolor{white}{H} & 3.06 & [3.00, 3.12] & 108.70 & < .001 & 3.03 & [2.97, 3.09] & 97.90 & < .001\\
\ \ \ Propensity score, $\hat{\gamma}_{04}$ \textcolor{white}{H} & 0.03 & [-0.04, 0.09] & 0.86 & .391 & 0.00 & [-0.06, 0.07] & 0.03 & .976\\
\ \ \ Before-slope, $\hat{\gamma}_{10}$ \textcolor{white}{H} & -0.02 & [-0.04, 0.00] & -2.44 & .015 & -0.01 & [-0.03, 0.00] & -1.90 & .058\\
\ \ \ After-slope, $\hat{\gamma}_{20}$ \textcolor{white}{H} & -0.03 & [-0.04, -0.02] & -5.75 & < .001 & -0.01 & [-0.02, 0.00] & -2.04 & .042\\
\ \ \ Shift, $\hat{\gamma}_{30}$ \textcolor{white}{H} & 0.11 & [0.07, 0.14] & 6.34 & < .001 & 0.00 & [-0.03, 0.03] & -0.29 & .772\\
\ \ \ Grandparent, $\hat{\gamma}_{01}$ \textcolor{white}{H} & -0.03 & [-0.12, 0.06] & -0.62 & .535 & 0.01 & [-0.08, 0.10] & 0.24 & .813\\
\ \ \ Female, $\hat{\gamma}_{02}$ \textcolor{white}{H} & -0.03 & [-0.09, 0.04] & -0.80 & .423 & -0.04 & [-0.11, 0.04] & -0.98 & .328\\
\ \ \ Before-slope * Grandparent, $\hat{\gamma}_{11}$ \textcolor{white}{H} & 0.01 & [-0.03, 0.05] & 0.41 & .685 & 0.00 & [-0.03, 0.04] & 0.05 & .960\\
\ \ \ After-slope * Grandparent, $\hat{\gamma}_{21}$ \textcolor{white}{H} & 0.03 & [0.01, 0.06] & 2.66 & .008 & 0.01 & [-0.01, 0.03] & 0.94 & .346\\
\ \ \ Shift * Grandparent, $\hat{\gamma}_{31}$ \textcolor{white}{H} & -0.15 & [-0.22, -0.07] & -3.93 & < .001 & -0.03 & [-0.10, 0.03] & -1.00 & .316\\
\ \ \ Before-slope * Female, $\hat{\gamma}_{12}$ \textcolor{white}{H} & 0.00 & [-0.02, 0.03] & 0.28 & .781 & 0.02 & [0.00, 0.04] & 1.97 & .049\\
\ \ \ After-slope * Female, $\hat{\gamma}_{22}$ \textcolor{white}{H} & 0.02 & [0.01, 0.04] & 3.05 & .002 & -0.01 & [-0.02, 0.00] & -1.47 & .141\\
\ \ \ Shift * Female, $\hat{\gamma}_{32}$ \textcolor{white}{H} & -0.09 & [-0.14, -0.05] & -4.11 & < .001 & 0.06 & [0.03, 0.10] & 3.21 & .001\\
\ \ \ Grandparent * Female, $\hat{\gamma}_{03}$ \textcolor{white}{H} & 0.02 & [-0.10, 0.13] & 0.30 & .763 & 0.03 & [-0.09, 0.14] & 0.45 & .652\\
\ \ \ Before-slope * Grandparent * Female, $\hat{\gamma}_{13}$ \textcolor{white}{H} & 0.02 & [-0.04, 0.07] & 0.67 & .504 & 0.00 & [-0.05, 0.05] & 0.08 & .939\\
\ \ \ After-slope * Grandparent * Female, $\hat{\gamma}_{23}$ \textcolor{white}{H} & -0.03 & [-0.06, 0.00] & -1.75 & .079 & 0.00 & [-0.03, 0.03] & 0.27 & .790\\
\ \ \ Shift * Grandparent * Female, $\hat{\gamma}_{33}$ \textcolor{white}{H} & 0.14 & [0.04, 0.23] & 2.71 & .007 & -0.02 & [-0.11, 0.06] & -0.52 & .603\\
\bottomrule
\addlinespace
\insertTableNotes
\end{longtable}

}

\end{lltable}




\begin{lltable}

\begin{TableNotes}[para]
\normalsize{\textit{Note.} Two models were computed for each of the two samples (LISS panel, HRS): grandparents matched with parent controls (models MXa) and with nonparent controls (models MXb). CI = confidence interval. \(R^2_{M1a} =\) 0.00, \(R^2_{M1b} =\) 0.00, \(R^2_{M2a} =\) 0.00, \(R^2_{M2b} =\) 0.01.}
\end{TableNotes}

\footnotesize{

\begin{longtable}{lrcrrrcrr}\noalign{\getlongtablewidth\global\LTcapwidth=\longtablewidth}
\caption{\label{tab:H1-swls-tab}Fixed Effects of Life Satisfaction Over the Transition to Grandparenthood.}\\
\toprule
 & \multicolumn{4}{c}{Parent controls} & \multicolumn{4}{c}{Nonparent controls} \\
\cmidrule(r){2-5} \cmidrule(r){6-9}
Parameter & $\hat{\gamma}$ & 95\% CI & $t$ & $p$ & $\hat{\gamma}$ & 95\% CI & $t$ & $p$\\
\midrule
\endfirsthead
\caption*{\normalfont{Table \ref{tab:H1-swls-tab} continued}}\\
\toprule
 & \multicolumn{4}{c}{Parent controls} & \multicolumn{4}{c}{Nonparent controls} \\
\cmidrule(r){2-5} \cmidrule(r){6-9}
Parameter & $\hat{\gamma}$ & 95\% CI & $t$ & $p$ & $\hat{\gamma}$ & 95\% CI & $t$ & $p$\\
\midrule
\endhead
LISS panel (M1a, M1b) &  &  &  &  &  &  &  & \\
\ \ \ Intercept, $\hat{\gamma}_{00}$ \textcolor{white}{L} & 5.11 & [4.99, 5.23] & 85.63 & < .001 & 5.13 & [4.99, 5.27] & 72.47 & < .001\\
\ \ \ Propensity score, $\hat{\gamma}_{02}$ \textcolor{white}{L} & 0.07 & [-0.10, 0.24] & 0.78 & .433 & 0.01 & [-0.15, 0.17] & 0.17 & .863\\
\ \ \ Before-slope, $\hat{\gamma}_{10}$ \textcolor{white}{L} & -0.01 & [-0.02, 0.01] & -1.06 & .288 & 0.02 & [0.00, 0.03] & 2.18 & .029\\
\ \ \ After-slope, $\hat{\gamma}_{20}$ \textcolor{white}{L} & 0.01 & [0.00, 0.02] & 2.13 & .033 & -0.01 & [-0.02, 0.01] & -0.93 & .351\\
\ \ \ Shift, $\hat{\gamma}_{30}$ \textcolor{white}{L} & 0.02 & [-0.04, 0.08] & 0.72 & .470 & -0.11 & [-0.17, -0.05] & -3.42 & .001\\
\ \ \ Grandparent, $\hat{\gamma}_{01}$ \textcolor{white}{L} & 0.07 & [-0.11, 0.25] & 0.73 & .464 & 0.07 & [-0.13, 0.26] & 0.66 & .510\\
\ \ \ Before-slope * Grandparent, $\hat{\gamma}_{11}$ \textcolor{white}{L} & 0.02 & [-0.01, 0.04] & 1.03 & .301 & -0.01 & [-0.04, 0.02] & -0.47 & .637\\
\ \ \ After-slope * Grandparent, $\hat{\gamma}_{21}$ \textcolor{white}{L} & -0.02 & [-0.05, 0.00] & -1.78 & .075 & 0.00 & [-0.03, 0.02] & -0.33 & .741\\
\ \ \ Shift * Grandparent, $\hat{\gamma}_{31}$ \textcolor{white}{L} & 0.05 & [-0.08, 0.18] & 0.79 & .428 & 0.18 & [0.04, 0.32] & 2.57 & .010\\
HRS (M2a, M2b) &  &  &  &  &  &  &  & \\
\ \ \ Intercept, $\hat{\gamma}_{00}$ \textcolor{white}{H} & 4.81 & [4.69, 4.92] & 82.17 & < .001 & 4.58 & [4.45, 4.72] & 66.89 & < .001\\
\ \ \ Propensity score, $\hat{\gamma}_{02}$ \textcolor{white}{H} & 0.40 & [0.19, 0.61] & 3.78 & < .001 & 0.33 & [0.11, 0.54] & 3.01 & .003\\
\ \ \ Before-slope, $\hat{\gamma}_{10}$ \textcolor{white}{H} & -0.03 & [-0.07, 0.01] & -1.53 & .125 & 0.05 & [0.01, 0.08] & 2.50 & .013\\
\ \ \ After-slope, $\hat{\gamma}_{20}$ \textcolor{white}{H} & 0.01 & [-0.01, 0.04] & 0.83 & .405 & 0.04 & [0.01, 0.06] & 3.14 & .002\\
\ \ \ Shift, $\hat{\gamma}_{30}$ \textcolor{white}{H} & 0.02 & [-0.05, 0.10] & 0.58 & .564 & -0.05 & [-0.12, 0.02] & -1.50 & .135\\
\ \ \ Grandparent, $\hat{\gamma}_{01}$ \textcolor{white}{H} & -0.02 & [-0.21, 0.16] & -0.24 & .812 & 0.20 & [0.00, 0.39] & 1.98 & .048\\
\ \ \ Before-slope * Grandparent, $\hat{\gamma}_{11}$ \textcolor{white}{H} & 0.12 & [0.03, 0.21] & 2.58 & .010 & 0.05 & [-0.04, 0.13] & 1.06 & .290\\
\ \ \ After-slope * Grandparent, $\hat{\gamma}_{21}$ \textcolor{white}{H} & 0.03 & [-0.02, 0.09] & 1.17 & .241 & 0.01 & [-0.05, 0.06] & 0.31 & .753\\
\ \ \ Shift * Grandparent, $\hat{\gamma}_{31}$ \textcolor{white}{H} & -0.08 & [-0.24, 0.09] & -0.93 & .351 & -0.01 & [-0.17, 0.15] & -0.13 & .897\\
\bottomrule
\addlinespace
\insertTableNotes
\end{longtable}

}

\end{lltable}




\begin{lltable}

\begin{TableNotes}[para]
\normalsize{\textit{Note.} Two models were computed for each of the two samples (LISS panel, HRS): grandparents matched with parent controls (models MXa) and with nonparent controls (models MXb). CI = confidence interval. \(R^2_{M1a} =\) 0.00, \(R^2_{M1b} =\) 0.00, \(R^2_{M2a} =\) 0.01, \(R^2_{M2b} =\) 0.01.}
\end{TableNotes}

\footnotesize{

\begin{longtable}{lrcrrrcrr}\noalign{\getlongtablewidth\global\LTcapwidth=\longtablewidth}
\caption{\label{tab:H1-swls-gender-tab}Fixed Effects of Life Satisfaction Over the Transition to Grandparenthood Moderated by Gender.}\\
\toprule
 & \multicolumn{4}{c}{Parent controls} & \multicolumn{4}{c}{Nonparent controls} \\
\cmidrule(r){2-5} \cmidrule(r){6-9}
Parameter & $\hat{\gamma}$ & 95\% CI & $t$ & $p$ & $\hat{\gamma}$ & 95\% CI & $t$ & $p$\\
\midrule
\endfirsthead
\caption*{\normalfont{Table \ref{tab:H1-swls-gender-tab} continued}}\\
\toprule
 & \multicolumn{4}{c}{Parent controls} & \multicolumn{4}{c}{Nonparent controls} \\
\cmidrule(r){2-5} \cmidrule(r){6-9}
Parameter & $\hat{\gamma}$ & 95\% CI & $t$ & $p$ & $\hat{\gamma}$ & 95\% CI & $t$ & $p$\\
\midrule
\endhead
LISS panel (M1a, M1b) &  &  &  &  &  &  &  & \\
\ \ \ Intercept, $\hat{\gamma}_{00}$ \textcolor{white}{L} & 5.05 & [4.89, 5.21] & 61.49 & < .001 & 5.05 & [4.86, 5.24] & 51.98 & < .001\\
\ \ \ Propensity score, $\hat{\gamma}_{04}$ \textcolor{white}{L} & 0.06 & [-0.11, 0.23] & 0.70 & .485 & 0.01 & [-0.15, 0.17] & 0.17 & .866\\
\ \ \ Before-slope, $\hat{\gamma}_{10}$ \textcolor{white}{L} & -0.01 & [-0.03, 0.01] & -1.13 & .258 & 0.02 & [0.00, 0.05] & 2.28 & .023\\
\ \ \ After-slope, $\hat{\gamma}_{20}$ \textcolor{white}{L} & 0.01 & [0.00, 0.03] & 1.55 & .122 & -0.03 & [-0.04, -0.01] & -2.76 & .006\\
\ \ \ Shift, $\hat{\gamma}_{30}$ \textcolor{white}{L} & 0.10 & [0.01, 0.18] & 2.25 & .025 & 0.00 & [-0.09, 0.09] & -0.01 & .988\\
\ \ \ Grandparent, $\hat{\gamma}_{01}$ \textcolor{white}{L} & 0.21 & [-0.04, 0.46] & 1.67 & .096 & 0.23 & [-0.04, 0.50] & 1.65 & .099\\
\ \ \ Female, $\hat{\gamma}_{02}$ \textcolor{white}{L} & 0.12 & [-0.08, 0.32] & 1.18 & .239 & 0.16 & [-0.08, 0.40] & 1.28 & .203\\
\ \ \ Before-slope * Grandparent, $\hat{\gamma}_{11}$ \textcolor{white}{L} & 0.00 & [-0.04, 0.04] & 0.10 & .922 & -0.03 & [-0.08, 0.01] & -1.38 & .168\\
\ \ \ After-slope * Grandparent, $\hat{\gamma}_{21}$ \textcolor{white}{L} & -0.03 & [-0.07, 0.01] & -1.62 & .104 & 0.01 & [-0.03, 0.05] & 0.36 & .718\\
\ \ \ Shift * Grandparent, $\hat{\gamma}_{31}$ \textcolor{white}{L} & 0.01 & [-0.18, 0.20] & 0.10 & .919 & 0.11 & [-0.10, 0.31] & 1.03 & .303\\
\ \ \ Before-slope * Female, $\hat{\gamma}_{12}$ \textcolor{white}{L} & 0.01 & [-0.02, 0.03] & 0.55 & .581 & -0.02 & [-0.04, 0.01] & -1.10 & .273\\
\ \ \ After-slope * Female, $\hat{\gamma}_{22}$ \textcolor{white}{L} & 0.00 & [-0.02, 0.02] & -0.11 & .913 & 0.04 & [0.01, 0.06] & 2.95 & .003\\
\ \ \ Shift * Female, $\hat{\gamma}_{32}$ \textcolor{white}{L} & -0.14 & [-0.26, -0.02] & -2.37 & .018 & -0.21 & [-0.33, -0.08] & -3.28 & .001\\
\ \ \ Grandparent * Female, $\hat{\gamma}_{03}$ \textcolor{white}{L} & -0.27 & [-0.59, 0.05] & -1.67 & .097 & -0.31 & [-0.66, 0.05] & -1.71 & .088\\
\ \ \ Before-slope * Grandparent * Female, $\hat{\gamma}_{13}$ \textcolor{white}{L} & 0.03 & [-0.03, 0.08] & 0.87 & .385 & 0.05 & [-0.02, 0.11] & 1.48 & .138\\
\ \ \ After-slope * Grandparent * Female, $\hat{\gamma}_{23}$ \textcolor{white}{L} & 0.01 & [-0.04, 0.07] & 0.51 & .607 & -0.03 & [-0.08, 0.03] & -0.90 & .369\\
\ \ \ Shift * Grandparent * Female, $\hat{\gamma}_{33}$ \textcolor{white}{L} & 0.08 & [-0.17, 0.34] & 0.63 & .530 & 0.15 & [-0.13, 0.43] & 1.07 & .283\\
HRS (M2a, M2b) &  &  &  &  &  &  &  & \\
\ \ \ Intercept, $\hat{\gamma}_{00}$ \textcolor{white}{H} & 4.67 & [4.52, 4.82] & 60.70 & < .001 & 4.54 & [4.37, 4.71] & 52.50 & < .001\\
\ \ \ Propensity score, $\hat{\gamma}_{04}$ \textcolor{white}{H} & 0.41 & [0.20, 0.62] & 3.84 & < .001 & 0.30 & [0.08, 0.51] & 2.71 & .007\\
\ \ \ Before-slope, $\hat{\gamma}_{10}$ \textcolor{white}{H} & 0.01 & [-0.04, 0.07] & 0.49 & .625 & 0.05 & [-0.01, 0.10] & 1.61 & .107\\
\ \ \ After-slope, $\hat{\gamma}_{20}$ \textcolor{white}{H} & 0.00 & [-0.04, 0.04] & 0.09 & .931 & 0.02 & [-0.01, 0.06] & 1.31 & .190\\
\ \ \ Shift, $\hat{\gamma}_{30}$ \textcolor{white}{H} & 0.07 & [-0.04, 0.18] & 1.23 & .220 & -0.16 & [-0.27, -0.05] & -2.91 & .004\\
\ \ \ Grandparent, $\hat{\gamma}_{01}$ \textcolor{white}{H} & 0.11 & [-0.15, 0.37] & 0.81 & .419 & 0.25 & [-0.02, 0.51] & 1.82 & .070\\
\ \ \ Female, $\hat{\gamma}_{02}$ \textcolor{white}{H} & 0.24 & [0.07, 0.41] & 2.75 & .006 & 0.10 & [-0.10, 0.29] & 0.98 & .329\\
\ \ \ Before-slope * Grandparent, $\hat{\gamma}_{11}$ \textcolor{white}{H} & 0.00 & [-0.13, 0.14] & 0.03 & .978 & -0.02 & [-0.15, 0.11] & -0.33 & .745\\
\ \ \ After-slope * Grandparent, $\hat{\gamma}_{21}$ \textcolor{white}{H} & 0.04 & [-0.04, 0.13] & 1.05 & .294 & 0.03 & [-0.05, 0.10] & 0.62 & .536\\
\ \ \ Shift * Grandparent, $\hat{\gamma}_{31}$ \textcolor{white}{H} & -0.08 & [-0.33, 0.16] & -0.65 & .514 & 0.14 & [-0.10, 0.37] & 1.16 & .246\\
\ \ \ Before-slope * Female, $\hat{\gamma}_{12}$ \textcolor{white}{H} & -0.08 & [-0.16, 0.00] & -2.08 & .037 & 0.01 & [-0.07, 0.08] & 0.14 & .887\\
\ \ \ After-slope * Female, $\hat{\gamma}_{22}$ \textcolor{white}{H} & 0.02 & [-0.03, 0.07] & 0.64 & .525 & 0.02 & [-0.03, 0.07] & 0.84 & .399\\
\ \ \ Shift * Female, $\hat{\gamma}_{32}$ \textcolor{white}{H} & -0.09 & [-0.24, 0.06] & -1.14 & .254 & 0.19 & [0.05, 0.33] & 2.59 & .010\\
\ \ \ Grandparent * Female, $\hat{\gamma}_{03}$ \textcolor{white}{H} & -0.23 & [-0.55, 0.09] & -1.42 & .156 & -0.08 & [-0.40, 0.25] & -0.47 & .637\\
\ \ \ Before-slope * Grandparent * Female, $\hat{\gamma}_{13}$ \textcolor{white}{H} & 0.21 & [0.03, 0.39] & 2.28 & .023 & 0.11 & [-0.05, 0.28] & 1.34 & .181\\
\ \ \ After-slope * Grandparent * Female, $\hat{\gamma}_{23}$ \textcolor{white}{H} & -0.02 & [-0.13, 0.09] & -0.37 & .714 & -0.03 & [-0.13, 0.08] & -0.50 & .615\\
\ \ \ Shift * Grandparent * Female, $\hat{\gamma}_{33}$ \textcolor{white}{H} & 0.01 & [-0.32, 0.34] & 0.06 & .954 & -0.26 & [-0.57, 0.05] & -1.63 & .103\\
\bottomrule
\addlinespace
\insertTableNotes
\end{longtable}

}

\end{lltable}



\begin{figure}
\centering
\includegraphics{Figs/H1-agree-fig-1.pdf}
\caption{\label{fig:H1-agree-fig}Change trajectories of agreeableness based on the basic models (1st column) and the models with the gender interaction (2nd column). The error bars are 95\% confidence intervals of the predicted values which only account for the fixed-effects portion of the model. The vertical line indicates the approximate time of the transition to grandparenthood.}
\end{figure}



\begin{figure}
\centering
\includegraphics{Figs/H1-con-fig-1.pdf}
\caption{\label{fig:H1-con-fig}Change trajectories of conscientiousness based on the basic models (1st column) and the models with the gender interaction (2nd column). The error bars are 95\% confidence intervals of the predicted values which only account for the fixed-effects portion of the model. The vertical line indicates the approximate time of the transition to grandparenthood.}
\end{figure}



\begin{figure}
\centering
\includegraphics{Figs/H1-extra-fig-1.pdf}
\caption{\label{fig:H1-extra-fig}Change trajectories of extraversion based on the basic models (1st column) and the models with the gender interaction (2nd column). The error bars are 95\% confidence intervals of the predicted values which only account for the fixed-effects portion of the model. The vertical line indicates the approximate time of the transition to grandparenthood.}
\end{figure}



\begin{figure}
\centering
\includegraphics{Figs/H1-neur-fig-1.pdf}
\caption{\label{fig:H1-neur-fig}Change trajectories of neuroticism based on the basic models (1st column) and the models with the gender interaction (2nd column). The error bars are 95\% confidence intervals of the predicted values which only account for the fixed-effects portion of the model. The vertical line indicates the approximate time of the transition to grandparenthood.}
\end{figure}



\begin{figure}
\centering
\includegraphics{Figs/H1-open-fig-1.pdf}
\caption{\label{fig:H1-open-fig}Change trajectories of openness based on the basic models (1st column) and the models with the gender interaction (2nd column). The error bars are 95\% confidence intervals of the predicted values which only account for the fixed-effects portion of the model. The vertical line indicates the approximate time of the transition to grandparenthood.}
\end{figure}



\begin{figure}
\centering
\includegraphics{Figs/H1-swls-fig-1.pdf}
\caption{\label{fig:H1-swls-fig}Change trajectories of life satisfaction based on the basic models (1st column) and the models with the gender interaction (2nd column). The error bars are 95\% confidence intervals of the predicted values which only account for the fixed-effects portion of the model. The vertical line indicates the approximate time of the transition to grandparenthood.}
\end{figure}

\hypertarget{discussion}{%
\section{Discussion}\label{discussion}}

Based on

\begin{itemize}
\item
  personality maturation cross-culturally: (Bleidorn et al., 2013; Chopik \& Kitayama, 2018)
\item
  facets / nuances (Mõttus \& Rozgonjuk, 2021)
\item
  arrival of grandchild associated with retirement decisions (Lumsdaine \& Vermeer, 2015); pers X WB interaction over retirement (Henning et al., 2017);
\item
  Does the Transition to Grandparenthood Deter Gray Divorce? A Test of the Braking Hypothesis (Brown et al., 2021)
\item
  prolonged period of grandparenthood? (Margolis \& Wright, 2017)
\item
  subjective experience of aging (Bordone \& Arpino, 2015)
\item
  policy relevance of personality (Bleidorn et al., 2019), e.g., health outcomes (Turiano et al., 2012), but not really evidence for healthy neuroticism (Turiano et al., 2020)
\item
  mortality \& grandparenthood(Christiansen, 2014); moderated by race? (Choi, 2020); but see HRS -\textgreater{} \enquote{Grandparenthood overall was unassociated with mortality risk in both women and men} (Ellwardt et al., 2021)
  --\textgreater{} (Hilbrand et al., n.d.): \enquote{Survival analyses based on data from the Berlin Aging Study revealed that mortality hazards for grandparents who provided non-custodial childcare were 37\% lower than for grandparents who did not provide childcare and for non-grandparents. These associations held after controlling for physical health, age, socioeconomic status and various characteristics of the children and grandchildren.}
\item
  \enquote{Older grandparents tended to provide financial assistance and more strongly identified with the role. When their grandchildren were younger, grandparents tended to interact more with them, share more activities, provide baby-sitting, and receive more symbolic rewards from the grandparent role.} (Silverstein \& Marenco, 2001)
\item
  \enquote{refutes the central claim of role theory according to which salient roles are more beneficial to the psychological well-being of the individual than are other roles, especially in old age. It also questions the theoretical framework of grandparent role meaning that is commonly cited in the literature} (Muller \& Litwin, 2011)
  --\textgreater{} see also (Condon et al., 2019): First-Time Grandparents' Role Satisfaction and Its Determinants
\item
  ``maternal grandmothers tend to invest the most in their grandchildren, followed by maternal grandfathers, then paternal grandmothers, with paternal grandfathers investing the least`` -\textgreater{} also: call for causally informed designs! (Coall \& Hertwig, 2011) --\textgreater{} discusses grandparental role investment from an evolutionary perspective --\textgreater{} see also (Danielsbacka et al., 2011)
\item
  factors determining grandparental investement: (Coall et al., 2014)
\item
  relation to well-being: (Danielsbacka \& Tanskanen, 2016)
\item
  \enquote{Over the last two decades, the share of U.S. children under age 18 who live in a multigenerational household (with a grandparent and parent) has increased dramatically`` (Pilkauskas et al., 2020) --\textgreater{} for Germany:}on the basis of the DEAS data, the share of grandparents who take care of their grandchildren increased between 2008 and 2014" (Mahne \& Klaus, 2017)
\item
  other countries with different childcare systems: (Bordone et al., 2017); \enquote{in countries with scarce publicly funded daycare services and parental leave grandparental care is often provided on a daily basis}; (Hank \& Buber, 2009)
\item
  differences in Big Five assessment: HRS adjectives vs.~LISS statements
\end{itemize}

\hypertarget{limitations}{%
\subsection{Limitations}\label{limitations}}

Despite

\hypertarget{conclusions}{%
\subsection{Conclusions}\label{conclusions}}

Our

\hypertarget{acknowledgements}{%
\subsection{Acknowledgements}\label{acknowledgements}}

We thank X for valuable feedback.

\newpage

\hypertarget{references}{%
\section{References}\label{references}}

\begingroup
\setlength{\parindent}{-0.5in}
\setlength{\leftskip}{0.5in}

\hypertarget{refs}{}
\leavevmode\hypertarget{ref-aassveFirstGlanceBlack2021}{}%
Aassve, A., Luppi, F., \& Mencarini, L. (2021). A first glance into the black box of life satisfaction surrounding childbearing. \emph{Journal of Population Research}. \url{https://doi.org/10.1007/s12546-021-09267-z}

\leavevmode\hypertarget{ref-allemandLongtermCorrelatedChange2008a}{}%
Allemand, M., Zimprich, D., \& Martin, M. (2008). Long-term correlated change in personality traits in old age. \emph{Psychology and Aging}, \emph{23}(3), 545--557. \url{https://doi.org/10.1037/a0013239}

\leavevmode\hypertarget{ref-anusicStabilityChangePersonality2016}{}%
Anusic, I., \& Schimmack, U. (2016). Stability and change of personality traits, self-esteem, and well-being: Introducing the meta-analytic stability and change model of retest correlations. \emph{Journal of Personality and Social Psychology}, \emph{110}(5), 766--781. \url{https://doi.org/10.1037/pspp0000066}

\leavevmode\hypertarget{ref-anusicDoesPersonalityModerate2014}{}%
Anusic, I., Yap, S., \& Lucas, R. E. (2014a). Does personality moderate reaction and adaptation to major life events? Analysis of life satisfaction and affect in an Australian national sample. \emph{Journal of Research in Personality}, \emph{51}, 69--77. \url{https://doi.org/10.1016/j.jrp.2014.04.009}

\leavevmode\hypertarget{ref-anusicTestingSetpointTheory2014}{}%
Anusic, I., Yap, S., \& Lucas, R. E. (2014b). Testing set-point theory in a Swiss national sample: Reaction and adaptation to major life events. \emph{Social Indicators Research}, \emph{119}(3), 1265--1288. \url{https://doi.org/10.1007/s11205-013-0541-2}

\leavevmode\hypertarget{ref-ardeltStillStableAll2000}{}%
Ardelt, M. (2000). Still stable after all these years? Personality stability theory revisited. \emph{Social Psychology Quarterly}, \emph{63}(4), 392--405. \url{https://doi.org/10.2307/2695848}

\leavevmode\hypertarget{ref-arpinoGrandparentingEducationSubjective2018}{}%
Arpino, B., Bordone, V., \& Balbo, N. (2018). Grandparenting, education and subjective well-being of older Europeans. \emph{European Journal of Ageing}, \emph{15}(3), 251--263. \url{https://doi.org/10.1007/s10433-018-0467-2}

\leavevmode\hypertarget{ref-arpinoFamilyHistoriesDemography2018}{}%
Arpino, B., Gumà, J., \& Julià, A. (2018). Family histories and the demography of grandparenthood. \emph{Demographic Research}, \emph{39}(42), 1105--1150. \url{https://doi.org/10.4054/DemRes.2018.39.42}

\leavevmode\hypertarget{ref-ashtonEmpiricalTheoreticalPractical2007}{}%
Ashton, M. C., \& Lee, K. (2007). Empirical, Theoretical, and Practical Advantages of the HEXACO Model of Personality Structure. \emph{Personality and Social Psychology Review}, \emph{11}(2), 150--166. \url{https://doi.org/10.1177/1088868306294907}

\leavevmode\hypertarget{ref-ashtonObjectionsHEXACOModel2020}{}%
Ashton, M. C., \& Lee, K. (2020). Objections to the HEXACO Model of Personality Structureand why those Objections Fail. \emph{European Journal of Personality}, \emph{34}(4), 492--510. \url{https://doi.org/10.1002/per.2242}

\leavevmode\hypertarget{ref-asselmannPersonalityMaturationPersonality2021}{}%
Asselmann, E., \& Specht, J. (2021). Personality maturation and personality relaxation: Differences of the Big Five personality traits in the years around the beginning and ending of working life. \emph{Journal of Personality}, \emph{n/a}(n/a). \url{https://doi.org/10.1111/jopy.12640}

\leavevmode\hypertarget{ref-asselmannTestingSocialInvestment2020}{}%
Asselmann, E., \& Specht, J. (2020). Testing the Social Investment Principle Around Childbirth: Little Evidence for Personality Maturation Before and After Becoming a Parent. \emph{European Journal of Personality}, \emph{n/a}(n/a). \url{https://doi.org/10.1002/per.2269}

\leavevmode\hypertarget{ref-atesDoesGrandchildCare2017}{}%
Ates, M. (2017). Does grandchild care influence grandparents' self-rated health? Evidence from a fixed effects approach. \emph{Social Science \& Medicine}, \emph{190}, 67--74. \url{https://doi.org/10.1016/j.socscimed.2017.08.021}

\leavevmode\hypertarget{ref-R-papaja}{}%
Aust, F., \& Barth, M. (2020). \emph{papaja: Prepare reproducible APA journal articles with R Markdown}. \url{https://github.com/crsh/papaja}

\leavevmode\hypertarget{ref-austinIntroductionPropensityScore2011}{}%
Austin, P. C. (2011). An introduction to propensity score methods for reducing the effects of confounding in observational studies. \emph{Multivariate Behavioral Research}, \emph{46}(3), 399--424. \url{https://doi.org/10.1080/00273171.2011.568786}

\leavevmode\hypertarget{ref-austinDoublePropensityscoreAdjustment2017}{}%
Austin, P. C. (2017). Double propensity-score adjustment: A solution to design bias or bias due to incomplete matching. \emph{Statistical Methods in Medical Research}, \emph{26}(1), 201--222. \url{https://doi.org/10.1177/0962280214543508}

\leavevmode\hypertarget{ref-bairdLifeSatisfactionLifespan2010}{}%
Baird, B. M., Lucas, R. E., \& Donnellan, M. B. (2010). Life satisfaction across the lifespan: Findings from two nationally representative panel studies. \emph{Social Indicators Research}, \emph{99}(2), 183--203. \url{https://doi.org/10.1007/s11205-010-9584-9}

\leavevmode\hypertarget{ref-balboRoleFamilyOrientations2016}{}%
Balbo, N., \& Arpino, B. (2016). The role of family orientations in shaping the effect of fertility on subjective well-being: A propensity score matching approach. \emph{Demography}, \emph{53}(4), 955--978. \url{https://doi.org/10.1007/s13524-016-0480-z}

\leavevmode\hypertarget{ref-baltesLifeSpanTheory2006}{}%
Baltes, P. B., Lindenberger, U., \& Staudinger, U. M. (2006). Life Span Theory in Developmental Psychology. In R. M. Lerner \& W. Damon (Eds.), \emph{Handbook of child psychology: Theoretical models of human development} (pp. 569--664). John Wiley \& Sons Inc.

\leavevmode\hypertarget{ref-R-lme4}{}%
Bates, D., Mächler, M., Bolker, B., \& Walker, S. (2015). Fitting linear mixed-effects models using lme4. \emph{Journal of Statistical Software}, \emph{67}(1), 1--48. \url{https://doi.org/10.18637/jss.v067.i01}

\leavevmode\hypertarget{ref-beckMegaAnalysisPersonalityPrediction2021}{}%
Beck, E. D., \& Jackson, J. J. (2021). A Mega-Analysis of Personality Prediction: Robustness and Boundary Conditions. \emph{Journal of Personality and Social Psychology}, \emph{In Press}. \url{https://doi.org/10.31234/osf.io/7pg9b}

\leavevmode\hypertarget{ref-bengtsonNuclearFamilyIncreasing2001}{}%
Bengtson, V. L. (2001). Beyond the Nuclear Family: The Increasing Importance of Multigenerational Bonds. \emph{Journal of Marriage and Family}, \emph{63}(1), 1--16. \url{https://doi.org/10.1111/j.1741-3737.2001.00001.x}

\leavevmode\hypertarget{ref-benjaminRedefineStatisticalSignificance2018}{}%
Benjamin, D. J., Berger, J. O., Clyde, M., Wolpert, R. L., Johnson, V. E., Johannesson, M., Dreber, A., Nosek, B. A., Wagenmakers, E. J., Berk, R., \& Brembs, B. (2018). Redefine statistical significance. \emph{Nature Human Behavior}, \emph{2}, 6--10. \url{https://doi.org/10.1038/s41562-017-0189-z}

\leavevmode\hypertarget{ref-bleidornPolicyRelevancePersonality2019}{}%
Bleidorn, W., Hill, P. L., Back, M. D., Denissen, J. J. A., Hennecke, M., Hopwood, C. J., Jokela, M., Kandler, C., Lucas, R. E., Luhmann, M., Orth, U., Wagner, J., Wrzus, C., Zimmermann, J., \& Roberts, B. W. (2019). The policy relevance of personality traits. \emph{American Psychologist}, \emph{74}(9), 1056--1067. \url{https://doi.org/10.1037/amp0000503}

\leavevmode\hypertarget{ref-bleidornPersonalityTraitStability2021}{}%
Bleidorn, W., Hopwood, C. J., Back, M. D., Denissen, J. J. A., Hennecke, M., Hill, P. L., Jokela, M., Kandler, C., Lucas, R. E., Luhmann, M., Orth, U., Roberts, B. W., Wagner, J., Wrzus, C., \& Zimmermann, J. (2021). Personality Trait Stability and Change. \emph{Personality Science}, \emph{2}(1), 1--20. \url{https://doi.org/10.5964/ps.6009}

\leavevmode\hypertarget{ref-bleidornLifeEventsPersonality2018}{}%
Bleidorn, W., Hopwood, C. J., \& Lucas, R. E. (2018). Life events and personality trait change. \emph{Journal of Personality}, \emph{86}(1), 83--96. \url{https://doi.org/10.1111/jopy.12286}

\leavevmode\hypertarget{ref-bleidornPersonalityMaturationWorld2013}{}%
Bleidorn, W., Klimstra, T. A., Denissen, J. J. A., Rentfrow, P. J., Potter, J., \& Gosling, S. D. (2013). Personality Maturation Around the World: A Cross-Cultural Examination of Social-Investment Theory. \emph{Psychological Science}, \emph{24}(12), 2530--2540. \url{https://doi.org/10.1177/0956797613498396}

\leavevmode\hypertarget{ref-bleidornRetirementAssociatedChange2018}{}%
Bleidorn, W., \& Schwaba, T. (2018). Retirement is associated with change in self-esteem. \emph{Psychology and Aging}, \emph{33}(4), 586--594. \url{https://doi.org/10.1037/pag0000253}

\leavevmode\hypertarget{ref-bleidornPersonalityDevelopmentEmerging2017}{}%
Bleidorn, W., \& Schwaba, T. (2017). Personality development in emerging adulthood. In J. Specht (Ed.), \emph{Personality Development Across the Lifespan} (pp. 39--51). Academic Press. \url{https://doi.org/10.1016/B978-0-12-804674-6.00004-1}

\leavevmode\hypertarget{ref-bordoneGrandchildrenInfluenceHow2015}{}%
Bordone, V., \& Arpino, B. (2015). Do Grandchildren Influence How Old You Feel? \emph{Journal of Aging and Health}, \emph{28}(6), 1055--1072. \url{https://doi.org/10.1177/0898264315618920}

\leavevmode\hypertarget{ref-bordonePatternsGrandparentalChild2017}{}%
Bordone, V., Arpino, B., \& Aassve, A. (2017). Patterns of grandparental child care across Europe: The role of the policy context and working mothers' need. \emph{Ageing and Society}, \emph{37}(4), 845--873. \url{https://doi.org/10.1017/S0144686X1600009X}

\leavevmode\hypertarget{ref-brownDoesTransitionGrandparenthood2021}{}%
Brown, S. L., Lin, I.-F., \& Mellencamp, K. A. (2021). Does the Transition to Grandparenthood Deter Gray Divorce? A Test of the Braking Hypothesis. \emph{Social Forces}, \emph{99}(3), 1209--1232. \url{https://doi.org/10.1093/sf/soaa030}

\leavevmode\hypertarget{ref-bruderlFixedEffectsPanelRegression2015}{}%
Brüderl, J., \& Ludwig, V. (2015). \emph{Fixed-Effects Panel Regression} (H. Best \& C. Wolf, Eds.). SAGE.

\leavevmode\hypertarget{ref-burgetteMultipleImputationMissing2010}{}%
Burgette, L. F., \& Reiter, J. P. (2010). Multiple Imputation for Missing Data via Sequential Regression Trees. \emph{American Journal of Epidemiology}, \emph{172}(9), 1070--1076. \url{https://doi.org/10.1093/aje/kwq260}

\leavevmode\hypertarget{ref-caspiWhenIndividualDifferences1993}{}%
Caspi, A., \& Moffitt, T. E. (1993). When do individual differences matter? A paradoxical theory of personality coherence. \emph{Psychological Inquiry}, \emph{4}(4), 247--271. \url{https://doi.org/10.1207/s15327965pli0404_1}

\leavevmode\hypertarget{ref-choiGrandparentingMortalityHow2020}{}%
Choi, S.-w. E. (2020). Grandparenting and Mortality: How Does Race-Ethnicity Matter? \emph{Journal of Health and Social Behavior}, \emph{61}(1), 96--112. \url{https://doi.org/10.1177/0022146520903282}

\leavevmode\hypertarget{ref-chopikDoesPersonalityChange2018}{}%
Chopik, W. J. (2018). Does personality change following spousal bereavement? \emph{Journal of Research in Personality}, \emph{72}, 10--21. \url{https://doi.org/10.1016/j.jrp.2016.08.010}

\leavevmode\hypertarget{ref-chopikPersonalityChangeLife2018}{}%
Chopik, W. J., \& Kitayama, S. (2018). Personality change across the life span: Insights from a cross-cultural, longitudinal study. \emph{Journal of Personality}, \emph{86}(3), 508--521. \url{https://doi.org/10.1111/jopy.12332}

\leavevmode\hypertarget{ref-chopikChangesOptimismPessimism2020}{}%
Chopik, W. J., Oh, J., Kim, E. S., Schwaba, T., Krämer, M. D., Richter, D., \& Smith, J. (2020). Changes in optimism and pessimism in response to life events: Evidence from three large panel studies. \emph{Journal of Research in Personality}, \emph{88}, 103985. \url{https://doi.org/10.1016/j.jrp.2020.103985}

\leavevmode\hypertarget{ref-christiansenAssociationGrandparenthoodMortality2014}{}%
Christiansen, S. G. (2014). The association between grandparenthood and mortality. \emph{Social Science \& Medicine}, \emph{118}, 89--96. \url{https://doi.org/10.1016/j.socscimed.2014.07.061}

\leavevmode\hypertarget{ref-chungLongitudinalEffectsGrandchild2018}{}%
Chung, S., \& Park, A. (2018). The longitudinal effects of grandchild care on depressive symptoms and physical health of grandmothers in South Korea: A latent growth approach. \emph{Aging \& Mental Health}, \emph{22}(12), 1556--1563. \url{https://doi.org/10.1080/13607863.2017.1376312}

\leavevmode\hypertarget{ref-coallGrandparentalInvestmentRelic2011}{}%
Coall, D. A., \& Hertwig, R. (2011). Grandparental Investment: A Relic of the Past or a Resource for the Future? \emph{Current Directions in Psychological Science}, \emph{20}(2), 93--98. \url{https://doi.org/10.1177/0963721411403269}

\leavevmode\hypertarget{ref-coallPredictorsGrandparentalInvestment2014}{}%
Coall, D. A., Hilbrand, S., \& Hertwig, R. (2014). Predictors of Grandparental Investment Decisions in Contemporary Europe: Biological Relatedness and Beyond. \emph{PLOS ONE}, \emph{9}(1), e84082. \url{https://doi.org/10.1371/journal.pone.0084082}

\leavevmode\hypertarget{ref-coallInterdisciplinaryPerspectivesGrandparental2018}{}%
Coall, D. A., Hilbrand, S., Sear, R., \& Hertwig, R. (2018). Interdisciplinary perspectives on grandparental investment: A journey towards causality. \emph{Contemporary Social Science}, \emph{13}(2), 159--174. \url{https://doi.org/10.1080/21582041.2018.1433317}

\leavevmode\hypertarget{ref-condonFirstTimeGrandparentsRole2019}{}%
Condon, J., Luszcz, M., \& McKee, I. (2019). First-Time Grandparents' Role Satisfaction and Its Determinants. \emph{The International Journal of Aging and Human Development}, Advance Online Publication. \url{https://doi.org/10.1177/0091415019882005}

\leavevmode\hypertarget{ref-condonTransitionGrandparenthoodProspective2018}{}%
Condon, J., Luszcz, M., \& McKee, I. (2018). The transition to grandparenthood: A prospective study of mental health implications. \emph{Aging \& Mental Health}, \emph{22}(3), 336--343. \url{https://doi.org/10.1080/13607863.2016.1248897}

\leavevmode\hypertarget{ref-cookHowMuchBias2020}{}%
Cook, T. D., Zhu, N., Klein, A., Starkey, P., \& Thomas, J. (2020). How much bias results if a quasi-experimental design combines local comparison groups, a pretest outcome measure and other covariates?: A within study comparison of preschool effects. \emph{Psychological Methods}, \emph{Advance Online Publication}, 0. \url{https://doi.org/10.1037/met0000260}

\leavevmode\hypertarget{ref-costaPersonalityLifeSpan2019}{}%
Costa, P. T., McCrae, R. R., \& Löckenhoff, C. E. (2019). Personality Across the Life Span. \emph{Annual Review of Psychology}, \emph{70}(1), 423--448. \url{https://doi.org/10.1146/annurev-psych-010418-103244}

\leavevmode\hypertarget{ref-damianSixteenGoingSixtysix2019}{}%
Damian, R. I., Spengler, M., Sutu, A., \& Roberts, B. W. (2019). Sixteen going on sixty-six: A longitudinal study of personality stability and change across 50 years. \emph{Journal of Personality and Social Psychology}, \emph{117}(3), 674--695. \url{https://doi.org/10.1037/pspp0000210}

\leavevmode\hypertarget{ref-danielsbackaAssociationGrandparentalInvestment2016}{}%
Danielsbacka, M., \& Tanskanen, A. O. (2016). The association between grandparental investment and grandparents' happiness in Finland. \emph{Personal Relationships}, \emph{23}(4), 787--800. \url{https://doi.org/10.1111/pere.12160}

\leavevmode\hypertarget{ref-danielsbackaGrandparentalChildcareHealth2019}{}%
Danielsbacka, M., Tanskanen, A. O., Coall, D. A., \& Jokela, M. (2019). Grandparental childcare, health and well-being in Europe: A within-individual investigation of longitudinal data. \emph{Social Science \& Medicine}, \emph{230}, 194--203. \url{https://doi.org/10.1016/j.socscimed.2019.03.031}

\leavevmode\hypertarget{ref-danielsbackaGrandparentalChildCare2011}{}%
Danielsbacka, M., Tanskanen, A. O., Jokela, M., \& Rotkirch, A. (2011). Grandparental Child Care in Europe: Evidence for Preferential Investment in More Certain Kin. \emph{Evolutionary Psychology}, \emph{9}(1), 147470491100900102. \url{https://doi.org/10.1177/147470491100900102}

\leavevmode\hypertarget{ref-denissenBigFiveInventory2020}{}%
Denissen, J. J. A., Geenen, R., Soto, C. J., John, O. P., \& van Aken, M. A. G. (2020). The Big Five Inventory2: Replication of Psychometric Properties in a Dutch Adaptation and First Evidence for the Discriminant Predictive Validity of the Facet Scales. \emph{Journal of Personality Assessment}, \emph{102}(3), 309--324. \url{https://doi.org/10.1080/00223891.2018.1539004}

\leavevmode\hypertarget{ref-denissenTransactionsLifeEvents2019}{}%
Denissen, J. J. A., Luhmann, M., Chung, J. M., \& Bleidorn, W. (2019). Transactions between life events and personality traits across the adult lifespan. \emph{Journal of Personality and Social Psychology}, \emph{116}(4), 612--633. \url{https://doi.org/10.1037/pspp0000196}

\leavevmode\hypertarget{ref-dienerSatisfactionLifeScale1985}{}%
Diener, E., Emmons, R. A., Larsen, R. J., \& Griffin, S. (1985). The Satisfaction With Life Scale. \emph{Journal of Personality Assessment}, \emph{49}(1), 71--75. \url{https://doi.org/10.1207/s15327752jpa4901_13}

\leavevmode\hypertarget{ref-digessaBecomingGrandparentIts2019}{}%
Di Gessa, G., Bordone, V., \& Arpino, B. (2019). Becoming a Grandparent and Its Effect on Well-Being: The Role of Order of Transitions, Time, and Gender. \emph{The Journals of Gerontology, Series B: Psychological Sciences and Social Sciences}, Advance Online Publication. \url{https://doi.org/10.1093/geronb/gbz135}

\leavevmode\hypertarget{ref-digessaHealthImpactIntensive2016}{}%
Di Gessa, G., Glaser, K., \& Tinker, A. (2016a). The Health Impact of Intensive and Nonintensive Grandchild Care in Europe: New Evidence From SHARE. \emph{The Journals of Gerontology, Series B: Psychological Sciences and Social Sciences}, \emph{71}(5), 867--879. \url{https://doi.org/10.1093/geronb/gbv055}

\leavevmode\hypertarget{ref-digessaImpactCaringGrandchildren2016}{}%
Di Gessa, G., Glaser, K., \& Tinker, A. (2016b). The impact of caring for grandchildren on the health of grandparents in Europe: A lifecourse approach. \emph{Social Science \& Medicine}, \emph{152}, 166--175. \url{https://doi.org/10.1016/j.socscimed.2016.01.041}

\leavevmode\hypertarget{ref-dorePopulationIndividuallevelChanges2018}{}%
Doré, B., \& Bolger, N. (2018). Population- and individual-level changes in life satisfaction surrounding major life stressors. \emph{Social Psychological and Personality Science}, \emph{9}(7), 875--884. \url{https://doi.org/10.1177/1948550617727589}

\leavevmode\hypertarget{ref-ellwardtGrandparenthoodRiskMortality2021}{}%
Ellwardt, L., Hank, K., \& Mendes de Leon, C. F. (2021). Grandparenthood and risk of mortality: Findings from the Health and Retirement Study. \emph{Social Science \& Medicine}, \emph{268}, 113371. \url{https://doi.org/10.1016/j.socscimed.2020.113371}

\leavevmode\hypertarget{ref-elwertEndogenousSelectionBias2014}{}%
Elwert, F., \& Winship, C. (2014). Endogenous Selection Bias: The Problem of Conditioning on a Collider Variable. \emph{Annual Review of Sociology}, \emph{40}(1), 31--53. \url{https://doi.org/10.1146/annurev-soc-071913-043455}

\leavevmode\hypertarget{ref-car2019}{}%
Fox, J., \& Weisberg, S. (2019). \emph{An R companion to applied regression} (Third). Sage.

\leavevmode\hypertarget{ref-goldbergDevelopmentMarkersBigFive1992}{}%
Goldberg, L. R. (1992). The development of markers for the Big-Five factor structure. \emph{Psychological Assessment}, \emph{4}(1), 26--42. \url{https://doi.org/10.1037/1040-3590.4.1.26}

\leavevmode\hypertarget{ref-grahamTrajectoriesBigFive2020}{}%
Graham, E. K., Weston, S. J., Gerstorf, D., Yoneda, T. B., Booth, T., Beam, C. R., Petkus, A. J., Drewelies, J., Hall, A. N., Bastarache, E. D., Estabrook, R., Katz, M. J., Turiano, N. A., Lindenberger, U., Smith, J., Wagner, G. G., Pedersen, N. L., Allemand, M., Spiro Iii, A., \ldots{} Mroczek, D. K. (2020). Trajectories of Big Five Personality Traits: A Coordinated Analysis of 16 Longitudinal Samples. \emph{European Journal of Personality}, \emph{n/a}(n/a). \url{https://doi.org/10.1002/per.2259}

\leavevmode\hypertarget{ref-greenlandQuantifyingBiasesCausal2003}{}%
Greenland, S. (2003). Quantifying biases in causal models: Classical confounding vs collider-stratification bias. \emph{Epidemiology}, \emph{14}(3), 300--306. \url{https://doi.org/10.1097/01.EDE.0000042804.12056.6C}

\leavevmode\hypertarget{ref-greenlandCriticalLookMethods1995}{}%
Greenland, S., \& Finkle, W. D. (1995). A Critical Look at Methods for Handling Missing Covariates in Epidemiologic Regression Analyses. \emph{American Journal of Epidemiology}, \emph{142}(12), 1255--1264. \url{https://doi.org/10.1093/oxfordjournals.aje.a117592}

\leavevmode\hypertarget{ref-hagestadAgeLifeCourse1985}{}%
Hagestad, G. O., \& Neugarten, B. L. (1985). Age and the life course. In E. Shanas \& R. Binstock (Eds.), \emph{Handbook of aging and the social sciences}. Van Nostrand and Reinhold.

\leavevmode\hypertarget{ref-hallbergPretestMeasuresStudy2018}{}%
Hallberg, K., Cook, T. D., Steiner, P. M., \& Clark, M. H. (2018). Pretest Measures of the Study Outcome and the Elimination of Selection Bias: Evidence from Three Within Study Comparisons. \emph{Prevention Science}, \emph{19}(3), 274--283. \url{https://doi.org/10.1007/s11121-016-0732-6}

\leavevmode\hypertarget{ref-hankGrandparentsCaringTheir2009}{}%
Hank, K., \& Buber, I. (2009). Grandparents Caring for their Grandchildren: Findings From the 2004 Survey of Health, Ageing, and Retirement in Europe. \emph{Journal of Family Issues}, \emph{30}(1), 53--73. \url{https://doi.org/10.1177/0192513X08322627}

\leavevmode\hypertarget{ref-hayslipGrandparentsRaisingGrandchildren2019}{}%
Hayslip, B., Jr, Fruhauf, C. A., \& Dolbin-MacNab, M. L. (2019). Grandparents Raising Grandchildren: What Have We Learned Over the Past Decade? \emph{The Gerontologist}, \emph{59}(3), e152--e163. \url{https://doi.org/10.1093/geront/gnx106}

\leavevmode\hypertarget{ref-henningRolePersonalitySubjective2017}{}%
Henning, G., Hansson, I., Berg, A. I., Lindwall, M., \& Johansson, B. (2017). The role of personality for subjective well-being in the retirement transition Comparing variable- and person-oriented models. \emph{Personality and Individual Differences}, \emph{116}, 385--392. \url{https://doi.org/10.1016/j.paid.2017.05.017}

\leavevmode\hypertarget{ref-hentschelInfluenceMajorLife2017}{}%
Hentschel, S., Eid, M., \& Kutscher, T. (2017). The Influence of Major Life Events and Personality Traits on the Stability of Affective Well-Being. \emph{Journal of Happiness Studies}, \emph{18}(3), 719--741. \url{https://doi.org/10.1007/s10902-016-9744-y}

\leavevmode\hypertarget{ref-hilbrandCaregivingFamilyAssociated2017}{}%
Hilbrand, S., Coall, D. A., Gerstorf, D., \& Hertwig, R. (n.d.). Caregiving within and beyond the family is associated with lower mortality for the caregiver: A prospective study. \emph{Evolution and Human Behavior}, \emph{38}(3), 397--403. \url{https://doi.org/10.1016/j.evolhumbehav.2016.11.010}

\leavevmode\hypertarget{ref-MatchIt2011}{}%
Ho, D. E., Imai, K., King, G., \& Stuart, E. A. (2011). MatchIt: Nonparametric preprocessing for parametric causal inference. \emph{Journal of Statistical Software}, \emph{42}(8), 1--28.

\leavevmode\hypertarget{ref-hoffmanLongitudinalAnalysisModeling2015}{}%
Hoffman, L. (2015). \emph{Longitudinal analysis: Modeling within-person fluctuation and change}. Routledge/Taylor \& Francis Group.

\leavevmode\hypertarget{ref-huttemanDevelopmentalTasksFramework2014}{}%
Hutteman, R., Hennecke, M., Orth, U., Reitz, A. K., \& Specht, J. (2014). Developmental Tasks as a Framework to Study Personality Development in Adulthood and Old Age. \emph{European Journal of Personality}, \emph{28}(3), 267--278. \url{https://doi.org/10.1002/per.1959}

\leavevmode\hypertarget{ref-infurnaMidlife2020sOpportunities2020}{}%
Infurna, F. J., Gerstorf, D., \& Lachman, M. E. (2020). Midlife in the 2020s: Opportunities and challenges. \emph{American Psychologist}, \emph{75}(4), 470--485. \url{https://doi.org/10.1037/amp0000591}

\leavevmode\hypertarget{ref-johnParadigmShiftIntegrative2008}{}%
John, O. P., Naumann, L. P., \& Soto, C. J. (2008). Paradigm shift to the integrative Big Five trait taxonomy: History, measurement, and conceptual issues. In O. P. John, R. W. Robins, \& L. A. Pervin (Eds.), \emph{Handbook of personality: Theory and research} (pp. 114--158). The Guilford Press.

\leavevmode\hypertarget{ref-johnsonImpactHavingChildren2006}{}%
Johnson, A. B., \& Rodgers, J. L. (2006). The impact of having children on the lives of women: The Effects of Children Questionnaire. \emph{Journal of Applied Social Psychology}, \emph{36}(11), 2685--2714. \url{https://doi.org/10.1111/j.0021-9029.2006.00123.x}

\leavevmode\hypertarget{ref-kandlerPatternsSourcesPersonality2015a}{}%
Kandler, C., Kornadt, A. E., Hagemeyer, B., \& Neyer, F. J. (2015). Patterns and sources of personality development in old age. \emph{Journal of Personality and Social Psychology}, \emph{109}(1), 175--191. \url{https://doi.org/10.1037/pspp0000028}

\leavevmode\hypertarget{ref-kramerImpactHavingChildren2020}{}%
Krämer, M. D., \& Rodgers, J. L. (2020). The impact of having children on domain-specific life satisfaction: A quasi-experimental longitudinal investigation using the Socio-Economic Panel (SOEP) data. \emph{Journal of Personality and Social Psychology}, \emph{119}(6), 1497--1514. \url{https://doi.org/10.1037/pspp0000279}

\leavevmode\hypertarget{ref-R-lmerTest}{}%
Kuznetsova, A., Brockhoff, P. B., \& Christensen, R. H. B. (2017). lmerTest package: Tests in linear mixed effects models. \emph{Journal of Statistical Software}, \emph{82}(13), 1--26. \url{https://doi.org/10.18637/jss.v082.i13}

\leavevmode\hypertarget{ref-lachmanMidlifeDevelopmentInventory1997}{}%
Lachman, M. E., \& Weaver, S. L. (1997). \emph{The Midlife Development Inventory (MIDI) personality scales: Scale construction and scoring}. Brandeis University.

\leavevmode\hypertarget{ref-leopoldDemographyGrandparenthoodInternational2015}{}%
Leopold, T., \& Skopek, J. (2015). The Demography of Grandparenthood: An International Profile. \emph{Social Forces}, \emph{94}(2), 801--832. \url{https://doi.org/10.1093/sf/sov066}

\leavevmode\hypertarget{ref-lodi-smithSocialInvestmentPersonality2007}{}%
Lodi-Smith, J., \& Roberts, B. W. (2007). Social Investment and Personality: A Meta-Analysis of the Relationship of Personality Traits to Investment in Work, Family, Religion, and Volunteerism. \emph{Personality and Social Psychology Review}, \emph{11}(1), 68--86. \url{https://doi.org/10.1177/1088868306294590}

\leavevmode\hypertarget{ref-lucasPersonalityDevelopmentLife2011}{}%
Lucas, R. E., \& Donnellan, M. B. (2011). Personality development across the life span: Longitudinal analyses with a national sample from Germany. \emph{Journal of Personality and Social Psychology}, \emph{101}(4), 847--861. \url{https://doi.org/10.1037/a0024298}

\leavevmode\hypertarget{ref-luhmannDimensionalTaxonomyPerceived2020}{}%
Luhmann, M., Fassbender, I., Alcock, M., \& Haehner, P. (2020). A dimensional taxonomy of perceived characteristics of major life events. \emph{Journal of Personality and Social Psychology}, No Pagination Specified--No Pagination Specified. \url{https://doi.org/10.1037/pspp0000291}

\leavevmode\hypertarget{ref-luhmannSubjectiveWellbeingAdaptation2012}{}%
Luhmann, M., Hofmann, W., Eid, M., \& Lucas, R. E. (2012). Subjective well-being and adaptation to life events: A meta-analysis. \emph{Journal of Personality and Social Psychology}, \emph{102}(3), 592--615. \url{https://doi.org/10.1037/a0025948}

\leavevmode\hypertarget{ref-luhmannStudyingChangesLife2014}{}%
Luhmann, M., Orth, U., Specht, J., Kandler, C., \& Lucas, R. E. (2014). Studying changes in life circumstances and personality: It's about time. \emph{European Journal of Personality}, \emph{28}(3), 256--266. \url{https://doi.org/10.1002/per.1951}

\leavevmode\hypertarget{ref-lumsdaineRetirementTimingWomen2015}{}%
Lumsdaine, R. L., \& Vermeer, S. J. C. (2015). Retirement timing of women and the role of care responsibilities for grandchildren. \emph{Demography}, \emph{52}(2), 433--454. \url{https://doi.org/10.1007/s13524-015-0382-5}

\leavevmode\hypertarget{ref-ludtkeRandomWalkUniversity2011}{}%
Lüdtke, O., Roberts, B. W., Trautwein, U., \& Nagy, G. (2011). A random walk down university avenue: Life paths, life events, and personality trait change at the transition to university life. \emph{Journal of Personality and Social Psychology}, \emph{101}(3), 620--637. \url{https://doi.org/10.1037/a0023743}

\leavevmode\hypertarget{ref-maccallumPracticeDichotomizationQuantitative2002}{}%
MacCallum, R. C., Zhang, S., Preacher, K. J., \& Rucker, D. D. (2002). On the practice of dichotomization of quantitative variables. \emph{Psychological Methods}, \emph{7}(1), 19--40. \url{https://doi.org/10.1037/1082-989X.7.1.19}

\leavevmode\hypertarget{ref-mahneGrandparenthoodSubjectiveWellBeing2014}{}%
Mahne, K., \& Huxhold, O. (2014). Grandparenthood and Subjective Well-Being: Moderating Effects of Educational Level. \emph{The Journals of Gerontology: Series B}, \emph{70}(5), 782--792. \url{https://doi.org/10.1093/geronb/gbu147}

\leavevmode\hypertarget{ref-mahneZwischenEnkelgluckUnd2017}{}%
Mahne, K., \& Klaus, D. (2017). Zwischen Enkelglück und (Groß-)Elternpflicht die Bedeutung und Ausgestaltung von Beziehungen zwischen Großeltern und Enkelkindern. In K. Mahne, J. K. Wolff, J. Simonson, \& C. Tesch-Römer (Eds.), \emph{Altern im Wandel: Zwei Jahrzehnte Deutscher Alterssurvey (DEAS)} (pp. 231--245). Springer Fachmedien Wiesbaden. \url{https://doi.org/10.1007/978-3-658-12502-8_15}

\leavevmode\hypertarget{ref-margolisCohortPerspectiveDemography2019}{}%
Margolis, R., \& Verdery, A. M. (2019). A Cohort Perspective on the Demography of Grandparenthood: Past, Present, and Future Changes in Race and Sex Disparities in the United States. \emph{Demography}, \emph{56}(4), 1495--1518. \url{https://doi.org/10.1007/s13524-019-00795-1}

\leavevmode\hypertarget{ref-margolisHealthyGrandparenthoodHow2017}{}%
Margolis, R., \& Wright, L. (2017). Healthy Grandparenthood: How Long Is It, and How Has It Changed? \emph{Demography}, \emph{54}(6), 2073--2099. \url{https://doi.org/10.1007/s13524-017-0620-0}

\leavevmode\hypertarget{ref-marshMeasurementInvarianceBigfive2013}{}%
Marsh, H. W., Nagengast, B., \& Morin, A. J. S. (2013). Measurement invariance of big-five factors over the life span: ESEM tests of gender, age, plasticity, maturity, and la dolce vita effects. \emph{Developmental Psychology}, \emph{49}(6), 1194--1218. \url{https://doi.org/10.1037/a0026913}

\leavevmode\hypertarget{ref-mccraeModeratedAnalysesLongitudinal1993}{}%
McCrae, R. R. (1993). Moderated analyses of longitudinal personality stability. \emph{Journal of Personality and Social Psychology}, \emph{65}(3), 577--585. \url{https://doi.org/10.1037/0022-3514.65.3.577}

\leavevmode\hypertarget{ref-mcneishThanksCoefficientAlpha2018}{}%
McNeish, D. (2018). Thanks coefficient alpha, we'll take it from here. \emph{Psychological Methods}, \emph{23}(3), 412--433. \url{https://doi.org/10.1037/met0000144}

\leavevmode\hypertarget{ref-mcneishFixedEffectsModels2019}{}%
McNeish, D., \& Kelley, K. (2019). Fixed effects models versus mixed effects models for clustered data: Reviewing the approaches, disentangling the differences, and making recommendations. \emph{Psychological Methods}, \emph{24}(1), 20--35. \url{https://doi.org/10.1037/met0000182}

\leavevmode\hypertarget{ref-meyerGrandparentingUnitedStates2017}{}%
Meyer, M. H., \& Kandic, A. (2017). Grandparenting in the United States. \emph{Innovation in Aging}, \emph{1}(2), 1--10. \url{https://doi.org/10.1093/geroni/igx023}

\leavevmode\hypertarget{ref-mitraComparisonTwoMethods2016}{}%
Mitra, R., \& Reiter, J. P. (2016). A comparison of two methods of estimating propensity scores after multiple imputation. \emph{Statistical Methods in Medical Research}, \emph{25}(1), 188--204. \url{https://doi.org/10.1177/0962280212445945}

\leavevmode\hypertarget{ref-mottusPersonalityTraitsOld2012}{}%
Mõttus, R., Johnson, W., \& Deary, I. J. (2012). Personality traits in old age: Measurement and rank-order stability and some mean-level change. \emph{Psychology and Aging}, \emph{27}(1), 243--249. \url{https://doi.org/10.1037/a0023690}

\leavevmode\hypertarget{ref-mottusPersonalityTraitsFacets2017}{}%
Mõttus, R., Kandler, C., Bleidorn, W., Riemann, R., \& McCrae, R. R. (2017). Personality traits below facets: The consensual validity, longitudinal stability, heritability, and utility of personality nuances. \emph{Journal of Personality and Social Psychology}, \emph{112}(3), 474--490. \url{https://doi.org/10.1037/pspp0000100}

\leavevmode\hypertarget{ref-mottusDevelopmentDetailsAge2021}{}%
Mõttus, R., \& Rozgonjuk, D. (2021). Development is in the details: Age differences in the Big Five domains, facets, and nuances. \emph{Journal of Personality and Social Psychology}, \emph{120}(4), 1035--1048. \url{https://doi.org/10.1037/pspp0000276}

\leavevmode\hypertarget{ref-muellerPersonalityDevelopmentOld2016}{}%
Mueller, S., Wagner, J., Drewelies, J., Duezel, S., Eibich, P., Specht, J., Demuth, I., Steinhagen-Thiessen, E., Wagner, G. G., \& Gerstorf, D. (2016). Personality development in old age relates to physical health and cognitive performance: Evidence from the Berlin Aging Study II. \emph{Journal of Research in Personality}, \emph{65}, 94--108. \url{https://doi.org/10.1016/j.jrp.2016.08.007}

\leavevmode\hypertarget{ref-mullerGrandparentingWellbeingHow2011}{}%
Muller, Z., \& Litwin, H. (2011). Grandparenting and well-being: How important is grandparent-role centrality? \emph{European Journal of Ageing}, \emph{8}, 109--118. \url{https://doi.org/10.1007/s10433-011-0185-5}

\leavevmode\hypertarget{ref-ozerPersonalityPredictionConsequential2005}{}%
Ozer, D. J., \& Benet-Martínez, V. (2005). Personality and the Prediction of Consequential Outcomes. \emph{Annual Review of Psychology}, \emph{57}(1), 401--421. \url{https://doi.org/10.1146/annurev.psych.57.102904.190127}

\leavevmode\hypertarget{ref-pearlCausalInferenceStatistics2009}{}%
Pearl, J. (2009). Causal inference in statistics: An overview. \emph{Statistics Surveys}, \emph{3}, 96--146. \url{https://doi.org/10.1214/09-SS057}

\leavevmode\hypertarget{ref-pilkauskasHistoricalTrendsChildren2020}{}%
Pilkauskas, N. V., Amorim, M., \& Dunifon, R. E. (2020). Historical Trends in Children Living in Multigenerational Households in the United States: 18702018. \emph{Demography}, \emph{57}(6), 2269--2296. \url{https://doi.org/10.1007/s13524-020-00920-5}

\leavevmode\hypertarget{ref-R-nlme}{}%
Pinheiro, J., Bates, D., \& R-core. (2021). \emph{Nlme: Linear and nonlinear mixed effects models} {[}Manual{]}.

\leavevmode\hypertarget{ref-R-base}{}%
R Core Team. (2021). \emph{R: A language and environment for statistical computing}. R Foundation for Statistical Computing. \url{https://www.R-project.org/}

\leavevmode\hypertarget{ref-robertsRankorderConsistencyPersonality2000}{}%
Roberts, B. W., \& DelVecchio, W. F. (2000). The rank-order consistency of personality traits from childhood to old age: A quantitative review of longitudinal studies. \emph{Psychological Bulletin}, \emph{126}(1), 3--25. \url{https://doi.org/10.1037/0033-2909.126.1.3}

\leavevmode\hypertarget{ref-robertsPowerPersonalityComparative2007}{}%
Roberts, B. W., Kuncel, N. R., Shiner, R., Caspi, A., \& Goldberg, L. R. (2007). The Power of Personality: The Comparative Validity of Personality Traits, Socioeconomic Status, and Cognitive Ability for Predicting Important Life Outcomes. \emph{Perspectives on Psychological Science}, \emph{2}(4), 313--345. \url{https://doi.org/10.1111/j.1745-6916.2007.00047.x}

\leavevmode\hypertarget{ref-robertsPatternsMeanlevelChange2006a}{}%
Roberts, B. W., Walton, K. E., \& Viechtbauer, W. (2006). Patterns of mean-level change in personality traits across the life course: A meta-analysis of longitudinal studies. \emph{Psychological Bulletin}, \emph{132}, 1--25. \url{https://doi.org/10.1037/0033-2909.132.1.1}

\leavevmode\hypertarget{ref-robertsPersonalityDevelopmentContext2006}{}%
Roberts, B. W., \& Wood, D. (2006). Personality Development in the Context of the Neo-Socioanalytic Model of Personality. In D. K. Mroczek \& T. D. Little (Eds.), \emph{Handbook of Personality Development}. Routledge.

\leavevmode\hypertarget{ref-robertsEvaluatingFiveFactor2005}{}%
Roberts, B. W., Wood, D., \& Smith, J. L. (2005). Evaluating Five Factor Theory and social investment perspectives on personality trait development. \emph{Journal of Research in Personality}, \emph{39}(1), 166--184. \url{https://doi.org/10.1016/j.jrp.2004.08.002}

\leavevmode\hypertarget{ref-robertsPersonalityPsychology2021}{}%
Roberts, B. W., \& Yoon, H. J. (2021). Personality Psychology. \emph{Annual Review of Psychology}, \emph{in press}. \url{https://doi.org/10.1146/annurev-psych-020821-114927}

\leavevmode\hypertarget{ref-rohrerThinkingClearlyCorrelations2018}{}%
Rohrer, J. M. (2018). Thinking Clearly About Correlations and Causation: Graphical Causal Models for Observational Data. \emph{Advances in Methods and Practices in Psychological Science}, \emph{1}(1), 27--42. \url{https://doi.org/10.1177/2515245917745629}

\leavevmode\hypertarget{ref-rosenbaumConsquencesAdjustmentConcomitant1984}{}%
Rosenbaum, P. (1984). The consquences of adjustment for a concomitant variable that has been affected by the treatment. \emph{Journal of the Royal Statistical Society. Series A (General)}, \emph{147}(5), 656--666. \url{https://doi.org/10.2307/2981697}

\leavevmode\hypertarget{ref-scherpenzeelDataCollectionProbabilityBased2011}{}%
Scherpenzeel, A. (2011). Data Collection in a Probability-Based Internet Panel: How the LISS Panel Was Built and How It Can Be Used. \emph{Bulletin of Sociological Methodology/Bulletin de Méthodologie Sociologique}, \emph{109}(1), 56--61. \url{https://doi.org/10.1177/0759106310387713}

\leavevmode\hypertarget{ref-scherpenzeelTrueLongitudinalProbabilitybased2010}{}%
Scherpenzeel, A. C., \& Das, M. (2010). True'' longitudinal and probability-based internet panels: Evidence from the Netherlands. In M. Das, P. Ester, \& L. Kaczmirek (Eds.), \emph{Social and behavioral research and the internet: Advances in applied methods and research strategies} (pp. 77--104). Taylor \& Francis.

\leavevmode\hypertarget{ref-schwabaPersonalityTraitDevelopment2019}{}%
Schwaba, T., \& Bleidorn, W. (2019). Personality trait development across the transition to retirement. \emph{Journal of Personality and Social Psychology}, \emph{116}(4), 651--665. \url{https://doi.org/10.1037/pspp0000179}

\leavevmode\hypertarget{ref-schwabaIndividualDifferencesPersonality2018}{}%
Schwaba, T., \& Bleidorn, W. (2018). Individual differences in personality change across the adult life span. \emph{Journal of Personality}, \emph{86}(3), 450--464. \url{https://doi.org/10.1111/jopy.12327}

\leavevmode\hypertarget{ref-shadishExperimentalQuasiexperimentalDesigns2002}{}%
Shadish, W. R., Cook, T. D., \& Campbell, D. T. (2002). \emph{Experimental and quasi-experimental designs for generalized causal inference}. Houghton, Mifflin and Company.

\leavevmode\hypertarget{ref-sheppardBecomingFirstTimeGrandparent2019}{}%
Sheppard, P., \& Monden, C. (2019). Becoming a First-Time Grandparent and Subjective Well-Being: A Fixed Effects Approach. \emph{Journal of Marriage and Family}, \emph{81}(4), 1016--1026. \url{https://doi.org/10.1111/jomf.12584}

\leavevmode\hypertarget{ref-silversteinHowAmericansEnact2001}{}%
Silverstein, M., \& Marenco, A. (2001). How Americans Enact the Grandparent Role Across the Family Life Course. \emph{Journal of Family Issues}, \emph{22}(4), 493--522. \url{https://doi.org/10.1177/019251301022004006}

\leavevmode\hypertarget{ref-skopekWhoBecomesGrandparent2017}{}%
Skopek, J., \& Leopold, T. (2017). Who becomes a grandparent and when? Educational differences in the chances and timing of grandparenthood. \emph{Demographic Research}, \emph{37}(29), 917--928. \url{https://doi.org/10.4054/DemRes.2017.37.29}

\leavevmode\hypertarget{ref-sonnegaCohortProfileHealth2014}{}%
Sonnega, A., Faul, J. D., Ofstedal, M. B., Langa, K. M., Phillips, J. W., \& Weir, D. R. (2014). Cohort Profile: The Health and Retirement Study (HRS). \emph{International Journal of Epidemiology}, \emph{43}(2), 576--585. \url{https://doi.org/10.1093/ije/dyu067}

\leavevmode\hypertarget{ref-sotoLinksPersonalityLife2021}{}%
Soto, C. J. (2021). Do Links Between Personality and Life Outcomes Generalize? Testing the Robustness of TraitOutcome Associations Across Gender, Age, Ethnicity, and Analytic Approaches. \emph{Social Psychological and Personality Science}, \emph{12}(1), 118--130. \url{https://doi.org/10.1177/1948550619900572}

\leavevmode\hypertarget{ref-sotoHowReplicableAre2019}{}%
Soto, C. J. (2019). How Replicable Are Links Between Personality Traits and Consequential Life Outcomes? The Life Outcomes of Personality Replication Project. \emph{Psychological Science}, \emph{30}(5), 711--727. \url{https://doi.org/10.1177/0956797619831612}

\leavevmode\hypertarget{ref-spechtPersonalityDevelopmentAdulthood2017}{}%
Specht, J. (2017). Personality development in adulthood and old age. In J. Specht (Ed.), \emph{Personality Development Across the Lifespan} (pp. 53--67). Academic Press. \url{https://doi.org/10.1016/B978-0-12-804674-6.00005-3}

\leavevmode\hypertarget{ref-spechtWhatDrivesAdult2014}{}%
Specht, J., Bleidorn, W., Denissen, J. J. A., Hennecke, M., Hutteman, R., Kandler, C., Luhmann, M., Orth, U., Reitz, A. K., \& Zimmermann, J. (2014). What Drives Adult Personality Development? A Comparison of Theoretical Perspectives and Empirical Evidence. \emph{European Journal of Personality}, \emph{28}(3), 216--230. \url{https://doi.org/10.1002/per.1966}

\leavevmode\hypertarget{ref-spechtStabilityChangePersonality2011}{}%
Specht, J., Egloff, B., \& Schmukle, S. C. (2011). Stability and change of personality across the life course: The impact of age and major life events on mean-level and rank-order stability of the Big Five. \emph{Journal of Personality and Social Psychology}, \emph{101}(4), 862--882. \url{https://doi.org/10.1037/a0024950}

\leavevmode\hypertarget{ref-steinerImportanceCovariateSelection2010}{}%
Steiner, P., Cook, T., Shadish, W., \& Clark, M. (2010). The Importance of Covariate Selection in Controlling for Selection Bias in Observational Studies. \emph{Psychological Methods}, \emph{15}, 250--267. \url{https://doi.org/10.1037/a0018719}

\leavevmode\hypertarget{ref-stephanPhysicalActivityPersonality2014}{}%
Stephan, Y., Sutin, A. R., \& Terracciano, A. (2014). Physical activity and personality development across adulthood and old age: Evidence from two longitudinal studies. \emph{Journal of Research in Personality}, \emph{49}, 1--7. \url{https://doi.org/10.1016/j.jrp.2013.12.003}

\leavevmode\hypertarget{ref-stuartMatchingMethodsCausal2010}{}%
Stuart, E. A. (2010). Matching methods for causal inference: A review and a look forward. \emph{Statistical Science: A Review Journal of the Institute of Mathematical Statistics}, \emph{25}(1), 1--21. \url{https://doi.org/10.1214/09-STS313}

\leavevmode\hypertarget{ref-tanskanenTransitionGrandparenthoodSubjective2019}{}%
Tanskanen, A. O., Danielsbacka, M., Coall, D. A., \& Jokela, M. (2019). Transition to Grandparenthood and Subjective Well-Being in Older Europeans: A Within-Person Investigation Using Longitudinal Data. \emph{Evolutionary Psychology}, \emph{17}(3), 1474704919875948. \url{https://doi.org/10.1177/1474704919875948}

\leavevmode\hypertarget{ref-thoemmesSystematicReviewPropensity2011}{}%
Thoemmes, F. J., \& Kim, E. S. (2011). A Systematic Review of Propensity Score Methods in the Social Sciences. \emph{Multivariate Behavioral Research}, \emph{46}(1), 90--118. \url{https://doi.org/10.1080/00273171.2011.540475}

\leavevmode\hypertarget{ref-triadoGrandparentsWhoProvide2014}{}%
Triadó, C., Villar, F., Celdrán, M., \& Solé, C. (2014). Grandparents Who Provide Auxiliary Care for Their Grandchildren: Satisfaction, Difficulties, and Impact on Their Health and Well-being. \emph{Journal of Intergenerational Relationships}, \emph{12}(2), 113--127. \url{https://doi.org/10.1080/15350770.2014.901102}

\leavevmode\hypertarget{ref-turianoHealthyNeuroticismAssociated2020}{}%
Turiano, N. A., Graham, E. K., Weston, S. J., Booth, T., Harrison, F., James, B. D., Lewis, N. A., Makkar, S. R., Mueller, S., Wisniewski, K. M., Zhaoyang, R., Spiro, A., Willis, S., Schaie, K. W., Lipton, R. B., Katz, M., Sliwinski, M., Deary, I. J., Zelinski, E. M., \ldots{} Mroczek, D. K. (2020). Is Healthy Neuroticism Associated with Longevity? A Coordinated Integrative Data Analysis. \emph{Collabra: Psychology}, \emph{6}(33). \url{https://doi.org/10.1525/collabra.268}

\leavevmode\hypertarget{ref-turianoPersonalityTraitLevel2012}{}%
Turiano, N. A., Pitzer, L., Armour, C., Karlamangla, A., Ryff, C. D., \& Mroczek, D. K. (2012). Personality Trait Level and Change as Predictors of Health Outcomes: Findings From a National Study of Americans (MIDUS). \emph{The Journals of Gerontology: Series B}, \emph{67B}(1), 4--12. \url{https://doi.org/10.1093/geronb/gbr072}

\leavevmode\hypertarget{ref-mice2011}{}%
van Buuren, S., \& Groothuis-Oudshoorn, K. (2011). mice: Multivariate imputation by chained equations in r. \emph{Journal of Statistical Software}, \emph{45}(3), 1--67.

\leavevmode\hypertarget{ref-vanderlaanRepresentativityLISSPanel2009}{}%
van der Laan, J. (2009). \emph{Representativity of the LISS panel (Discussion Paper 09041)}. Statistics Netherlands.

\leavevmode\hypertarget{ref-vanderweelePrinciplesConfounderSelection2019}{}%
VanderWeele, T. J. (2019). Principles of confounder selection. \emph{European Journal of Epidemiology}, \emph{34}(3), 211--219. \url{https://doi.org/10.1007/s10654-019-00494-6}

\leavevmode\hypertarget{ref-vanderweeleOutcomeWideLongitudinalDesigns2020}{}%
VanderWeele, T. J., Mathur, M. B., \& Chen, Y. (2020). Outcome-Wide Longitudinal Designs for Causal Inference: A New Template for Empirical Studies. \emph{Statistical Science}, \emph{35}(3), 437--466. \url{https://doi.org/10.1214/19-STS728}

\leavevmode\hypertarget{ref-vanscheppingenLongitudinalActorPartner2019}{}%
van Scheppingen, M. A., Chopik, W. J., Bleidorn, W., \& Denissen, J. J. A. (2019). Longitudinal actor, partner, and similarity effects of personality on well-being. \emph{Journal of Personality and Social Psychology}, \emph{117}(4), e51--e70. \url{https://doi.org/10.1037/pspp0000211}

\leavevmode\hypertarget{ref-vanscheppingenPersonalityTraitDevelopment2016}{}%
van Scheppingen, M. A., Jackson, J. J., Specht, J., Hutteman, R., Denissen, J. J. A., \& Bleidorn, W. (2016). Personality Trait Development During the Transition to Parenthood: A Test of Social Investment Theory. \emph{Social Psychological and Personality Science}, \emph{7}(5), 452--462. \url{https://doi.org/10.1177/1948550616630032}

\leavevmode\hypertarget{ref-vanscheppingenTrajectoriesLifeSatisfaction2020}{}%
van Scheppingen, M. A., \& Leopold, T. (2020). Trajectories of life satisfaction before, upon, and after divorce: Evidence from a new matching approach. \emph{Journal of Personality and Social Psychology}, \emph{119}(6), 1444--1458. \url{https://doi.org/10.1037/pspp0000270}

\leavevmode\hypertarget{ref-wagnerFirstPartnershipExperience2015}{}%
Wagner, J., Becker, M., Lüdtke, O., \& Trautwein, U. (2015). The First Partnership Experience and Personality Development: A Propensity Score Matching Study in Young Adulthood. \emph{Social Psychological and Personality Science}, \emph{6}(4), 455--463. \url{https://doi.org/10.1177/1948550614566092}

\leavevmode\hypertarget{ref-wagnerIntegrativeModelSources2020}{}%
Wagner, J., Orth, U., Bleidorn, W., Hopwood, C. J., \& Kandler, C. (2020). Toward an Integrative Model of Sources of Personality Stability and Change. \emph{Current Directions in Psychological Science}, \emph{29}(5), 438--444. \url{https://doi.org/10.1177/0963721420924751}

\leavevmode\hypertarget{ref-wagnerPersonalityTraitDevelopment2016}{}%
Wagner, J., Ram, N., Smith, J., \& Gerstorf, D. (2016). Personality trait development at the end of life: Antecedents and correlates of mean-level trajectories. \emph{Journal of Personality and Social Psychology}, \emph{111}(3), 411--429. \url{https://doi.org/10.1037/pspp0000071}

\leavevmode\hypertarget{ref-tidyverse2019}{}%
Wickham, H., Averick, M., Bryan, J., Chang, W., McGowan, L. D., François, R., Grolemund, G., Hayes, A., Henry, L., Hester, J., Kuhn, M., Pedersen, T. L., Miller, E., Bache, S. M., Müller, K., Ooms, J., Robinson, D., Seidel, D. P., Spinu, V., \ldots{} Yutani, H. (2019). Welcome to the tidyverse. \emph{Journal of Open Source Software}, \emph{4}(43), 1686. \url{https://doi.org/10.21105/joss.01686}

\leavevmode\hypertarget{ref-wortmanStabilityChangeBig2012}{}%
Wortman, J., Lucas, R. E., \& Donnellan, M. B. (2012). Stability and change in the Big Five personality domains: Evidence from a longitudinal study of Australians. \emph{Psychology and Aging}, \emph{27}(4), 867--874. \url{https://doi.org/10.1037/a0029322}

\leavevmode\hypertarget{ref-wrzusProcessesPersonalityDevelopment2017}{}%
Wrzus, C., \& Roberts, B. W. (2017). Processes of personality development in adulthood: The TESSERA framework. \emph{Personality and Social Psychology Review}, \emph{21}(3), 253--277. \url{https://doi.org/10.1177/1088868316652279}

\leavevmode\hypertarget{ref-yapDoesPersonalityModerate2012}{}%
Yap, S., Anusic, I., \& Lucas, R. E. (2012). Does personality moderate reaction and adaptation to major life events? Evidence from the British Household Panel Survey. \emph{Journal of Research in Personality}, \emph{46}(5), 477--488. \url{https://doi.org/10.1016/j.jrp.2012.05.005}

\endgroup


\clearpage
\makeatletter
\efloat@restorefloats
\makeatother


\begin{appendix}
\renewcommand{\appendixname}{\textcolor{white}{.}}
\renewcommand{\thefigure}{S\arabic{figure}} \setcounter{figure}{0}
\renewcommand{\thetable}{S\arabic{table}} \setcounter{table}{0}

\setcounter{page}{1}

\hypertarget{supplemental-material}{%
\section{Supplemental Material}\label{supplemental-material}}

\noindent \textbf{Supplemental Tables}

\begin{table}[h]

\begin{center}
\begin{threeparttable}

\caption{\label{tab:icc-table}Intra-Class Correlations}

\begin{tabular}{lcccccc}
\toprule
& \multicolumn{1}{c}{\textcolor{white}{xxx}A\textcolor{white}{xxx}} & \multicolumn{1}{c}{\textcolor{white}{xxx}C\textcolor{white}{xxx}} & \multicolumn{1}{c}{\textcolor{white}{xxx}E\textcolor{white}{xxx}} & \multicolumn{1}{c}{\textcolor{white}{xxx}N\textcolor{white}{xxx}} & \multicolumn{1}{c}{\textcolor{white}{xxx}O\textcolor{white}{xxx}} & \multicolumn{1}{c}{\textcolor{white}{xxx}LS\textcolor{white}{xxx}}\\
\midrule
LISS: Parent controls &  &  &  &  &  & \\
\ \ \ $ICC_{pid}$ \textcolor{white}{LP} & 0.74 & 0.77 & 0.81 & 0.71 & 0.78 & 0.35\\
\ \ \ $ICC_{hid}$ \textcolor{white}{LP} & 0.05 & 0.01 & 0.02 & 0.07 & 0.00 & 0.37\\
\ \ \ $ICC_{pid/hid}$ \textcolor{white}{LP} & 0.79 & 0.78 & 0.83 & 0.78 & 0.78 & 0.71\\
LISS: Nonparent controls &  &  &  &  &  & \\
\ \ \ $ICC_{pid}$ \textcolor{white}{LN} & 0.76 & 0.76 & 0.64 & 0.67 & 0.79 & 0.32\\
\ \ \ $ICC_{hid}$ \textcolor{white}{LN} & 0.00 & 0.00 & 0.22 & 0.10 & 0.02 & 0.36\\
\ \ \ $ICC_{pid/hid}$ \textcolor{white}{LN} & 0.76 & 0.77 & 0.85 & 0.77 & 0.81 & 0.67\\
HRS: Parent controls &  &  &  &  &  & \\
\ \ \ $ICC_{pid}$ \textcolor{white}{HP} & 0.76 & 0.69 & 0.79 & 0.73 & 0.57 & 0.31\\
\ \ \ $ICC_{hid}$ \textcolor{white}{HP} & 0.00 & 0.07 & 0.00 & 0.01 & 0.21 & 0.35\\
\ \ \ $ICC_{pid/hid}$ \textcolor{white}{HP} & 0.76 & 0.76 & 0.79 & 0.74 & 0.78 & 0.67\\
HRS: Nonparent controls &  &  &  &  &  & \\
\ \ \ $ICC_{pid}$ \textcolor{white}{HN} & 0.71 & 0.73 & 0.77 & 0.76 & 0.59 & 0.33\\
\ \ \ $ICC_{hid}$ \textcolor{white}{HN} & 0.07 & 0.06 & 0.04 & 0.00 & 0.23 & 0.38\\
\ \ \ $ICC_{pid/hid}$ \textcolor{white}{HN} & 0.78 & 0.79 & 0.80 & 0.76 & 0.82 & 0.71\\
\bottomrule
\addlinespace
\end{tabular}

\begin{tablenotes}[para]
\normalsize{\textit{Note.} A = agreeableness, C = conscientiousness, E = extraversion, N = neuroticism, O = openness, LS = life satisfaction. Intra-class correlations are the proportion of total variation that is explained by the respective blocking factor. $ICC_{pid}$ is the proportion of total variance explained by nesting in respondents which corresponds to the correlation between two randomly selected observations from the same respondent. $ICC_{hid}$ is the proportion of total variance explained by nesting in households which corresponds to the correlation between two randomly selected observations from the same household. $ICC_{pid/hid}$ is the proportion of total variance explained by nesting in respondents and in households which corresponds to the correlation between two randomly selected observations from the same respondent and the same household.}
\end{tablenotes}

\end{threeparttable}
\end{center}

\end{table}










\begin{lltable}

\begin{TableNotes}[para]
\normalsize{\textit{Note.} obs. = observations. \(time=0\) marks
the first year where the transition to grandparenthood has been
reported. The number of grandparent participants is \(N_{LISS}=\) 250
and \(N_{HRS}=\) 846.}
\end{TableNotes}

\small{

\begin{longtable}{lccccccccccccc}\noalign{\getlongtablewidth\global\LTcapwidth=\longtablewidth}
\caption{\label{tab:piecewise-coding-scheme}Longitudinal sample size in the
analysis samples and coding scheme for the piecewise regression
coefficients}\\
\toprule
& \multicolumn{6}{c}{Pre-transition years} & \multicolumn{7}{c}{Post-transition years} \\
\cmidrule(r){2-7} \cmidrule(r){8-14}
& -6 & -5 & -4 & -3 & -2 & -1 & 0 & 1 & 2 & 3 & 4 & 5 & 6\\
\midrule
\endfirsthead
\caption*{\normalfont{Table \ref{tab:piecewise-coding-scheme} continued}}\\
\toprule
& \multicolumn{6}{c}{Pre-transition years} & \multicolumn{7}{c}{Post-transition years} \\
\cmidrule(r){2-7} \cmidrule(r){8-14}
& -6 & -5 & -4 & -3 & -2 & -1 & 0 & 1 & 2 & 3 & 4 & 5 & 6\\
\midrule
\endhead
LISS: Analysis samples &  &  &  &  &  &  &  &  &  &  &  &  & \\
\ \ \ Grandparents: obs. \textcolor{white}{L} & 92 & 105 & 108 & 121 & 156 & 116 & 133 & 138 & 108 & 108 & 69 & 62 & 52\\
\ \ \ Grandparents: \% women \textcolor{white}{L} & 51.09 & 48.57 & 52.78 & 51.24 & 56.41 & 62.93 & 47.37 & 52.90 & 51.85 & 50.00 & 56.52 & 66.13 & 53.85\\
\ \ \ Parent controls: obs. \textcolor{white}{L} & 335 & 425 & 381 & 540 & 740 & 351 & 450 & 488 & 333 & 394 & 365 & 164 & 201\\
\ \ \ Parent controls: \% women \textcolor{white}{L} & 57.61 & 51.06 & 55.12 & 51.48 & 55.00 & 56.13 & 53.11 & 54.10 & 56.76 & 51.27 & 56.99 & 59.76 & 48.76\\
\ \ \ Nonparent controls: obs. \textcolor{white}{L} & 331 & 399 & 407 & 554 & 739 & 354 & 473 & 516 & 367 & 477 & 375 & 146 & 202\\
\ \ \ Nonparent controls: \% women \textcolor{white}{L} & 52.57 & 54.89 & 57.99 & 52.71 & 55.21 & 54.52 & 49.26 & 54.46 & 52.86 & 52.83 & 54.67 & 48.63 & 51.49\\
LISS: Coding scheme &  &  &  &  &  &  &  &  &  &  &  &  & \\
\ \ \ Before-slope \textcolor{white}{L} & 0 & 1 & 2 & 3 & 4 & 5 & 5 & 5 & 5 & 5 & 5 & 5 & 5\\
\ \ \ After-slope \textcolor{white}{L} & 0 & 0 & 0 & 0 & 0 & 0 & 1 & 2 & 3 & 4 & 5 & 6 & 7\\
\ \ \ Jump \textcolor{white}{L} & 0 & 0 & 0 & 0 & 0 & 0 & 1 & 1 & 1 & 1 & 1 & 1 & 1\\
HRS: Analysis samples &  &  &  &  &  &  &  &  &  &  &  &  & \\
\ \ \ Grandparents: obs. \textcolor{white}{H} & 162 &  & 388 &  & 461 &  & 380 &  & 444 &  & 195 &  & 232\\
\ \ \ Grandparents: \% women \textcolor{white}{H} & 57.41 &  & 54.12 &  & 55.53 &  & 53.95 &  & 55.41 &  & 56.41 &  & 53.45\\
\ \ \ Parent controls: obs. \textcolor{white}{H} & 619 &  & 1540 &  & 1844 &  & 1228 &  & 1504 &  & 658 &  & 864\\
\ \ \ Parent controls: \% women \textcolor{white}{H} & 55.41 &  & 54.03 &  & 55.53 &  & 54.64 &  & 56.45 &  & 56.08 &  & 57.64\\
\ \ \ Nonparent controls: obs. \textcolor{white}{H} & 620 &  & 1541 &  & 1844 &  & 1205 &  & 1448 &  & 688 &  & 821\\
\ \ \ Nonparent controls: \% women \textcolor{white}{H} & 56.45 &  & 54.06 &  & 55.53 &  & 56.10 &  & 58.91 &  & 57.56 &  & 60.54\\
HRS: Coding scheme &  &  &  &  &  &  &  &  &  &  &  &  & \\
\ \ \ Before-slope \textcolor{white}{H} & 0 &  & 1 &  & 2 &  & 2 &  & 2 &  & 2 &  & 2\\
\ \ \ After-slope \textcolor{white}{H} & 0 &  & 0 &  & 0 &  & 1 &  & 2 &  & 3 &  & 4\\
\ \ \ Jump \textcolor{white}{H} & 0 &  & 0 &  & 0 &  & 1 &  & 1 &  & 1 &  & 1\\
\bottomrule
\addlinespace
\insertTableNotes
\end{longtable}

}

\end{lltable}








\begin{lltable}

\begin{TableNotes}[para]
\normalsize{\textit{Note.} Standard deviation shown in brackets; \(time=0\)
marks the first year where the transition to grandparenthood has been
reported.}
\end{TableNotes}

\small{

\begin{longtable}{lccccccccccccc}\noalign{\getlongtablewidth\global\LTcapwidth=\longtablewidth}
\caption{\label{tab:descriptives-liss}Mean and Standard Deviation of the Big Five
and Life Satisfaction over Time in the LISS Panel}\\
\toprule
& \multicolumn{6}{c}{Pre-transition years} & \multicolumn{7}{c}{Post-transition years} \\
\cmidrule(r){2-7} \cmidrule(r){8-14}
& -6 & -5 & -4 & -3 & -2 & -1 & 0 & 1 & 2 & 3 & 4 & 5 & 6\\
\midrule
\endfirsthead
\caption*{\normalfont{Table \ref{tab:descriptives-liss} continued}}\\
\toprule
& \multicolumn{6}{c}{Pre-transition years} & \multicolumn{7}{c}{Post-transition years} \\
\cmidrule(r){2-7} \cmidrule(r){8-14}
& -6 & -5 & -4 & -3 & -2 & -1 & 0 & 1 & 2 & 3 & 4 & 5 & 6\\
\midrule
\endhead
Agreeableness &  &  &  &  &  &  &  &  &  &  &  &  & \\
\ \ \ Grandparents \textcolor{white}{A} & 3.85 & 3.87 & 3.93 & 3.87 & 3.90 & 3.93 & 3.87 & 3.92 & 3.91 & 3.91 & 3.89 & 4.01 & 3.98\\
\ \ \ \textcolor{white}{Ag} & (0.52) & (0.50) & (0.46) & (0.49) & (0.54) & (0.47) & (0.49) & (0.52) & (0.52) & (0.51) & (0.52) & (0.49) & (0.37)\\
\ \ \ Parent controls \textcolor{white}{A} & 3.93 & 3.89 & 3.90 & 3.87 & 3.91 & 3.95 & 3.91 & 3.89 & 3.90 & 3.92 & 3.86 & 3.86 & 3.81\\
\ \ \ \textcolor{white}{Ap} & (0.52) & (0.51) & (0.47) & (0.50) & (0.48) & (0.48) & (0.47) & (0.51) & (0.53) & (0.48) & (0.50) & (0.43) & (0.43)\\
\ \ \ Nonparent controls \textcolor{white}{A} & 3.95 & 3.94 & 3.98 & 3.98 & 3.94 & 3.91 & 3.94 & 3.95 & 3.94 & 3.94 & 3.92 & 3.92 & 3.88\\
\ \ \ \textcolor{white}{An} & (0.47) & (0.50) & (0.45) & (0.50) & (0.49) & (0.47) & (0.44) & (0.45) & (0.46) & (0.47) & (0.41) & (0.44) & (0.42)\\
Conscientiousness &  &  &  &  &  &  &  &  &  &  &  &  & \\
\ \ \ Grandparents \textcolor{white}{C} & 3.76 & 3.84 & 3.74 & 3.75 & 3.77 & 3.79 & 3.77 & 3.78 & 3.75 & 3.79 & 3.84 & 3.74 & 3.76\\
\ \ \ \textcolor{white}{Cg} & (0.50) & (0.45) & (0.49) & (0.46) & (0.53) & (0.48) & (0.49) & (0.51) & (0.49) & (0.51) & (0.44) & (0.48) & (0.43)\\
\ \ \ Parent controls \textcolor{white}{C} & 3.80 & 3.78 & 3.80 & 3.77 & 3.79 & 3.83 & 3.82 & 3.79 & 3.80 & 3.79 & 3.78 & 3.76 & 3.77\\
\ \ \ \textcolor{white}{Cp} & (0.52) & (0.50) & (0.52) & (0.49) & (0.49) & (0.50) & (0.49) & (0.47) & (0.47) & (0.46) & (0.43) & (0.44) & (0.45)\\
\ \ \ Nonparent controls \textcolor{white}{C} & 3.77 & 3.79 & 3.76 & 3.80 & 3.74 & 3.75 & 3.77 & 3.72 & 3.82 & 3.81 & 3.78 & 3.84 & 3.80\\
\ \ \ \textcolor{white}{Cn} & (0.53) & (0.50) & (0.51) & (0.50) & (0.51) & (0.53) & (0.50) & (0.50) & (0.50) & (0.51) & (0.48) & (0.46) & (0.50)\\
Extraversion &  &  &  &  &  &  &  &  &  &  &  &  & \\
\ \ \ Grandparents \textcolor{white}{E} & 3.23 & 3.20 & 3.31 & 3.32 & 3.28 & 3.30 & 3.19 & 3.24 & 3.22 & 3.19 & 3.33 & 3.34 & 3.19\\
\ \ \ \textcolor{white}{Eg} & (0.66) & (0.74) & (0.54) & (0.58) & (0.64) & (0.57) & (0.61) & (0.69) & (0.65) & (0.60) & (0.60) & (0.58) & (0.55)\\
\ \ \ Parent controls \textcolor{white}{E} & 3.32 & 3.30 & 3.28 & 3.27 & 3.26 & 3.30 & 3.25 & 3.20 & 3.22 & 3.28 & 3.19 & 3.19 & 3.14\\
\ \ \ \textcolor{white}{Ep} & (0.58) & (0.59) & (0.58) & (0.59) & (0.59) & (0.59) & (0.64) & (0.62) & (0.59) & (0.61) & (0.58) & (0.53) & (0.56)\\
\ \ \ Nonparent controls \textcolor{white}{E} & 3.31 & 3.27 & 3.21 & 3.32 & 3.32 & 3.28 & 3.30 & 3.27 & 3.31 & 3.31 & 3.28 & 3.13 & 3.26\\
\ \ \ \textcolor{white}{En} & (0.74) & (0.70) & (0.79) & (0.75) & (0.69) & (0.70) & (0.72) & (0.73) & (0.77) & (0.78) & (0.73) & (0.75) & (0.74)\\
Neuroticism &  &  &  &  &  &  &  &  &  &  &  &  & \\
\ \ \ Grandparents \textcolor{white}{N} & 2.39 & 2.31 & 2.33 & 2.41 & 2.45 & 2.47 & 2.30 & 2.39 & 2.30 & 2.36 & 2.33 & 2.44 & 2.53\\
\ \ \ \textcolor{white}{Ng} & (0.71) & (0.64) & (0.60) & (0.64) & (0.65) & (0.71) & (0.67) & (0.76) & (0.68) & (0.66) & (0.67) & (0.80) & (0.67)\\
\ \ \ Parent controls \textcolor{white}{N} & 2.43 & 2.42 & 2.42 & 2.38 & 2.40 & 2.37 & 2.35 & 2.35 & 2.30 & 2.28 & 2.35 & 2.31 & 2.33\\
\ \ \ \textcolor{white}{Np} & (0.59) & (0.63) & (0.56) & (0.58) & (0.58) & (0.60) & (0.63) & (0.65) & (0.56) & (0.56) & (0.60) & (0.55) & (0.56)\\
\ \ \ Nonparent controls \textcolor{white}{N} & 2.41 & 2.44 & 2.47 & 2.36 & 2.43 & 2.37 & 2.33 & 2.37 & 2.34 & 2.33 & 2.35 & 2.48 & 2.35\\
\ \ \ \textcolor{white}{Nn} & (0.64) & (0.63) & (0.69) & (0.70) & (0.69) & (0.63) & (0.69) & (0.71) & (0.74) & (0.68) & (0.70) & (0.82) & (0.83)\\
Openness &  &  &  &  &  &  &  &  &  &  &  &  & \\
\ \ \ Grandparents \textcolor{white}{O} & 3.43 & 3.50 & 3.54 & 3.49 & 3.49 & 3.50 & 3.48 & 3.48 & 3.50 & 3.45 & 3.50 & 3.43 & 3.36\\
\ \ \ \textcolor{white}{Og} & (0.51) & (0.50) & (0.49) & (0.45) & (0.49) & (0.50) & (0.48) & (0.54) & (0.43) & (0.46) & (0.50) & (0.53) & (0.56)\\
\ \ \ Parent controls \textcolor{white}{O} & 3.53 & 3.46 & 3.43 & 3.48 & 3.48 & 3.48 & 3.50 & 3.49 & 3.44 & 3.51 & 3.42 & 3.37 & 3.42\\
\ \ \ \textcolor{white}{Op} & (0.52) & (0.52) & (0.50) & (0.53) & (0.51) & (0.51) & (0.52) & (0.50) & (0.48) & (0.48) & (0.49) & (0.48) & (0.42)\\
\ \ \ Nonparent controls \textcolor{white}{O} & 3.53 & 3.57 & 3.53 & 3.58 & 3.52 & 3.51 & 3.52 & 3.55 & 3.54 & 3.59 & 3.53 & 3.51 & 3.51\\
\ \ \ \textcolor{white}{On} & (0.52) & (0.51) & (0.51) & (0.52) & (0.52) & (0.51) & (0.51) & (0.51) & (0.52) & (0.51) & (0.50) & (0.47) & (0.53)\\
Life satisfaction &  &  &  &  &  &  &  &  &  &  &  &  & \\
\ \ \ Grandparents \textcolor{white}{L} & 5.18 & 5.29 & 5.23 & 5.16 & 5.28 & 5.24 & 5.31 & 5.24 & 5.37 & 5.38 & 5.39 & 5.25 & 5.15\\
\ \ \ \textcolor{white}{Lg} & (1.06) & (0.93) & (1.13) & (0.95) & (0.93) & (1.10) & (0.93) & (1.03) & (1.09) & (0.90) & (1.10) & (1.10) & (1.00)\\
\ \ \ Parent controls \textcolor{white}{L} & 5.21 & 5.30 & 5.26 & 5.23 & 5.28 & 5.29 & 5.36 & 5.25 & 5.26 & 5.45 & 5.33 & 5.40 & 5.41\\
\ \ \ \textcolor{white}{Lp} & (1.11) & (1.03) & (1.01) & (0.97) & (1.01) & (1.07) & (0.99) & (1.03) & (1.04) & (0.93) & (1.04) & (1.05) & (1.05)\\
\ \ \ Nonparent controls \textcolor{white}{L} & 5.27 & 5.19 & 5.10 & 5.21 & 5.26 & 5.18 & 5.24 & 5.09 & 5.10 & 5.07 & 5.23 & 4.98 & 5.19\\
\ \ \ \textcolor{white}{Ln} & (0.92) & (0.87) & (0.90) & (0.92) & (0.95) & (0.90) & (0.96) & (1.04) & (1.12) & (1.13) & (1.08) & (1.30) & (1.18)\\
\bottomrule
\addlinespace
\insertTableNotes
\end{longtable}

}

\end{lltable}




\begin{lltable}

\begin{TableNotes}[para]
\normalsize{\textit{Note.} Standard deviation shown in brackets; \(time=0\)
marks the first year where the transition to grandparenthood has been
reported.}
\end{TableNotes}

\small{

\begin{longtable}{lccccccccccccc}\noalign{\getlongtablewidth\global\LTcapwidth=\longtablewidth}
\caption{\label{tab:descriptives-hrs}Mean and Standard Deviation of the Big Five
and Life Satisfaction over Time in the HRS}\\
\toprule
& \multicolumn{6}{c}{Pre-transition years} & \multicolumn{7}{c}{Post-transition years} \\
\cmidrule(r){2-7} \cmidrule(r){8-14}
& -6 & -5 & -4 & -3 & -2 & -1 & 0 & 1 & 2 & 3 & 4 & 5 & 6\\
\midrule
\endfirsthead
\caption*{\normalfont{Table \ref{tab:descriptives-hrs} continued}}\\
\toprule
& \multicolumn{6}{c}{Pre-transition years} & \multicolumn{7}{c}{Post-transition years} \\
\cmidrule(r){2-7} \cmidrule(r){8-14}
& -6 & -5 & -4 & -3 & -2 & -1 & 0 & 1 & 2 & 3 & 4 & 5 & 6\\
\midrule
\endhead
Agreeableness &  &  &  &  &  &  &  &  &  &  &  &  & \\
\ \ \ Grandparents \textcolor{white}{A} & 3.46 &  & 3.51 &  & 3.51 &  & 3.52 &  & 3.52 &  & 3.50 &  & 3.56\\
\ \ \ \textcolor{white}{Ag} & (0.47) &  & (0.48) &  & (0.49) &  & (0.49) &  & (0.48) &  & (0.53) &  & (0.44)\\
\ \ \ Parent controls \textcolor{white}{A} & 3.50 &  & 3.48 &  & 3.50 &  & 3.49 &  & 3.49 &  & 3.44 &  & 3.47\\
\ \ \ \textcolor{white}{Ap} & (0.48) &  & (0.49) &  & (0.46) &  & (0.50) &  & (0.48) &  & (0.52) &  & (0.51)\\
\ \ \ Nonparent controls \textcolor{white}{A} & 3.50 &  & 3.50 &  & 3.50 &  & 3.52 &  & 3.52 &  & 3.44 &  & 3.48\\
\ \ \ \textcolor{white}{An} & (0.50) &  & (0.50) &  & (0.51) &  & (0.50) &  & (0.50) &  & (0.53) &  & (0.53)\\
Conscientiousness &  &  &  &  &  &  &  &  &  &  &  &  & \\
\ \ \ Grandparents \textcolor{white}{C} & 3.47 &  & 3.46 &  & 3.47 &  & 3.46 &  & 3.45 &  & 3.44 &  & 3.49\\
\ \ \ \textcolor{white}{Cg} & (0.46) &  & (0.45) &  & (0.44) &  & (0.45) &  & (0.44) &  & (0.43) &  & (0.44)\\
\ \ \ Parent controls \textcolor{white}{C} & 3.45 &  & 3.45 &  & 3.45 &  & 3.47 &  & 3.46 &  & 3.43 &  & 3.44\\
\ \ \ \textcolor{white}{Cp} & (0.45) &  & (0.45) &  & (0.45) &  & (0.45) &  & (0.46) &  & (0.50) &  & (0.50)\\
\ \ \ Nonparent controls \textcolor{white}{C} & 3.50 &  & 3.48 &  & 3.49 &  & 3.50 &  & 3.48 &  & 3.46 &  & 3.49\\
\ \ \ \textcolor{white}{Cn} & (0.44) &  & (0.44) &  & (0.44) &  & (0.42) &  & (0.45) &  & (0.45) &  & (0.43)\\
Extraversion &  &  &  &  &  &  &  &  &  &  &  &  & \\
\ \ \ Grandparents \textcolor{white}{E} & 3.15 &  & 3.22 &  & 3.20 &  & 3.21 &  & 3.19 &  & 3.22 &  & 3.22\\
\ \ \ \textcolor{white}{Eg} & (0.56) &  & (0.56) &  & (0.54) &  & (0.56) &  & (0.58) &  & (0.59) &  & (0.58)\\
\ \ \ Parent controls \textcolor{white}{E} & 3.20 &  & 3.18 &  & 3.19 &  & 3.21 &  & 3.21 &  & 3.17 &  & 3.19\\
\ \ \ \textcolor{white}{Ep} & (0.51) &  & (0.56) &  & (0.54) &  & (0.54) &  & (0.54) &  & (0.55) &  & (0.56)\\
\ \ \ Nonparent controls \textcolor{white}{E} & 3.19 &  & 3.20 &  & 3.20 &  & 3.23 &  & 3.22 &  & 3.23 &  & 3.24\\
\ \ \ \textcolor{white}{En} & (0.55) &  & (0.54) &  & (0.56) &  & (0.54) &  & (0.54) &  & (0.56) &  & (0.57)\\
Neuroticism &  &  &  &  &  &  &  &  &  &  &  &  & \\
\ \ \ Grandparents \textcolor{white}{N} & 2.00 &  & 1.97 &  & 2.06 &  & 1.91 &  & 1.96 &  & 1.91 &  & 1.91\\
\ \ \ \textcolor{white}{Ng} & (0.56) &  & (0.63) &  & (0.62) &  & (0.60) &  & (0.58) &  & (0.59) &  & (0.61)\\
\ \ \ Parent controls \textcolor{white}{N} & 2.01 &  & 2.05 &  & 2.01 &  & 2.03 &  & 2.00 &  & 2.01 &  & 1.95\\
\ \ \ \textcolor{white}{Np} & (0.59) &  & (0.60) &  & (0.59) &  & (0.61) &  & (0.61) &  & (0.61) &  & (0.60)\\
\ \ \ Nonparent controls \textcolor{white}{N} & 2.05 &  & 2.00 &  & 2.02 &  & 1.92 &  & 1.97 &  & 1.84 &  & 1.90\\
\ \ \ \textcolor{white}{Nn} & (0.56) &  & (0.58) &  & (0.60) &  & (0.57) &  & (0.59) &  & (0.55) &  & (0.58)\\
Openness &  &  &  &  &  &  &  &  &  &  &  &  & \\
\ \ \ Grandparents \textcolor{white}{O} & 3.00 &  & 3.02 &  & 3.04 &  & 3.01 &  & 3.00 &  & 2.96 &  & 3.04\\
\ \ \ \textcolor{white}{Og} & (0.51) &  & (0.53) &  & (0.51) &  & (0.52) &  & (0.52) &  & (0.59) &  & (0.51)\\
\ \ \ Parent controls \textcolor{white}{O} & 3.03 &  & 3.00 &  & 2.98 &  & 3.03 &  & 3.00 &  & 2.96 &  & 2.96\\
\ \ \ \textcolor{white}{Op} & (0.51) &  & (0.56) &  & (0.54) &  & (0.54) &  & (0.52) &  & (0.58) &  & (0.56)\\
\ \ \ Nonparent controls \textcolor{white}{O} & 3.06 &  & 3.05 &  & 3.05 &  & 3.07 &  & 3.06 &  & 3.02 &  & 3.04\\
\ \ \ \textcolor{white}{On} & (0.54) &  & (0.53) &  & (0.55) &  & (0.54) &  & (0.55) &  & (0.57) &  & (0.57)\\
Life satisfaction &  &  &  &  &  &  &  &  &  &  &  &  & \\
\ \ \ Grandparents \textcolor{white}{L} & 5.14 &  & 5.08 &  & 5.15 &  & 5.17 &  & 5.16 &  & 5.29 &  & 5.28\\
\ \ \ \textcolor{white}{Lg} & (1.44) &  & (1.45) &  & (1.46) &  & (1.40) &  & (1.44) &  & (1.38) &  & (1.50)\\
\ \ \ Parent controls \textcolor{white}{L} & 5.14 &  & 4.98 &  & 5.01 &  & 5.11 &  & 5.10 &  & 5.06 &  & 5.12\\
\ \ \ \textcolor{white}{Lp} & (1.52) &  & (1.57) &  & (1.57) &  & (1.52) &  & (1.53) &  & (1.47) &  & (1.47)\\
\ \ \ Nonparent controls \textcolor{white}{L} & 5.10 &  & 5.14 &  & 5.09 &  & 5.26 &  & 5.21 &  & 5.40 &  & 5.40\\
\ \ \ \textcolor{white}{Ln} & (1.49) &  & (1.50) &  & (1.52) &  & (1.44) &  & (1.51) &  & (1.30) &  & (1.36)\\
\bottomrule
\addlinespace
\insertTableNotes
\end{longtable}

}

\end{lltable}













\begin{lltable}

\begin{TableNotes}[para]
\normalsize{\textit{Note.} PSM = propensity score matching, ref. =
reference category, f.~= female, m. = male, NA = covariate not used in
this sample. The standardized difference in means between the
grandparent and the two control groups (parent and nonparent) was
computed by \((\bar{x}_{gp}-\bar{x}_{c})/ (\hat\sigma_{gp})\). Rules of
thumb say that this measure should ideally be below \(.25\) (Stuart,
2010) or below \(.10\) (Austin, 2011).}
\end{TableNotes}

\footnotesize{

\begin{longtable}{lllrrrr}\noalign{\getlongtablewidth\global\LTcapwidth=\longtablewidth}
\caption{\label{tab:stddiffmeans-balance-liss}Standardized Difference in Means for
Covariates Used in Propensity Score Matching and the Propensity Score in
the LISS panel}\\
\toprule
&  &  & \multicolumn{2}{c}{Parent control group} & \multicolumn{2}{c}{Nonparent control group} \\
\cmidrule(r){4-5} \cmidrule(r){6-7}
Covariate & Description & Raw variable & Before PSM & After PSM & Before PSM & After PSM\\
\midrule
\endfirsthead
\caption*{\normalfont{Table \ref{tab:stddiffmeans-balance-liss} continued}}\\
\toprule
&  &  & \multicolumn{2}{c}{Parent control group} & \multicolumn{2}{c}{Nonparent control group} \\
\cmidrule(r){4-5} \cmidrule(r){6-7}
Covariate & Description & Raw variable & Before PSM & After PSM & Before PSM & After PSM\\
\midrule
\endhead
pscore & Propensity score & / & 1.14 & 0.02 & 1.34 & 0.04\\
female & Gender (f.=1, m.=0) & geslacht & 0.05 & 0.00 & 0.05 & 0.00\\
age & Age & gebjaar & 0.85 & -0.10 & 4.05 & -0.01\\
degreehighersec & Higher secondary/preparatory university education & oplmet & 0.07 & -0.06 & -0.07 & 0.12\\
degreevocational & Intermediate vocational education & oplmet & -0.20 & -0.06 & -0.02 & 0.00\\
degreecollege & Higher vocational education & oplmet & 0.00 & 0.05 & 0.02 & -0.09\\
degreeuniversity & University degree & oplmet & -0.08 & 0.14 & -0.15 & -0.05\\
religion & Member of religion/church & cr*012 & 0.10 & 0.08 & 0.33 & 0.07\\
speakdutch & Dutch spoken at home (primarily) & cr*089 & -0.02 & -0.06 & 0.00 & -0.02\\
divorced & Divorced (marital status) & burgstat & 0.02 & -0.03 & 0.29 & -0.02\\
widowed & Widowed (marital status) & burgstat & 0.09 & -0.12 & 0.13 & -0.07\\
livetogether & Live together with partner & cf*025 & -0.08 & 0.04 & 1.05 & -0.02\\
rooms & Rooms in dwelling & cd*034 & -0.03 & 0.05 & 0.63 & -0.11\\
logincome & Personal net monthly income in Euros (logarithm) & nettoink & -0.01 & 0.04 & 0.59 & -0.14\\
rental & Live for rent (vs. self-owned dwelling) & woning & -0.08 & -0.09 & -0.47 & -0.03\\
financialsit & Financial situation of household (scale from 1-5) & ci*252 & 0.08 & 0.00 & -0.03 & 0.00\\
jobhours & Average work hours per week & cw*127 & 0.02 & 0.08 & 0.11 & -0.04\\
mobility & Mobility problems (walking, staircase, shopping) & ch*023/027/041 & 0.07 & 0.04 & 0.09 & -0.02\\
dep & Depression items from Mental Health Inventory & ch*011 - ch*015 & -0.01 & 0.08 & -0.22 & -0.08\\
betterhealth & Poor/moderate health status (ref.: good) & ch*004 & 0.00 & -0.01 & -0.26 & 0.07\\
worsehealth & Very good/excellent health status (ref.: good) & ch*004 & 0.04 & -0.02 & 0.11 & -0.04\\
totalchildren & Number living children & cf*455 / cf*036 & 0.25 & 0.02 & NA & NA\\
totalresidentkids & Number of living-at-home children in household & aantalki & -0.71 & 0.02 & NA & NA\\
secondkid & Has two or more children & cf*455 / cf*036 & 0.20 & 0.04 & NA & NA\\
thirdkid & Has three or more children & cf*455 / cf*036 & 0.26 & 0.01 & NA & NA\\
kid1female & Gender of first child (f.=1, m.=0) & cf*068 & 0.04 & 0.04 & NA & NA\\
kid2female & Gender of second child (f.=1, m.=0) & cf*069 & 0.01 & -0.06 & NA & NA\\
kid3female & Gender of third child (f.=1, m.=0) & cf*070 & 0.17 & 0.02 & NA & NA\\
kid1age & Age of first child & cf*456 / cf*037 & 1.70 & -0.17 & NA & NA\\
kid2age & Age of second child & cf*457 / cf*038 & 0.87 & -0.01 & NA & NA\\
kid3age & Age of third child & cf*458 / cf*039 & 0.40 & 0.01 & NA & NA\\
kid1home & First child living at home & cf*083 & -1.56 & 0.05 & NA & NA\\
kid2home & Second child living at home & cf*084 & -1.05 & 0.04 & NA & NA\\
kid3home & Third child living at home & cf*085 & -0.05 & 0.00 & NA & NA\\
swls & Satisfaction with Life Scale & cp*014 - cp*018 & 0.10 & -0.03 & 0.25 & -0.06\\
agree & Agreeableness & cp*021 - cp*066 & 0.05 & -0.01 & 0.13 & -0.13\\
con & Conscientiousness & cp*022 - cp*067 & -0.06 & -0.05 & 0.16 & 0.00\\
extra & Extraversion & cp*020 - cp*065 & 0.05 & 0.02 & 0.02 & -0.07\\
neur & Neuroticism & cp*023 - cp*068 & -0.02 & 0.02 & -0.26 & 0.03\\
open & Openness & cp*024 - cp*069 & 0.06 & 0.05 & -0.16 & -0.08\\
participation & Waves participated & / & -0.27 & -0.09 & 0.09 & -0.03\\
year & Year of assessment & wave & -0.23 & -0.07 & 0.08 & -0.06\\
\bottomrule
\addlinespace
\insertTableNotes
\end{longtable}

}

\end{lltable}





\begin{lltable}

\begin{TableNotes}[para]
\normalsize{\textit{Note.} PSM = propensity score matching, ref. =
reference category, f.~= female, m. = male, NA = covariate not used in
this sample. The standardized difference in means between the
grandparent and the two control groups (parent and nonparent) was
computed by \((\bar{x}_{gp}-\bar{x}_{c})/ (\hat\sigma_{gp})\). Rules of
thumb say that this measure should ideally be below \(.25\) (Stuart,
2010) or below \(.10\) (Austin, 2011).}
\end{TableNotes}

\footnotesize{

\begin{longtable}{lllrrrr}\noalign{\getlongtablewidth\global\LTcapwidth=\longtablewidth}
\caption{\label{tab:stddiffmeans-balance-hrs}Standardized Difference in Means for
Covariates Used in Propensity Score Matching and the Propensity Score in
the HRS}\\
\toprule
&  &  & \multicolumn{2}{c}{Parent control group} & \multicolumn{2}{c}{Nonparent control group} \\
\cmidrule(r){4-5} \cmidrule(r){6-7}
Covariate & Description & Raw variable & Before PSM & After PSM & Before PSM & After PSM\\
\midrule
\endfirsthead
\caption*{\normalfont{Table \ref{tab:stddiffmeans-balance-hrs} continued}}\\
\toprule
&  &  & \multicolumn{2}{c}{Parent control group} & \multicolumn{2}{c}{Nonparent control group} \\
\cmidrule(r){4-5} \cmidrule(r){6-7}
Covariate & Description & Raw variable & Before PSM & After PSM & Before PSM & After PSM\\
\midrule
\endhead
pscore & Propensity score & / & 0.92 & 0.01 & 1.45 & 0.00\\
female & Gender (f.=1, m.=0) & RAGENDER & -0.07 & 0.00 & 0.01 & 0.00\\
age & Age & RABYEAR & -0.46 & -0.01 & -1.02 & 0.11\\
schlyrs & Years of education & RAEDYRS & 0.11 & 0.03 & 0.25 & -0.04\\
religyear & Religious attendance: yearly & *B082 & 0.04 & 0.01 & 0.13 & 0.00\\
religmonth & Religious attendance: monthly & *B082 & 0.01 & -0.02 & 0.10 & 0.05\\
religweek & Religious attendance: weekly & *B082 & 0.06 & 0.02 & 0.04 & 0.03\\
religmore & Religious attendance: more & *B082 & 0.09 & -0.04 & 0.06 & -0.01\\
notusaborn & Not born in the US & *Z230 & -0.05 & 0.03 & 0.13 & -0.02\\
black & Race: black/african american (ref.: white) & RARACEM & -0.13 & -0.08 & -0.22 & 0.01\\
raceother & Race: other (ref.: white) & RARACEM & -0.09 & -0.06 & 0.01 & -0.05\\
divorced & Divorced (marital status) & R*MSTAT & -0.06 & 0.01 & 0.01 & 0.03\\
widowed & Widowed (marital status) & R*MSTAT & -0.31 & 0.02 & -0.41 & 0.04\\
livetogether & Live together with partner & *A030 / *XF065\_R & 0.25 & -0.02 & 1.05 & -0.04\\
roomslessthree & Number of rooms (in housing unit) & *H147 / *066 & -0.15 & -0.05 & -0.59 & -0.01\\
roomsfourfive & Number of rooms (in housing unit) & *H147 / *066 & 0.00 & -0.02 & -0.25 & -0.03\\
roomsmoreeight & Number of rooms (in housing unit) & *H147 / *066 & 0.07 & -0.03 & 0.28 & 0.00\\
loghhincome & Household income (logarithm) & *ITOT & 0.03 & 0.03 & 0.41 & 0.00\\
loghhwealth & Household wealth (logarithm) & *ATOTB & 0.07 & 0.05 & 0.34 & -0.02\\
renter & Live for rent (vs. self-owned dwelling) & *H004 & -0.10 & -0.08 & -0.51 & -0.02\\
jobhours & Hours worked/week main job & R*JHOURS & 0.25 & 0.08 & 0.59 & 0.00\\
paidwork & Working for pay & *J020 & 0.28 & 0.07 & 0.62 & -0.04\\
mobilitydiff & Difficulty in mobility rated from 0-5 & R*MOBILA & -0.16 & -0.04 & -0.52 & 0.00\\
cesd & CESD score (depression) & R*CESD & -0.13 & -0.04 & -0.26 & -0.04\\
conde & Sum of health conditions & R*CONDE & -0.22 & -0.03 & -0.51 & 0.04\\
healthexcellent & Self-report of health - excellent (ref: good) & R*SHLT & 0.05 & 0.02 & 0.15 & -0.03\\
healthverygood & Self-report of health - very good (ref: good) & R*SHLT & 0.23 & 0.02 & 0.31 & -0.02\\
healthfair & Self-report of health - fair (ref: good) & R*SHLT & -0.16 & -0.02 & -0.29 & 0.00\\
healthpoor & Self-report of health - poor (ref: good) & R*SHLT & -0.07 & -0.03 & -0.24 & 0.02\\
totalnonresidentkids & Number of nonresident kids & *A100 & 0.66 & -0.05 & NA & NA\\
totalresidentkids & Number of resident children & *A099 & -0.22 & 0.00 & NA & NA\\
secondkid & Has two or more children & KIDID & 0.52 & -0.03 & NA & NA\\
thirdkid & Has three or more children & KIDID & 0.38 & -0.03 & NA & NA\\
kid1female & Gender of first child (f.=1, m.=0) & KAGENDERBG & 0.11 & 0.03 & NA & NA\\
kid2female & Gender of second child (f.=1, m.=0) & KAGENDERBG & 0.17 & -0.01 & NA & NA\\
kid3female & Gender of third child (f.=1, m.=0) & KAGENDERBG & 0.24 & 0.02 & NA & NA\\
kid1age & Age of first child & KABYEARBG & -0.35 & -0.02 & NA & NA\\
kid2age & Age of second child & KABYEARBG & 0.36 & -0.03 & NA & NA\\
kid3age & Age of third child & KABYEARBG & 0.35 & -0.01 & NA & NA\\
kid1educ & Education of first child (years) & KAEDUC & 0.30 & 0.02 & NA & NA\\
kid2educ & Education of second child (years) & KAEDUC & 0.57 & 0.00 & NA & NA\\
kid3educ & Education of third child (years) & KAEDUC & 0.40 & -0.02 & NA & NA\\
childrenclose & Children live within 10 miles & *E012 & 0.14 & 0.01 & NA & NA\\
siblings & Number of living siblings & R*LIVSIB & 0.05 & -0.04 & 0.21 & 0.03\\
swls & Satisfaction with Life Scale & *LB003* & 0.17 & 0.08 & 0.30 & 0.00\\
agree & Agreeableness & *LB033* & 0.06 & 0.04 & 0.11 & 0.02\\
con & Conscientiousness & *LB033* & 0.14 & 0.04 & 0.26 & -0.04\\
extra & Extraversion & *LB033* & 0.04 & 0.04 & 0.18 & 0.01\\
neur & Neuroticism & *LB033* & -0.06 & 0.00 & -0.04 & 0.01\\
open & Openness & *LB033* & 0.04 & 0.07 & 0.05 & -0.04\\
participation & Waves participated (2006-2018) & / & -0.36 & -0.01 & -0.26 & -0.04\\
interviewyear & Date of interview - year & *A501 & -0.33 & -0.05 & -0.18 & -0.05\\
\bottomrule
\addlinespace
\insertTableNotes
\end{longtable}

}

\end{lltable}

\newpage

\noindent \textbf{Supplemental Figures}

\newpage

\noindent  \textbf{Complete Software and Session Information}

We used R (Version 4.0.4; R Core Team, 2021) and the R-packages
\emph{car} (Version 3.0.10; Fox et al., 2020a, 2020b; Yentes \& Wilhelm,
2018), \emph{carData} (Version 3.0.4; Fox et al., 2020b),
\emph{careless} (Version 1.1.3; Yentes \& Wilhelm, 2018), \emph{citr}
(Version 0.3.2; Aust, 2019), \emph{corrplot2017} (Wei \& Simko, 2017),
\emph{cowplot} (Version 1.1.0; Wilke, 2020), \emph{dplyr} (Version
1.0.2; Wickham, François, et al., 2020), \emph{effects} (Version 4.2.0;
Fox \& Weisberg, 2018; Fox, 2003; Fox \& Hong, 2009), \emph{forcats}
(Version 0.5.0; Wickham, 2020a), \emph{foreign} (Version 0.8.81; R Core
Team, 2020), \emph{ggplot2} (Version 3.3.4; Wickham, 2016),
\emph{GPArotation} (Version 2014.11.1; Bernaards \& I.Jennrich, 2005),
\emph{interactions} (Version 1.1.3; Long, 2019), \emph{jtools} (Version
2.1.1; Long, 2020), \emph{knitr} (Version 1.30; Xie, 2015), \emph{lme4}
(Version 1.1.26; Bates et al., 2015), \emph{lmerTest} (Version 3.1.3;
Kuznetsova et al., 2017), \emph{magick} (Version 2.6.0; Ooms, 2021),
\emph{MatchIt} (Version 4.1.0; Ho et al., 2020), \emph{Matrix} (Version
1.3.2; Bates \& Maechler, 2021), \emph{papaja} (Version 0.1.0.9997; Aust
\& Barth, 2020), \emph{patchwork} (Version 1.1.0.9000; Pedersen, 2020),
\emph{png} (Version 0.1.7; Urbanek, 2013), \emph{psych} (Version 2.0.9;
Revelle, 2020), \emph{purrr} (Version 0.3.4; Henry \& Wickham, 2020),
\emph{readr} (Version 1.4.0; Wickham \& Hester, 2020), \emph{robustlmm}
(Version 2.3; Koller, 2016), \emph{scales} (Version 1.1.1; Wickham \&
Seidel, 2020), \emph{stringr} (Version 1.4.0; Wickham, 2019),
\emph{tibble} (Version 3.1.2; Müller \& Wickham, 2020), \emph{tidyr}
(Version 1.1.2; Wickham, 2020b), \emph{tidyverse} (Version 1.3.0;
Wickham, Averick, et al., 2019), and \emph{tinylabels} (Version 0.1.0;
Barth, 2020) for data wrangling, analyses, and plots.\\
The following is the output of R's \emph{sessionInfo()} command, which
shows information to aid analytic reproducibility of the analyses.

R version 4.0.4 (2021-02-15) Platform: x86\_64-apple-darwin17.0 (64-bit)
Running under: macOS Big Sur 10.16

Matrix products: default BLAS:
/Library/Frameworks/R.framework/Versions/4.0/Resources/lib/libRblas.dylib
LAPACK:
/Library/Frameworks/R.framework/Versions/4.0/Resources/lib/libRlapack.dylib

locale: {[}1{]}
en\_US.UTF-8/en\_US.UTF-8/en\_US.UTF-8/C/en\_US.UTF-8/en\_US.UTF-8

attached base packages: {[}1{]} stats graphics grDevices utils datasets
methods base

other attached packages: {[}1{]} cowplot\_1.1.0 lmerTest\_3.1-3
lme4\_1.1-26\\
{[}4{]} Matrix\_1.3-2 GPArotation\_2014.11-1 psych\_2.0.9\\
{[}7{]} forcats\_0.5.0 stringr\_1.4.0 dplyr\_1.0.2\\
{[}10{]} purrr\_0.3.4 readr\_1.4.0 tidyr\_1.1.2\\
{[}13{]} tibble\_3.1.2 ggplot2\_3.3.4 tidyverse\_1.3.0\\
{[}16{]} citr\_0.3.2 papaja\_0.1.0.9997 tinylabels\_0.1.0

loaded via a namespace (and not attached): {[}1{]} nlme\_3.1-152
fs\_1.5.0 lubridate\_1.7.9.2\\
{[}4{]} RColorBrewer\_1.1-2 httr\_1.4.2 numDeriv\_2016.8-1.1 {[}7{]}
tools\_4.0.4 backports\_1.2.1 utf8\_1.2.1\\
{[}10{]} R6\_2.5.0 DBI\_1.1.0 colorspace\_2.0-1\\
{[}13{]} withr\_2.4.2 tidyselect\_1.1.0 mnormt\_2.0.2\\
{[}16{]} compiler\_4.0.4 cli\_2.5.0 rvest\_0.3.6\\
{[}19{]} xml2\_1.3.2 labeling\_0.4.2 bookdown\_0.21\\
{[}22{]} scales\_1.1.1 digest\_0.6.27 minqa\_1.2.4\\
{[}25{]} rmarkdown\_2.5 base64enc\_0.1-3 pkgconfig\_2.0.3\\
{[}28{]} htmltools\_0.5.0 highr\_0.8 dbplyr\_1.4.4\\
{[}31{]} fastmap\_1.0.1 rlang\_0.4.11 readxl\_1.3.1\\
{[}34{]} rstudioapi\_0.13 shiny\_1.5.0 farver\_2.1.0\\
{[}37{]} generics\_0.1.0 jsonlite\_1.7.2 magrittr\_2.0.1\\
{[}40{]} Rcpp\_1.0.6 munsell\_0.5.0 fansi\_0.5.0\\
{[}43{]} lifecycle\_1.0.0 stringi\_1.5.3 yaml\_2.2.1\\
{[}46{]} MASS\_7.3-53 grid\_4.0.4 blob\_1.2.1\\
{[}49{]} parallel\_4.0.4 promises\_1.1.1 crayon\_1.4.1\\
{[}52{]} miniUI\_0.1.1.1 lattice\_0.20-41 haven\_2.3.1\\
{[}55{]} splines\_4.0.4 hms\_0.5.3 tmvnsim\_1.0-2\\
{[}58{]} knitr\_1.30 pillar\_1.6.1 boot\_1.3-26\\
{[}61{]} reprex\_0.3.0 glue\_1.4.2 evaluate\_0.14\\
{[}64{]} modelr\_0.1.8 vctrs\_0.3.8 nloptr\_1.2.2.2\\
{[}67{]} httpuv\_1.5.4 cellranger\_1.1.0 gtable\_0.3.0\\
{[}70{]} assertthat\_0.2.1 xfun\_0.19 mime\_0.9\\
{[}73{]} xtable\_1.8-4 broom\_0.7.6 later\_1.1.0.1\\
{[}76{]} statmod\_1.4.35 ellipsis\_0.3.2

\newpage

\noindent  \textbf{References}

\begingroup
\setlength{\parindent}{-0.5in}
\setlength{\leftskip}{0.5in}

\hypertarget{refs}{}
\leavevmode\hypertarget{ref-R-citr}{}%
Aust, F. (2019). \emph{Citr: 'RStudio' add-in to insert markdown
citations}. \url{https://github.com/crsh/citr}

\leavevmode\hypertarget{ref-R-papaja}{}%
Aust, F., \& Barth, M. (2020). \emph{papaja: Prepare reproducible APA
journal articles with R Markdown}. \url{https://github.com/crsh/papaja}

\leavevmode\hypertarget{ref-austinIntroductionPropensityScore2011}{}%
Austin, P. C. (2011). An introduction to propensity score methods for
reducing the effects of confounding in observational studies.
\emph{Multivariate Behavioral Research}, \emph{46}(3), 399--424.
\url{https://doi.org/10.1080/00273171.2011.568786}

\leavevmode\hypertarget{ref-R-tinylabels}{}%
Barth, M. (2020). \emph{Tinylabels: Lightweight variable labels}.
\url{https://CRAN.R-project.org/package=tinylabels}

\leavevmode\hypertarget{ref-R-Matrix}{}%
Bates, D., \& Maechler, M. (2021). \emph{Matrix: Sparse and dense matrix
classes and methods}. \url{https://CRAN.R-project.org/package=Matrix}

\leavevmode\hypertarget{ref-R-lme4}{}%
Bates, D., Mächler, M., Bolker, B., \& Walker, S. (2015). Fitting linear
mixed-effects models using lme4. \emph{Journal of Statistical Software},
\emph{67}(1), 1--48. \url{https://doi.org/10.18637/jss.v067.i01}

\leavevmode\hypertarget{ref-R-GPArotation}{}%
Bernaards, C. A., \& I.Jennrich, R. (2005). Gradient projection
algorithms and software for arbitrary rotation criteria in factor
analysis. \emph{Educational and Psychological Measurement}, \emph{65},
676--696.

\leavevmode\hypertarget{ref-R-effects_b}{}%
Fox, J. (2003). Effect displays in R for generalised linear models.
\emph{Journal of Statistical Software}, \emph{8}(15), 1--27.
\url{https://www.jstatsoft.org/article/view/v008i15}

\leavevmode\hypertarget{ref-R-effects_c}{}%
Fox, J., \& Hong, J. (2009). Effect displays in R for multinomial and
proportional-odds logit models: Extensions to the effects package.
\emph{Journal of Statistical Software}, \emph{32}(1), 1--24.
\url{https://www.jstatsoft.org/article/view/v032i01}

\leavevmode\hypertarget{ref-R-effects_a}{}%
Fox, J., \& Weisberg, S. (2018). Visualizing fit and lack of fit in
complex regression models with predictor effect plots and partial
residuals. \emph{Journal of Statistical Software}, \emph{87}(9), 1--27.
\url{https://doi.org/10.18637/jss.v087.i09}

\leavevmode\hypertarget{ref-R-car}{}%
Fox, J., Weisberg, S., \& Price, B. (2020a). \emph{Car: Companion to
applied regression} {[}Manual{]}.

\leavevmode\hypertarget{ref-R-carData}{}%
Fox, J., Weisberg, S., \& Price, B. (2020b). \emph{CarData: Companion to
applied regression data sets}.
\url{https://CRAN.R-project.org/package=carData}

\leavevmode\hypertarget{ref-R-purrr}{}%
Henry, L., \& Wickham, H. (2020). \emph{Purrr: Functional programming
tools}. \url{https://CRAN.R-project.org/package=purrr}

\leavevmode\hypertarget{ref-R-MatchIt}{}%
Ho, D., Imai, K., King, G., Stuart, E., \& Greifer, N. (2020).
\emph{MatchIt: Nonparametric preprocessing for parametric causal
inference} {[}Manual{]}.

\leavevmode\hypertarget{ref-R-robustlmm}{}%
Koller, M. (2016). robustlmm: An R package for robust estimation of
linear mixed-effects models. \emph{Journal of Statistical Software},
\emph{75}(6), 1--24. \url{https://doi.org/10.18637/jss.v075.i06}

\leavevmode\hypertarget{ref-R-lmerTest}{}%
Kuznetsova, A., Brockhoff, P. B., \& Christensen, R. H. B. (2017).
lmerTest package: Tests in linear mixed effects models. \emph{Journal of
Statistical Software}, \emph{82}(13), 1--26.
\url{https://doi.org/10.18637/jss.v082.i13}

\leavevmode\hypertarget{ref-R-interactions}{}%
Long, J. A. (2019). \emph{Interactions: Comprehensive, user-friendly
toolkit for probing interactions}.
\url{https://cran.r-project.org/package=interactions}

\leavevmode\hypertarget{ref-R-jtools}{}%
Long, J. A. (2020). \emph{Jtools: Analysis and presentation of social
scientific data}. \url{https://cran.r-project.org/package=jtools}

\leavevmode\hypertarget{ref-R-tibble}{}%
Müller, K., \& Wickham, H. (2020). \emph{Tibble: Simple data frames}.
\url{https://CRAN.R-project.org/package=tibble}

\leavevmode\hypertarget{ref-R-magick}{}%
Ooms, J. (2021). \emph{Magick: Advanced graphics and image-processing in
r}. \url{https://CRAN.R-project.org/package=magick}

\leavevmode\hypertarget{ref-R-patchwork}{}%
Pedersen, T. L. (2020). \emph{Patchwork: The composer of plots}.

\leavevmode\hypertarget{ref-R-foreign}{}%
R Core Team. (2020). \emph{Foreign: Read data stored by 'minitab', 's',
'sas', 'spss', 'stata', 'systat', 'weka', 'dBase', ...}
\url{https://CRAN.R-project.org/package=foreign}

\leavevmode\hypertarget{ref-R-base}{}%
R Core Team. (2021). \emph{R: A language and environment for statistical
computing}. R Foundation for Statistical Computing.
\url{https://www.R-project.org/}

\leavevmode\hypertarget{ref-R-psych}{}%
Revelle, W. (2020). \emph{Psych: Procedures for psychological,
psychometric, and personality research}. Northwestern University.
\url{https://CRAN.R-project.org/package=psych}

\leavevmode\hypertarget{ref-stuartMatchingMethodsCausal2010}{}%
Stuart, E. A. (2010). Matching methods for causal inference: A review
and a look forward. \emph{Statistical Science: A Review Journal of the
Institute of Mathematical Statistics}, \emph{25}(1), 1--21.
\url{https://doi.org/10.1214/09-STS313}

\leavevmode\hypertarget{ref-R-png}{}%
Urbanek, S. (2013). \emph{Png: Read and write png images}.
\url{https://CRAN.R-project.org/package=png}

\leavevmode\hypertarget{ref-R-corrplot2017}{}%
Wei, T., \& Simko, V. (2017). \emph{R package "corrplot": Visualization
of a correlation matrix}. \url{https://github.com/taiyun/corrplot}

\leavevmode\hypertarget{ref-R-ggplot2}{}%
Wickham, H. (2016). \emph{Ggplot2: Elegant graphics for data analysis}.
Springer-Verlag New York. \url{https://ggplot2.tidyverse.org}

\leavevmode\hypertarget{ref-R-stringr}{}%
Wickham, H. (2019). \emph{Stringr: Simple, consistent wrappers for
common string operations}.
\url{https://CRAN.R-project.org/package=stringr}

\leavevmode\hypertarget{ref-R-forcats}{}%
Wickham, H. (2020a). \emph{Forcats: Tools for working with categorical
variables (factors)}. \url{https://CRAN.R-project.org/package=forcats}

\leavevmode\hypertarget{ref-R-tidyr}{}%
Wickham, H. (2020b). \emph{Tidyr: Tidy messy data}.
\url{https://CRAN.R-project.org/package=tidyr}

\leavevmode\hypertarget{ref-R-tidyverse}{}%
Wickham, H., Averick, M., Bryan, J., Chang, W., McGowan, L. D.,
François, R., Grolemund, G., Hayes, A., Henry, L., Hester, J., Kuhn, M.,
Pedersen, T. L., Miller, E., Bache, S. M., Müller, K., Ooms, J.,
Robinson, D., Seidel, D. P., Spinu, V., \ldots{} Yutani, H. (2019).
Welcome to the tidyverse. \emph{Journal of Open Source Software},
\emph{4}(43), 1686. \url{https://doi.org/10.21105/joss.01686}

\leavevmode\hypertarget{ref-R-dplyr}{}%
Wickham, H., François, R., Henry, L., \& Müller, K. (2020). \emph{Dplyr:
A grammar of data manipulation}.
\url{https://CRAN.R-project.org/package=dplyr}

\leavevmode\hypertarget{ref-R-readr}{}%
Wickham, H., \& Hester, J. (2020). \emph{Readr: Read rectangular text
data}. \url{https://CRAN.R-project.org/package=readr}

\leavevmode\hypertarget{ref-R-scales}{}%
Wickham, H., \& Seidel, D. (2020). \emph{Scales: Scale functions for
visualization}. \url{https://CRAN.R-project.org/package=scales}

\leavevmode\hypertarget{ref-R-cowplot}{}%
Wilke, C. O. (2020). \emph{Cowplot: Streamlined plot theme and plot
annotations for 'ggplot2'}.
\url{https://CRAN.R-project.org/package=cowplot}

\leavevmode\hypertarget{ref-R-knitr}{}%
Xie, Y. (2015). \emph{Dynamic documents with R and knitr} (2nd ed.).
Chapman; Hall/CRC. \url{https://yihui.org/knitr/}

\leavevmode\hypertarget{ref-R-careless}{}%
Yentes, R. D., \& Wilhelm, F. (2018). \emph{Careless: Procedures for
computing indices of careless responding}.

\endgroup
\end{appendix}

\end{document}
