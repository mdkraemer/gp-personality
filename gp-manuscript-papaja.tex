% Options for packages loaded elsewhere
\PassOptionsToPackage{unicode}{hyperref}
\PassOptionsToPackage{hyphens}{url}
%
\documentclass[
  english,
  man, noextraspace]{apa7}
\usepackage{lmodern}
\usepackage{amssymb,amsmath}
\usepackage{ifxetex,ifluatex}
\ifnum 0\ifxetex 1\fi\ifluatex 1\fi=0 % if pdftex
  \usepackage[T1]{fontenc}
  \usepackage[utf8]{inputenc}
  \usepackage{textcomp} % provide euro and other symbols
\else % if luatex or xetex
  \usepackage{unicode-math}
  \defaultfontfeatures{Scale=MatchLowercase}
  \defaultfontfeatures[\rmfamily]{Ligatures=TeX,Scale=1}
\fi
% Use upquote if available, for straight quotes in verbatim environments
\IfFileExists{upquote.sty}{\usepackage{upquote}}{}
\IfFileExists{microtype.sty}{% use microtype if available
  \usepackage[]{microtype}
  \UseMicrotypeSet[protrusion]{basicmath} % disable protrusion for tt fonts
}{}
\makeatletter
\@ifundefined{KOMAClassName}{% if non-KOMA class
  \IfFileExists{parskip.sty}{%
    \usepackage{parskip}
  }{% else
    \setlength{\parindent}{0pt}
    \setlength{\parskip}{6pt plus 2pt minus 1pt}}
}{% if KOMA class
  \KOMAoptions{parskip=half}}
\makeatother
\usepackage{xcolor}
\IfFileExists{xurl.sty}{\usepackage{xurl}}{} % add URL line breaks if available
\IfFileExists{bookmark.sty}{\usepackage{bookmark}}{\usepackage{hyperref}}
\hypersetup{
  pdftitle={The Transition to Grandparenthood and its Impact on the Big Five Personality Traits and Life Satisfaction},
  pdfauthor={Michael D. Krämer1,2, Manon A. van Scheppingen3, William J. Chopik4, \& David Richter1,4},
  pdflang={en-EN},
  pdfkeywords={grandparenthood, Big Five, life satisfaction, development, propensity score matching},
  hidelinks,
  pdfcreator={LaTeX via pandoc}}
\urlstyle{same} % disable monospaced font for URLs
\usepackage{graphicx,grffile}
\makeatletter
\def\maxwidth{\ifdim\Gin@nat@width>\linewidth\linewidth\else\Gin@nat@width\fi}
\def\maxheight{\ifdim\Gin@nat@height>\textheight\textheight\else\Gin@nat@height\fi}
\makeatother
% Scale images if necessary, so that they will not overflow the page
% margins by default, and it is still possible to overwrite the defaults
% using explicit options in \includegraphics[width, height, ...]{}
\setkeys{Gin}{width=\maxwidth,height=\maxheight,keepaspectratio}
% Set default figure placement to htbp
\makeatletter
\def\fps@figure{htbp}
\makeatother
\setlength{\emergencystretch}{3em} % prevent overfull lines
\providecommand{\tightlist}{%
  \setlength{\itemsep}{0pt}\setlength{\parskip}{0pt}}
\setcounter{secnumdepth}{-\maxdimen} % remove section numbering
% Make \paragraph and \subparagraph free-standing
\ifx\paragraph\undefined\else
  \let\oldparagraph\paragraph
  \renewcommand{\paragraph}[1]{\oldparagraph{#1}\mbox{}}
\fi
\ifx\subparagraph\undefined\else
  \let\oldsubparagraph\subparagraph
  \renewcommand{\subparagraph}[1]{\oldsubparagraph{#1}\mbox{}}
\fi
% Manuscript styling
\usepackage{upgreek}
\captionsetup{font=singlespacing,justification=justified}

% Table formatting
\usepackage{longtable}
\usepackage{lscape}
% \usepackage[counterclockwise]{rotating}   % Landscape page setup for large tables
\usepackage{multirow}		% Table styling
\usepackage{tabularx}		% Control Column width
\usepackage[flushleft]{threeparttable}	% Allows for three part tables with a specified notes section
\usepackage{threeparttablex}            % Lets threeparttable work with longtable

% Create new environments so endfloat can handle them
% \newenvironment{ltable}
%   {\begin{landscape}\begin{center}\begin{threeparttable}}
%   {\end{threeparttable}\end{center}\end{landscape}}
\newenvironment{lltable}{\begin{landscape}\begin{center}\begin{ThreePartTable}}{\end{ThreePartTable}\end{center}\end{landscape}}

% Enables adjusting longtable caption width to table width
% Solution found at http://golatex.de/longtable-mit-caption-so-breit-wie-die-tabelle-t15767.html
\makeatletter
\newcommand\LastLTentrywidth{1em}
\newlength\longtablewidth
\setlength{\longtablewidth}{1in}
\newcommand{\getlongtablewidth}{\begingroup \ifcsname LT@\roman{LT@tables}\endcsname \global\longtablewidth=0pt \renewcommand{\LT@entry}[2]{\global\advance\longtablewidth by ##2\relax\gdef\LastLTentrywidth{##2}}\@nameuse{LT@\roman{LT@tables}} \fi \endgroup}

% \setlength{\parindent}{0.5in}
% \setlength{\parskip}{0pt plus 0pt minus 0pt}

% Overwrite redefinition of paragraph and subparagraph by the default LaTeX template
% See https://github.com/crsh/papaja/issues/292
\makeatletter
\renewcommand{\paragraph}{\@startsection{paragraph}{4}{\parindent}%
  {0\baselineskip \@plus 0.2ex \@minus 0.2ex}%
  {-1em}%
  {\normalfont\normalsize\bfseries\itshape\typesectitle}}

\renewcommand{\subparagraph}[1]{\@startsection{subparagraph}{5}{1em}%
  {0\baselineskip \@plus 0.2ex \@minus 0.2ex}%
  {-\z@\relax}%
  {\normalfont\normalsize\itshape\hspace{\parindent}{#1}\textit{\addperi}}{\relax}}
\makeatother

% \usepackage{etoolbox}
\makeatletter
\patchcmd{\HyOrg@maketitle}
  {\section{\normalfont\normalsize\abstractname}}
  {\section*{\normalfont\normalsize\abstractname}}
  {}{\typeout{Failed to patch abstract.}}
\patchcmd{\HyOrg@maketitle}
  {\section{\protect\normalfont{\@title}}}
  {\section*{\protect\normalfont{\@title}}}
  {}{\typeout{Failed to patch title.}}
\makeatother
\shorttitle{Grandparenthood, Big Five, and Life Satisfaction}
\keywords{grandparenthood, Big Five, life satisfaction, development, propensity score matching\newline\indent Word count: abc}
\DeclareDelayedFloatFlavor{ThreePartTable}{table}
\DeclareDelayedFloatFlavor{lltable}{table}
\DeclareDelayedFloatFlavor*{longtable}{table}
\makeatletter
\renewcommand{\efloat@iwrite}[1]{\immediate\expandafter\protected@write\csname efloat@post#1\endcsname{}}
\makeatother
\usepackage{lineno}

\linenumbers
\usepackage{csquotes}
\usepackage{setspace}
\usepackage{amsmath}
\AtBeginEnvironment{tabular}{\singlespacing}
\AtBeginEnvironment{lltable}{\singlespacing}
\AtBeginEnvironment{tablenotes}{\doublespacing}
\captionsetup[table]{font={stretch=1.5}}
\captionsetup[figure]{font={stretch=1.5}}
\ifxetex
  % Load polyglossia as late as possible: uses bidi with RTL langages (e.g. Hebrew, Arabic)
  \usepackage{polyglossia}
  \setmainlanguage[]{english}
\else
  \usepackage[shorthands=off,main=english]{babel}
\fi

\title{The Transition to Grandparenthood and its Impact on the Big Five Personality Traits and Life Satisfaction}
\author{Michael D. Krämer\textsuperscript{1,2}, Manon A. van Scheppingen\textsuperscript{3}, William J. Chopik\textsuperscript{4}, \& David Richter\textsuperscript{1,4}}
\date{}


\note{\clearpage}

\authornote{

\addORCIDlink{Michael D. Krämer}{0000-0002-9883-5676}, Socio-Economic Panel (SOEP), German Institute for Economic Research (DIW Berlin); International Max Planck Research School on the Life Course (LIFE), Max Planck Institute for Human Development\\
Manon A. van Scheppingen, Department of Developmental Psychology, Tilburg School of Social and Behavioral Sciences, Tilburg University\\
William J. Chopik, Department of Psychology, Michigan State University\\
David Richter, Socio-Economic Panel (SOEP), German Institute for Economic Research (DIW Berlin); Survey Research Division, Department of Education and Psychology, Freie Universität Berlin

The authors made the following contributions. Michael D. Krämer: Conceptualization, Data Curation, Formal Analysis, Methodology, Visualization, Writing - Original Draft Preparation, Writing - Review \& Editing; Manon A. van Scheppingen: Methodology, Writing - Review \& Editing; William J. Chopik: Methodology, Writing - Review \& Editing; David Richter: Supervision, Methodology, Writing - Review \& Editing.

Correspondence concerning this article should be addressed to Michael D. Krämer, German Institute for Economic Research, Mohrenstr. 58, 10117 Berlin, Germany. E-mail: \href{mailto:mkraemer@diw.de}{\nolinkurl{mkraemer@diw.de}}

}

\affiliation{\vspace{0.5cm}\textsuperscript{1} German Institute for Economic Research, Germany\\\textsuperscript{2} International Max Planck Research School on the Life Course (LIFE), Max Planck Institute for Human Development, Germany\\\textsuperscript{3} Tilburg University, Netherlands\\\textsuperscript{4} Michigan State University, USA\\\textsuperscript{5} Freie Universität Berlin, Germany}

\abstract{
abc
}



\begin{document}
\maketitle

In view of an aging demographic and an increased share of childcare functions being fulfilled by grandparents, intergenerational relations have received heightened attention from psychological and sociological research in recent years (Bengtson, 2001). With regard to personality development, the transition to grandparenthood has been posited as an important developmental task in old age (Hutteman et al., 2014). However, empirical research into the psychological consequences of this transition is sparse. Testing hypotheses derived from neo-socioanalytic theory (Roberts \& Wood, 2006) in a matched control-group design (see Luhmann et al., 2014), we aim to investigate whether the transition to grandparenthood affects the Big Five personality traits and life satisfaction.\\

\hypertarget{personality-development-in-middle-adulthood-and-old-age}{%
\subsection{Personality Development in Middle Adulthood and Old Age}\label{personality-development-in-middle-adulthood-and-old-age}}

In accordance with the life span perspective characterizing aging as a lifelong process of development and adaptation (Baltes et al., 2006), personality traits are subject to change throughout the entire life span (Costa et al., 2019; Specht, 2017; Specht et al., 2014). Although a major portion of development takes place in adolescence and emerging adulthood (Bleidorn \& Schwaba, 2017; Schwaba \& Bleidorn, 2018), evidence has accumulated that the Big Five personality traits also undergo changes in middle and old adulthood (e.g., Kandler et al., 2015; Lucas \& Donnellan, 2011; Mõttus et al., 2012; Wagner et al., 2016; for a review, see Specht, 2017).\\
Changes over time occur both in mean trait levels of these age groups (i.e., mean-level change; Roberts et al., 2006) and in the relative ordering of people to each other on trait dimensions (i.e., rank-order stability; Anusic \& Schimmack, 2016; Roberts \& DelVecchio, 2000). Mean-level changes in middle adulthood (ca. 30--60 years old; Hutteman et al., 2014) are typically characterized in terms of greater maturity as evidenced by increased agreeableness and conscientiousness, and decreased neuroticism (Roberts et al., 2006). In old age (ca. 60 years and older; Hutteman et al., 2014), research is generally more sparse but there is some evidence for a reversal of the maturity effect, especially following retirement (sometimes termed \emph{La dolce vita} effect; Marsh et al., 2013; cf.~Schwaba \& Bleidorn, 2019) and at the end of life in ill health (Wagner et al., 2016). In terms of rank-order stability, some prior studies have shown support for an inverted U-shape trajectory (Ardelt, 2000; Lucas \& Donnellan, 2011; Specht et al., 2011; Wortman et al., 2012): Rank-order stability rises until reaching a plateau in midlife, and decreases, again, in old age. However, evidence is mixed whether rank-order stability actually decreases again in old age (see Costa et al., 2019). Nonetheless, the historical view that personality is stable, or \enquote{set like plaster} (Specht, 2017, p. 64) after one reaches adulthood (or leaves emerging adulthood behind; Bleidorn \& Schwaba, 2017) can be largely abandoned (Specht et al., 2014).\\
Theories explaining the mechanisms of personality development in middle adulthood and old age emphasize as interdependent sources of stability and change both genetic influences and life experiences (Specht et al., 2014; Wagner et al., 2020). Here, we focus on the latter\footnote{In a behavior-genetic twin study, Kandler et al.~(2015) found that environmental factors were the main source of personality development in old age.} and conceptualize the transition to grandparenthood as a life experience that offers the adoption of a new social role according to the social investment principle of neo-socioanalytic theory (Lodi-Smith \& Roberts, 2007; Roberts \& Wood, 2006). According to the social investment principle, normative life events or transitions such as entering the work force or becoming a parent lead to personality maturation through the adoption of new social roles (Roberts et al., 2005). These new roles encourage or compel people to act in a more agreeable, conscientious, and emotionally stable way, and the experiences in these role as well as societal expectations towardds them are hypothesized to drive long-term personality development (Lodi-Smith \& Roberts, 2007). Conversely, consistent social roles foster personality stability. The paradoxical theory of personality coherence (Caspi \& Moffitt, 1993) offers another explanation for personality development stating that trait change is more likely whenever people transition into unknown environments where pre-existing behavioral responses are no longer appropriate and societal norms or social expectations give clear indications how to behave instead (vs.~environments where no such guidance is available). This supports the view that age-graded, normative life experiences such as possibly the transition to grandparenthood drive personality development (see also Specht et al., 2014).\\
Certain life events such as the first romantic relationship (Wagner et al., 2015) or the transition from high school to university (Lüdtke et al., 2011) have (partly) been found to be accompanied by mean-level increases in line with the social investment principle (for a review, see Bleidorn et al., 2018). However, recent evidence regarding the transition to parenthood failed to empirically support the social investment principle (Asselmann \& Specht, 2020; van Scheppingen et al., 2016). An analysis of monthly trajectories of the Big Five before and after nine major life events only found limited support for the social investment principle, that is, small increases were only found in emotional stability following the transition to employment but not for the other traits or for the other life events theoretically linked to social investment (Denissen et al., 2019). It has also been emphasized recently that effects of life events on the Big Five personality trends generally tend to be small, and need to be properly analyzed using robust, prospective designs and appropriate control groups (Bleidorn et al., 2018; Luhmann et al., 2014).\\
Overall, much remains unknown regarding the environmental factors underlying personality development in middle adulthood and old age. One indication that age-graded, normative life experiences contribute to change following a period of relative stability is recent research on retirement (Bleidorn \& Schwaba, 2018; Schwaba \& Bleidorn, 2019). While these results were only partly in line with the social investment principle in terms of mean-level changes and displayed substantial individual differences in change trajectories, the authors also discuss that as social role \enquote{divestment} (Schwaba \& Bleidorn, 2019, p. X) retirement functions differently compared to social investment which adds a role. The transition to grandparenthood could represent such an investment in older adulthood---given that grandparents have regular contact with their grandchild and actively take part in childcare (i.e., invest psychologically in the new grandparent role; Lodi-Smith \& Roberts, 2007), to some degree.

\hypertarget{grandparenthood}{%
\subsection{Grandparenthood}\label{grandparenthood}}

The transition to grandparenthood, that is, the birth of the first grandchild, can be described as a time-discrete life event marking the beginning of one's status as a grandparent (Luhmann et al., 2012). In terms of characteristics of major life events (Luhmann et al., 2020), the transition to grandparenthood stands out in that it is externally caused (by one's own children), while at the same time predictable (as soon as one's children reveal their family planning or pregnancy), as well as generally positive in valence and emotionally significant.\\
Grandparenthood can also be characterized as a developmental task (Hutteman et al., 2014) mostly associated with the period of (early) old age---although considerable variation in the age at the transition to grandparenthood exists both within and across cultures (Leopold \& Skopek, 2015; Skopek \& Leopold, 2017). Still, the period where parents on average experience the birth of their first grandchild coincides with the end of midlife stability in terms of personality development (Specht, 2017), where retirement, shifting social roles, and initial cognitive and health declines can potentially be disruptive to life circumstances putting personality development into motion (e.g., Mueller et al., 2016; Stephan et al., 2014). As a developmental task, grandparenthood is expected to follow a normative sequence of aging that is subject to societal expectations and values differing across cultures and historical time (Hutteman et al., 2014). Mastering developmental tasks to a high degree is hypothesized to drive personality development towards maturation similarly to propositions by the social investment principle, that is, leading to higher levels of agreeableness and conscientiousness, and lower levels of neuroticism (Roberts et al., 2005; Roberts \& Wood, 2006). In comparison to the transition to parenthood which has been found to be ambivalent in terms of both personality maturation and life satisfaction (Krämer \& Rodgers, 2020; van Scheppingen et al., 2016), Hutteman et al.~(2014) hypothesize that the transition to grandparenthood is generally seen as positive because it (usually) does not impose the stressful daily demands of childcare on grandparents.\\
While we could not find prior studies investigating development of the Big Five over the transition to grandparenthood, there is some evidence on life satisfaction although it is conflicting: Past research on associations of grandparenthood with life satisfaction often relied on cross-sectional designs (e.g., Mahne \& Huxhold, 2014; Triadó et al., 2014). Longitudinal studies utilizing panel data from the Survey of Health, Ageing and Retirement in Europe (SHARE) showed that the birth of a grandchild was followed by improvements to quality of life and life satisfaction only among women (Tanskanen et al., 2019), and only in first-time grandmothers via their daughters (Di Gessa et al., 2019). Several studies emphasized that grandparents actively involved in childcare experienced larger positive effects to life satisfaction (Arpino, Bordone, et al., 2018; Danielsbacka et al., 2019; Danielsbacka \& Tanskanen, 2016). On the other hand, fixed effects regression models\footnote{Fixed effects regression models exclusively rely on within-person variance (see Brüderl \& Ludwig, 2015; McNeish \& Kelley, 2019).} using SHARE data did not find any effects of first-time grandparenthood on life satisfaction regardless of grandparental investment and only minor decreases of grandmothers' depressive symptoms (Sheppard \& Monden, 2019).
In a similar vein, some prospective studies reported beneficial effects of the transition to grandparenthood and of grandparental childcare investment on various health measures, especially in women (Chung \& Park, 2018; Condon et al., 2018; Di Gessa et al., 2016a, 2016b). Again, effects on self-rated health did not persevere in fixed effects analyses as reported in Ates (2017) who used longitudinal data from the German Aging Survey (DEAS).

\hypertarget{current-study}{%
\subsection{Current Study}\label{current-study}}

Three research questions motivate the current study which is the first to analyze personality development over the transition to grandparenthood with regards to the Big Five traits:

\begin{enumerate}
\def\labelenumi{\arabic{enumi}.}
\tightlist
\item
  What are the effects of the transition to grandparenthood on mean-level trajectories of the Big Five traits and life satisfaction?
\item
  How large are interindividual differences in intraindividual change for the Big Five traits and life satisfaction over the transition to grandparenthood?
\item
  How does the transition to grandparenthood affect rank-order stability of the Big Five traits and life satisfaction?
\end{enumerate}

To address these questions, we will compare development over the transition to grandparenthood with that of matched participants that do not experience this transition during the study period (Luhmann et al., 2014). This is necessary because pre-existing differences in variables related to the development of the Big Five or life satisfaction between those who are observed to become a grandparent and those who are not introduce confounding bias when trying to estimate the effect of the transition to grandparenthood (e.g., VanderWeele et al., 2020). Propensity score matching accounts for confounding through equating the groups in their propensity to experience the event in question, which is calculated from a broad range of covariates related to the event and the outcomes. Thereby, to address confounding balance between the covariates used to calculate the propensity score is also aimed for (Stuart, 2010).\\
We adopt a prospective design that tests effects of first-time grandparents against two propensity-score-matched control groups: first, a matched control group of parents (but not grandparents) with at least their oldest child in reproductive age, and, second, a matched control group of nonparents. This allows us to disentangle potential effects attributable to becoming a grandparent from effects attributable to being a parent, thus, addressing selection effects into grandparenthood and confounding more comprehensively than previous research. Our comparative design also controls for average age-related and historical trends in the Big Five traits and life satisfaction (Luhmann et al., 2014), and enables us to report effects of the transition to grandparenthood unconfounded by instrumentation effects, which describe the tendency of reporting lower well-being scores with each repeated measurement (Baird et al., 2010). We go beyond previous studies utilizing matched control groups (Anusic et al., 2014a, 2014b; Yap et al., 2012) in that we performed the matching at a specific time point preceding the transition to grandparenthood (at least two years before) and not based on individual survey years. This design choice ensures that the covariates involved in the matching procedure are not already influenced by the event or anticipation of it (Elwert \& Winship, 2014; Greenland, 2003; Rosenbaum, 1984; VanderWeele, 2019; VanderWeele et al., 2020), thereby also reducing the risk of confounding through collider bias (Elwert \& Winship, 2014). Similar approaches in the study of life events have recently been adopted (Balbo \& Arpino, 2016; Krämer \& Rodgers, 2020; van Scheppingen \& Leopold, 2020).\\
Informed by the social investment principle and previous research on personality development in middle adulthood and old age, we preregistered the following hypotheses (prior to data analysis; osf.io/):

\begin{itemize}
\tightlist
\item
  H1a: Following the birth of their first grandchild, grandparents increase slightly in agreeableness and conscientiousness, and decrease in neuroticism as compared to the matched control groups of parents (but not grandparents) and nonparents, but do not differ in their trajectories of extraversion and openness to experience.
\item
  H1b: Grandmothers increase in life satisfaction following the transition to grandparenthood as compared to the matched control groups (but grandfathers do not).
\item
  H2: Individual differences in intraindividual change in the Big Five and life satisfaction are larger in the grandparent group than the control group.
\item
  H3a: Compared to the matched control groups, grandparents' rank-order stability of the Big Five decreases over the transition to grandparenthood.
\item
  H3b: Grandparents' rank-order stability of life satisfaction is comparatively stable over the transition to grandparenthood.
\end{itemize}

Exploratorily, we further probe the social investment principle by testing two moderators of potential social investment and role conflict, hours of grandchild care and performing paid work.

\hypertarget{methods}{%
\section{Methods}\label{methods}}

\hypertarget{samples}{%
\subsection{Samples}\label{samples}}

To evaluate these hypotheses, we used data from two population-representative panel studies: the Longitudinal Internet Studies for the Social Sciences (LISS) panel from the Netherlands and the Health and Retirement Study (HRS) from the United States.\\
The LISS panel is a representative sample of the Dutch population initiated in 2008 with data collection still ongoing (Scherpenzeel, 2011; van der Laan, 2009). It is administered by CentERdata (Tilburg University, The Netherlands). Included households are a true probability sample of households drawn from the population register (Scherpenzeel \& Das, 2010). While originally roughly half of invited households consented to participate, refreshment samples were drawn in order to oversample previously underrepresented groups using information about response rates and their association with demographic variables (household type, age, ethnicity; see \url{https://www.lissdata.nl/about-panel/sample-and-recruitment}). Data collection was carried out online and participants lacking the necessary technical equipment were outfitted with it. We included yearly assessments from 2008 to 2020 from several different modules (see \emph{Measures}) as well as data on basic demographics which was assessed on a monthly rate. For later coding of covariates from these monthly demographic data we used the first available assessment in each year.\\
The HRS is a longitudinal population-representative study of older adults in the US (Sonnega et al., 2014) administered by the Survey Research Center (University of Michigan, United States). Initiated in 1992 with a first cohort of individuals aged 51-61 and their spouses, the study has since been extended with additional cohorts in the 1990s. In addition to the HRS core interview every two years (in-person or as a telephone survey), the study has since 2006 included a leave-behind questionnaire covering a broad range of psychosocial topics including the Big Five personality traits and life satisfaction. These topics, however, were only administered every four years starting in 2006 for one half of the sample and in 2008 for the other half. We included personality data from 2006 to 2016, all available data for the coding of the transition to grandparenthood from 1996 to 2016, as well as covariate data from 2006 to 2016 including variables drawn from the Imputations File and the Family Data (available up to 2014).\\
These two panel studies provided the advantage that they contained several waves of personality data as well as information on grandparent status and a broad range of covariates at each wave. While the HRS provided a large sample with a wider age range, the LISS panel was smaller and younger\footnote{The reason for the included grandparents from the LISS panel being younger was that grandparenthood questions were part of the \emph{Work and Schooling} module and---for reasons unknown to us---filtered to participants performing paid work. Thus, older, retired first-time grandparents from the LISS panel could not be identified.} but provided more frequent personality assessments spaced every one to two years. Note that M. van Scheppingen has previously used the LISS panel to analyze ???. B. Chopik has previously used the HRS to analyze ???. These publications do not overlap with the current study in the central focus of grandparenthood.\footnote{Publications using LISS panel data can be searched at \url{https://www.dataarchive.lissdata.nl/publications/}. Publications using HRS data can be searched at \url{https://hrs.isr.umich.edu/publications/biblio/}.} The present study used de-identified archival data in the public domain, and, thus, it was not necessary to obtain ethical approval from an IRB.

\hypertarget{measures}{%
\subsection{Measures}\label{measures}}

\hypertarget{personality}{%
\subsubsection{Personality}\label{personality}}

In the LISS panel, the Big Five personality traits were assessed using the 50-item version of the IPIP Big-Five Inventory scales (Goldberg, 1992). For each Big Five trait, ten 5-point Likert-scale items were answered (1 = \emph{very inaccurate}, 2 = \emph{moderately inaccurate}, 3 = \emph{neither inaccurate nor accurate}, 4 = \emph{moderately accurate}, 5 = \emph{very accurate}). Example items included \enquote{Like order} (conscientiousness), \enquote{Sympathize with others' feelings} (agreeableness), \enquote{Worry about things} (neuroticism), \enquote{Have a vivid imagination} (openness to experience), and \enquote{Start conversations} (extraversion). At each wave, we took a participant's mean of each subscale as their trait score. Internal consistencies, as indicated by McDonald's \(\omega\) (McNeish, 2018), averaged XX over all traits and years ranging from XX (X) in year to XX (X) in year. Another study has shown measurement invariance for these scales across time and age groups (Schwaba \& Bleidorn, 2018). The Big Five (and life satisfaction) were contained in the \emph{Personality} module which was administered yearly but with planned missingness in some years for certain cohorts (see Denissen et al., 2019). Thus, there are one to two years between included assessments, given no other sources of missingness.\\
In the HRS, the Midlife Development Inventory (MIDI) scales were administered to measure the Big Five (Lachman \& Weaver, 1997). This scale was constructed for use in large-scale panel studies of adults and consisted of 26 adjectives (five each for conscientiousness, agreeableness, and extraversion, four for neuroticism, and seven for openness to experience). Participants were asked to rate on a 4-point scale how well each item described them (1 = \emph{a lot}, 2 = \emph{some}, 3 = \emph{a little}, 4 = \emph{not at all}). Example items included \enquote{Organized} (conscientiousness), \enquote{Sympathetic} (agreeableness), \enquote{Worrying} (neuroticism), \enquote{Imaginative} (openness to experience), and \enquote{Talkative} (extraversion). For better comparability with the LISS panel, we reverse scored all items so that higher values corresponded to higher trait levels and, at each wave, took the mean of each subscale as the trait score. Big Five trait scores showed satisfactory internal consistencies which averaged XX over all traits and years ranging from XX (X) in year to XX (X) in year.

\hypertarget{life-satisfaction}{%
\subsubsection{Life satisfaction}\label{life-satisfaction}}

In both samples, life satisfaction was assessed using the 5-item Satisfaction with Life Scale (SWLS; Diener et al., 1985) which participants answered on a 7-point Likert scale (1 = \emph{strongly disagree}, 2 = \emph{somewhat disagree}, 3 = \emph{slightly disagree}, 4 = \emph{neither agree or disagree}, 5 = \emph{slightly agree}, 6 = \emph{somewhat agree}, 7 = \emph{strongly agree})\footnote{In the LISS panel, the \enquote{somewhat} was omitted and instead of \enquote{or} \enquote{nor} was used.}. An example item was \enquote{I am satisfied with my life}. In the LISS panel, internal consistencies averaged XX over all years ranging from XX (X) in year to XX (X) in year. In the HRS, internal consistencies averaged XX over all years ranging from XX (X) in year to XX (X) in year.

\hypertarget{transition-to-grandparenthood}{%
\subsubsection{Transition to Grandparenthood}\label{transition-to-grandparenthood}}

The procedure to obtain information on grandparents' transition to grandparenthood generally followed the same steps in both samples. The items this coding was based on, however, differed slightly: In the LISS panel, participants were asked \enquote{Do you have children and/or grandchildren?} with \enquote{children}, \enquote{grandchildren}, and \enquote{no children or grandchildren} as possible answer categories. This question was part of the \emph{Work and Schooling} module and filtered to participants performing paid work. In the HRS, all participants were asked for the total number of grandchildren: \enquote{Altogether, how many grandchildren do you (or your husband / wife / partner, or your late husband / wife / partner) have? Include as grandchildren any children of your (or your {[}late{]} husband's / wife's / partner's) biological, step- or adopted children}.\footnote{The reference to step- or adopted children has been added since wave 2006.}\\
In both samples, we tracked grandparenthood status (0 = \emph{no grandchildren}, 1 = \emph{at least one grandchild}) over time. Due to longitudinally inconsistent data in some cases, we included in the grandparent group only participants with exactly one transition from 0 to 1 in this grandparenthood status variable, and no transitions back (see Fig. SX). We marked participants who continually indicated that they had no grandchildren as potential members of the control groups.

\hypertarget{covariates}{%
\subsubsection{Covariates}\label{covariates}}

For propensity score matching, we used a broad set of covariates (VanderWeele et al., 2020) covering participants' demographics (e.g., education), economic situation (e.g., income), and health (e.g., mobility difficulties). We also included the pre-transition outcome variables as covariates---as recommended in the literature (Cook et al., 2020; Hallberg et al., 2018; Steiner et al., 2010; VanderWeele et al., 2020), as well as the panel wave participation count and the assessment year in order to control for instrumentation effects and historical trends (e.g., 2008 financial crisis; Baird et al., 2010; Luhmann et al., 2014). For matching grandparents with the parent control group we additionally included as covariates variables related to fertility and family history (e.g., number of children, age of first three children) which were causally related to the timing of the transition to grandparenthood (i.e., entry into treatment; Arpino, Gumà, et al., 2018; Margolis \& Verdery, 2019).\\
Covariate selection has seldom been explicitly discussed in previous longitudinal studies estimating treatment effects of life events (e.g., through a matching design). We see two (in part conflicting) traditions that address covariate selection: First, classical recommendations from psychology argue to include all available variables that are to associated with both the treatment assignment process (i.e., selection into treatment) and the outcome (e.g., Steiner et al., 2010; Stuart, 2010). Second, recommendations from a structural causal modeling perspective (see Elwert \& Winship, 2014; Rohrer, 2018) are more cautious aiming to avoid pitfalls such as conditioning on a pre-treatment collider (collider bias) or a mediator (overcontrol bias). Structural causal modeling, however, requires advanced knowledge of the causal structures underlying all involved variables (Pearl, 2009).\\
In selecting covariates, we followed guidelines laid out by VanderWeele et al.~(2019; 2020) which reconcile both views and offer practical guidance when complete knowledge of the underlying causal structures is unknown: They propose a \enquote{modified disjunctive cause criterion} (VanderWeele, 2019, p. 218) recommending to select all available covariates which are assumed to be causes of the outcomes, treatment exposure (i.e., the transition to grandparenthood), or both, as well as any proxies for an unmeasured common cause of the outcomes and treatment exposure. To be excluded from this list are variables assumed to be instrumental variables (i.e., assumed causes of treatment exposure that are unrelated to the outcomes except through the exposure) and collider variables (Elwert \& Winship, 2014). Because all our covariates were measured at the time of matching (i.e, at least two years before the birth of the grandchild), we judge the risk of covariates introducing collider bias and overcontrol bias to be relatively small.\\
An overview of the variables we used to compute the propensity scores for matching can be found in the Supplemental Material, alongside justification for each covariate on whether we assume it to be causally related to treatment assignment, the outcomes, or both. Generally, we tried to find substantively equivalent covariates in both samples but had to compromise in a few cases (e.g., children's educational level only in HRS vs.~children living at home only in LISS).\\
Estimating propensity scores requires complete covariate data. Therefore, before computing propensity scores, we performed multiple imputations in order to account for missingness in our covariates (Greenland \& Finkle, 1995). Using five imputed data sets computed by classification and regression trees (CART; Burgette \& Reiter, 2010) in the \emph{mice} R package (van Buuren \& Groothuis-Oudshoorn, 2011), we predicted treatment assignment (i.e., the transition to grandparenthood) five times per observation in logistic regressions with a logit link function.\footnote{In these logistic regressions we included all covariates listed above as predictors except for \emph{female} which was later used for exact matching and health-related covariates in LISS-wave 2014 which altogether were not assessed in that wave.} We averaged these five scores to create the final propensity score to be used for matching (Mitra \& Reiter, 2016). We only used imputed data for propensity score computation and not in later analyses because missing data in the outcome variables due to nonresponse was negligible.

\hypertarget{moderators}{%
\subsubsection{Moderators}\label{moderators}}

Based on insights from previous research, we tested three variables as potential moderators of the mean-level trajectories of the Big Five and life satisfaction over the transition to grandparenthood: First, we analyzed whether gender acted as a moderator as indicated by research on life satisfaction (see Tanskanen et al., 2019; Di Gessa et al., 2019). We coded a dummy variable indicating female gender (0 = \emph{male}, 1 = \emph{female}). Second, we tested whether performing paid work or not was associated with divergent trajectories of the Big Five and life satisfaction (see Schwaba \& Bleidorn, 2019). Since the LISS subsample of grandparents we identified was based exclusively on participants performing paid work, we performed these analyses only in the HRS subsample. This served two purposes: first, to test how participants involved in the workforce (even if officially retired) differed from those not working, which might shed light on role conflict. Second, to assess whether potential differences in the main results between the LISS and HRS samples disappeared once we constrained the HRS sample in the same way that the LISS sample had already been constrained through filtering.\\
Third, we tested how the involvement in grandchild care affected trajectories of the Big Five and life satisfaction in grandparents after the transition to grandparenthood (see Arpino, Bordone, et al., 2018; Danielsbacka et al., 2019; Danielsbacka \& Tanskanen, 2016). We coded a dummy variable (0 = \emph{provided less than 100 hours of grandchild care}, 1 = \emph{provided 100 or more hours of grandchild care}) as a moderator based on the question \enquote{Did you (or your {[}late{]} husband / wife / partner) spend 100 or more hours in total since the last interview / in the last two years taking care of grand- or great grandchildren?}.\footnote{Although dichotomization of a continuous construct (hours of care) is not ideal for moderation analysis (MacCallum et al., 2002), there were too many missing values in the variable assessing hours of care directly (variables *E063).} This information was only available in the HRS; in the LISS panel only very few participants answered follow-up questions on intensity of care (\textgreater50 in the final analysis sample).

\hypertarget{procedure}{%
\subsection{Procedure}\label{procedure}}

Drawing on all available data, three main restrictions defined the final analysis samples of grandparents (see Fig. X for participant flowcharts): First, we identified participants who indicated having grandchildren for the first time during study participation (see \emph{Measures}; \(N_{LISS} =\) 337; \(N_{HRS} =\) 2982, including HRS waves 1996-2004 before personality assessments were introduced). Second, we restricted the sample to participants with at least one valid personality assessment (\(N_{LISS} =\) 335; \(N_{HRS} =\) 1577).\footnote{For the HRS subsample, we also excluded \(N =\) 30 grandparents in a previous step who reported unrealistically high numbers of grandchildren (\(>\) 10) in their first assessment following the transition to grandparenthood.} Third, we included in the analysis samples only participants with both a valid personality assessment before and one after the transition to grandparenthood (\(N_{LISS} =\) 253; \(N_{HRS} =\) 721). Lastly, few participants were excluded because of inconsistent or missing information regarding their children\footnote{We opted not to use multiple imputation for these child-related variables such as number of children which defined the control groups and were also later used for computing the propensity scores.} resulting the final analysis samples of first-time grandparents, \(N_{LISS} =\) 250 (XX\% female; age at transition to grandparenthood \(M =\) XX, \(SD =\) XX) and \(N_{HRS} =\) 712 (XX\% female; age at transition to grandparenthood \(M =\) XX, \(SD =\) XX).\\

To disentangle effects of the transition to grandparenthood from effects of being a parent, we defined two pools of potential control subjects to be involved in the matching procedure: The first pool of potential control subjects comprised parents who had at least one child in reproductive age (defined as \(15 \leq age_{firstborn}\leq65\)) but no grandchildren throughout the observation period (\(N_{LISS} =\) 844 with 3,040 longitudinal observations; \(N_{HRS} =\) 1,891 with 3,300 longitudinal observations). The second pool of potential matches comprised participants who reported being childless throughout the observation period (\(N_{LISS} =\) 1077 with 4,337 longitudinal observations; \(N_{HRS} =\) 1,577 with 2,357 longitudinal observations). The two control groups were, thus, by definition mutually exclusive.\\
In order to match each grandparent with a control participant who was most similar in terms of the included covariates we utilized propensity score matching. Propensity score matching of grandparents was performed in a grandparent's survey year which preceded the first wave after reporting the transition by at least two years. This served the purpose to ensure that the covariates used for matching were not affected by the event itself or its anticipation (i.e., when one's child was already pregnant with the grandchild; Greenland, 2003; Rosenbaum, 1984; VanderWeele et al., 2020). Propensity score matching was performed using the \emph{MatchIt} R package (Ho et al., 2011) with exact matching on gender combined with Mahalanobis distance matching on the propensity score. In total, four matchings were performed; two per sample (LISS; HRS) and two per control group (parents but not grandparents; nonparents). We matched 1:1 with replacement because of the relatively small pools of available non-grandparent controls. This meant that control observations were allowed to be used multiple times for matching (i.e., duplicated in the analysis samples\footnote{In the LISS data, 250 grandparent observations were matched with 250 control observations corresponding to 186 unique person-year observations stemming from 130 unique participants for the parent control group and to 174 unique person-year observations stemming from 107 unique participants for the nonparent control group. In the HRS data, 712 grandparent observations were matched with 712 control observations corresponding to 503 unique person-year observations stemming from 442 unique participants for the parent control group and to 418 unique person-year observations stemming from 350 unique participants for the nonparent control group.}). We did not specify a caliper because our goal was to find matches for all grandparents, and because we achieved satisfactory covariate balance this way.\\
We evaluated the matching procedure in terms of covariate balance and, graphically, in terms of overlap of the distributions of the propensity scores and (non-categorical) covariates (Stuart, 2010). Covariate balance as indicated by the standardized difference in means between the grandparent and the controls after matching was satisfactory (see Table X) lying below 0.25 as recommended in the literature (Stuart, 2010). Graphically, the differences between the distributions of the propensity score and the covariates were also small and indicated no missing overlap (see Fig. SX).\\
After matching, each matched control observation received the same value as their matched grandparent in the \emph{time} variable describing the temporal relation to treatment, and the control subject's other longitudinal observations were centered around this matched observation. Thereby, we coded a counterfactual transition time frame for each control subject. Due to left- and right censored longitudinal data (i.e., panel entry or attrition), we restricted the final analysis samples to six years before and six years after the transition as shown in Table X. We analyzed unbalanced panel data where not every participant provided all person-year observations. The final LISS analysis samples, thus, contained 250 grandparents with XXXX longitudinal observations, matched with 250 control subjects with either XXXX (parent control group) or XXXX lognitudinal observations (nonparent control group). The final HRS analysis samples contained 712 grandparents with XXXX longitudinal observations, matched with 250 control subjects with either XXXX (parent control group) or XXXX lognitudinal observations (nonparent control group).

\hypertarget{analytical-strategy}{%
\subsection{Analytical Strategy}\label{analytical-strategy}}

Our design can be referred to as an interrupted time-series with a \enquote{nonequivalent no-treatment control group} (Shadish et al., 2002, p. 182) where treatment, that is, the transition to grandparenthood, is not deliberately manipulated.\\
First, to analyze mean-level changes, we used linear piecewise regression coefficients in multilevel regression models with person-year observations nested within participants (Hoffman, 2015). To model change over time in relation to the birth of the first grandchild, we coded three piecewise regression coefficients: a \emph{before-slope} representing linear change in the years leading up to the transition to grandparenthood, an \emph{after-slope} representing linear change in the years after the transition, and a \emph{jump} coefficient shifting the intercept directly after the transition was first reported, thus representing sudden changes that go beyond changes already modeled by the \emph{after-slope} (see Table SX for the coding scheme of these coefficients). Similar piecewise growth-curve models have recently been adopted to study personality development (e.g., Bleidorn \& Schwaba, 2018; Krämer \& Rodgers, 2020; Schwaba \& Bleidorn, 2019; van Scheppingen \& Leopold, 2020).\\
All effects of the transition to grandparenthood on the Big Five and life satisfaction were modeled as deviations from patterns in the matched control groups by interacting the three piecewise coefficients with the binary treatment variable (0 = \emph{control subject}, 1 = \emph{grandparent}). In additional models, we interacted these coefficients with the binary gender variable (0 = \emph{male}, 1 = \emph{female}) resulting in three-way interactions that tested whether effects varied significantly by gender. To test differences in the growth parameters between two groups in cases where these differences were represented by multiple fixed-effects coefficients, we defined linear contrasts using the \enquote{linearHypothesis} command from the \emph{car} R package (Fox \& Weisberg, 2019). All models of mean-level changes were estimated using maximum likelihood and included random intercepts but no random slopes of the piecewise regression coefficients.\\
Second, to assess interindividual differences in intraindividual change in the Big Five and life satisfaction we added random slopes to the models assessing mean-level changes (see Denissen et al., 2019 for a similar approach). In other words, we allowed for differences between individuals in their trajectories of change to be modeled, that is, differences in the \emph{before-slope}, \emph{after-slope}, and \emph{jump} coefficients. Because multiple simultaneous random slopes are often not computationally feasible, we added random slopes one at a time and used likelihood ratio test to determine whether the addition of the respective random slope led to a significant improvement in model fit. We plotted distributions of random slopes (for a similar approach, see Denissen et al., 2019; Doré \& Bolger, 2018). To test differences in the random slopes between the grandparent group and the control groups, we ???.\\
Third, to examine rank-order stability in the Big Five and life satisfaction over the transition to grandparenthood, we computed the test-retest correlation of measurements prior to the transition to grandparenthood (at the time of matching) with the first available measurement after the transition. To test the difference in test-retest stability between grandparents and either of the control groups, we entered the pre-treatment measure as well as the treatment variable (0 = \emph{controls}, 1 = \emph{grandparents}) and their interaction into regression models predicting the Big Five and life satisfaction. The interaction tests for significant differences in the test-retest stability between those who experienced the transition to grandparenthood and those who did not (for a similar approach, see Denissen et al., 2019; McCrae, 1993).\\
We used R (Version 4.0.4; R Core Team, 2021) and the R-packages \emph{lme4} (Version 1.1.26; Bates et al., 2015), and \emph{lmerTest} (Version 3.1.3; Kuznetsova et al., 2017) for multilevel modeling, as well as \emph{tidyverse} (Wickham et al., 2019) for data wrangling, and \emph{papaja} (Aust \& Barth, 2020) for reproducible manuscript production. Additional modeling details and a list of all software we used is provided in the Supplemental Material. In line with Benjamin et al.~(n.d.), we set the \(\alpha\)-level for all confirmatory analyses to \(.005\).

\begin{enumerate}
\def\labelenumi{\arabic{enumi}.}
\setcounter{enumi}{1}
\tightlist
\item
  How large are interindividual differences in intraindividual change for the Big Five traits and life satisfaction over the transition to grandparenthood?
\end{enumerate}

\hypertarget{results}{%
\section{Results}\label{results}}

\hypertarget{discussion}{%
\section{Discussion}\label{discussion}}

Based on

\begin{itemize}
\item
  personality maturation cross-culturally: (Bleidorn et al., 2013; Chopik \& Kitayama, 2018)
\item
  facets / nuances (Mõttus \& Rozgonjuk, 2021)
\item
  arrival of grandchild associated with retirement decisions (Lumsdaine \& Vermeer, 2015); pers X WB interaction over retirement (Henning et al., 2017);
\item
  Does the Transition to Grandparenthood Deter Gray Divorce? A Test of the Braking Hypothesis (Brown et al., 2021)
\item
  prolonged period of grandparenthood? (Margolis \& Wright, 2017)
\item
  subjective experience of aging (Bordone \& Arpino, 2015)
\item
  policy relevance of personality (Bleidorn et al., 2019), e.g., health outcomes (Turiano et al., 2012), but not really evidence for healthy neuroticism (Turiano et al., 2020)
\end{itemize}

\hypertarget{limitations}{%
\subsection{Limitations}\label{limitations}}

Despite

\hypertarget{conclusions}{%
\subsection{Conclusions}\label{conclusions}}

Our

\hypertarget{acknowledgements}{%
\subsection{Acknowledgements}\label{acknowledgements}}

We thank X for valuable feedback.

\newpage

\hypertarget{references}{%
\section{References}\label{references}}

\begingroup
\setlength{\parindent}{-0.5in}
\setlength{\leftskip}{0.5in}

\hypertarget{refs}{}
\leavevmode\hypertarget{ref-anusicStabilityChangePersonality2016}{}%
Anusic, I., \& Schimmack, U. (2016). Stability and change of personality traits, self-esteem, and well-being: Introducing the meta-analytic stability and change model of retest correlations. \emph{Journal of Personality and Social Psychology}, \emph{110}(5), 766--781. \url{https://doi.org/10.1037/pspp0000066}

\leavevmode\hypertarget{ref-anusicDoesPersonalityModerate2014}{}%
Anusic, I., Yap, S., \& Lucas, R. E. (2014a). Does personality moderate reaction and adaptation to major life events? Analysis of life satisfaction and affect in an Australian national sample. \emph{Journal of Research in Personality}, \emph{51}, 69--77. \url{https://doi.org/10.1016/j.jrp.2014.04.009}

\leavevmode\hypertarget{ref-anusicTestingSetpointTheory2014}{}%
Anusic, I., Yap, S., \& Lucas, R. E. (2014b). Testing set-point theory in a Swiss national sample: Reaction and adaptation to major life events. \emph{Social Indicators Research}, \emph{119}(3), 1265--1288. \url{https://doi.org/10.1007/s11205-013-0541-2}

\leavevmode\hypertarget{ref-ardeltStillStableAll2000}{}%
Ardelt, M. (2000). Still stable after all these years? Personality stability theory revisited. \emph{Social Psychology Quarterly}, \emph{63}(4), 392--405. \url{https://doi.org/10.2307/2695848}

\leavevmode\hypertarget{ref-arpinoGrandparentingEducationSubjective2018}{}%
Arpino, B., Bordone, V., \& Balbo, N. (2018). Grandparenting, education and subjective well-being of older Europeans. \emph{European Journal of Ageing}, \emph{15}(3), 251--263. \url{https://doi.org/10.1007/s10433-018-0467-2}

\leavevmode\hypertarget{ref-arpinoFamilyHistoriesDemography2018}{}%
Arpino, B., Gumà, J., \& Julià, A. (2018). Family histories and the demography of grandparenthood. \emph{Demographic Research}, \emph{39}(42), 1105--1150. \url{https://doi.org/10.4054/DemRes.2018.39.42}

\leavevmode\hypertarget{ref-asselmannTestingSocialInvestment2020}{}%
Asselmann, E., \& Specht, J. (2020). Testing the Social Investment Principle Around Childbirth: Little Evidence for Personality Maturation Before and After Becoming a Parent. \emph{European Journal of Personality}, \emph{n/a}(n/a). \url{https://doi.org/10.1002/per.2269}

\leavevmode\hypertarget{ref-atesDoesGrandchildCare2017}{}%
Ates, M. (2017). Does grandchild care influence grandparents' self-rated health? Evidence from a fixed effects approach. \emph{Social Science \& Medicine}, \emph{190}, 67--74. \url{https://doi.org/10.1016/j.socscimed.2017.08.021}

\leavevmode\hypertarget{ref-R-papaja}{}%
Aust, F., \& Barth, M. (2020). \emph{papaja: Prepare reproducible APA journal articles with R Markdown}. \url{https://github.com/crsh/papaja}

\leavevmode\hypertarget{ref-bairdLifeSatisfactionLifespan2010}{}%
Baird, B. M., Lucas, R. E., \& Donnellan, M. B. (2010). Life satisfaction across the lifespan: Findings from two nationally representative panel studies. \emph{Social Indicators Research}, \emph{99}(2), 183--203. \url{https://doi.org/10.1007/s11205-010-9584-9}

\leavevmode\hypertarget{ref-balboRoleFamilyOrientations2016}{}%
Balbo, N., \& Arpino, B. (2016). The role of family orientations in shaping the effect of fertility on subjective well-being: A propensity score matching approach. \emph{Demography}, \emph{53}(4), 955--978. \url{https://doi.org/10.1007/s13524-016-0480-z}

\leavevmode\hypertarget{ref-baltesLifeSpanTheory2006}{}%
Baltes, P. B., Lindenberger, U., \& Staudinger, U. M. (2006). Life Span Theory in Developmental Psychology. In R. M. Lerner \& W. Damon (Eds.), \emph{Handbook of child psychology: Theoretical models of human development} (pp. 569--664). John Wiley \& Sons Inc.

\leavevmode\hypertarget{ref-R-lme4}{}%
Bates, D., Mächler, M., Bolker, B., \& Walker, S. (2015). Fitting linear mixed-effects models using lme4. \emph{Journal of Statistical Software}, \emph{67}(1), 1--48. \url{https://doi.org/10.18637/jss.v067.i01}

\leavevmode\hypertarget{ref-bengtsonNuclearFamilyIncreasing2001}{}%
Bengtson, V. L. (2001). Beyond the Nuclear Family: The Increasing Importance of Multigenerational Bonds. \emph{Journal of Marriage and Family}, \emph{63}(1), 1--16. \url{https://doi.org/10.1111/j.1741-3737.2001.00001.x}

\leavevmode\hypertarget{ref-benjaminRedefineStatisticalSignificance2018}{}%
Benjamin, D. J., Berger, J. O., Clyde, M., Wolpert, R. L., Johnson, V. E., Johannesson, M., Dreber, A., Nosek, B. A., Wagenmakers, E. J., Berk, R., \& Brembs, B. (n.d.). Redefine statistical significance. \emph{Nature Human Behavior}, \emph{2}, 6--10. \url{https://doi.org/10.1038/s41562-017-0189-z}

\leavevmode\hypertarget{ref-bleidornPolicyRelevancePersonality2019}{}%
Bleidorn, W., Hill, P. L., Back, M. D., Denissen, J. J. A., Hennecke, M., Hopwood, C. J., Jokela, M., Kandler, C., Lucas, R. E., Luhmann, M., Orth, U., Wagner, J., Wrzus, C., Zimmermann, J., \& Roberts, B. W. (2019). The policy relevance of personality traits. \emph{American Psychologist}, \emph{74}(9), 1056--1067. \url{https://doi.org/10.1037/amp0000503}

\leavevmode\hypertarget{ref-bleidornLifeEventsPersonality2018}{}%
Bleidorn, W., Hopwood, C. J., \& Lucas, R. E. (2018). Life events and personality trait change. \emph{Journal of Personality}, \emph{86}(1), 83--96. \url{https://doi.org/10.1111/jopy.12286}

\leavevmode\hypertarget{ref-bleidornPersonalityMaturationWorld2013}{}%
Bleidorn, W., Klimstra, T. A., Denissen, J. J. A., Rentfrow, P. J., Potter, J., \& Gosling, S. D. (2013). Personality Maturation Around the World: A Cross-Cultural Examination of Social-Investment Theory. \emph{Psychological Science}, \emph{24}(12), 2530--2540. \url{https://doi.org/10.1177/0956797613498396}

\leavevmode\hypertarget{ref-bleidornRetirementAssociatedChange2018}{}%
Bleidorn, W., \& Schwaba, T. (2018). Retirement is associated with change in self-esteem. \emph{Psychology and Aging}, \emph{33}(4), 586--594. \url{https://doi.org/10.1037/pag0000253}

\leavevmode\hypertarget{ref-bleidornPersonalityDevelopmentEmerging2017}{}%
Bleidorn, W., \& Schwaba, T. (2017). Personality development in emerging adulthood. In J. Specht (Ed.), \emph{Personality Development Across the Lifespan} (pp. 39--51). Academic Press. \url{https://doi.org/10.1016/B978-0-12-804674-6.00004-1}

\leavevmode\hypertarget{ref-bordoneGrandchildrenInfluenceHow2015}{}%
Bordone, V., \& Arpino, B. (2015). Do Grandchildren Influence How Old You Feel? \emph{Journal of Aging and Health}, \emph{28}(6), 1055--1072. \url{https://doi.org/10.1177/0898264315618920}

\leavevmode\hypertarget{ref-brownDoesTransitionGrandparenthood2021}{}%
Brown, S. L., Lin, I.-F., \& Mellencamp, K. A. (2021). Does the Transition to Grandparenthood Deter Gray Divorce? A Test of the Braking Hypothesis. \emph{Social Forces}, \emph{99}(3), 1209--1232. \url{https://doi.org/10.1093/sf/soaa030}

\leavevmode\hypertarget{ref-bruderlFixedEffectsPanelRegression2015}{}%
Brüderl, J., \& Ludwig, V. (2015). \emph{Fixed-Effects Panel Regression} (H. Best \& C. Wolf, Eds.). SAGE.

\leavevmode\hypertarget{ref-burgetteMultipleImputationMissing2010}{}%
Burgette, L. F., \& Reiter, J. P. (2010). Multiple Imputation for Missing Data via Sequential Regression Trees. \emph{American Journal of Epidemiology}, \emph{172}(9), 1070--1076. \url{https://doi.org/10.1093/aje/kwq260}

\leavevmode\hypertarget{ref-caspiWhenIndividualDifferences1993}{}%
Caspi, A., \& Moffitt, T. E. (1993). When do individual differences matter? A paradoxical theory of personality coherence. \emph{Psychological Inquiry}, \emph{4}(4), 247--271. \url{https://doi.org/10.1207/s15327965pli0404_1}

\leavevmode\hypertarget{ref-chopikPersonalityChangeLife2018}{}%
Chopik, W. J., \& Kitayama, S. (2018). Personality change across the life span: Insights from a cross-cultural, longitudinal study. \emph{Journal of Personality}, \emph{86}(3), 508--521. \url{https://doi.org/10.1111/jopy.12332}

\leavevmode\hypertarget{ref-chungLongitudinalEffectsGrandchild2018}{}%
Chung, S., \& Park, A. (2018). The longitudinal effects of grandchild care on depressive symptoms and physical health of grandmothers in South Korea: A latent growth approach. \emph{Aging \& Mental Health}, \emph{22}(12), 1556--1563. \url{https://doi.org/10.1080/13607863.2017.1376312}

\leavevmode\hypertarget{ref-condonTransitionGrandparenthoodProspective2018}{}%
Condon, J., Luszcz, M., \& McKee, I. (2018). The transition to grandparenthood: A prospective study of mental health implications. \emph{Aging \& Mental Health}, \emph{22}(3), 336--343. \url{https://doi.org/10.1080/13607863.2016.1248897}

\leavevmode\hypertarget{ref-cookHowMuchBias2020}{}%
Cook, T. D., Zhu, N., Klein, A., Starkey, P., \& Thomas, J. (2020). How much bias results if a quasi-experimental design combines local comparison groups, a pretest outcome measure and other covariates?: A within study comparison of preschool effects. \emph{Psychological Methods}, \emph{Advance Online Publication}, 0. \url{https://doi.org/10.1037/met0000260}

\leavevmode\hypertarget{ref-costaPersonalityLifeSpan2019}{}%
Costa, P. T., McCrae, R. R., \& Löckenhoff, C. E. (2019). Personality Across the Life Span. \emph{Annual Review of Psychology}, \emph{70}(1), 423--448. \url{https://doi.org/10.1146/annurev-psych-010418-103244}

\leavevmode\hypertarget{ref-danielsbackaAssociationGrandparentalInvestment2016}{}%
Danielsbacka, M., \& Tanskanen, A. O. (2016). The association between grandparental investment and grandparents' happiness in Finland. \emph{Personal Relationships}, \emph{23}(4), 787--800. \url{https://doi.org/10.1111/pere.12160}

\leavevmode\hypertarget{ref-danielsbackaGrandparentalChildcareHealth2019}{}%
Danielsbacka, M., Tanskanen, A. O., Coall, D. A., \& Jokela, M. (2019). Grandparental childcare, health and well-being in Europe: A within-individual investigation of longitudinal data. \emph{Social Science \& Medicine}, \emph{230}, 194--203. \url{https://doi.org/10.1016/j.socscimed.2019.03.031}

\leavevmode\hypertarget{ref-denissenTransactionsLifeEvents2019}{}%
Denissen, J. J. A., Luhmann, M., Chung, J. M., \& Bleidorn, W. (2019). Transactions between life events and personality traits across the adult lifespan. \emph{Journal of Personality and Social Psychology}, \emph{116}(4), 612--633. \url{https://doi.org/10.1037/pspp0000196}

\leavevmode\hypertarget{ref-dienerSatisfactionLifeScale1985}{}%
Diener, E., Emmons, R. A., Larsen, R. J., \& Griffin, S. (1985). The Satisfaction With Life Scale. \emph{Journal of Personality Assessment}, \emph{49}(1), 71--75. \url{https://doi.org/10.1207/s15327752jpa4901_13}

\leavevmode\hypertarget{ref-digessaBecomingGrandparentIts2019}{}%
Di Gessa, G., Bordone, V., \& Arpino, B. (2019). Becoming a Grandparent and Its Effect on Well-Being: The Role of Order of Transitions, Time, and Gender. \emph{The Journals of Gerontology, Series B: Psychological Sciences and Social Sciences}, Advance Online Publication. \url{https://doi.org/10.1093/geronb/gbz135}

\leavevmode\hypertarget{ref-digessaHealthImpactIntensive2016}{}%
Di Gessa, G., Glaser, K., \& Tinker, A. (2016a). The Health Impact of Intensive and Nonintensive Grandchild Care in Europe: New Evidence From SHARE. \emph{The Journals of Gerontology, Series B: Psychological Sciences and Social Sciences}, \emph{71}(5), 867--879. \url{https://doi.org/10.1093/geronb/gbv055}

\leavevmode\hypertarget{ref-digessaImpactCaringGrandchildren2016}{}%
Di Gessa, G., Glaser, K., \& Tinker, A. (2016b). The impact of caring for grandchildren on the health of grandparents in Europe: A lifecourse approach. \emph{Social Science \& Medicine}, \emph{152}, 166--175. \url{https://doi.org/10.1016/j.socscimed.2016.01.041}

\leavevmode\hypertarget{ref-dorePopulationIndividuallevelChanges2018}{}%
Doré, B., \& Bolger, N. (2018). Population- and individual-level changes in life satisfaction surrounding major life stressors. \emph{Social Psychological and Personality Science}, \emph{9}(7), 875--884. \url{https://doi.org/10.1177/1948550617727589}

\leavevmode\hypertarget{ref-elwertEndogenousSelectionBias2014}{}%
Elwert, F., \& Winship, C. (2014). Endogenous Selection Bias: The Problem of Conditioning on a Collider Variable. \emph{Annual Review of Sociology}, \emph{40}(1), 31--53. \url{https://doi.org/10.1146/annurev-soc-071913-043455}

\leavevmode\hypertarget{ref-car2019}{}%
Fox, J., \& Weisberg, S. (2019). \emph{An R companion to applied regression} (Third). Sage.

\leavevmode\hypertarget{ref-goldbergDevelopmentMarkersBigFive1992}{}%
Goldberg, L. R. (1992). The development of markers for the Big-Five factor structure. \emph{Psychological Assessment}, \emph{4}(1), 26--42. \url{https://doi.org/10.1037/1040-3590.4.1.26}

\leavevmode\hypertarget{ref-greenlandQuantifyingBiasesCausal2003}{}%
Greenland, S. (2003). Quantifying biases in causal models: Classical confounding vs collider-stratification bias. \emph{Epidemiology}, \emph{14}(3), 300--306. \url{https://doi.org/10.1097/01.EDE.0000042804.12056.6C}

\leavevmode\hypertarget{ref-greenlandCriticalLookMethods1995}{}%
Greenland, S., \& Finkle, W. D. (1995). A Critical Look at Methods for Handling Missing Covariates in Epidemiologic Regression Analyses. \emph{American Journal of Epidemiology}, \emph{142}(12), 1255--1264. \url{https://doi.org/10.1093/oxfordjournals.aje.a117592}

\leavevmode\hypertarget{ref-hallbergPretestMeasuresStudy2018}{}%
Hallberg, K., Cook, T. D., Steiner, P. M., \& Clark, M. H. (2018). Pretest Measures of the Study Outcome and the Elimination of Selection Bias: Evidence from Three Within Study Comparisons. \emph{Prevention Science}, \emph{19}(3), 274--283. \url{https://doi.org/10.1007/s11121-016-0732-6}

\leavevmode\hypertarget{ref-henningRolePersonalitySubjective2017}{}%
Henning, G., Hansson, I., Berg, A. I., Lindwall, M., \& Johansson, B. (2017). The role of personality for subjective well-being in the retirement transition Comparing variable- and person-oriented models. \emph{Personality and Individual Differences}, \emph{116}, 385--392. \url{https://doi.org/10.1016/j.paid.2017.05.017}

\leavevmode\hypertarget{ref-MatchIt2011}{}%
Ho, D. E., Imai, K., King, G., \& Stuart, E. A. (2011). MatchIt: Nonparametric preprocessing for parametric causal inference. \emph{Journal of Statistical Software}, \emph{42}(8), 1--28.

\leavevmode\hypertarget{ref-hoffmanLongitudinalAnalysisModeling2015}{}%
Hoffman, L. (2015). \emph{Longitudinal analysis: Modeling within-person fluctuation and change}. Routledge/Taylor \& Francis Group.

\leavevmode\hypertarget{ref-huttemanDevelopmentalTasksFramework2014}{}%
Hutteman, R., Hennecke, M., Orth, U., Reitz, A. K., \& Specht, J. (2014). Developmental Tasks as a Framework to Study Personality Development in Adulthood and Old Age. \emph{European Journal of Personality}, \emph{28}(3), 267--278. \url{https://doi.org/10.1002/per.1959}

\leavevmode\hypertarget{ref-kandlerPatternsSourcesPersonality2015a}{}%
Kandler, C., Kornadt, A. E., Hagemeyer, B., \& Neyer, F. J. (2015). Patterns and sources of personality development in old age. \emph{Journal of Personality and Social Psychology}, \emph{109}(1), 175--191. \url{https://doi.org/10.1037/pspp0000028}

\leavevmode\hypertarget{ref-kramerImpactHavingChildren2020}{}%
Krämer, M. D., \& Rodgers, J. L. (2020). The impact of having children on domain-specific life satisfaction: A quasi-experimental longitudinal investigation using the Socio-Economic Panel (SOEP) data. \emph{Journal of Personality and Social Psychology}, \emph{119}(6), 1497--1514. \url{https://doi.org/10.1037/pspp0000279}

\leavevmode\hypertarget{ref-R-lmerTest}{}%
Kuznetsova, A., Brockhoff, P. B., \& Christensen, R. H. B. (2017). lmerTest package: Tests in linear mixed effects models. \emph{Journal of Statistical Software}, \emph{82}(13), 1--26. \url{https://doi.org/10.18637/jss.v082.i13}

\leavevmode\hypertarget{ref-lachmanMidlifeDevelopmentInventory1997}{}%
Lachman, M. E., \& Weaver, S. L. (1997). \emph{The Midlife Development Inventory (MIDI) personality scales: Scale construction and scoring}. Brandeis University.

\leavevmode\hypertarget{ref-leopoldDemographyGrandparenthoodInternational2015}{}%
Leopold, T., \& Skopek, J. (2015). The Demography of Grandparenthood: An International Profile. \emph{Social Forces}, \emph{94}(2), 801--832. \url{https://doi.org/10.1093/sf/sov066}

\leavevmode\hypertarget{ref-lodi-smithSocialInvestmentPersonality2007}{}%
Lodi-Smith, J., \& Roberts, B. W. (2007). Social Investment and Personality: A Meta-Analysis of the Relationship of Personality Traits to Investment in Work, Family, Religion, and Volunteerism. \emph{Personality and Social Psychology Review}, \emph{11}(1), 68--86. \url{https://doi.org/10.1177/1088868306294590}

\leavevmode\hypertarget{ref-lucasPersonalityDevelopmentLife2011}{}%
Lucas, R. E., \& Donnellan, M. B. (2011). Personality development across the life span: Longitudinal analyses with a national sample from Germany. \emph{Journal of Personality and Social Psychology}, \emph{101}(4), 847--861. \url{https://doi.org/10.1037/a0024298}

\leavevmode\hypertarget{ref-luhmannDimensionalTaxonomyPerceived2020}{}%
Luhmann, M., Fassbender, I., Alcock, M., \& Haehner, P. (2020). A dimensional taxonomy of perceived characteristics of major life events. \emph{Journal of Personality and Social Psychology}, No Pagination Specified--No Pagination Specified. \url{https://doi.org/10.1037/pspp0000291}

\leavevmode\hypertarget{ref-luhmannSubjectiveWellbeingAdaptation2012}{}%
Luhmann, M., Hofmann, W., Eid, M., \& Lucas, R. E. (2012). Subjective well-being and adaptation to life events: A meta-analysis. \emph{Journal of Personality and Social Psychology}, \emph{102}(3), 592--615. \url{https://doi.org/10.1037/a0025948}

\leavevmode\hypertarget{ref-luhmannStudyingChangesLife2014}{}%
Luhmann, M., Orth, U., Specht, J., Kandler, C., \& Lucas, R. E. (2014). Studying changes in life circumstances and personality: It's about time. \emph{European Journal of Personality}, \emph{28}(3), 256--266. \url{https://doi.org/10.1002/per.1951}

\leavevmode\hypertarget{ref-lumsdaineRetirementTimingWomen2015}{}%
Lumsdaine, R. L., \& Vermeer, S. J. C. (2015). Retirement timing of women and the role of care responsibilities for grandchildren. \emph{Demography}, \emph{52}(2), 433--454. \url{https://doi.org/10.1007/s13524-015-0382-5}

\leavevmode\hypertarget{ref-ludtkeRandomWalkUniversity2011}{}%
Lüdtke, O., Roberts, B. W., Trautwein, U., \& Nagy, G. (2011). A random walk down university avenue: Life paths, life events, and personality trait change at the transition to university life. \emph{Journal of Personality and Social Psychology}, \emph{101}(3), 620--637. \url{https://doi.org/10.1037/a0023743}

\leavevmode\hypertarget{ref-maccallumPracticeDichotomizationQuantitative2002}{}%
MacCallum, R. C., Zhang, S., Preacher, K. J., \& Rucker, D. D. (2002). On the practice of dichotomization of quantitative variables. \emph{Psychological Methods}, \emph{7}(1), 19--40. \url{https://doi.org/10.1037/1082-989X.7.1.19}

\leavevmode\hypertarget{ref-mahneGrandparenthoodSubjectiveWellBeing2014}{}%
Mahne, K., \& Huxhold, O. (2014). Grandparenthood and Subjective Well-Being: Moderating Effects of Educational Level. \emph{The Journals of Gerontology: Series B}, \emph{70}(5), 782--792. \url{https://doi.org/10.1093/geronb/gbu147}

\leavevmode\hypertarget{ref-margolisCohortPerspectiveDemography2019}{}%
Margolis, R., \& Verdery, A. M. (2019). A Cohort Perspective on the Demography of Grandparenthood: Past, Present, and Future Changes in Race and Sex Disparities in the United States. \emph{Demography}, \emph{56}(4), 1495--1518. \url{https://doi.org/10.1007/s13524-019-00795-1}

\leavevmode\hypertarget{ref-margolisHealthyGrandparenthoodHow2017}{}%
Margolis, R., \& Wright, L. (2017). Healthy Grandparenthood: How Long Is It, and How Has It Changed? \emph{Demography}, \emph{54}(6), 2073--2099. \url{https://doi.org/10.1007/s13524-017-0620-0}

\leavevmode\hypertarget{ref-marshMeasurementInvarianceBigfive2013}{}%
Marsh, H. W., Nagengast, B., \& Morin, A. J. S. (2013). Measurement invariance of big-five factors over the life span: ESEM tests of gender, age, plasticity, maturity, and la dolce vita effects. \emph{Developmental Psychology}, \emph{49}(6), 1194--1218. \url{https://doi.org/10.1037/a0026913}

\leavevmode\hypertarget{ref-mccraeModeratedAnalysesLongitudinal1993}{}%
McCrae, R. R. (1993). Moderated analyses of longitudinal personality stability. \emph{Journal of Personality and Social Psychology}, \emph{65}(3), 577--585. \url{https://doi.org/10.1037/0022-3514.65.3.577}

\leavevmode\hypertarget{ref-mcneishThanksCoefficientAlpha2018}{}%
McNeish, D. (2018). Thanks coefficient alpha, we'll take it from here. \emph{Psychological Methods}, \emph{23}(3), 412--433. \url{https://doi.org/10.1037/met0000144}

\leavevmode\hypertarget{ref-mcneishFixedEffectsModels2019}{}%
McNeish, D., \& Kelley, K. (2019). Fixed effects models versus mixed effects models for clustered data: Reviewing the approaches, disentangling the differences, and making recommendations. \emph{Psychological Methods}, \emph{24}(1), 20--35. \url{https://doi.org/10.1037/met0000182}

\leavevmode\hypertarget{ref-mitraComparisonTwoMethods2016}{}%
Mitra, R., \& Reiter, J. P. (2016). A comparison of two methods of estimating propensity scores after multiple imputation. \emph{Statistical Methods in Medical Research}, \emph{25}(1), 188--204. \url{https://doi.org/10.1177/0962280212445945}

\leavevmode\hypertarget{ref-mottusPersonalityTraitsOld2012}{}%
Mõttus, R., Johnson, W., \& Deary, I. J. (2012). Personality traits in old age: Measurement and rank-order stability and some mean-level change. \emph{Psychology and Aging}, \emph{27}(1), 243--249. \url{https://doi.org/10.1037/a0023690}

\leavevmode\hypertarget{ref-mottusDevelopmentDetailsAge2021}{}%
Mõttus, R., \& Rozgonjuk, D. (2021). Development is in the details: Age differences in the Big Five domains, facets, and nuances. \emph{Journal of Personality and Social Psychology}, \emph{120}(4), 1035--1048. \url{https://doi.org/10.1037/pspp0000276}

\leavevmode\hypertarget{ref-muellerPersonalityDevelopmentOld2016}{}%
Mueller, S., Wagner, J., Drewelies, J., Duezel, S., Eibich, P., Specht, J., Demuth, I., Steinhagen-Thiessen, E., Wagner, G. G., \& Gerstorf, D. (2016). Personality development in old age relates to physical health and cognitive performance: Evidence from the Berlin Aging Study II. \emph{Journal of Research in Personality}, \emph{65}, 94--108. \url{https://doi.org/10.1016/j.jrp.2016.08.007}

\leavevmode\hypertarget{ref-pearlCausalInferenceStatistics2009}{}%
Pearl, J. (2009). Causal inference in statistics: An overview. \emph{Statistics Surveys}, \emph{3}, 96--146. \url{https://doi.org/10.1214/09-SS057}

\leavevmode\hypertarget{ref-R-base}{}%
R Core Team. (2021). \emph{R: A language and environment for statistical computing}. R Foundation for Statistical Computing. \url{https://www.R-project.org/}

\leavevmode\hypertarget{ref-robertsRankorderConsistencyPersonality2000}{}%
Roberts, B. W., \& DelVecchio, W. F. (2000). The rank-order consistency of personality traits from childhood to old age: A quantitative review of longitudinal studies. \emph{Psychological Bulletin}, \emph{126}(1), 3--25. \url{https://doi.org/10.1037/0033-2909.126.1.3}

\leavevmode\hypertarget{ref-robertsPatternsMeanlevelChange2006a}{}%
Roberts, B. W., Walton, K. E., \& Viechtbauer, W. (2006). Patterns of mean-level change in personality traits across the life course: A meta-analysis of longitudinal studies. \emph{Psychological Bulletin}, \emph{132}, 1--25. \url{https://doi.org/10.1037/0033-2909.132.1.1}

\leavevmode\hypertarget{ref-robertsPersonalityDevelopmentContext2006}{}%
Roberts, B. W., \& Wood, D. (2006). Personality Development in the Context of the Neo-Socioanalytic Model of Personality. In D. K. Mroczek \& T. D. Little (Eds.), \emph{Handbook of Personality Development}. Routledge.

\leavevmode\hypertarget{ref-robertsEvaluatingFiveFactor2005}{}%
Roberts, B. W., Wood, D., \& Smith, J. L. (2005). Evaluating Five Factor Theory and social investment perspectives on personality trait development. \emph{Journal of Research in Personality}, \emph{39}(1), 166--184. \url{https://doi.org/10.1016/j.jrp.2004.08.002}

\leavevmode\hypertarget{ref-rohrerThinkingClearlyCorrelations2018}{}%
Rohrer, J. M. (2018). Thinking Clearly About Correlations and Causation: Graphical Causal Models for Observational Data. \emph{Advances in Methods and Practices in Psychological Science}, \emph{1}(1), 27--42. \url{https://doi.org/10.1177/2515245917745629}

\leavevmode\hypertarget{ref-rosenbaumConsquencesAdjustmentConcomitant1984}{}%
Rosenbaum, P. (1984). The consquences of adjustment for a concomitant variable that has been affected by the treatment. \emph{Journal of the Royal Statistical Society. Series A (General)}, \emph{147}(5), 656--666. \url{https://doi.org/10.2307/2981697}

\leavevmode\hypertarget{ref-scherpenzeelDataCollectionProbabilityBased2011}{}%
Scherpenzeel, A. (2011). Data Collection in a Probability-Based Internet Panel: How the LISS Panel Was Built and How It Can Be Used. \emph{Bulletin of Sociological Methodology/Bulletin de Méthodologie Sociologique}, \emph{109}(1), 56--61. \url{https://doi.org/10.1177/0759106310387713}

\leavevmode\hypertarget{ref-scherpenzeelTrueLongitudinalProbabilitybased2010}{}%
Scherpenzeel, A. C., \& Das, M. (2010). True'' longitudinal and probability-based internet panels: Evidence from the Netherlands. In M. Das, P. Ester, \& L. Kaczmirek (Eds.), \emph{Social and behavioral research and the internet: Advances in applied methods and research strategies} (pp. 77--104). Taylor \& Francis.

\leavevmode\hypertarget{ref-schwabaPersonalityTraitDevelopment2019}{}%
Schwaba, T., \& Bleidorn, W. (2019). Personality trait development across the transition to retirement. \emph{Journal of Personality and Social Psychology}, \emph{116}(4), 651--665. \url{https://doi.org/10.1037/pspp0000179}

\leavevmode\hypertarget{ref-schwabaIndividualDifferencesPersonality2018}{}%
Schwaba, T., \& Bleidorn, W. (2018). Individual differences in personality change across the adult life span. \emph{Journal of Personality}, \emph{86}(3), 450--464. \url{https://doi.org/10.1111/jopy.12327}

\leavevmode\hypertarget{ref-shadishExperimentalQuasiexperimentalDesigns2002}{}%
Shadish, W. R., Cook, T. D., \& Campbell, D. T. (2002). \emph{Experimental and quasi-experimental designs for generalized causal inference}. Houghton, Mifflin and Company.

\leavevmode\hypertarget{ref-sheppardBecomingFirstTimeGrandparent2019}{}%
Sheppard, P., \& Monden, C. (2019). Becoming a First-Time Grandparent and Subjective Well-Being: A Fixed Effects Approach. \emph{Journal of Marriage and Family}, \emph{81}(4), 1016--1026. \url{https://doi.org/10.1111/jomf.12584}

\leavevmode\hypertarget{ref-skopekWhoBecomesGrandparent2017}{}%
Skopek, J., \& Leopold, T. (2017). Who becomes a grandparent and when? Educational differences in the chances and timing of grandparenthood. \emph{Demographic Research}, \emph{37}(29), 917--928. \url{https://doi.org/10.4054/DemRes.2017.37.29}

\leavevmode\hypertarget{ref-sonnegaCohortProfileHealth2014}{}%
Sonnega, A., Faul, J. D., Ofstedal, M. B., Langa, K. M., Phillips, J. W., \& Weir, D. R. (2014). Cohort Profile: The Health and Retirement Study (HRS). \emph{International Journal of Epidemiology}, \emph{43}(2), 576--585. \url{https://doi.org/10.1093/ije/dyu067}

\leavevmode\hypertarget{ref-spechtPersonalityDevelopmentAdulthood2017}{}%
Specht, J. (2017). Personality development in adulthood and old age. In J. Specht (Ed.), \emph{Personality Development Across the Lifespan} (pp. 53--67). Academic Press. \url{https://doi.org/10.1016/B978-0-12-804674-6.00005-3}

\leavevmode\hypertarget{ref-spechtWhatDrivesAdult2014}{}%
Specht, J., Bleidorn, W., Denissen, J. J. A., Hennecke, M., Hutteman, R., Kandler, C., Luhmann, M., Orth, U., Reitz, A. K., \& Zimmermann, J. (2014). What Drives Adult Personality Development? A Comparison of Theoretical Perspectives and Empirical Evidence. \emph{European Journal of Personality}, \emph{28}(3), 216--230. \url{https://doi.org/10.1002/per.1966}

\leavevmode\hypertarget{ref-spechtStabilityChangePersonality2011}{}%
Specht, J., Egloff, B., \& Schmukle, S. C. (2011). Stability and change of personality across the life course: The impact of age and major life events on mean-level and rank-order stability of the Big Five. \emph{Journal of Personality and Social Psychology}, \emph{101}(4), 862--882. \url{https://doi.org/10.1037/a0024950}

\leavevmode\hypertarget{ref-steinerImportanceCovariateSelection2010}{}%
Steiner, P., Cook, T., Shadish, W., \& Clark, M. (2010). The Importance of Covariate Selection in Controlling for Selection Bias in Observational Studies. \emph{Psychological Methods}, \emph{15}, 250--267. \url{https://doi.org/10.1037/a0018719}

\leavevmode\hypertarget{ref-stephanPhysicalActivityPersonality2014}{}%
Stephan, Y., Sutin, A. R., \& Terracciano, A. (2014). Physical activity and personality development across adulthood and old age: Evidence from two longitudinal studies. \emph{Journal of Research in Personality}, \emph{49}, 1--7. \url{https://doi.org/10.1016/j.jrp.2013.12.003}

\leavevmode\hypertarget{ref-stuartMatchingMethodsCausal2010}{}%
Stuart, E. A. (2010). Matching methods for causal inference: A review and a look forward. \emph{Statistical Science: A Review Journal of the Institute of Mathematical Statistics}, \emph{25}(1), 1--21. \url{https://doi.org/10.1214/09-STS313}

\leavevmode\hypertarget{ref-tanskanenTransitionGrandparenthoodSubjective2019}{}%
Tanskanen, A. O., Danielsbacka, M., Coall, D. A., \& Jokela, M. (2019). Transition to Grandparenthood and Subjective Well-Being in Older Europeans: A Within-Person Investigation Using Longitudinal Data. \emph{Evolutionary Psychology}, \emph{17}(3), 1474704919875948. \url{https://doi.org/10.1177/1474704919875948}

\leavevmode\hypertarget{ref-triadoGrandparentsWhoProvide2014}{}%
Triadó, C., Villar, F., Celdrán, M., \& Solé, C. (2014). Grandparents Who Provide Auxiliary Care for Their Grandchildren: Satisfaction, Difficulties, and Impact on Their Health and Well-being. \emph{Journal of Intergenerational Relationships}, \emph{12}(2), 113--127. \url{https://doi.org/10.1080/15350770.2014.901102}

\leavevmode\hypertarget{ref-turianoHealthyNeuroticismAssociated2020}{}%
Turiano, N. A., Graham, E. K., Weston, S. J., Booth, T., Harrison, F., James, B. D., Lewis, N. A., Makkar, S. R., Mueller, S., Wisniewski, K. M., Zhaoyang, R., Spiro, A., Willis, S., Schaie, K. W., Lipton, R. B., Katz, M., Sliwinski, M., Deary, I. J., Zelinski, E. M., \ldots{} Mroczek, D. K. (2020). Is Healthy Neuroticism Associated with Longevity? A Coordinated Integrative Data Analysis. \emph{Collabra: Psychology}, \emph{6}(33). \url{https://doi.org/10.1525/collabra.268}

\leavevmode\hypertarget{ref-turianoPersonalityTraitLevel2012}{}%
Turiano, N. A., Pitzer, L., Armour, C., Karlamangla, A., Ryff, C. D., \& Mroczek, D. K. (2012). Personality Trait Level and Change as Predictors of Health Outcomes: Findings From a National Study of Americans (MIDUS). \emph{The Journals of Gerontology: Series B}, \emph{67B}(1), 4--12. \url{https://doi.org/10.1093/geronb/gbr072}

\leavevmode\hypertarget{ref-mice2011}{}%
van Buuren, S., \& Groothuis-Oudshoorn, K. (2011). mice: Multivariate imputation by chained equations in r. \emph{Journal of Statistical Software}, \emph{45}(3), 1--67.

\leavevmode\hypertarget{ref-vanderlaanRepresentativityLISSPanel2009}{}%
van der Laan, J. (2009). \emph{Representativity of the LISS panel (Discussion Paper 09041)}. Statistics Netherlands.

\leavevmode\hypertarget{ref-vanderweelePrinciplesConfounderSelection2019}{}%
VanderWeele, T. J. (2019). Principles of confounder selection. \emph{European Journal of Epidemiology}, \emph{34}(3), 211--219. \url{https://doi.org/10.1007/s10654-019-00494-6}

\leavevmode\hypertarget{ref-vanderweeleOutcomeWideLongitudinalDesigns2020}{}%
VanderWeele, T. J., Mathur, M. B., \& Chen, Y. (2020). Outcome-Wide Longitudinal Designs for Causal Inference: A New Template for Empirical Studies. \emph{Statistical Science}, \emph{35}(3), 437--466. \url{https://doi.org/10.1214/19-STS728}

\leavevmode\hypertarget{ref-vanscheppingenPersonalityTraitDevelopment2016}{}%
van Scheppingen, M. A., Jackson, J. J., Specht, J., Hutteman, R., Denissen, J. J. A., \& Bleidorn, W. (2016). Personality Trait Development During the Transition to Parenthood: A Test of Social Investment Theory. \emph{Social Psychological and Personality Science}, \emph{7}(5), 452--462. \url{https://doi.org/10.1177/1948550616630032}

\leavevmode\hypertarget{ref-vanscheppingenTrajectoriesLifeSatisfaction2020}{}%
van Scheppingen, M. A., \& Leopold, T. (2020). Trajectories of life satisfaction before, upon, and after divorce: Evidence from a new matching approach. \emph{Journal of Personality and Social Psychology}, \emph{119}(6), 1444--1458. \url{https://doi.org/10.1037/pspp0000270}

\leavevmode\hypertarget{ref-wagnerFirstPartnershipExperience2015}{}%
Wagner, J., Becker, M., Lüdtke, O., \& Trautwein, U. (2015). The First Partnership Experience and Personality Development: A Propensity Score Matching Study in Young Adulthood. \emph{Social Psychological and Personality Science}, \emph{6}(4), 455--463. \url{https://doi.org/10.1177/1948550614566092}

\leavevmode\hypertarget{ref-wagnerIntegrativeModelSources2020}{}%
Wagner, J., Orth, U., Bleidorn, W., Hopwood, C. J., \& Kandler, C. (2020). Toward an Integrative Model of Sources of Personality Stability and Change. \emph{Current Directions in Psychological Science}, \emph{29}(5), 438--444. \url{https://doi.org/10.1177/0963721420924751}

\leavevmode\hypertarget{ref-wagnerPersonalityTraitDevelopment2016}{}%
Wagner, J., Ram, N., Smith, J., \& Gerstorf, D. (2016). Personality trait development at the end of life: Antecedents and correlates of mean-level trajectories. \emph{Journal of Personality and Social Psychology}, \emph{111}(3), 411--429. \url{https://doi.org/10.1037/pspp0000071}

\leavevmode\hypertarget{ref-tidyverse2019}{}%
Wickham, H., Averick, M., Bryan, J., Chang, W., McGowan, L. D., François, R., Grolemund, G., Hayes, A., Henry, L., Hester, J., Kuhn, M., Pedersen, T. L., Miller, E., Bache, S. M., Müller, K., Ooms, J., Robinson, D., Seidel, D. P., Spinu, V., \ldots{} Yutani, H. (2019). Welcome to the tidyverse. \emph{Journal of Open Source Software}, \emph{4}(43), 1686. \url{https://doi.org/10.21105/joss.01686}

\leavevmode\hypertarget{ref-wortmanStabilityChangeBig2012}{}%
Wortman, J., Lucas, R. E., \& Donnellan, M. B. (2012). Stability and change in the Big Five personality domains: Evidence from a longitudinal study of Australians. \emph{Psychology and Aging}, \emph{27}(4), 867--874. \url{https://doi.org/10.1037/a0029322}

\leavevmode\hypertarget{ref-yapDoesPersonalityModerate2012}{}%
Yap, S., Anusic, I., \& Lucas, R. E. (2012). Does personality moderate reaction and adaptation to major life events? Evidence from the British Household Panel Survey. \emph{Journal of Research in Personality}, \emph{46}(5), 477--488. \url{https://doi.org/10.1016/j.jrp.2012.05.005}

\endgroup


\end{document}
