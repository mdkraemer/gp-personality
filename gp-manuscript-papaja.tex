% Options for packages loaded elsewhere
\PassOptionsToPackage{unicode}{hyperref}
\PassOptionsToPackage{hyphens}{url}
%
\documentclass[
  english,
  man, noextraspace]{apa7}
\usepackage{lmodern}
\usepackage{amssymb,amsmath}
\usepackage{ifxetex,ifluatex}
\ifnum 0\ifxetex 1\fi\ifluatex 1\fi=0 % if pdftex
  \usepackage[T1]{fontenc}
  \usepackage[utf8]{inputenc}
  \usepackage{textcomp} % provide euro and other symbols
\else % if luatex or xetex
  \usepackage{unicode-math}
  \defaultfontfeatures{Scale=MatchLowercase}
  \defaultfontfeatures[\rmfamily]{Ligatures=TeX,Scale=1}
\fi
% Use upquote if available, for straight quotes in verbatim environments
\IfFileExists{upquote.sty}{\usepackage{upquote}}{}
\IfFileExists{microtype.sty}{% use microtype if available
  \usepackage[]{microtype}
  \UseMicrotypeSet[protrusion]{basicmath} % disable protrusion for tt fonts
}{}
\makeatletter
\@ifundefined{KOMAClassName}{% if non-KOMA class
  \IfFileExists{parskip.sty}{%
    \usepackage{parskip}
  }{% else
    \setlength{\parindent}{0pt}
    \setlength{\parskip}{6pt plus 2pt minus 1pt}}
}{% if KOMA class
  \KOMAoptions{parskip=half}}
\makeatother
\usepackage{xcolor}
\IfFileExists{xurl.sty}{\usepackage{xurl}}{} % add URL line breaks if available
\IfFileExists{bookmark.sty}{\usepackage{bookmark}}{\usepackage{hyperref}}
\hypersetup{
  pdftitle={The Transition to Grandparenthood: No Consistent Evidence for Change in the Big Five Personality Traits and Life Satisfaction},
  pdfauthor={Michael D. Krämer1,2, Manon A. van Scheppingen3, William J. Chopik4, \& David Richter1,4},
  pdflang={en-EN},
  pdfkeywords={grandparenthood, Big Five, life satisfaction, development, propensity score matching},
  hidelinks,
  pdfcreator={LaTeX via pandoc}}
\urlstyle{same} % disable monospaced font for URLs
\usepackage{graphicx,grffile}
\makeatletter
\def\maxwidth{\ifdim\Gin@nat@width>\linewidth\linewidth\else\Gin@nat@width\fi}
\def\maxheight{\ifdim\Gin@nat@height>\textheight\textheight\else\Gin@nat@height\fi}
\makeatother
% Scale images if necessary, so that they will not overflow the page
% margins by default, and it is still possible to overwrite the defaults
% using explicit options in \includegraphics[width, height, ...]{}
\setkeys{Gin}{width=\maxwidth,height=\maxheight,keepaspectratio}
% Set default figure placement to htbp
\makeatletter
\def\fps@figure{htbp}
\makeatother
\setlength{\emergencystretch}{3em} % prevent overfull lines
\providecommand{\tightlist}{%
  \setlength{\itemsep}{0pt}\setlength{\parskip}{0pt}}
\setcounter{secnumdepth}{-\maxdimen} % remove section numbering
% Make \paragraph and \subparagraph free-standing
\ifx\paragraph\undefined\else
  \let\oldparagraph\paragraph
  \renewcommand{\paragraph}[1]{\oldparagraph{#1}\mbox{}}
\fi
\ifx\subparagraph\undefined\else
  \let\oldsubparagraph\subparagraph
  \renewcommand{\subparagraph}[1]{\oldsubparagraph{#1}\mbox{}}
\fi
% Manuscript styling
\usepackage{upgreek}
\captionsetup{font=singlespacing,justification=justified}

% Table formatting
\usepackage{longtable}
\usepackage{lscape}
% \usepackage[counterclockwise]{rotating}   % Landscape page setup for large tables
\usepackage{multirow}		% Table styling
\usepackage{tabularx}		% Control Column width
\usepackage[flushleft]{threeparttable}	% Allows for three part tables with a specified notes section
\usepackage{threeparttablex}            % Lets threeparttable work with longtable

% Create new environments so endfloat can handle them
% \newenvironment{ltable}
%   {\begin{landscape}\begin{center}\begin{threeparttable}}
%   {\end{threeparttable}\end{center}\end{landscape}}
\newenvironment{lltable}{\begin{landscape}\begin{center}\begin{ThreePartTable}}{\end{ThreePartTable}\end{center}\end{landscape}}

% Enables adjusting longtable caption width to table width
% Solution found at http://golatex.de/longtable-mit-caption-so-breit-wie-die-tabelle-t15767.html
\makeatletter
\newcommand\LastLTentrywidth{1em}
\newlength\longtablewidth
\setlength{\longtablewidth}{1in}
\newcommand{\getlongtablewidth}{\begingroup \ifcsname LT@\roman{LT@tables}\endcsname \global\longtablewidth=0pt \renewcommand{\LT@entry}[2]{\global\advance\longtablewidth by ##2\relax\gdef\LastLTentrywidth{##2}}\@nameuse{LT@\roman{LT@tables}} \fi \endgroup}

% \setlength{\parindent}{0.5in}
% \setlength{\parskip}{0pt plus 0pt minus 0pt}

% Overwrite redefinition of paragraph and subparagraph by the default LaTeX template
% See https://github.com/crsh/papaja/issues/292
\makeatletter
\renewcommand{\paragraph}{\@startsection{paragraph}{4}{\parindent}%
  {0\baselineskip \@plus 0.2ex \@minus 0.2ex}%
  {-1em}%
  {\normalfont\normalsize\bfseries\itshape\typesectitle}}

\renewcommand{\subparagraph}[1]{\@startsection{subparagraph}{5}{1em}%
  {0\baselineskip \@plus 0.2ex \@minus 0.2ex}%
  {-\z@\relax}%
  {\normalfont\normalsize\itshape\hspace{\parindent}{#1}\textit{\addperi}}{\relax}}
\makeatother

% \usepackage{etoolbox}
\makeatletter
\patchcmd{\HyOrg@maketitle}
  {\section{\normalfont\normalsize\abstractname}}
  {\section*{\normalfont\normalsize\abstractname}}
  {}{\typeout{Failed to patch abstract.}}
\patchcmd{\HyOrg@maketitle}
  {\section{\protect\normalfont{\@title}}}
  {\section*{\protect\normalfont{\@title}}}
  {}{\typeout{Failed to patch title.}}
\makeatother
\shorttitle{Grandparenthood, Big Five, and Life Satisfaction}
\keywords{grandparenthood, Big Five, life satisfaction, development, propensity score matching\newline\indent Word count: abc}
\DeclareDelayedFloatFlavor{ThreePartTable}{table}
\DeclareDelayedFloatFlavor{lltable}{table}
\DeclareDelayedFloatFlavor*{longtable}{table}
\makeatletter
\renewcommand{\efloat@iwrite}[1]{\immediate\expandafter\protected@write\csname efloat@post#1\endcsname{}}
\makeatother
\usepackage{lineno}

\linenumbers
\usepackage{csquotes}
\usepackage{setspace}
\usepackage{amsmath}
\AtBeginEnvironment{tabular}{\singlespacing}
\AtBeginEnvironment{lltable}{\singlespacing}
\AtBeginEnvironment{tablenotes}{\doublespacing}
\captionsetup[table]{font={stretch=1.5}}
\captionsetup[figure]{font={stretch=1.5}}
\renewcommand\author[1]{}
\renewcommand\affiliation[1]{}
\authorsnames[{1,2}, 3, 4, {1,5}]{Michael D. Krämer, Manon A. van Scheppingen, William J. Chopik, David Richter}
\authorsaffiliations{{German Institute for Economic Research, Germany}, {International Max Planck Research School on the Life Course (LIFE), Germany}, {Tilburg University, Netherlands}, {Michigan State University, USA}, {Freie Universität Berlin, Germany}}
\ifxetex
  % Load polyglossia as late as possible: uses bidi with RTL langages (e.g. Hebrew, Arabic)
  \usepackage{polyglossia}
  \setmainlanguage[]{english}
\else
  \usepackage[shorthands=off,main=english]{babel}
\fi

\title{The Transition to Grandparenthood: No Consistent Evidence for Change in the Big Five Personality Traits and Life Satisfaction}
\author{Michael D. Krämer\textsuperscript{1,2}, Manon A. van Scheppingen\textsuperscript{3}, William J. Chopik\textsuperscript{4}, \& David Richter\textsuperscript{1,4}}
\date{}


\note{\clearpage}

\authornote{

\addORCIDlink{Michael D. Krämer}{0000-0002-9883-5676}, Socio-Economic Panel (SOEP), German Institute for Economic Research (DIW Berlin); International Max Planck Research School on the Life Course (LIFE), Max Planck Institute for Human Development\\
\addORCIDlink{Manon A. van Scheppingen}{0000-0003-0133-2069}, Department of Developmental Psychology, Tilburg School of Social and Behavioral Sciences, Tilburg University\\
\addORCIDlink{William J. Chopik}{0000-0003-1748-8738}, Department of Psychology, Michigan State University\\
\addORCIDlink{David Richter}{0000-0003-2811-8652}, Socio-Economic Panel (SOEP), German Institute for Economic Research (DIW Berlin); Survey Research Division, Department of Education and Psychology, Freie Universität Berlin

The authors made the following contributions. Michael D. Krämer: Conceptualization, Data Curation, Formal Analysis, Methodology, Visualization, Writing - Original Draft Preparation, Writing - Review \& Editing; Manon A. van Scheppingen: Methodology, Writing - Review \& Editing; William J. Chopik: Methodology, Writing - Review \& Editing; David Richter: Supervision, Methodology, Writing - Review \& Editing.

Correspondence concerning this article should be addressed to Michael D. Krämer, German Institute for Economic Research, Mohrenstr. 58, 10117 Berlin, Germany. E-mail: \href{mailto:mkraemer@diw.de}{\nolinkurl{mkraemer@diw.de}}

}

\affiliation{\vspace{0.5cm}\textsuperscript{1} German Institute for Economic Research, Germany\\\textsuperscript{2} International Max Planck Research School on the Life Course (LIFE), Germany\\\textsuperscript{3} Tilburg University, Netherlands\\\textsuperscript{4} Michigan State University, USA\\\textsuperscript{5} Freie Universität Berlin, Germany}

\abstract{
Intergenerational relations have received increased attention amidst an aging demographic and increased childcare responsibilities taken on by grandparents. However, few studies have investigated the psychological consequences of becoming a grandparent. For the Big Five personality traits, the transition to grandparenthood has been proposed as a developmental task in middle adulthood and old age contributing to personality development through this new role adoption---in line with the social investment principle. In this preregistered study, we used nationally representative panel data from the Netherlands (\emph{N} = 250) and the USA (\emph{N} = 846) to analyze first-time grandparents' development of the Big Five and life satisfaction in terms of mean-level changes, interindividual differences in change, and rank-order stability. We tested gender, paid work, and grandchild care as moderators of change trajectories. To address confounding bias, we employed propensity score matching in two procedures: matching grandparents with parents as well as nonparents in order to achieve balance in different sets of carefully selected covariates. Longitudinal multilevel models demonstrated mostly stability of the Big Five and life satisfaction over the transition to grandparenthood, and no consistent moderation effects. A few small effects that suggested personality development did not replicate across analysis samples. Contrary to expectations, we also found no consistent evidence for larger interindividual differences in change in the grandparents, or for smaller rank-order stability compared to the controls. Our findings add to recent failed tests of the social investment principle and are discussed in light of characteristics specific to grandparenthood that might moderate personality development.
}



\begin{document}
\maketitle

Becoming a grandparent is an important life event for many people in midlife or old age (Infurna et al., 2020). At the same time, there is considerable heterogeneity in how intensely grandparents are involved in their grandchildren's lives and care (Meyer \& Kandic, 2017). In the context of an aging demographic, the time that grandparents are alive and in good health during grandparenthood is prolonged compared to previous generations (Bengtson, 2001; Leopold \& Skopek, 2015; Margolis \& Wright, 2017). In addition, an increased share of childcare functions are being fulfilled by grandparents (Hayslip et al., 2019; Pilkauskas et al., 2020). Thus, intergenerational relations have received heightened attention from psychological and sociological research in recent years (Bengtson, 2001; Coall \& Hertwig, 2011; Fingerman et al., 2020). With regard to personality development, the transition to grandparenthood has been posited as an important developmental task in old age (Hutteman et al., 2014). However, empirical research into the psychological consequences of becoming a grandparent is sparse. Testing hypotheses derived from neo-socioanalytic theory (Roberts \& Wood, 2006) in a prospective matched control-group design (see Luhmann et al., 2014), we investigate whether the transition to grandparenthood affects the Big Five personality traits and life satisfaction using data from two nationally representative panel studies.

\hypertarget{personality-development-in-middle-adulthood-and-old-age}{%
\subsection{Personality Development in Middle Adulthood and Old Age}\label{personality-development-in-middle-adulthood-and-old-age}}

The life span perspective characterizes aging as a lifelong process of development and adaptation (Baltes et al., 2006). In accordance with this perspective, research has found personality traits to be subject to change throughout the entire life span (Costa et al., 2019; Graham et al., 2020; Specht, 2017; Specht et al., 2014; for recent reviews, see Bleidorn et al., 2021; Roberts \& Yoon, 2021). Although a major portion of personality development takes place in adolescence and emerging adulthood (Bleidorn \& Schwaba, 2017; Pusch et al., 2019; Schwaba \& Bleidorn, 2018), evidence has accumulated that personality traits also undergo changes in middle and old adulthood (e.g., Allemand et al., 2008; Damian et al., 2019; Kandler et al., 2015; Lucas \& Donnellan, 2011; Mõttus et al., 2012; Mueller et al., 2016; Seifert et al., 2021; Wagner et al., 2016; for a review, see Specht, 2017).\\
Here, we examine the Big Five personality traits---agreeableness, conscientiousness, extraversion, neuroticism, and openness to experience---which constitute a broad categorization of universal patterns of thought, affect, and behavior (John et al., 2008; John \& Srivastava, 1999). Changes over time in the Big Five occur both in mean trait levels (i.e., mean-level change; Roberts et al., 2006) and in the relative ordering of people to each other on trait dimensions (i.e., rank-order stability; Anusic \& Schimmack, 2016; Roberts \& DelVecchio, 2000). No observed changes in mean trait levels do not necessarily mean that individual trait levels are stable over time, and perfect rank-order stability does not preclude mean-level changes. Mean-level changes in early to middle adulthood (ca. 30--60 years old; Hutteman et al., 2014) are typically characterized in terms of greater maturity as evidenced by increased agreeableness and conscientiousness, and decreased neuroticism (Damian et al., 2019; Roberts et al., 2006). In old age (ca. 60 years and older; Hutteman et al., 2014), research is generally more sparse but there is some evidence for a \emph{reversal} of the maturity effect following retirement (sometimes termed \emph{la dolce vita} effect; Asselmann \& Specht, 2021; Marsh et al., 2013; cf.~Schwaba \& Bleidorn, 2019) and at the end of life in ill health (Wagner et al., 2016).\\
In terms of rank-order stability, most prior studies have shown support for an inverted U-shape trajectory (Ardelt, 2000; Lucas \& Donnellan, 2011; Seifert et al., 2021; Specht et al., 2011; Wortman et al., 2012): Rank-order stability rises until reaching a plateau in midlife, and decreases, again, in old age. However, evidence is mixed whether rank-order stability actually decreases again in old age (see Costa et al., 2019; Wagner et al., 2019). Nonetheless, the historical view that personality is stable, or \enquote{set like plaster} (Specht, 2017, p. 64) after one reaches adulthood (or leaves emerging adulthood behind; Bleidorn \& Schwaba, 2017) can largely be abandoned (Specht et al., 2014).\\
Theories explaining the mechanisms of personality development in middle adulthood and old age emphasize both genetic influences and life experiences as interdependent sources of stability and change (Bleidorn et al., 2021; Specht et al., 2014; Wagner et al., 2020). We conceptualize the transition to grandparenthood as a life experience offering the adoption of a new social role according to the social investment principle of neo-socioanalytic theory (Lodi-Smith \& Roberts, 2007; Roberts \& Wood, 2006). The social investment principle states that normative life events or transitions such as entering the work force or becoming a parent lead to personality maturation through the adoption of new social roles (Roberts et al., 2005). These new roles encourage or compel people to act in a more agreeable, conscientious, and emotionally stable (i.e., less neurotic) way, and the experiences in these roles as well as societal expectations towards them are hypothesized to drive long-term personality development (Lodi-Smith \& Roberts, 2007; Wrzus \& Roberts, 2017). Conversely, consistent social roles foster personality stability.\\
The paradoxical theory of personality coherence (Caspi \& Moffitt, 1993) offers a complimentary perspective for personality development through role transitions: trait change is supposedly more likely whenever people transition into unknown environments where pre-existing behavioral responses are no longer appropriate and social expectations give clear indications how to behave instead. On the other hand, stability is favored in environments where no clear guidance for how to behave is available. Thus, the finding that age-graded, normative life experiences, such as the transition to grandparenthood, drive personality development would also be in line with the paradoxical theory of personality coherence (see Specht et al., 2014).\\
Empirically, certain life events such as the first romantic relationship (Wagner et al., 2015) or the transition from high school to university or the first job (Asselmann \& Specht, 2021; Golle et al., 2019; Lüdtke et al., 2011) have been found to co-occur with mean-level increases (partly) consistent with the social investment principle (for a review, see Bleidorn et al., 2018). However, recent evidence regarding the transition to parenthood failed to support the social investment principle (Asselmann \& Specht, 2020b; van Scheppingen et al., 2016). An analysis of trajectories of the Big Five before and after eight life events only found limited support for the social investment principle: small increases were found in emotional stability following the transition to employment but not for the other traits or for the other life events theoretically linked to social investment (Denissen et al., 2019).\\
Overall, much remains unknown regarding the environmental factors underlying personality development in middle adulthood and old age. One indication that age-graded, normative life experiences contribute to change following a period of relative stability in midlife is offered by recent research on retirement (Bleidorn \& Schwaba, 2018; Schwaba \& Bleidorn, 2019). These results were only partly in line with the social investment principle in terms of mean-level changes and displayed substantial interindividual differences in change trajectories. Schwaba and Bleidorn discuss that as a social role \enquote{divestment} (2019, p. 660) retirement functions differently compared to social \emph{investment} which adds a role (another paper introduced the term personality \emph{relaxation} in this context, see Asselmann \& Specht, 2021). Grandparenthood could represent such an investment into a new role in middle adulthood and old age---given that grandparents have regular contact with their grandchild and actively take part in childcare to some degree (i.e., invest psychologically in the new role; Lodi-Smith \& Roberts, 2007).

\hypertarget{grandparenthood}{%
\subsection{Grandparenthood}\label{grandparenthood}}

The transition to grandparenthood, that is, the birth of the first grandchild, can be described as a time-discrete life event marking the beginning of one's status as a grandparent (Luhmann et al., 2012). In terms of characteristics of major life events (Luhmann et al., 2020), the transition to grandparenthood stands out in that it is externally caused (by one's own children; see also Arpino, Gumà, et al., 2018; Margolis \& Verdery, 2019), while at the same time being predictable as soon as one's children reveal their family planning or pregnancy. The transition to grandparenthood has been labeled a countertransition due to this lack of direct control over if and when someone has their first grandchild (Hagestad \& Neugarten, 1985; as cited in Arpino, Gumà, et al., 2018). Grandparenthood is also generally positive in valence and emotionally significant---given one maintains a good relationship with their child.\\
Grandparenthood can be characterized as a developmental task (Hutteman et al., 2014) in (early) old age---although the age at the transition to grandparenthood varies considerably both within and between cultures (Leopold \& Skopek, 2015; Skopek \& Leopold, 2017). Still, the period where parents on average experience the birth of their first grandchild coincides with the end of (relative) personality stability in midlife (Specht, 2017), where retirement, shifting social roles, and initial cognitive and health declines can be disruptive to life circumstances putting personality development into motion (e.g., Mueller et al., 2016; Stephan et al., 2014). As a developmental task, grandparenthood is expected to be part of a normative sequence of aging that is subject to societal expectations and values differing across cultures and historical time (Baltes et al., 2006; Hutteman et al., 2014). Mastering developmental tasks (i.e., fulfilling roles and expectations to a high degree) is hypothesized to drive personality development towards maturation similarly to propositions by the social investment principle, that is, leading to higher levels of agreeableness and conscientiousness, and lower levels of neuroticism (Roberts et al., 2005; Roberts \& Wood, 2006). Grandparental investment in their grandchildren has been discussed as beneficial in terms of the evolutionary, economic, and sociological advantages it provides for the whole intergenerational family structure (Coall et al., 2018; Coall \& Hertwig, 2011).\\
In comparison to the transition to parenthood which has been found to be ambivalent in terms of both personality maturation and life satisfaction (Aassve et al., 2021; Johnson \& Rodgers, 2006; Krämer \& Rodgers, 2020; van Scheppingen et al., 2016), Hutteman et al.~(2014) hypothesize that the transition to grandparenthood is generally seen as positive because it (usually) does not impose the stressful demands of daily childcare on grandparents. However, societal expectations on how grandparents should behave (e.g., \enquote{Grandparents should help parents with childcare if needed}) are less clearly defined compared to parenthood, and strongly depend on the degree of (possible) grandparental investment (Lodi-Smith \& Roberts, 2007). Thus, societal expectations and role demands differ depending on how close grandparents live to their children, the quality of the relationship with their children, and sociodemographic factors that exert conflicting role demands (Bordone et al., 2017; Lumsdaine \& Vermeer, 2015; Silverstein \& Marenco, 2001; cf.~Muller \& Litwin, 2011). In the whole population of first-time grandparents this diversity of role investment might generate pronounced interindividual differences in intraindividual personality change.\\
While we could not find prior studies investigating development of the Big Five over the transition to grandparenthood, there is some evidence for changes in life satisfaction across the transition to grandparenthood. In cross-sectional studies, grandparents who provide grandchild care or have close relationships with their older grandchildren often have higher life satisfaction (e.g., Mahne \& Huxhold, 2014; Triadó et al., 2014). There are a few longitudinal studies, albeit they offer conflicting conclusions: Data from the Survey of Health, Ageing and Retirement in Europe (SHARE) showed that the birth of a grandchild was followed by improvements in quality of life and life satisfaction, but only among women (Tanskanen et al., 2019) and only in first-time grandmothers via their daughters (Di Gessa et al., 2019). Several studies emphasized that grandparents actively involved in childcare experienced larger increases in life satisfaction (Arpino, Bordone, et al., 2018; Danielsbacka et al., 2019; Danielsbacka \& Tanskanen, 2016). On the other hand, fixed effects regression models\footnote{Fixed effects regression models exclusively rely on within-person variance (see Brüderl \& Ludwig, 2015; McNeish \& Kelley, 2019).} using SHARE data did not find any effects of first-time grandparenthood on life satisfaction regardless of grandparental investment and only minor decreases of grandmothers' depressive symptoms (Sheppard \& Monden, 2019).\\
In a similar vein, some prospective studies reported beneficial effects of the transition to grandparenthood and of grandparental childcare investment on various health measures, especially in women (Chung \& Park, 2018; Condon et al., 2018; Di Gessa et al., 2016a, 2016b). Again, the beneficial effects of grandparenthood on self-rated health did not persevere in fixed effects analyses, such as those reported in Ates's (2017) analyses of longitudinal data from the German Aging Survey (DEAS).\\
We are not aware of any study investigating trait rank-order stability over the transition to grandparenthood. The occurrence of other life events has been shown to be associated with rank-order stability of personality and well-being, although only for certain events and traits (e.g., Denissen et al., 2019; Hentschel et al., 2017; Specht et al., 2011). Altogether, prior evidence is lacking for the Big Five and inconclusive for life satisfaction (and related measures) which might be due to different methodological approaches that did not always account for confounding (i.e., selection effects).

\hypertarget{methodological-considerations}{%
\subsection{Methodological Considerations}\label{methodological-considerations}}

Effects of life events on psychological traits generally tend to be small and need to be properly analyzed using robust, prospective designs, and appropriate control groups (Bleidorn et al., 2018; Luhmann et al., 2014). This is necessary because pre-existing differences between prospective grandparents and non-grandparents in variables related to the development of the Big Five or life satisfaction introduce confounding bias when estimating the effects of an event such as the transition to grandparenthood (VanderWeele et al., 2020). The impact of adjusting (or not adjusting) for pre-existing differences, or background characteristics, has recently been emphasized in the prediction of life outcomes from personality in a mega-analytic framework of ten large panel studies (Beck \& Jackson, 2021). Propensity score matching is one technique to account for confounding bias by equating groups in their estimated propensity to experience the event (Thoemmes \& Kim, 2011). This propensity is calculated from regressing the so-called treatment variable (indicating whether someone experienced the event) on covariates related to the likelihood of experiencing the event and to the outcomes. This approach addresses confounding bias by creating balance between the groups in the covariates used to calculate the propensity score (Stuart, 2010).\\
We adopt a prospective design that tests the effects of becoming first-time grandparents separately against two propensity-score-matched control groups: first, parents (but not grandparents) with at least one child in reproductive age, and, second, nonparents. Adopting two control groups allows us to disentangle potential effects attributable to becoming a grandparent from effects attributable to being a parent already (i.e., parents who eventually become grandparents might share additional similarity with parents who do not). Thus, we are able to address selection effects into grandparenthood more comprehensively than previous research and we cover the first two of three causal pathways to not experiencing grandparenthood pointed out by demographic research (Margolis \& Verdery, 2019): one's own childlessness, childlessness of one's children, and not living long enough to become a grandparent. Our comparative design controls for average age-related and historical trends in the Big Five traits and life satisfaction (Luhmann et al., 2014). The design also enables us to report effects of the transition to grandparenthood unconfounded by instrumentation effects, which describe the tendency of reporting lower well-being scores with each repeated measurement (Baird et al., 2010).\\
We improve upon previous longitudinal studies using matched control groups (e.g., Anusic et al., 2014a, 2014b; Yap et al., 2012) by matching at a specific time point preceding the transition to grandparenthood (i.e., at least two years beforehand) and not based on individual survey years. This design choice ensures that the covariates involved in the matching procedure are not already influenced by the event or anticipation of it (Greenland, 2003; Rosenbaum, 1984; VanderWeele, 2019; VanderWeele et al., 2020), thereby reducing the risk of introducing confounding through collider bias (Elwert \& Winship, 2014). Similar approaches in the study of life events have recently been adopted (Balbo \& Arpino, 2016; Krämer \& Rodgers, 2020; van Scheppingen \& Leopold, 2020).

\hypertarget{current-study}{%
\subsection{Current Study}\label{current-study}}

In the current study, we examine development of the Big Five personality traits across the transition to grandparenthood in a prospective, quasi-experimental design, thereby extending previous research on effects of this transition on well-being to psychological development in a more general sense. We also revisit the development of life satisfaction which we define as the general, cognitive appraisal of one's well-being in life based on subjective criteria (Eid \& Larsen, 2008). Three research questions motivate the current study which---to our knowledge---is the first to analyze Big Five personality development over the transition to grandparenthood:

\begin{enumerate}
\def\labelenumi{\arabic{enumi}.}
\tightlist
\item
  What are the effects of the transition to grandparenthood on mean-level trajectories of the Big Five traits and life satisfaction?
\item
  How large are interindividual differences in intraindividual change for the Big Five traits and life satisfaction over the transition to grandparenthood?
\item
  How does the transition to grandparenthood affect rank-order stability of the Big Five traits and life satisfaction?
\end{enumerate}

To address these questions, we used two nationally representative panel data sets and compared grandparents' development over the transition to grandparenthood with that of matched respondents who did not become grandparents during the study period (Luhmann et al., 2014). Informed by the social investment principle and previous research on personality development in middle adulthood and old age, we preregistered the following hypotheses (see blinded file \emph{Preregistration.pdf} on \url{https://osf.io/75a4r/?view_only=ac929a2c41fb4afd9d1a64a3909848d0/}):

\begin{itemize}
\tightlist
\item
  H1a: Following the birth of their first grandchild, grandparents increase in agreeableness and conscientiousness, and decrease in neuroticism compared to the matched control groups of parents (but not grandparents) and nonparents. We do not expect the groups to differ in their trajectories of extraversion and openness to experience.
\item
  H1b: Grandparents' post-transition increases in agreeableness and conscientiousness, and decreases in neuroticism are more pronounced among those who provide substantial grandchild care.
\item
  H1c: Grandmothers increase in life satisfaction following the transition to grandparenthood as compared to the matched control groups but grandfathers do not.
\item
  H2: Individual differences in intraindividual change in the Big Five and life satisfaction are larger in the grandparent group than the control groups.
\item
  H3: Compared to the matched control groups, grandparents' rank-order stability of the Big Five and life satisfaction over the transition to grandparenthood is smaller.
\end{itemize}

Finally, commitments to other institutions necessarily constrain the amount of possible grandparental investment. Thus, exploratorily, we probe the moderator \emph{performing paid work} which could constitute a potential role conflict among grandparents.

\hypertarget{methods}{%
\section{Methods}\label{methods}}

\hypertarget{samples}{%
\subsection{Samples}\label{samples}}

To evaluate these hypotheses, we used data from two population-representative panel studies: the Longitudinal Internet Studies for the Social Sciences (LISS) panel from the Netherlands and the Health and Retirement Study (HRS) from the United States.\\
The LISS panel is a representative sample of the Dutch population initiated in 2008 with data collection still ongoing (Scherpenzeel, 2011; van der Laan, 2009). It is administered by Centerdata (Tilburg University, The Netherlands). Included households are a true probability sample of households drawn from the population register (Scherpenzeel \& Das, 2010). While roughly half of invited households consented to participate, refreshment samples were drawn in order to oversample previously underrepresented groups using information about response rates and their association with demographic variables (see \url{https://www.lissdata.nl/about-panel/sample-and-recruitment/}). Data collection was carried out online and respondents lacking the necessary technical equipment were outfitted with it. We included yearly assessments from 2008 to 2020 as well as basic demographics assessed monthly. For later coding of covariates from these monthly demographic data we used the first available assessment in each year.\\
The HRS is an ongoing population-representative study of older adults in the US (Sonnega et al., 2014) administered by the Survey Research Center (University of Michigan, United States). Initiated in 1992 with a first cohort of individuals aged 51-61 and their spouses, the study has since been extended with additional cohorts (see \url{https://hrs.isr.umich.edu/documentation/survey-design/}). In addition to the biennial in-person or telephone interview, the study has since 2006 included a leave-behind questionnaire covering psychosocial topics including the Big Five personality traits and life satisfaction. These topics, however, were only administered every four years starting in 2006 for one half of the sample and in 2008 for the other half. We included personality data from 2006 to 2018, all available data for the coding of the transition to grandparenthood from 1996 to 2018, as well as covariate data from 2006 to 2018 including variables drawn from the Imputations File and the Family Data (only available up to 2014).\\
These two panel studies provided the advantage that they contained several waves of personality data as well as information on grandparent status and a broad range of covariates. While the HRS provided a large sample with a wider age range, the LISS was smaller and younger but provided more frequent personality assessments spaced every one to two years. Included grandparents from the LISS were younger because grandparenthood questions were part of the \emph{Work and Schooling} module and---for reasons unknown to us---filtered to respondents performing paid work. Thus, older, retired first-time grandparents from the LISS could not be identified.
Even though we have published using the LISS and HRS data before (see preregistration, \url{https://osf.io/75a4r/?view_only=ac929a2c41fb4afd9d1a64a3909848d0/}), these publications do not overlap with the current study in the focus of grandparenthood.\footnote{Publications using LISS data can be searched at \url{https://www.dataarchive.lissdata.nl/publications/}. Publications using HRS data can be searched at \url{https://hrs.isr.umich.edu/publications/biblio/}.} The present study used de-identified archival data in the public domain, and, thus, it was not necessary to obtain ethical approval from an IRB.

\hypertarget{measures}{%
\subsection{Measures}\label{measures}}

\hypertarget{personality}{%
\subsubsection{Personality}\label{personality}}

In the LISS, the Big Five personality traits were assessed using the 50-item version of the IPIP Big-Five Inventory scales (Goldberg, 1992). For each trait, ten 5-point Likert-scale items were answered (1 = \emph{very inaccurate}, 2 = \emph{moderately inaccurate}, 3 = \emph{neither inaccurate nor accurate}, 4 = \emph{moderately accurate}, 5 = \emph{very accurate}). Example items included \enquote{Like order} (conscientiousness), \enquote{Sympathize with others' feelings} (agreeableness), \enquote{Worry about things} (neuroticism), \enquote{Have a vivid imagination} (openness to experience), and \enquote{Start conversations} (extraversion). At each wave, we took a respondent's mean of each subscale as their trait score. Internal consistencies at the time of matching, as indicated by McDonald's \(\omega\) (McNeish, 2018), averaged \(\omega =\) 0.83 over all traits ranging from \(\omega =\) 0.77 (conscientiousness in the parent control group) to \(\omega =\) 0.90 (extraversion in the nonparent control group). Other studies have shown measurement invariance for these scales across time and age groups, and convergent validity with the Big Five Inventory (BFI-2) (Denissen et al., 2020; Schwaba \& Bleidorn, 2018). The Big Five and life satisfaction were administered yearly but with planned missingness in some years for certain cohorts (see Denissen et al., 2019). \\
In the HRS, the Midlife Development Inventory (MIDI) scales measured the Big Five (Lachman \& Weaver, 1997) consisting of 26 adjectives (five each for conscientiousness, agreeableness, and extraversion, four for neuroticism, and seven for openness to experience). Respondents were asked to rate on a 4-point scale how well each item described them (1 = \emph{a lot}, 2 = \emph{some}, 3 = \emph{a little}, 4 = \emph{not at all}). Example adjectives included \enquote{Organized} (conscientiousness), \enquote{Sympathetic} (agreeableness), \enquote{Worrying} (neuroticism), \enquote{Imaginative} (openness to experience), and \enquote{Talkative} (extraversion). For better comparability with the LISS panel, we reverse scored all items so that higher values corresponded to higher trait levels and, at each wave, took the mean of each subscale as the trait score. Big Five trait scores showed satisfactory internal consistencies at the time of matching which averaged \(\omega =\) 0.75 over all traits ranging from \(\omega =\) 0.68 (conscientiousness in the nonparent control group) to \(\omega =\) 0.81 (agreeableness in the nonparent control group).

\hypertarget{life-satisfaction}{%
\subsubsection{Life Satisfaction}\label{life-satisfaction}}

In both samples, life satisfaction was assessed using the 5-item Satisfaction with Life Scale (SWLS; Diener et al., 1985) which respondents answered on a 7-point Likert scale (1 = \emph{strongly disagree}, 2 = \emph{somewhat disagree}, 3 = \emph{slightly disagree}, 4 = \emph{neither agree or disagree}, 5 = \emph{slightly agree}, 6 = \emph{somewhat agree}, 7 = \emph{strongly agree})\footnote{In the LISS, the \enquote{somewhat} was omitted and instead of \enquote{or}, \enquote{nor} was used.}. An example item was \enquote{I am satisfied with my life}. Internal consistency at the time of matching was \(\omega =\) 0.90 in the LISS with the parent control sample (\(\omega =\) 0.88 with the nonparent control sample), and \(\omega =\) 0.91 in the HRS with the parent control sample (\(\omega =\) 0.91 with the nonparent control sample).

\hypertarget{transition-to-grandparenthood}{%
\subsubsection{Transition to Grandparenthood}\label{transition-to-grandparenthood}}

The procedure to obtain information on the transition to grandparenthood generally followed the same steps in both samples. The items this coding was based on, however, differed slightly: In the LISS, respondents performing paid work were asked \enquote{Do you have children and/or grandchildren?} with \enquote{children}, \enquote{grandchildren}, and \enquote{no children or grandchildren} as possible answer categories. In the HRS, all respondents were asked for the total number of grandchildren: \enquote{Altogether, how many grandchildren do you (or your husband / wife / partner, or your late husband / wife / partner) have? Include as grandchildren any children of your (or your {[}late{]} husband's / wife's / partner's) biological, step- or adopted children}.\footnote{The listing of biological, step-, or adopted children has been added since wave 2006.}\\
In both samples, we tracked grandparenthood status over time. Due to longitudinally inconsistent data in some cases, we included in the grandparent group only respondents with one transition from 0 (\emph{no grandchildren}) to 1 (\emph{at least one grandchild}) in this status variable, and no transitions backwards (see Figure \ref{fig:flowchart-participant}). We marked respondents who consistently indicated that they had no grandchildren as potential members of the control groups.

\hypertarget{moderators}{%
\subsubsection{Moderators}\label{moderators}}

Based on insights from previous research, we tested three variables as potential moderators of the mean-level trajectories of the Big Five and life satisfaction over the transition to grandparenthood: First, we analyzed whether female gender (0 = \emph{male}, 1 = \emph{female}) acted as a moderator as indicated by research on life satisfaction (Di Gessa et al., 2019; Tanskanen et al., 2019).\\
Second, we tested whether performing paid work (0 = \emph{no}, 1 = \emph{yes}) was associated with divergent trajectories of the Big Five and life satisfaction (Schwaba \& Bleidorn, 2019). Since the LISS subsample exclusively comprised respondents performing paid work, we performed these analyses only in the HRS. This served two purposes: to test how respondents involved in the workforce differed from those not working, which might shed light on role conflict and have implications for social investment mechanisms. These moderation analyses also allowed us to assess whether potential differences in results between the LISS and HRS samples could be accounted for by including performing paid work as a moderator in HRS analyses. In other words, perhaps the results in the HRS respondents performing paid work are similar to those seen in the LISS sample, which had already been conditioned on this variable through filtering in the questionnaire.\\
Third, we examined how involvement in grandchild care moderated trajectories of the Big Five and life satisfaction (Arpino, Bordone, et al., 2018; Danielsbacka et al., 2019; Danielsbacka \& Tanskanen, 2016). We coded a moderator variable (0 = \emph{provided less than 100 hours of grandchild care}, 1 = \emph{provided 100 or more hours of grandchild care}) based on the question \enquote{Did you (or your {[}late{]} husband / wife / partner) spend 100 or more hours in total since the last interview / in the last two years taking care of grand- or great grandchildren?}.\footnote{Dichotomization of a continuous construct (hours of care) is not ideal for moderation analysis (MacCallum et al., 2002). However, there were too many missing values in the variable assessing hours of care continuously (variables *E063).} This information was only available for grandparents in the HRS; in the LISS, too few respondents answered respective follow-up questions to be included in analyses.

\hypertarget{procedure}{%
\subsection{Procedure}\label{procedure}}

Drawing on all available data, three main restrictions defined the final analysis samples of grandparents (see Figure \ref{fig:flowchart-participant}): First, we identified respondents who indicated having grandchildren for the first time during study participation (\(N_{LISS} =\) 337; \(N_{HRS} =\) 3272, including HRS waves 1996-2004 before personality assessments were introduced). Second, we restricted the sample to respondents with at least one valid personality assessment (valid in the sense that at least one of the six outcomes was non-missing; \(N_{LISS} =\) 335; \(N_{HRS} =\) 1702).\footnote{We also excluded \(N =\) 30 HRS grandparents in a previous step who reported unrealistically high numbers of grandchildren (\(>\) 10) in their first assessment following the transition to grandparenthood.} Third, we included only respondents with both a valid personality assessment before and one after the transition to grandparenthood (\(N_{LISS} =\) 253; \(N_{HRS} =\) 859). Lastly, few respondents were excluded because of inconsistent or missing information regarding their children resulting in the final analysis samples of first-time grandparents, \(N_{LISS} =\) 250 (53.60\(\%\) female; age at transition to grandparenthood \(M =\) 57.94, \(SD =\) 4.87) and \(N_{HRS} =\) 846 (54.85\(\%\) female; age at transition to grandparenthood \(M =\) 61.80, \(SD =\) 6.88).\\
We defined two pools of potential control subjects to be involved in the matching procedure: The first comprised parents who had at least one child in reproductive age (defined as \(15 \leq age_{firstborn}\leq65\)) but no grandchildren throughout the observation period (\(N_{LISS} =\) 844 with 3040 longitudinal observations; \(N_{HRS} =\) 1485 with 2703 longitudinal observations). The second comprised respondents who reported being childless throughout the observation period (\(N_{LISS} =\) 1077 with 4337 longitudinal observations; \(N_{HRS} =\) 1340 with 2346 longitudinal observations). The two control groups were, thus, by definition mutually exclusive.

\hypertarget{covariates}{%
\subsubsection{Covariates}\label{covariates}}

In order to match each grandparent with the control respondent from each pool of potential controls who was most similar in terms of the included covariates we utilized propensity score matching.\\
Although critical to the design, covariate selection has seldom been explicitly discussed in studies estimating effects of life events (e.g., in matching designs). We see two (in part conflicting) traditions that address covariate selection: First, classical recommendations from psychology argue to include all available variables that are associated with both the treatment assignment process (i.e., selection into treatment) and the outcome (e.g., Steiner et al., 2010; Stuart, 2010). Second, recommendations from a structural causal modeling perspective (Elwert \& Winship, 2014; Rohrer, 2018) are more cautious aiming to avoid pitfalls such as conditioning on a pre-treatment collider (collider bias) or a mediator (overcontrol bias). Structural causal modeling, however, requires advanced knowledge of the causal structures underlying the involved variables (Pearl, 2009).\\
In selecting covariates, we followed guidelines by VanderWeele et al.~(2019; 2020) which reconcile both views and offer practical guidance when complete knowledge of the underlying causal structures is unknown, and when using large archival datasets. The \enquote{modified disjunctive cause criterion} (VanderWeele, 2019, p. 218) recommends to select all available covariates which are assumed to be causes of the outcomes, treatment exposure (i.e., the transition to grandparenthood), or both, as well as any proxies for an unmeasured common cause of the outcomes and treatment exposure. To be excluded from this selection are variables assumed to be instrumental variables (i.e., assumed causes of treatment exposure that are unrelated to the outcomes except through the exposure) and collider variables (Elwert \& Winship, 2014). Because all covariates we used for matching were measured at least two years before the birth of the grandchild, we judge the risk of introducing collider bias or overcontrol bias to be relatively small. In addition, as mentioned in the \emph{Introduction}, the event transition to grandparenthood is not planned by or under direct control of grandparents which further reduces the risk of these biases.\\
Following these guidelines, we selected covariates covering respondents' demographics (e.g., age, education), economic situation (e.g., income), and health (e.g., mobility difficulties). We also included the pre-transition outcome variables as covariates---as recommended in the literature (Cook et al., 2020; Hallberg et al., 2018; Steiner et al., 2010; VanderWeele et al., 2020), as well as wave participation count and assessment year in order to control for instrumentation effects and historical trends (e.g., 2008/2009 financial crisis; Baird et al., 2010; Luhmann et al., 2014). For matching grandparents with the parent control group we additionally selected covariates containing information on fertility and family history (e.g., number of children, age of first three children) which were causally related to the timing of the transition to grandparenthood (Arpino, Gumà, et al., 2018; Margolis \& Verdery, 2019).\\
An overview of all covariates we used to compute the propensity scores can be found in the supplemental materials (see Tables \ref{tab:stddiffmeans-balance-liss} \& \ref{tab:stddiffmeans-balance-hrs}). Importantly, as part of our preregistration we also provided justification for each covariate on whether we assume it to be related to treatment assignment, the outcomes, or both (see \emph{gp-covariates-overview.xlsx} on \url{https://osf.io/75a4r/?view_only=ac929a2c41fb4afd9d1a64a3909848d0}). We tried to find substantively equivalent covariates in both samples but had to compromise in a few cases (e.g., children's educational level only in HRS vs.~children living at home only in LISS).\\
Estimating propensity scores required complete covariate data. Therefore, we performed multiple imputations in order to account for missingness in our covariates (Greenland \& Finkle, 1995). Using five imputed data sets computed by classification and regression trees (CART; Burgette \& Reiter, 2010) in the \emph{mice} R package (van Buuren \& Groothuis-Oudshoorn, 2011), we predicted treatment assignment (i.e., the transition to grandparenthood) five times per observation in logistic regressions with a logit link function.\footnote{In these logistic regressions we included all covariates listed above as predictors except for \emph{female} which was later used for exact matching and health-related covariates in LISS-wave 2014 which were not assessed in that wave.} We averaged these five scores per observation to compute the final propensity score to be used for matching (Mitra \& Reiter, 2016). We used imputed data only for propensity score computation and not in later analyses because nonresponse in the outcome variables was negligible.

\hypertarget{propensity-score-matching}{%
\subsubsection{Propensity Score Matching}\label{propensity-score-matching}}

The time of matching preceded the survey year when the transition to grandparenthood was first reported by at least two years (aside from that choosing the smallest available gap between matching and transition). This ensured that the covariates were not affected by the event itself or its anticipation (i.e., matching occurred well prior to one's child being pregnant with their first child; Greenland, 2003; Rosenbaum, 1984; VanderWeele et al., 2020). Propensity score matching was performed using the \emph{MatchIt} R package (Ho et al., 2011) with exact matching on gender combined with Mahalanobis distance matching on the propensity score. Four matchings were performed; two per sample (LISS; HRS) and two per control group (parents; nonparents). We matched 1:4 with replacement because of the relatively small pools of available controls. This meant that each grandparent was matched with four control observations in each matching procedure, and that control observations were allowed to be used multiple times for matching\footnote{In the LISS, 250 grandparent observations were matched with 1000 control observations; these control observations corresponded to 523 unique person-year observations stemming from 270 unique respondents for the parent control group, and to 464 unique person-year observations stemming from 189 unique respondents for the nonparent control group. In the HRS, 846 grandparent observations were matched with 3384 control observations; these control observations corresponded to 1393 unique person-year observations stemming from 982 unique respondents for the parent control group, and to 1008 unique person-year observations stemming from 704 unique respondents for the nonparent control group.}. We did not specify a caliper because our goal was to find matches for all grandparents, and because we achieved good covariate balance this way.\\
We evaluated the matching procedure in terms of covariate balance and, graphically, in terms of overlap of the distributions of the propensity score (Stuart, 2010). Covariate balance as indicated by the standardized difference in means between the grandparent and the controls after matching was good (see Tables \ref{tab:stddiffmeans-balance-liss} \& \ref{tab:stddiffmeans-balance-hrs}) lying below 0.25 as recommended in the literature (Stuart, 2010), and below 0.10 with few exceptions (Austin, 2011). Graphically, group differences in the distribution of propensity scores were small and indicated no substantial missing overlap (see Figure \ref{fig:pscore-overlap}).\\
After matching, each matched control observation received the same value as their matched grandparent in the \emph{time} variable describing the temporal relation to treatment, and the control respondent's other longitudinal observations were centered around this matched observation. Thereby, we coded a counterfactual transition time frame for each control respondent. Due to left- and right-censored longitudinal data (i.e., panel entry or attrition), we restricted the final analysis samples to six years before and six years after the transition as shown in Table \ref{tab:piecewise-coding-scheme}.\footnote{As an additional robustness check, we also re-estimated mean-level trajectories after further restricting the analysis time frame by excluding time points earlier than two years before the transition (i.e., before the latest time of matching). This served the purpose to assess whether including time points from before matching (as preregistered) would distort the trajectories in any way. However, results were highly similar across all outcomes (see \emph{gp\_restricted\_models.pdf} on \url{https://osf.io/75a4r/?view_only=ac929a2c41fb4afd9d1a64a3909848d0}).}\\
The final LISS analysis samples (see Figure \ref{fig:flowchart-participant}) contained 250 grandparents with 1368 longitudinal observations, matched with 1000 control respondents with either 5167 (parent control group) or 5340 longitudinal observations (nonparent control group). The final HRS analysis samples contained 846 grandparents with 2262 longitudinal observations, matched with 3384 control respondents with either 8257 (parent control group) or 8167 longitudinal observations (nonparent control group). In the HRS, there were a few additional missing values in the outcomes ranging from 18 to 105 longitudinal observations which were listwise deleted in the respective analyses.

\hypertarget{analytical-strategy}{%
\subsection{Analytical Strategy}\label{analytical-strategy}}

We used R (Version 4.0.4; R Core Team, 2021) and the R-packages \emph{lme4} (Version 1.1.26; Bates et al., 2015), and \emph{lmerTest} (Version 3.1.3; Kuznetsova et al., 2017) for multilevel modeling, as well as \emph{tidyverse} (Wickham et al., 2019) for data wrangling, and \emph{papaja} (Aust \& Barth, 2020) for reproducible manuscript production. A complete list of software we used is provided in the supplemental materials. Scripts for data wrangling, analyses, and to reproduce this manuscript can be found on the OSF (\url{https://osf.io/75a4r/?view_only=ac929a2c41fb4afd9d1a64a3909848d0/}) and on GitHub (\url{https://github.com/} {[}blinded{]}). Following Benjamin et al.~(2018), we set the \(\alpha\)-level for confirmatory analyses to \(.005\).\\
Our design can be referred to as an interrupted time-series with a \enquote{nonequivalent no-treatment control group} (Shadish et al., 2002, p. 182) where treatment, that is, the transition to grandparenthood, is not deliberately manipulated. First, to analyze mean-level changes (research question 1), we used linear piecewise regression coefficients in multilevel models with person-year observations nested within respondents and households (Hoffman, 2015). To model change over time in relation to the transition to grandparenthood, we coded three piecewise regression coefficients: a \emph{before-slope} representing linear change in the years leading up to the transition to grandparenthood, an \emph{after-slope} representing linear change in the years after the transition, and a \emph{shift} coefficient shifting the intercept directly after the transition was first reported, thus representing sudden changes that go beyond changes already modeled by the \emph{after-slope} {[}see Table \ref{tab:piecewise-coding-scheme} for the coding scheme of these coefficients{]}. Other studies of personality development have recently adopted similar piecewise coefficients (e.g., Schwaba \& Bleidorn, 2019; Krämer \& Rodgers, 2020; van Scheppingen \& Leopold, 2020).\\
All effects of the transition to grandparenthood on the Big Five and life satisfaction were modeled as deviations from patterns in the matched control groups by interacting the three piecewise coefficients with the treatment variable (0 = \emph{control}, 1 = \emph{grandparent}). In additional models, we interacted these coefficients with the moderator variables resulting in two- and three-way interactions. To test differences in the growth parameters between two groups in cases where these differences were represented by multiple fixed-effects coefficients, we defined linear contrasts using the \emph{linearHypothesis} command from the \emph{car} package (Fox \& Weisberg, 2019). All models of mean-level changes were estimated using maximum likelihood and included random intercepts but no random slopes. We included the propensity score as a level-2 covariate for a double-robust approach (Austin, 2017). Model equations can be found in the supplemental materials.\\
Second, to assess interindividual differences in change (research question 2), we added random slopes to the models. In other words, we allowed for differences between individuals in their trajectories of change to be modeled, that is, differences in the \emph{before-slope}, \emph{after-slope}, and \emph{shift} coefficients. Because multiple simultaneous random slopes are often not computationally feasible, we added random slopes one at a time and used likelihood ratio tests to determine whether the addition of the respective random slope led to a significant improvement in model fit. To statistically test differences in the random slope variance between the grandparent group and each control group, we respecified the models as heterogeneous variance models using the \emph{nlme} R package (Pinheiro et al., 2021), which allowed for separate random slope variances to be estimated in the grandparent group and the control group within the same model. Model fit of these heterogeneous variance models was compared to corresponding models with a homogeneous (single) random slope variance via likelihood ratio tests.\\
Third, to examine rank-order stability in the Big Five and life satisfaction over the transition to grandparenthood (research question 3), we computed the test-retest correlation of measurements prior to the transition to grandparenthood (at the time of matching) and the first available measurement afterwards. To test differences in test-retest correlations between grandparents and either of the control groups, we entered the pre-treatment measure, the treatment variable (0 = \emph{control}, 1 = \emph{grandparent}), and their interaction into regression models predicting the Big Five and life satisfaction. The interaction tests for significant differences in the rank-order stability between those who experienced the transition to grandparenthood and those who did not (see Denissen et al., 2019; McCrae, 1993).

\hypertarget{results}{%
\section{Results}\label{results}}

Throughout the results section, we referred to statistical tests with \(.005 < p < .05\) as \emph{suggestive evidence} as stated in our preregistration.

\hypertarget{descriptive-results}{%
\subsection{Descriptive Results}\label{descriptive-results}}

Means and standard deviations of the Big Five and life satisfaction over the analyzed time points are presented in Tables \ref{tab:descriptives-liss} and \ref{tab:descriptives-hrs}. Visually represented (see Figures \ref{fig:loess-agree}-\ref{fig:loess-swls}), all six outcomes display marked stability over time in both LISS and HRS. Intra-class correlations (see Table \ref{tab:icc-table}) show that large portions of the total variance in the Big Five could be explained by nesting in respondents (\emph{median} = 0.75), while nesting in households only accounted for minor portions (\(ICC_{hid}\), \emph{median} = 0.03). For outcome--subsample combinations with \(ICC_{hid}\) below \(0.05\) we omitted the household nesting factor from all models to bypass computational errors---a small deviation from our preregistration. For life satisfaction the nesting in households accounted for slightly larger portions of the total variance (\emph{median} = 0.36) than nesting in respondents (\emph{median} = 0.32). Across all outcomes, the proportion of variance due to within-person factors was relatively low (\emph{median} = 0.22).

\hypertarget{mean-level-changes}{%
\subsection{Mean-Level Changes}\label{mean-level-changes}}

Figures \ref{fig:effects-basic-plot} and \ref{fig:effects-gender-plot} summarize the effects of the basic (i.e., unmoderated) models and those including the gender interaction for all outcomes and across the four analysis samples.

\hypertarget{agreeableness}{%
\subsubsection{Agreeableness}\label{agreeableness}}

In the basic models (see Tables \ref{tab:H1-agree-tab} \& \ref{tab:H1-agree-contrasts} and Figure \ref{fig:H1-agree-fig}), grandparents in the LISS increased slightly in agreeableness in the years after the transition to grandparenthood as compared to the parent controls, \(\hat{\gamma}_{21} = 0.02\), 95\% CI \([0.01, 0.03]\), \(p\) = .003. However, this effect was quite small and not significant when compared against the nonparent controls, or against either control sample in the HRS sample (suggestive evidence in the HRS nonparents: \(\hat{\gamma}_{21} = 0.02\), 95\% CI \([0.01, 0.04]\), \(p\) = .006). The models including the gender interaction (see Tables \ref{tab:H1-agree-gender-tab} \& \ref{tab:H1-agree-gender-contrasts} and Figure \ref{fig:H1-agree-fig}) indicated that grandfathers' post-transition increases in agreeableness were more pronounced as compared to parent (LISS: \(\hat{\gamma}_{21} = 0.03\), 95\% CI \([0.01, 0.05]\), \(p\) \textless{} .001; HRS: \(\hat{\gamma}_{21} = 0.04\), 95\% CI \([0.01, 0.06]\), \(p\) = .003) and nonparent controls (HRS: \(\hat{\gamma}_{21} = 0.03\), 95\% CI \([0.01, 0.05]\), \(p\) = .004), whereas grandmothers did not differ from female controls.\\
There was no consistent evidence for moderation by paid work (see Tables \ref{tab:H1-agree-work-tab} \& \ref{tab:H1-agree-work-contrasts} and Figure \ref{fig:H1-agree-work-fig}). Grandparents providing substantial grandchild care increased in agreeableness after the transition to grandparenthood compared to matched nonparent controls (difference in \emph{after} parameter: {[}\(\hat{\gamma}_{21}\) + \(\hat{\gamma}_{31}\){]} = 0.04, 95\% CI {[}0.01, 0.06{]}, \(p\) = .002; suggestive evidence in the parent sample: {[}\(\hat{\gamma}_{21}\) + \(\hat{\gamma}_{31}\){]} = 0.04, 95\% CI {[}0.01, 0.06{]}, \(p\) = .006; see Tables \ref{tab:H1-agree-care-tab} \& \ref{tab:H1-agree-care-contrasts} and Figure \ref{fig:H1-agree-care-fig}). However, differences between caring and non-caring grandparents---as specified in hypothesis H1b---were not significant in either sample.

\hypertarget{conscientiousness}{%
\subsubsection{Conscientiousness}\label{conscientiousness}}

We found a slight post-transition increase in grandparents' conscientiousness in comparison to the controls in the HRS (parents: \(\hat{\gamma}_{21} = 0.02\), 95\% CI \([0.01, 0.04]\), \(p\) = .002; nonparents: \(\hat{\gamma}_{21} = 0.02\), 95\% CI \([0.01, 0.04]\), \(p\) = .003; suggestive evidence in the LISS parent sample: \(\hat{\gamma}_{21} = 0.02\), 95\% CI \([0.00, 0.03]\), \(p\) = .006; see Tables \ref{tab:H1-con-tab} \& \ref{tab:H1-con-contrasts} and Figure \ref{fig:H1-con-fig}). Grandparents' conscientiousness trajectories were not significantly moderated by gender (see Tables \ref{tab:H1-con-gender-tab} \& \ref{tab:H1-con-gender-contrasts} and Figure \ref{fig:H1-con-fig}).\\
There were significant differences in conscientiousness depending on grandparents' work status (see Tables \ref{tab:H1-con-work-tab} \& \ref{tab:H1-con-work-contrasts} and Figure \ref{fig:H1-con-work-fig}): non-working grandparents saw more pronounced increases in conscientiousness in the years before the transition to grandparenthood compared to non-working parent, \(\hat{\gamma}_{21} = 0.08\), 95\% CI \([0.04, 0.13]\), \(p\) \textless{} .001, and nonparent controls, \(\hat{\gamma}_{21} = 0.07\), 95\% CI \([0.03, 0.12]\), \(p\) = .002, and compared to working grandparents (difference in \emph{before} parameter; parents: {[}\(\hat{\gamma}_{30}\) + \(\hat{\gamma}_{31}\){]} = -0.08, 95\% CI {[}-0.13, -0.03{]}, \(p\) = .002; nonparents: {[}\(\hat{\gamma}_{30}\) + \(\hat{\gamma}_{31}\){]} = -0.08, 95\% CI {[}-0.12, -0.03{]}, \(p\) = .001). Grandparents providing substantial grandchild care increased in conscientiousness to a greater degree than the matched respondents (difference in \emph{after} parameter; parents: {[}\(\hat{\gamma}_{21}\) + \(\hat{\gamma}_{31}\){]} = 0.04, 95\% CI {[}0.02, 0.07{]}, \(p\) \textless{} .001; nonparents: {[}\(\hat{\gamma}_{21}\) + \(\hat{\gamma}_{31}\){]} = 0.05, 95\% CI {[}0.03, 0.07{]}, \(p\) \textless{} .001; see Tables \ref{tab:H1-con-care-tab} \& \ref{tab:H1-con-care-contrasts} and Figure \ref{fig:H1-con-care-fig}). There was only suggestive evidence that grandparents who provided substantial grandchild care increased more strongly in conscientiousness after the transition compared to grandparents who did not (difference in \emph{after} parameter; parents: {[}\(\hat{\gamma}_{30}\) + \(\hat{\gamma}_{31}\){]} = 0.03, 95\% CI {[}0.00, 0.06{]}, \(p\) = .034; nonparents: {[}\(\hat{\gamma}_{30}\) + \(\hat{\gamma}_{31}\){]} = 0.03, 95\% CI {[}0.00, 0.06{]}, \(p\) = .022).

\hypertarget{extraversion}{%
\subsubsection{Extraversion}\label{extraversion}}

The trajectories of grandparents' extraversion closely followed those of the matched controls. There were no significant effects indicating differences between grandparents and controls in the basic models (see Tables \ref{tab:H1-extra-tab} \& \ref{tab:H1-extra-contrasts} and Figure \ref{fig:H1-extra-fig}), the models including the gender interaction (see Tables \ref{tab:H1-extra-gender-tab} \& \ref{tab:H1-extra-gender-contrasts} and Figure \ref{fig:H1-extra-fig}), or the models of moderation by paid work (see Tables \ref{tab:H1-extra-work-tab} \& \ref{tab:H1-extra-work-contrasts} and Figure \ref{fig:H1-extra-work-fig}). The only significant effect for extraversion was found in the analysis of moderation by grandchild care (see Tables \ref{tab:H1-extra-care-tab} \& \ref{tab:H1-extra-care-contrasts} and Figure \ref{fig:H1-extra-care-fig}): compared to matched parent controls, grandparents providing substantial grandchild care increased slightly more strongly in extraversion after the transition to grandparenthood (difference in \emph{after} parameter: {[}\(\hat{\gamma}_{21}\) + \(\hat{\gamma}_{31}\){]} = 0.04, 95\% CI {[}0.02, 0.07{]}, \(p\) = .001; suggestive evidence in the nonparent sample: {[}\(\hat{\gamma}_{21}\) + \(\hat{\gamma}_{31}\){]} = 0.04, 95\% CI {[}0.01, 0.06{]}, \(p\) = .007).

\hypertarget{neuroticism}{%
\subsubsection{Neuroticism}\label{neuroticism}}

The basic models for neuroticism (see Tables \ref{tab:H1-neur-tab} \& \ref{tab:H1-neur-contrasts} and Figure \ref{fig:H1-neur-fig}) showed only minor differences between grandparents and matched controls: Compared to HRS parent controls, HRS grandparents shifted slightly downward in their neuroticism immediately after the transition to grandparenthood (difference in \emph{shift} parameter: {[}\(\hat{\gamma}_{21}\) + \(\hat{\gamma}_{31}\){]} = -0.08, 95\% CI {[}-0.12, -0.03{]}, \(p\) \textless{} .001), which was not the case in the three other samples (HRS nonparents, LISS parents, and LISS nonparents). The models including the gender interaction (see Tables \ref{tab:H1-neur-gender-tab} \& \ref{tab:H1-neur-gender-contrasts} and Figure \ref{fig:H1-neur-fig}) showed one significant effect in the comparison of grandparents and controls: In the HRS, grandfathers, compared to male parent controls, shifted downward in neuroticism directly after the transition to grandparenthood (difference in \emph{shift} parameter: {[}\(\hat{\gamma}_{21}\) + \(\hat{\gamma}_{31}\){]} = -0.16, 95\% CI {[}-0.22, -0.09{]}, \(p\) \textless{} .001; suggestive evidence in the nonparent sample: {[}\(\hat{\gamma}_{21}\) + \(\hat{\gamma}_{31}\){]} = -0.07, 95\% CI {[}-0.14, -0.01{]}, \(p\) = .024). Thus, the effect present in the basic models seemed to be mostly due to differences in the grandfathers (vs.~male controls).\\
Grandparents' trajectories of neuroticism as compared to the controls were significantly moderated by paid work (see Tables \ref{tab:H1-neur-work-tab} \& \ref{tab:H1-neur-work-contrasts} and Figure \ref{fig:H1-neur-work-fig}): Compared to working nonparent controls, working grandparents increased more strongly in neuroticism in the years before the transition to grandparenthood (difference in \emph{before} parameter: {[}\(\hat{\gamma}_{21}\) + \(\hat{\gamma}_{31}\){]} = 0.06, 95\% CI {[}0.03, 0.10{]}, \(p\) \textless{} .001; suggestive evidence in the parent sample: {[}\(\hat{\gamma}_{21}\) + \(\hat{\gamma}_{31}\){]} = 0.05, 95\% CI {[}0.01, 0.08{]}, \(p\) = .015). At the first post-transition assessment, working grandparents shifted downward in neuroticism compared to working parent controls (difference in \emph{shift} parameter: {[}\(\hat{\gamma}_{41}\) + \(\hat{\gamma}_{61}\) + \(\hat{\gamma}_{51}\) + \(\hat{\gamma}_{71}\){]} = -0.08, 95\% CI {[}-0.14, -0.03{]}, \(p\) = .004; suggestive evidence in the nonparent sample: {[}\(\hat{\gamma}_{41}\) + \(\hat{\gamma}_{61}\) + \(\hat{\gamma}_{51}\) + \(\hat{\gamma}_{71}\){]} = -0.06, 95\% CI {[}-0.11, 0.00{]}, \(p\) = .034). There was suggestive evidence that grandparents providing substantial grandchild care decreased more strongly in neuroticism after the transition to grandparenthood than grandparents who did not (difference in \emph{after} parameter; parents: {[}\(\hat{\gamma}_{30}\) + \(\hat{\gamma}_{31}\){]} = -0.04, 95\% CI {[}-0.07, 0.00{]}, \(p\) = .044; nonparents: {[}\(\hat{\gamma}_{30}\) + \(\hat{\gamma}_{31}\){]} = -0.04, 95\% CI {[}-0.07, 0.00{]}, \(p\) = .048; see Tables \ref{tab:H1-neur-care-tab} \& \ref{tab:H1-neur-care-contrasts} and Figure \ref{fig:H1-neur-care-fig}).

\hypertarget{openness}{%
\subsubsection{Openness}\label{openness}}

For openness, we also found a high degree of similarity between grandparents and matched control respondents in their trajectories based on the basic models (see Tables \ref{tab:H1-open-tab} \& \ref{tab:H1-open-contrasts} and Figure \ref{fig:H1-open-fig}) and models including the gender interaction (see Tables \ref{tab:H1-open-gender-tab} \& \ref{tab:H1-open-gender-contrasts} and Figure \ref{fig:H1-open-fig}). Grandparents in the HRS shifted downward in openness in the first assessment after the transition to grandparenthood compared to the parent controls (difference in \emph{shift} parameter: {[}\(\hat{\gamma}_{21}\) + \(\hat{\gamma}_{31}\){]} = -0.05, 95\% CI {[}-0.09, -0.02{]}, \(p\) = .004; suggestive evidence in the nonparent sample: {[}\(\hat{\gamma}_{21}\) + \(\hat{\gamma}_{31}\){]} = -0.04, 95\% CI {[}-0.07, 0.00{]}, \(p\) = .039), which was due to significant differences between grandfathers and male parent controls (difference in \emph{shift} parameter: {[}\(\hat{\gamma}_{21}\) + \(\hat{\gamma}_{31}\){]} = -0.11, 95\% CI {[}-0.17, -0.06{]}, \(p\) \textless{} .001).\\
Performing paid work moderated grandparents' openness trajectories in subtle ways (see Tables \ref{tab:H1-open-work-tab} \& \ref{tab:H1-open-work-contrasts} and Figure \ref{fig:H1-open-work-fig}): Non-working grandparents increased more strongly in openness post-transition than non-working controls (parents: \(\hat{\gamma}_{41} = 0.05\), 95\% CI \([0.02, 0.07]\), \(p\) \textless{} .001; nonparents: \(\hat{\gamma}_{41} = 0.04\), 95\% CI \([0.02, 0.06]\), \(p\) \textless{} .001). Further, there was suggestive evidence that openness of non-working grandparents shifted downward directly after the transition compared to non-working controls (difference in \emph{shift} parameter; parents: {[}\(\hat{\gamma}_{41}\) + \(\hat{\gamma}_{61}\){]} = -0.09, 95\% CI {[}-0.15, -0.02{]}, \(p\) = .007; nonparents: {[}\(\hat{\gamma}_{41}\) + \(\hat{\gamma}_{61}\){]} = -0.07, 95\% CI {[}-0.13, -0.01{]}, \(p\) = .014). However, compared to non-working grandparents, working grandparents shifted upward in openness directly after the transition (suggestive evidence for difference in \emph{shift} parameter; parents: {[}\(\hat{\gamma}_{50}\) + \(\hat{\gamma}_{70}\) + \(\hat{\gamma}_{51}\) + \(\hat{\gamma}_{71}\){]} = 0.08, 95\% CI {[}0.00, 0.15{]}, \(p\) = .038; nonparents: {[}\(\hat{\gamma}_{50}\) + \(\hat{\gamma}_{70}\) + \(\hat{\gamma}_{51}\) + \(\hat{\gamma}_{71}\){]} = 0.08, 95\% CI {[}0.01, 0.14{]}, \(p\) = .023) and decreased afterwards (suggestive evidence for difference in \emph{after} parameter; parents: {[}\(\hat{\gamma}_{50}\) + \(\hat{\gamma}_{51}\){]} = -0.04, 95\% CI {[}-0.07, -0.01{]}, \(p\) = .016; nonparents: {[}\(\hat{\gamma}_{50}\) + \(\hat{\gamma}_{51}\){]} = -0.04, 95\% CI {[}-0.07, -0.01{]}, \(p\) = .007). The analysis of moderation by grandchild care (see Tables \ref{tab:H1-open-care-tab} \& \ref{tab:H1-open-care-contrasts} and Figure \ref{fig:H1-open-care-fig}) revealed that grandparents providing substantial grandchild care increased more strongly in openness after the transition to grandparenthood than the matched nonparent controls (difference in \emph{after} parameter: {[}\(\hat{\gamma}_{21}\) + \(\hat{\gamma}_{31}\){]} = 0.04, 95\% CI {[}0.01, 0.06{]}, \(p\) = .002; suggestive evidence in the parent sample: {[}\(\hat{\gamma}_{21}\) + \(\hat{\gamma}_{31}\){]} = 0.04, 95\% CI {[}0.01, 0.07{]}, \(p\) = .005). At the same time, the plotted trajectories demonstrated that the described moderation effects for openness were all quite small.

\hypertarget{life-satisfaction-1}{%
\subsubsection{Life Satisfaction}\label{life-satisfaction-1}}

The basic models for life satisfaction (see Tables \ref{tab:H1-swls-tab} \& \ref{tab:H1-swls-contrasts} and Figure \ref{fig:H1-swls-fig}) showed that grandparents in the LISS increased more strongly in life satisfaction directly following the transition compared to nonparent controls (difference in \emph{shift} parameter: {[}\(\hat{\gamma}_{21}\) + \(\hat{\gamma}_{31}\){]} = 0.18, 95\% CI {[}0.06, 0.30{]}, \(p\) = .004). There was evidence in the models including the gender interaction (see Tables \ref{tab:H1-swls-gender-tab} \& \ref{tab:H1-swls-gender-contrasts} and Figure \ref{fig:H1-swls-fig}) that this difference was due to grandmothers, who increased more strongly in life satisfaction directly following the transition to grandparenthood than female nonparent controls in the LISS (difference in \emph{shift} parameter: {[}\(\hat{\gamma}_{21}\) + \(\hat{\gamma}_{31}\) + \(\hat{\gamma}_{23}\) + \(\hat{\gamma}_{33}\){]} = 0.24, 95\% CI {[}0.08, 0.41{]}, \(p\) = .004). HRS grandmothers increased more strongly before the transition to grandparenthood compared to female parent controls (difference in \emph{before} parameter: {[}\(\hat{\gamma}_{11}\) + \(\hat{\gamma}_{13}\){]} = 0.21, 95\% CI {[}0.09, 0.33{]}, \(p\) \textless{} .001).\\
There was no consistent evidence for a moderation of life satisfaction by performing paid work (see Tables \ref{tab:H1-swls-work-tab} \& \ref{tab:H1-swls-work-contrasts} and Figure \ref{fig:H1-swls-work-fig}) or grandchild care (see Tables \ref{tab:H1-swls-care-tab} \& \ref{tab:H1-swls-care-contrasts} and Figure \ref{fig:H1-swls-care-fig}).

\hypertarget{interindividual-differences-in-change}{%
\subsection{Interindividual Differences in Change}\label{interindividual-differences-in-change}}

First, we conducted comparisons of model fit between the random intercept models reported previously and models where a random slope variance was estimated, separately for each change parameter. These comparisons showed a substantial amount of interindividual differences in change for all random slopes in all models as indicated by increases in model fit significant at \(p\) \textless{} .001.\\
Second, we estimated models with heterogeneous random slope variances of the grandparents and each control group in order to test whether interindividual differences in change were significantly larger in the grandparents. Contrary to hypothesis H2, for agreeableness, conscientiousness, and extraversion, interindividual differences in intraindividual change were greater in the control group for all tested effects (see Tables \ref{tab:H2-hetvar-tab-agree}, \ref{tab:H2-hetvar-tab-con}, \& \ref{tab:H2-hetvar-tab-extra}). In the two HRS samples, assuming group heterogeneity in the random slope variances lead to significant improvements in model fit in all model comparisons. In the two LISS samples, this was the case for around half the tests.\\
Interindividual differences in change in neuroticism before the transition to grandparenthood were significantly greater in the HRS grandparents than the nonparent controls (random slope variances of the \emph{before} parameter), \emph{likelihood ratio} = 73.45, \(p\) \textless{} .001. However, this was not the case in the comparison of grandparents with parent controls in the HRS or either control group in the LISS (see Table \ref{tab:H2-hetvar-tab-neur}). The other parameters of change in neuroticism did not differ significantly between groups in their random slope variances or---in the HRS---displayed significantly larger random slope variances in the respective control group.\\
For openness, interindividual differences in change before the transition to grandparenthood were significantly greater in the LISS grandparents than the nonparent controls (random slope variances of the \emph{before} parameter), \emph{likelihood ratio} = 25.90, \(p\) \textless{} .001. Again, this result could not be replicated in the other three samples, and the other parameters of change did either not differ between groups in their random slope variances or had significantly larger random slope variances in the respective control group (see Table \ref{tab:H2-hetvar-tab-open}).\\
We found partial evidence for larger interindividual differences in grandparents' changes in life satisfaction (see Table \ref{tab:H2-hetvar-tab-swls}): In the LISS grandparents, changes before the transition to grandparenthood varied interindividually to a larger extent compared to the parent controls (random slope variances of the \emph{before} parameter), \emph{likelihood ratio} = 41.47, \(p\) \textless{} .001, and in the HRS compared to the nonparent controls, \emph{likelihood ratio} = 111.97, \(p\) \textless{} .001. Still, the majority of tests for heterogeneous random slope variances in life satisfaction indicated either non-significant differences or significantly larger random slope variances in the control sample.

\hypertarget{rank-order-stability}{%
\subsection{Rank-Order Stability}\label{rank-order-stability}}

As indicators of rank-order stability, we computed test-retest correlations for the Big Five and life satisfaction for the matched sample, as well as separately for grandparents only and controls only (see Table \ref{tab:H3-rankorder-tab}). In 6 out of 24 comparisons grandparents' test-retest correlation was lower than that of the respective control group. However, differences in rank-order stability between grandparents and control respondents did not reach significance in any of these comparisons. Overall, we found no confirmatory evidence in support of hypothesis H3.\footnote{In addition to the preregistered retest interval, we have also computed a maximally large retest interval between the first available pre-transition assessment and the last available post-transition assessment within the observation period. Here, 5 out of 24 comparisons indicated that rank-order stability was lower in the grandparents. There was only one significant difference in rank-order stability in accordance with our hypothesis: HRS grandparents' rank-order stability in openness was lower than that of the nonparents, \(p\) \textless{} .001 (see Table \ref{tab:H3-rankordermax-tab}). Another analysis also failed to provide convincing evidence that grandparents' rank-order stability was lower: we followed the preregistered approach but then excluded any duplicate control respondents resulting from matching with replacement who might bias results towards greater stability in the controls. Descriptively, 14 out of 24 comparisons showed lower rank-order stability in the grandparents compared to either control group (see Table \ref{tab:H3-rankorderuni-tab}). However, differences between groups were small and nonsignificant throughout.}

\hypertarget{discussion}{%
\section{Discussion}\label{discussion}}

In an analysis of first-time grandparents in comparison with both parent and nonparent matched control respondents we found pronounced stability in the Big Five and life satisfaction over the transition to grandparenthood. Although there were a few isolated effects in line with our hypotheses on mean-level increases in agreeableness and conscientiousness, and decreases in neuroticism (H1a), they were very small in size and also not consistent over the two analyzed panel studies (LISS and HRS) or the two matched control groups (parents and nonparents). We found suggestive evidence that grandparents providing substantial grandchild care increased slightly more strongly in conscientiousness and decreased slightly more strongly in neuroticism than those grandparents who did not (H1b), as well as partial evidence for moderation of mean-level trajectories of conscientiousness, neuroticism, and openness by performing paid work. There was no consistent evidence that grandmothers reached higher levels of life satisfaction following the transition to grandparenthood (H1c). Although interindividual differences in change were present for all parameters of change, they were only greater in the grandparents compared to the controls in a stark minority of conducted model comparisons (H2). Lastly, rank-order stability did not differ between grandparents and either control group, or was larger in the control group---contrary to expectations (H3).

\hypertarget{social-investment-principle}{%
\subsection{Social Investment Principle}\label{social-investment-principle}}

We conducted a preregistered, cross-study, and multi-comparison test of the social investment principle (Lodi-Smith \& Roberts, 2007; Roberts \& Wood, 2006) in middle adulthood and old age where the transition to grandparenthood has been put forward as a potentially important developmental task driving personality development of the Big Five (Hutteman et al., 2014). Across all analyzed traits, we found more evidence for trait stability than change.\\
Still, whereas we did not find \emph{consistent} evidence for personality development across the transition to grandparenthood, the direction of the (sparse) effects we uncovered generally supported the social investment principle---in contrast to development following parenthood (Asselmann \& Specht, 2020b; van Scheppingen et al., 2016). Below we summarize our findings in support of the social investment principle because even small psychological effects may be meaningful and involve real world consequences (Götz et al., 2021): For agreeableness and conscientiousness we found slight post-transition increases in comparison to the matched control groups which were line with the social investment principle. However, the effects were not only small but also inconsistent across samples. Agreeableness only increased in the LISS (compared to parents) and conscientiousness only in the HRS (compared to both parents and nonparents). In the HRS, neuroticism decreased in grandparents directly following the transition to grandparenthood when compared to matched parent respondents. This was not the case in the LISS or compared to HRS nonparents.\\
In the case of agreeableness and neuroticism, these effects were only present in the comparison of grandfathers and male controls, whereas no effects were found for grandmothers. In contrast, past research---mostly in the domains of well-being and health---found more pronounced effects of the transition to grandparenthood for grandmothers (Di Gessa et al., 2016b, 2019; Sheppard \& Monden, 2019; Tanskanen et al., 2019). This was discussed in the context of grandmothers spending more time with their grandchildren than grandfathers and providing more hours of care (Condon et al., 2013; Di Gessa et al., 2020), thus making a higher social investment.\footnote{In the HRS analysis sample, the proportion of grandparents reporting that they have provided at least 100 hours of grandchild care since the last assessment was also slightly higher in grandmothers (\emph{M} = 0.45, \emph{SD} = 0.50) than grandfathers (\emph{M} = 0.41, \emph{SD} = 0.49).} We found partial support for this for life satisfaction (see below). Yet our results for the Big Five were not in agreement with this line of thought. Instead, one possible explanation is that (future) grandfathers have on average been previously more invested in their work lives than in child rearing, and at the end of their career or after retirement found investment in grandchild care to be a more novel and meaningful transition than grandmothers (StGeorge \& Fletcher, 2014; Tanskanen et al., 2021). Currently, however, empirical research specifically into the grandfather role is sparse (for a qualitative approach, see Mann \& Leeson, 2010), while the demography of grandparenthood is undergoing swift changes toward a higher proportion of actively involved grandfathers (see Coall et al., 2016; Mann, 2007). Thus, more research into grandfathers' experience of the transition to grandparenthood is needed to substantiate our tentative findings.\\
To gain more insight into social investment mechanisms, we tested paid work and grandchild care as moderators. For conscientiousness, we found that grandparents who were not gainfully employed increased more strongly in anticipation of the transition to grandparenthood than working grandparents (and than the matched nonworking controls). Although this could imply that working grandparents did not find as much time for social investment because of the role conflict with the employee/worker role (see Tanskanen et al., 2021), we would have expected these moderation effects after the transition where grandparents were indeed able to spend time with their grandchild. However, such post-transition differences did not surface. Results for neuroticism were even less clearly in line with the social investment principle: Working grandparents increased in neuroticism in anticipation of the transition to grandparenthood (compared to nonparents), and decreased immediately following the transition (compared to parents). Regarding moderation by grandchild care, our results suggested that grandparents who provided substantial grandchild care increased more in conscientiousness and decreased more in neuroticism compared to grandparents who did not. However, the strength of evidence was weak and indicates a need for temporally more fine-grained assessments with more extensive instruments of grandchild care (e.g., Vermote et al., 2021; see also Fingerman et al., 2020).\\
In total, evidence in favor of the social investment principle in our analyses was rather thin. This adds to other recent empirical tests in the context of parenthood and romantic relationships (Asselmann \& Specht, 2020a, 2020b; Spikic et al., 2021; van Scheppingen et al., 2016) that have challenged the original core assumption of personality maturation through age-graded social role transitions. It now seems likely that distinct (or additional) theoretical assumptions and mechanisms are required to explain empirical findings of personality development in middle adulthood and old age. First steps in that direction include the recent distinction between social investment and divestment (Schwaba \& Bleidorn, 2019) in the context of retirement (for the related distinction between personality maturation and relaxation, see Asselmann \& Specht, 2021), as well as the hypothesis that personality development is more closely tied to the subjective perceptions of adult role competency than to the transitions per se (Roberts \& Davis, 2016).\\
Nonetheless, the possibility remains that preconditions we have not considered have to be met for grandparents to undergo personality development after the transition to grandparenthood. For example, grandparents might need to live in close proximity to their grandchild, see them on a regular basis, and provide grandchild care above a certain quantity and quality (e.g., level of responsibility). To our knowledge, however, there are presently no datasets with such detailed information regarding the grandparent role in conjunction with multiple waves of Big Five personality data. Studies in the well-being literature have provided initial evidence that more frequent contact with grandchildren was associated with higher grandparental well-being (Arpino, Bordone, et al., 2018; Danielsbacka et al., 2019; Danielsbacka \& Tanskanen, 2016). However, Danielsbacka et al.~(2019) noted that this effect was due to between-person differences in grandparents, thus limiting a causal interpretation of frequency of grandchild care as a mechanism of development in psychological characteristics like life satisfaction and personality.

\hypertarget{life-satisfaction-2}{%
\subsection{Life Satisfaction}\label{life-satisfaction-2}}

Related, we did not find convincing evidence that life satisfaction changed as a consequence of the transition to grandparenthood. Only in the LISS in comparison with the nonparent control group did grandparents' life satisfaction increase slightly at the first assessment following the transition to grandparenthood. This difference was present in grandmothers but not grandfathers. While this pattern of effects is in line with several studies reporting increases associated with women becoming grandmothers (e.g., Di Gessa et al., 2019; Tanskanen et al., 2019), we did not uncover it reliably in both samples or with both comparison groups and also did not see consistent effects in the linear trajectories after the transition to grandparenthood. As mentioned in the introduction, a study into the effects of the transition on first-time grandparents' life satisfaction that used fixed effects regressions also did not discover any positive within-person effects of the transition (Sheppard \& Monden, 2019). Further, in line with this study, we did not find evidence that grandparents who provided substantial grandchild care increased more strongly in life satisfaction than those who did not, and, likewise, grandparents' life satisfaction trajectories were not moderated by employment status (Sheppard \& Monden, 2019).\\
Overall, research has accumulated that there is an association between having grandchildren and higher life satisfaction on the between-person level---especially for (maternal) grandmothers who provide frequent grandchild care (Danielsbacka et al., 2011; Danielsbacka \& Tanskanen, 2016)---but no within-person effect of the transition. The main reason for this divergence is the presence of \emph{selection} effects, that is, confounding which we have accounted for via the propensity score matching design, but which was present in previous within-person estimates of change (Luhmann et al., 2014; Thoemmes \& Kim, 2011; VanderWeele et al., 2020).

\hypertarget{interindividual-differences-in-change-1}{%
\subsection{Interindividual Differences in Change}\label{interindividual-differences-in-change-1}}

Analyzing how grandparents differed interindividually in their trajectories of change provided additional insight beyond the analysis of mean-level change. All parameters of change exhibited considerable interindividual differences. Similar to Denissen et al.~(2019) who found significant model fit improvements of random slopes in most models (see also Doré \& Bolger, 2018) this pattern indicates that respondents---both grandparents and matched controls---deviated to a considerable extent from the average trajectories that we reported on previously.\\
We expected larger interindividual differences in the grandparents because life events differ in the impact they have on people's daily lives and in the degree that those who experience them perceive them as meaningful or emotionally significant (Doré \& Bolger, 2018; Luhmann et al., 2020). Our results, however, indicated that interindividual differences were larger in the controls than the grandparents for many models, or not significantly different between groups. Only in a stark minority of tests were interindividual differences significantly larger in grandparents (concerning the linear slope in anticipation of grandparenthood for neuroticism, openness, and life satisfaction). Overall, we did not find evidence supporting the hypothesis that interindividual differences in change would be larger in the grandparents than the controls (H2).\\
Integrating this result into the literature, it is important to point out that most previous studies did not compare interindividual differences in personality change between the event group and a comparison group (even if they did use comparison groups for the main analyses; Denissen et al., 2019; Schwaba \& Bleidorn, 2019; cf.~Jackson \& Beck, 2021). As demonstrated by an analysis across the entire life span (i.e., irrespective of life events; Schwaba \& Bleidorn, 2018), interindividual differences in personality change---although largest in emerging adulthood---were substantial up until circa 70 years of age for most domains. Regarding the substantive question of how the transition to grandparenthood affects interindividual differences in change, we, therefore, propose that it is more informative to test grandparents' degree of variability in change against well-matched control groups than against no groups as often done previously.\\
Recently, Jackson and Beck (2021) have presented evidence that the experience of sixteen commonly analyzed life events was mostly associated with decreases in interindividual variation in the Big Five compared to those not experiencing the respective event. They used a comparable approach to ours but in a SEM latent growth curve framework and not accounting for covariates related to pre-existing group differences (i.e., without matching). Their results based on the German SOEP data suggested---counter to their expectations---that most life events made people \emph{more} similar to each other (Jackson \& Beck, 2021). Thus, coupled with our results it seems that the assumption that life events and transitions ostensibly produce increased heterogeneity between people needs to be scrutinized in future studies.

\hypertarget{rank-order-stability-1}{%
\subsection{Rank-Order Stability}\label{rank-order-stability-1}}

We also investigated whether grandparents' rank-order stability in the Big Five personality traits and life satisfaction over the transition to grandparenthood was lower than that of the matched controls. Conceptually, rank-order changes are possible in the absence of mean-level changes. Empirically, though, we did not find evidence supporting our hypothesis (H3): Rank-order stability did not differ significantly between grandparents and controls and, descriptively, was larger in the grandparents in the majority of comparisons. In a recent study of the effects of eight different life events on the development of the Big Five personality traits and life satisfaction (Denissen et al., 2019), comparably high rank-order stability was reported in the event groups. Only particularly adverse events such as widowhood and disability significantly lowered respondents' rank-order stability (Chopik, 2018; Denissen et al., 2019).\\
Regarding the Big Five's general age trajectories of rank-order stability, support for inverted U-shape trajectories was recently strengthened in a study of two panel data sets (Seifert et al., 2021). This study also explored that health deterioration accounted for parts of the decline of personality stability in old age. Therefore, it is possible that in later developmental phases (see also Hutteman et al., 2014) rank-order stability of personality is largely influenced by health status and less by normative life events. In the context of grandparenthood, this relates to research into health benefits (Chung \& Park, 2018; Condon et al., 2018; Di Gessa et al., 2016a, 2016b; cf.~Ates, 2017) and decreases to mortality risk associated with grandparenthood or grandchild care (Choi, 2020; Christiansen, 2014; Hilbrand et al., 2017; cf.~Ellwardt et al., 2021). Thereby, grandparenthood might have a time-lagged effect on personality stability through protective effects on health. However, with the currently available data such a mediating effect cannot be reliably recovered (under realistic assumptions; Rohrer et al., 2021).

\hypertarget{limitations-and-future-directions}{%
\subsection{Limitations and Future Directions}\label{limitations-and-future-directions}}

The current study has a number of strengths that bolster the robustness of its inferences: It features a preregistered analysis of archival data with an internal cross-study replication, a propensity score matching design that carefully deliberated covariate choice, and a twofold comparison of all effects of the grandparents---against matched parents (with children in reproductive age) and nonparents. To obtain a comprehensive picture of personality development, we analyzed mean-level changes, interindividual differences in change, and changes to rank-order stability. Both of the panel studies we used had their strengths and weaknesses: The HRS had a larger sample of first-time grandparents besides information on important moderators but assessed personality and life satisfaction only every four years. The LISS assessed the outcomes every year (apart from a few waves with planned missingness) but restricted the grandparent sample through filtering of the relevant questions to employed respondents resulting in a smaller and younger sample. Together, the strengths of one dataset partially compensated for the limitations of the other.\\
Still, a number of limitations need to be addressed: First, there remains some doubt whether we were able to follow truly socially invested grandparents over time. More detailed information regarding a grandparent's relationship with their first and later grandchildren and the level of care a grandparent provides would be a valuable source of information on social investment, as would be information on possible constraining factors such as length and cost of travel between grandparent and grandchild. Lacking such precise contextual information, the multidimensionality of the grandparent role (Buchanan \& Rotkirch, 2018; Findler et al., 2013; Thiele \& Whelan, 2006) might lend itself to future investigations into grandparents' personality development using growth mixture models (Grimm \& Ram, 2009; Infurna, 2021; Ram \& Grimm, 2009). On a similar note, we did not consider grandparents' subjective perception of the transition to grandparenthood in terms of the emotional significance, meaningfulness, and impact to daily lives which might be responsible for differential individual change trajectories (Kritzler et al., 2021; Luhmann et al., 2020).\\
Second, we relied on self-report personality data and did not include other-reports by family members or close friends (Luan et al., 2017; McCrae, 2018; McCrae \& Mõttus, 2019; Mõttus et al., 2019). Thus, our results might be influenced by common method bias (Podsakoff et al., 2003). Large-scale panel data incorporating both self- and other-reports of personality over time would be needed to address this issue (e.g., Oltmanns et al., 2020).\\
Third, a causal interpretation of our results rests on a number of assumptions that are not directly testable with the data (Li, 2013; Stuart, 2010): most importantly, we assumed that we picked the right sets of covariates, that our model to estimate the propensity score was correctly specified, and that there was no substantial remaining bias due to unmeasured confounding. Working with archival data meant that we had no influence on data collection, and we also aimed for roughly equivalent sets of covariates across both data sets. Therefore, we had to make some compromises to covariate choice. Still, we believe that our procedure to select covariates following state-of-the-art recommendations (see \emph{Methods}; VanderWeele et al., 2020), and to substantiate each covariate's selection explicitly within our preregistration improved upon previously applied practices. Regarding the propensity score estimation, we opted to estimate the grandparents' propensity scores at a specific time point at least two years before the transition to grandparenthood which had the advantages that (\emph{1}) the covariates were uncontaminated by anticipation of the transition, and (\emph{2}) the matched controls had a clear counterfactual timeline of transition (for similar recent approaches analyzing life events, see Balbo \& Arpino, 2016; Krämer \& Rodgers, 2020; van Scheppingen \& Leopold, 2020). Regarding the timing of measurements and the transition to grandparenthood, it also has to be emphasized that we might have missed more short-term effects playing out over months instead of years.\\
Fourth, our results only pertain to the countries for which our data are representative on a population-level, the Netherlands and the United States. Personality development, and more specifically personality maturation, have been examined cross-culturally (e.g., Bleidorn et al., 2013; Chopik \& Kitayama, 2018). On the one hand, these studies showed universal average patterns of change towards greater maturity over the life span. On the other hand, they emphasized cultural differences regarding norms and values and the temporal onset of social roles. For grandparenthood, there are substantial demographic differences between countries (Leopold \& Skopek, 2015), as well as differences in public child care systems which may demand different levels of grandparental involvement (Bordone et al., 2017; Hank \& Buber, 2009). Compared to the US, Dutch people on average become grandparents six years later (Leopold \& Skopek, 2015) and, although both countries have largely market-based systems for early child care, Dutch parents on average have access to more fully developed child care systems through (capped) governmental benefits (OECD, 2020). Despite these differences, our results from the Dutch and US samples did not indicate systematic discrepancies.\\
Lastly, while we assessed our dependent variables through highly reliable scales, there was a conceptual difference in the Big Five measures (see John \& Srivastava, 1999) in the two studies: In the LISS, the IPIP Big-Five Inventory (Goldberg, 1992) presented as items statements to which respondents indicated how accurately they described them (using a bipolar response scale). However, in the HRS, the Midlife Development Inventory (Lachman \& Weaver, 1997) used adjectives as items to ask respondents how well they described them (using a unipolar response scale). This discrepancy hindered the between-sample comparison somewhat and also resulted in different distributions of the Big Five across samples (see Figures \ref{fig:loess-agree}-\ref{fig:loess-swls}). The possibility should also be pointed out that our analyses on the domain-level of the Big Five could be too broad conceptually to identify patterns of personality development over the transition to grandparenthood that are discernible on the level of facets or nuances (Mõttus \& Rozgonjuk, 2021).

\hypertarget{conclusion}{%
\subsection{Conclusion}\label{conclusion}}

Do personality traits change over the transition to grandparenthood? Using data from two nationally representative panel studies in a preregistered propensity score matching design, the current study revealed that trajectories of the Big Five personality traits and life satisfaction remained predominantly stable in first-time grandparents over this transition compared to matched parents and nonparents. We found slight post-transition increases to grandparents' agreeableness and conscientiousness in line with our hypothesis of personality development based on the social investment principle. However, these effects were minuscule and inconsistent across analysis samples. In addition, our analyses revealed (\emph{1}) a lack of consistent moderation of personality development by grandparents providing substantial grandchild care, (\emph{2}) interindividual differences in change that were mostly smaller in grandparents than in matched respondents, and (\emph{3}) comparable rank-order stability in grandparents and matched respondents. Thus, we conclude that the transition to grandparenthood did not act as a straightforwardly important developmental task driving personality development in middle adulthood and old age (as previously proposed, see Hutteman et al., 2014). With more detailed assessment of the grandparent role, future research could investigate if personality development occurs in a subset of grandparents who are highly socially invested.

\hypertarget{acknowledgements}{%
\subsection{Acknowledgements}\label{acknowledgements}}

We thank Joe Rodgers, Jaap Denissen, and Julia Rohrer for helpful comments on earlier versions of this paper.

\newpage

\hypertarget{references}{%
\section{References}\label{references}}

\begingroup
\setlength{\parindent}{-0.5in}
\setlength{\leftskip}{0.5in}

\hypertarget{refs}{}
\leavevmode\hypertarget{ref-aassveFirstGlanceBlack2021}{}%
Aassve, A., Luppi, F., \& Mencarini, L. (2021). A first glance into the black box of life satisfaction surrounding childbearing. \emph{Journal of Population Research}. \url{https://doi.org/10.1007/s12546-021-09267-z}

\leavevmode\hypertarget{ref-allemandLongtermCorrelatedChange2008a}{}%
Allemand, M., Zimprich, D., \& Martin, M. (2008). Long-term correlated change in personality traits in old age. \emph{Psychology and Aging}, \emph{23}(3), 545--557. \url{https://doi.org/10.1037/a0013239}

\leavevmode\hypertarget{ref-anusicStabilityChangePersonality2016}{}%
Anusic, I., \& Schimmack, U. (2016). Stability and change of personality traits, self-esteem, and well-being: Introducing the meta-analytic stability and change model of retest correlations. \emph{Journal of Personality and Social Psychology}, \emph{110}(5), 766--781. \url{https://doi.org/10.1037/pspp0000066}

\leavevmode\hypertarget{ref-anusicDoesPersonalityModerate2014}{}%
Anusic, I., Yap, S., \& Lucas, R. E. (2014a). Does personality moderate reaction and adaptation to major life events? Analysis of life satisfaction and affect in an Australian national sample. \emph{Journal of Research in Personality}, \emph{51}, 69--77. \url{https://doi.org/10.1016/j.jrp.2014.04.009}

\leavevmode\hypertarget{ref-anusicTestingSetpointTheory2014}{}%
Anusic, I., Yap, S., \& Lucas, R. E. (2014b). Testing set-point theory in a Swiss national sample: Reaction and adaptation to major life events. \emph{Social Indicators Research}, \emph{119}(3), 1265--1288. \url{https://doi.org/10.1007/s11205-013-0541-2}

\leavevmode\hypertarget{ref-ardeltStillStableAll2000}{}%
Ardelt, M. (2000). Still stable after all these years? Personality stability theory revisited. \emph{Social Psychology Quarterly}, \emph{63}(4), 392--405. \url{https://doi.org/10.2307/2695848}

\leavevmode\hypertarget{ref-arpinoGrandparentingEducationSubjective2018}{}%
Arpino, B., Bordone, V., \& Balbo, N. (2018). Grandparenting, education and subjective well-being of older Europeans. \emph{European Journal of Ageing}, \emph{15}(3), 251--263. \url{https://doi.org/10.1007/s10433-018-0467-2}

\leavevmode\hypertarget{ref-arpinoFamilyHistoriesDemography2018}{}%
Arpino, B., Gumà, J., \& Julià, A. (2018). Family histories and the demography of grandparenthood. \emph{Demographic Research}, \emph{39}(42), 1105--1150. \url{https://doi.org/10.4054/DemRes.2018.39.42}

\leavevmode\hypertarget{ref-asselmannTakingUpsDowns2020}{}%
Asselmann, E., \& Specht, J. (2020a). Taking the ups and downs at the rollercoaster of love: Associations between major life events in the domain of romantic relationships and the Big Five personality traits. \emph{Developmental Psychology}, \emph{56}(9), 1803--1816. \url{https://doi.org/10.1037/dev0001047}

\leavevmode\hypertarget{ref-asselmannPersonalityMaturationPersonality2021}{}%
Asselmann, E., \& Specht, J. (2021). Personality maturation and personality relaxation: Differences of the Big Five personality traits in the years around the beginning and ending of working life. \emph{Journal of Personality}, Advance Online Publication. \url{https://doi.org/10.1111/jopy.12640}

\leavevmode\hypertarget{ref-asselmannTestingSocialInvestment2020}{}%
Asselmann, E., \& Specht, J. (2020b). Testing the Social Investment Principle Around Childbirth: Little Evidence for Personality Maturation Before and After Becoming a Parent. \emph{European Journal of Personality}, Advance Online Publication. \url{https://doi.org/10.1002/per.2269}

\leavevmode\hypertarget{ref-atesDoesGrandchildCare2017}{}%
Ates, M. (2017). Does grandchild care influence grandparents' self-rated health? Evidence from a fixed effects approach. \emph{Social Science \& Medicine}, \emph{190}, 67--74. \url{https://doi.org/10.1016/j.socscimed.2017.08.021}

\leavevmode\hypertarget{ref-R-papaja}{}%
Aust, F., \& Barth, M. (2020). \emph{papaja: Prepare reproducible APA journal articles with R Markdown}. \url{https://github.com/crsh/papaja}

\leavevmode\hypertarget{ref-austinIntroductionPropensityScore2011}{}%
Austin, P. C. (2011). An introduction to propensity score methods for reducing the effects of confounding in observational studies. \emph{Multivariate Behavioral Research}, \emph{46}(3), 399--424. \url{https://doi.org/10.1080/00273171.2011.568786}

\leavevmode\hypertarget{ref-austinDoublePropensityscoreAdjustment2017}{}%
Austin, P. C. (2017). Double propensity-score adjustment: A solution to design bias or bias due to incomplete matching. \emph{Statistical Methods in Medical Research}, \emph{26}(1), 201--222. \url{https://doi.org/10.1177/0962280214543508}

\leavevmode\hypertarget{ref-bairdLifeSatisfactionLifespan2010}{}%
Baird, B. M., Lucas, R. E., \& Donnellan, M. B. (2010). Life satisfaction across the lifespan: Findings from two nationally representative panel studies. \emph{Social Indicators Research}, \emph{99}(2), 183--203. \url{https://doi.org/10.1007/s11205-010-9584-9}

\leavevmode\hypertarget{ref-balboRoleFamilyOrientations2016}{}%
Balbo, N., \& Arpino, B. (2016). The role of family orientations in shaping the effect of fertility on subjective well-being: A propensity score matching approach. \emph{Demography}, \emph{53}(4), 955--978. \url{https://doi.org/10.1007/s13524-016-0480-z}

\leavevmode\hypertarget{ref-baltesLifeSpanTheory2006}{}%
Baltes, P. B., Lindenberger, U., \& Staudinger, U. M. (2006). Life Span Theory in Developmental Psychology. In R. M. Lerner \& W. Damon (Eds.), \emph{Handbook of child psychology: Theoretical models of human development} (pp. 569--664). John Wiley \& Sons Inc.

\leavevmode\hypertarget{ref-R-lme4}{}%
Bates, D., Mächler, M., Bolker, B., \& Walker, S. (2015). Fitting linear mixed-effects models using lme4. \emph{Journal of Statistical Software}, \emph{67}(1), 1--48. \url{https://doi.org/10.18637/jss.v067.i01}

\leavevmode\hypertarget{ref-beckMegaAnalysisPersonalityPrediction2021}{}%
Beck, E. D., \& Jackson, J. J. (2021). A Mega-Analysis of Personality Prediction: Robustness and Boundary Conditions. \emph{Journal of Personality and Social Psychology}, \emph{In Press}. \url{https://doi.org/10.31234/osf.io/7pg9b}

\leavevmode\hypertarget{ref-bengtsonNuclearFamilyIncreasing2001}{}%
Bengtson, V. L. (2001). Beyond the Nuclear Family: The Increasing Importance of Multigenerational Bonds. \emph{Journal of Marriage and Family}, \emph{63}(1), 1--16. \url{https://doi.org/10.1111/j.1741-3737.2001.00001.x}

\leavevmode\hypertarget{ref-benjaminRedefineStatisticalSignificance2018}{}%
Benjamin, D. J., Berger, J. O., Clyde, M., Wolpert, R. L., Johnson, V. E., Johannesson, M., Dreber, A., Nosek, B. A., Wagenmakers, E. J., Berk, R., \& Brembs, B. (2018). Redefine statistical significance. \emph{Nature Human Behavior}, \emph{2}, 6--10. \url{https://doi.org/10.1038/s41562-017-0189-z}

\leavevmode\hypertarget{ref-bleidornPersonalityTraitStability2021}{}%
Bleidorn, W., Hopwood, C. J., Back, M. D., Denissen, J. J. A., Hennecke, M., Hill, P. L., Jokela, M., Kandler, C., Lucas, R. E., Luhmann, M., Orth, U., Roberts, B. W., Wagner, J., Wrzus, C., \& Zimmermann, J. (2021). Personality Trait Stability and Change. \emph{Personality Science}, \emph{2}(1), 1--20. \url{https://doi.org/10.5964/ps.6009}

\leavevmode\hypertarget{ref-bleidornLifeEventsPersonality2018}{}%
Bleidorn, W., Hopwood, C. J., \& Lucas, R. E. (2018). Life events and personality trait change. \emph{Journal of Personality}, \emph{86}(1), 83--96. \url{https://doi.org/10.1111/jopy.12286}

\leavevmode\hypertarget{ref-bleidornPersonalityMaturationWorld2013}{}%
Bleidorn, W., Klimstra, T. A., Denissen, J. J. A., Rentfrow, P. J., Potter, J., \& Gosling, S. D. (2013). Personality Maturation Around the World: A Cross-Cultural Examination of Social-Investment Theory. \emph{Psychological Science}, \emph{24}(12), 2530--2540. \url{https://doi.org/10.1177/0956797613498396}

\leavevmode\hypertarget{ref-bleidornRetirementAssociatedChange2018}{}%
Bleidorn, W., \& Schwaba, T. (2018). Retirement is associated with change in self-esteem. \emph{Psychology and Aging}, \emph{33}(4), 586--594. \url{https://doi.org/10.1037/pag0000253}

\leavevmode\hypertarget{ref-bleidornPersonalityDevelopmentEmerging2017}{}%
Bleidorn, W., \& Schwaba, T. (2017). Personality development in emerging adulthood. In J. Specht (Ed.), \emph{Personality Development Across the Lifespan} (pp. 39--51). Academic Press. \url{https://doi.org/10.1016/B978-0-12-804674-6.00004-1}

\leavevmode\hypertarget{ref-bordonePatternsGrandparentalChild2017}{}%
Bordone, V., Arpino, B., \& Aassve, A. (2017). Patterns of grandparental child care across Europe: The role of the policy context and working mothers' need. \emph{Ageing and Society}, \emph{37}(4), 845--873. \url{https://doi.org/10.1017/S0144686X1600009X}

\leavevmode\hypertarget{ref-bruderlFixedEffectsPanelRegression2015}{}%
Brüderl, J., \& Ludwig, V. (2015). \emph{Fixed-Effects Panel Regression} (H. Best \& C. Wolf, Eds.). SAGE.

\leavevmode\hypertarget{ref-buchananTwentyfirstCenturyGrandparents2018}{}%
Buchanan, A., \& Rotkirch, A. (2018). Twenty-first century grandparents: Global perspectives on changing roles and consequences. \emph{Contemporary Social Science}, \emph{13}(2), 131--144. \url{https://doi.org/10.1080/21582041.2018.1467034}

\leavevmode\hypertarget{ref-burgetteMultipleImputationMissing2010}{}%
Burgette, L. F., \& Reiter, J. P. (2010). Multiple Imputation for Missing Data via Sequential Regression Trees. \emph{American Journal of Epidemiology}, \emph{172}(9), 1070--1076. \url{https://doi.org/10.1093/aje/kwq260}

\leavevmode\hypertarget{ref-caspiWhenIndividualDifferences1993}{}%
Caspi, A., \& Moffitt, T. E. (1993). When do individual differences matter? A paradoxical theory of personality coherence. \emph{Psychological Inquiry}, \emph{4}(4), 247--271. \url{https://doi.org/10.1207/s15327965pli0404_1}

\leavevmode\hypertarget{ref-choiGrandparentingMortalityHow2020}{}%
Choi, S.-w. E. (2020). Grandparenting and Mortality: How Does Race-Ethnicity Matter? \emph{Journal of Health and Social Behavior}, \emph{61}(1), 96--112. \url{https://doi.org/10.1177/0022146520903282}

\leavevmode\hypertarget{ref-chopikDoesPersonalityChange2018}{}%
Chopik, W. J. (2018). Does personality change following spousal bereavement? \emph{Journal of Research in Personality}, \emph{72}, 10--21. \url{https://doi.org/10.1016/j.jrp.2016.08.010}

\leavevmode\hypertarget{ref-chopikPersonalityChangeLife2018}{}%
Chopik, W. J., \& Kitayama, S. (2018). Personality change across the life span: Insights from a cross-cultural, longitudinal study. \emph{Journal of Personality}, \emph{86}(3), 508--521. \url{https://doi.org/10.1111/jopy.12332}

\leavevmode\hypertarget{ref-christiansenAssociationGrandparenthoodMortality2014}{}%
Christiansen, S. G. (2014). The association between grandparenthood and mortality. \emph{Social Science \& Medicine}, \emph{118}, 89--96. \url{https://doi.org/10.1016/j.socscimed.2014.07.061}

\leavevmode\hypertarget{ref-chungLongitudinalEffectsGrandchild2018}{}%
Chung, S., \& Park, A. (2018). The longitudinal effects of grandchild care on depressive symptoms and physical health of grandmothers in South Korea: A latent growth approach. \emph{Aging \& Mental Health}, \emph{22}(12), 1556--1563. \url{https://doi.org/10.1080/13607863.2017.1376312}

\leavevmode\hypertarget{ref-coallGrandparentalInvestmentRelic2011}{}%
Coall, D. A., \& Hertwig, R. (2011). Grandparental Investment: A Relic of the Past or a Resource for the Future? \emph{Current Directions in Psychological Science}, \emph{20}(2), 93--98. \url{https://doi.org/10.1177/0963721411403269}

\leavevmode\hypertarget{ref-coallNewNicheTheory2016}{}%
Coall, D. A., Hilbrand, S., Sear, R., \& Hertwig, R. (2016). A New Niche? The Theory of Grandfather Involvement. In A. Buchanan \& A. Rotkirch (Eds.), \emph{Grandfathers: Global Perspectives} (pp. 21--44). Palgrave Macmillan UK. \url{https://doi.org/10.1057/978-1-137-56338-5_2}

\leavevmode\hypertarget{ref-coallInterdisciplinaryPerspectivesGrandparental2018}{}%
Coall, D. A., Hilbrand, S., Sear, R., \& Hertwig, R. (2018). Interdisciplinary perspectives on grandparental investment: A journey towards causality. \emph{Contemporary Social Science}, \emph{13}(2), 159--174. \url{https://doi.org/10.1080/21582041.2018.1433317}

\leavevmode\hypertarget{ref-condonAustralianFirsttimeGrandparents2013}{}%
Condon, J., Corkindale, C., Luszcz, M., \& Gamble, E. (2013). The Australian First-time Grandparents Study: Time spent with the grandchild and its predictors. \emph{Australasian Journal on Ageing}, \emph{32}(1), 21--27. \url{https://doi.org/10.1111/j.1741-6612.2011.00588.x}

\leavevmode\hypertarget{ref-condonTransitionGrandparenthoodProspective2018}{}%
Condon, J., Luszcz, M., \& McKee, I. (2018). The transition to grandparenthood: A prospective study of mental health implications. \emph{Aging \& Mental Health}, \emph{22}(3), 336--343. \url{https://doi.org/10.1080/13607863.2016.1248897}

\leavevmode\hypertarget{ref-cookHowMuchBias2020}{}%
Cook, T. D., Zhu, N., Klein, A., Starkey, P., \& Thomas, J. (2020). How much bias results if a quasi-experimental design combines local comparison groups, a pretest outcome measure and other covariates?: A within study comparison of preschool effects. \emph{Psychological Methods}, Advance Online Publication. \url{https://doi.org/10.1037/met0000260}

\leavevmode\hypertarget{ref-costaPersonalityLifeSpan2019}{}%
Costa, P. T., McCrae, R. R., \& Löckenhoff, C. E. (2019). Personality Across the Life Span. \emph{Annual Review of Psychology}, \emph{70}(1), 423--448. \url{https://doi.org/10.1146/annurev-psych-010418-103244}

\leavevmode\hypertarget{ref-damianSixteenGoingSixtysix2019}{}%
Damian, R. I., Spengler, M., Sutu, A., \& Roberts, B. W. (2019). Sixteen going on sixty-six: A longitudinal study of personality stability and change across 50 years. \emph{Journal of Personality and Social Psychology}, \emph{117}(3), 674--695. \url{https://doi.org/10.1037/pspp0000210}

\leavevmode\hypertarget{ref-danielsbackaAssociationGrandparentalInvestment2016}{}%
Danielsbacka, M., \& Tanskanen, A. O. (2016). The association between grandparental investment and grandparents' happiness in Finland. \emph{Personal Relationships}, \emph{23}(4), 787--800. \url{https://doi.org/10.1111/pere.12160}

\leavevmode\hypertarget{ref-danielsbackaGrandparentalChildcareHealth2019}{}%
Danielsbacka, M., Tanskanen, A. O., Coall, D. A., \& Jokela, M. (2019). Grandparental childcare, health and well-being in Europe: A within-individual investigation of longitudinal data. \emph{Social Science \& Medicine}, \emph{230}, 194--203. \url{https://doi.org/10.1016/j.socscimed.2019.03.031}

\leavevmode\hypertarget{ref-danielsbackaGrandparentalChildCare2011}{}%
Danielsbacka, M., Tanskanen, A. O., Jokela, M., \& Rotkirch, A. (2011). Grandparental Child Care in Europe: Evidence for Preferential Investment in More Certain Kin. \emph{Evolutionary Psychology}, \emph{9}(1), 147470491100900102. \url{https://doi.org/10.1177/147470491100900102}

\leavevmode\hypertarget{ref-denissenBigFiveInventory2020}{}%
Denissen, J. J. A., Geenen, R., Soto, C. J., John, O. P., \& van Aken, M. A. G. (2020). The Big Five Inventory2: Replication of Psychometric Properties in a Dutch Adaptation and First Evidence for the Discriminant Predictive Validity of the Facet Scales. \emph{Journal of Personality Assessment}, \emph{102}(3), 309--324. \url{https://doi.org/10.1080/00223891.2018.1539004}

\leavevmode\hypertarget{ref-denissenTransactionsLifeEvents2019}{}%
Denissen, J. J. A., Luhmann, M., Chung, J. M., \& Bleidorn, W. (2019). Transactions between life events and personality traits across the adult lifespan. \emph{Journal of Personality and Social Psychology}, \emph{116}(4), 612--633. \url{https://doi.org/10.1037/pspp0000196}

\leavevmode\hypertarget{ref-dienerSatisfactionLifeScale1985}{}%
Diener, E., Emmons, R. A., Larsen, R. J., \& Griffin, S. (1985). The Satisfaction With Life Scale. \emph{Journal of Personality Assessment}, \emph{49}(1), 71--75. \url{https://doi.org/10.1207/s15327752jpa4901_13}

\leavevmode\hypertarget{ref-digessaBecomingGrandparentIts2019}{}%
Di Gessa, G., Bordone, V., \& Arpino, B. (2019). Becoming a Grandparent and Its Effect on Well-Being: The Role of Order of Transitions, Time, and Gender. \emph{The Journals of Gerontology, Series B: Psychological Sciences and Social Sciences}, Advance Online Publication. \url{https://doi.org/10.1093/geronb/gbz135}

\leavevmode\hypertarget{ref-digessaHealthImpactIntensive2016}{}%
Di Gessa, G., Glaser, K., \& Tinker, A. (2016a). The Health Impact of Intensive and Nonintensive Grandchild Care in Europe: New Evidence From SHARE. \emph{The Journals of Gerontology, Series B: Psychological Sciences and Social Sciences}, \emph{71}(5), 867--879. \url{https://doi.org/10.1093/geronb/gbv055}

\leavevmode\hypertarget{ref-digessaImpactCaringGrandchildren2016}{}%
Di Gessa, G., Glaser, K., \& Tinker, A. (2016b). The impact of caring for grandchildren on the health of grandparents in Europe: A lifecourse approach. \emph{Social Science \& Medicine}, \emph{152}, 166--175. \url{https://doi.org/10.1016/j.socscimed.2016.01.041}

\leavevmode\hypertarget{ref-digessaLookingGrandchildrenGender2020}{}%
Di Gessa, G., Zaninotto, P., \& Glaser, K. (2020). Looking after grandchildren: Gender differences in ``when,'' ``what,'' and ``why'': Evidence from the English Longitudinal Study of Ageing. \emph{Demographic Research}, \emph{43}(53), 1545--1562. \url{https://doi.org/10.4054/DemRes.2020.43.53}

\leavevmode\hypertarget{ref-dorePopulationIndividuallevelChanges2018}{}%
Doré, B., \& Bolger, N. (2018). Population- and individual-level changes in life satisfaction surrounding major life stressors. \emph{Social Psychological and Personality Science}, \emph{9}(7), 875--884. \url{https://doi.org/10.1177/1948550617727589}

\leavevmode\hypertarget{ref-eidScienceSubjectiveWellbeing2008}{}%
Eid, M., \& Larsen, R. J. (2008). \emph{The science of subjective well-being}. Guilford Press.

\leavevmode\hypertarget{ref-ellwardtGrandparenthoodRiskMortality2021}{}%
Ellwardt, L., Hank, K., \& Mendes de Leon, C. F. (2021). Grandparenthood and risk of mortality: Findings from the Health and Retirement Study. \emph{Social Science \& Medicine}, \emph{268}, 113371. \url{https://doi.org/10.1016/j.socscimed.2020.113371}

\leavevmode\hypertarget{ref-elwertEndogenousSelectionBias2014}{}%
Elwert, F., \& Winship, C. (2014). Endogenous Selection Bias: The Problem of Conditioning on a Collider Variable. \emph{Annual Review of Sociology}, \emph{40}(1), 31--53. \url{https://doi.org/10.1146/annurev-soc-071913-043455}

\leavevmode\hypertarget{ref-findlerConstructionValidationMultidimensional2013}{}%
Findler, L., Taubman - Ben-Ari, O., Nuttman-Shwartz, O., \& Lazar, R. (2013). Construction and Validation of the Multidimensional Experience of Grandparenthood Set of Inventories. \emph{Social Work Research}, \emph{37}(3), 237--253. \url{https://doi.org/10.1093/swr/svt025}

\leavevmode\hypertarget{ref-fingermanDecadeResearchIntergenerational2020}{}%
Fingerman, K. L., Huo, M., \& Birditt, K. S. (2020). A Decade of Research on Intergenerational Ties: Technological, Economic, Political, and Demographic Changes. \emph{Journal of Marriage and Family}, \emph{82}(1), 383--403. \url{https://doi.org/10.1111/jomf.12604}

\leavevmode\hypertarget{ref-car2019}{}%
Fox, J., \& Weisberg, S. (2019). \emph{An R companion to applied regression} (Third). Sage.

\leavevmode\hypertarget{ref-goldbergDevelopmentMarkersBigFive1992}{}%
Goldberg, L. R. (1992). The development of markers for the Big-Five factor structure. \emph{Psychological Assessment}, \emph{4}(1), 26--42. \url{https://doi.org/10.1037/1040-3590.4.1.26}

\leavevmode\hypertarget{ref-golleSchoolWorkChoice2019}{}%
Golle, J., Rose, N., Göllner, R., Spengler, M., Stoll, G., Hübner, N., Rieger, S., Trautwein, U., Lüdtke, O., Roberts, B. W., \& Nagengast, B. (2019). School or Work? The Choice May Change Your Personality. \emph{Psychological Science}, \emph{30}(1), 32--42. \url{https://doi.org/10.1177/0956797618806298}

\leavevmode\hypertarget{ref-gotzSmallEffectsIndispensable2021}{}%
Götz, F. M., Gosling, S. D., \& Rentfrow, P. J. (2021). Small Effects: The Indispensable Foundation for a Cumulative Psychological Science. \emph{Perspectives on Psychological Science}, Advance Online Publication. \url{https://doi.org/10.1177/1745691620984483}

\leavevmode\hypertarget{ref-grahamTrajectoriesBigFive2020}{}%
Graham, E. K., Weston, S. J., Gerstorf, D., Yoneda, T. B., Booth, T., Beam, C. R., Petkus, A. J., Drewelies, J., Hall, A. N., Bastarache, E. D., Estabrook, R., Katz, M. J., Turiano, N. A., Lindenberger, U., Smith, J., Wagner, G. G., Pedersen, N. L., Allemand, M., Spiro Iii, A., \ldots{} Mroczek, D. K. (2020). Trajectories of Big Five Personality Traits: A Coordinated Analysis of 16 Longitudinal Samples. \emph{European Journal of Personality}, Advance Online Publication. \url{https://doi.org/10.1002/per.2259}

\leavevmode\hypertarget{ref-greenlandQuantifyingBiasesCausal2003}{}%
Greenland, S. (2003). Quantifying biases in causal models: Classical confounding vs collider-stratification bias. \emph{Epidemiology}, \emph{14}(3), 300--306. \url{https://doi.org/10.1097/01.EDE.0000042804.12056.6C}

\leavevmode\hypertarget{ref-greenlandCriticalLookMethods1995}{}%
Greenland, S., \& Finkle, W. D. (1995). A Critical Look at Methods for Handling Missing Covariates in Epidemiologic Regression Analyses. \emph{American Journal of Epidemiology}, \emph{142}(12), 1255--1264. \url{https://doi.org/10.1093/oxfordjournals.aje.a117592}

\leavevmode\hypertarget{ref-grimmSecondorderGrowthMixture2009}{}%
Grimm, K. J., \& Ram, N. (2009). A second-order growth mixture model for developmental research. \emph{Research in Human Development}, \emph{6}(2-3), 121--143. \url{https://doi.org/10.1080/15427600902911221}

\leavevmode\hypertarget{ref-hagestadAgeLifeCourse1985}{}%
Hagestad, G. O., \& Neugarten, B. L. (1985). Age and the life course. In E. Shanas \& R. Binstock (Eds.), \emph{Handbook of aging and the social sciences}. Van Nostrand and Reinhold.

\leavevmode\hypertarget{ref-hallbergPretestMeasuresStudy2018}{}%
Hallberg, K., Cook, T. D., Steiner, P. M., \& Clark, M. H. (2018). Pretest Measures of the Study Outcome and the Elimination of Selection Bias: Evidence from Three Within Study Comparisons. \emph{Prevention Science}, \emph{19}(3), 274--283. \url{https://doi.org/10.1007/s11121-016-0732-6}

\leavevmode\hypertarget{ref-hankGrandparentsCaringTheir2009}{}%
Hank, K., \& Buber, I. (2009). Grandparents Caring for their Grandchildren: Findings From the 2004 Survey of Health, Ageing, and Retirement in Europe. \emph{Journal of Family Issues}, \emph{30}(1), 53--73. \url{https://doi.org/10.1177/0192513X08322627}

\leavevmode\hypertarget{ref-hayslipGrandparentsRaisingGrandchildren2019}{}%
Hayslip, B., Jr, Fruhauf, C. A., \& Dolbin-MacNab, M. L. (2019). Grandparents Raising Grandchildren: What Have We Learned Over the Past Decade? \emph{The Gerontologist}, \emph{59}(3), e152--e163. \url{https://doi.org/10.1093/geront/gnx106}

\leavevmode\hypertarget{ref-hentschelInfluenceMajorLife2017}{}%
Hentschel, S., Eid, M., \& Kutscher, T. (2017). The Influence of Major Life Events and Personality Traits on the Stability of Affective Well-Being. \emph{Journal of Happiness Studies}, \emph{18}(3), 719--741. \url{https://doi.org/10.1007/s10902-016-9744-y}

\leavevmode\hypertarget{ref-hilbrandCaregivingFamilyAssociated2017}{}%
Hilbrand, S., Coall, D. A., Gerstorf, D., \& Hertwig, R. (2017). Caregiving within and beyond the family is associated with lower mortality for the caregiver: A prospective study. \emph{Evolution and Human Behavior}, \emph{38}(3), 397--403. \url{https://doi.org/10.1016/j.evolhumbehav.2016.11.010}

\leavevmode\hypertarget{ref-MatchIt2011}{}%
Ho, D. E., Imai, K., King, G., \& Stuart, E. A. (2011). MatchIt: Nonparametric preprocessing for parametric causal inference. \emph{Journal of Statistical Software}, \emph{42}(8), 1--28.

\leavevmode\hypertarget{ref-hoffmanLongitudinalAnalysisModeling2015}{}%
Hoffman, L. (2015). \emph{Longitudinal analysis: Modeling within-person fluctuation and change}. Routledge/Taylor \& Francis Group.

\leavevmode\hypertarget{ref-huttemanDevelopmentalTasksFramework2014}{}%
Hutteman, R., Hennecke, M., Orth, U., Reitz, A. K., \& Specht, J. (2014). Developmental Tasks as a Framework to Study Personality Development in Adulthood and Old Age. \emph{European Journal of Personality}, \emph{28}(3), 267--278. \url{https://doi.org/10.1002/per.1959}

\leavevmode\hypertarget{ref-infurnaUtilizingPrinciplesLifeSpan2021}{}%
Infurna, F. J. (2021). Utilizing Principles of Life-Span Developmental Psychology to Study the Complexities of Resilience Across the Adult Life Span. \emph{The Gerontologist}, \emph{61}(6), 807--818. \url{https://doi.org/10.1093/geront/gnab086}

\leavevmode\hypertarget{ref-infurnaMidlife2020sOpportunities2020}{}%
Infurna, F. J., Gerstorf, D., \& Lachman, M. E. (2020). Midlife in the 2020s: Opportunities and challenges. \emph{American Psychologist}, \emph{75}(4), 470--485. \url{https://doi.org/10.1037/amp0000591}

\leavevmode\hypertarget{ref-jacksonPersonalityDevelopmentMean2021}{}%
Jackson, J. J., \& Beck, E. D. (2021). Personality Development Beyond the Mean: Do Life Events Shape Personality Variability, Structure, and Ipsative Continuity? \emph{The Journals of Gerontology: Series B}, \emph{76}(1), 20--30. \url{https://doi.org/10.1093/geronb/gbaa093}

\leavevmode\hypertarget{ref-johnParadigmShiftIntegrative2008}{}%
John, O. P., Naumann, L. P., \& Soto, C. J. (2008). Paradigm shift to the integrative Big Five trait taxonomy: History, measurement, and conceptual issues. In O. P. John, R. W. Robins, \& L. A. Pervin (Eds.), \emph{Handbook of personality: Theory and research} (pp. 114--158). The Guilford Press.

\leavevmode\hypertarget{ref-johnBigFiveTrait1999}{}%
John, O. P., \& Srivastava, S. (1999). The Big Five Trait taxonomy: History, measurement, and theoretical perspectives. In L. A. Pervin \& O. P. John (Eds.), \emph{Handbook of personality: Theory and research, 2nd ed.} (pp. 102--138). Guilford Press.

\leavevmode\hypertarget{ref-johnsonImpactHavingChildren2006}{}%
Johnson, A. B., \& Rodgers, J. L. (2006). The impact of having children on the lives of women: The Effects of Children Questionnaire. \emph{Journal of Applied Social Psychology}, \emph{36}(11), 2685--2714. \url{https://doi.org/10.1111/j.0021-9029.2006.00123.x}

\leavevmode\hypertarget{ref-kandlerPatternsSourcesPersonality2015a}{}%
Kandler, C., Kornadt, A. E., Hagemeyer, B., \& Neyer, F. J. (2015). Patterns and sources of personality development in old age. \emph{Journal of Personality and Social Psychology}, \emph{109}(1), 175--191. \url{https://doi.org/10.1037/pspp0000028}

\leavevmode\hypertarget{ref-kramerImpactHavingChildren2020}{}%
Krämer, M. D., \& Rodgers, J. L. (2020). The impact of having children on domain-specific life satisfaction: A quasi-experimental longitudinal investigation using the Socio-Economic Panel (SOEP) data. \emph{Journal of Personality and Social Psychology}, \emph{119}(6), 1497--1514. \url{https://doi.org/10.1037/pspp0000279}

\leavevmode\hypertarget{ref-kritzlerHowAreCommon2021}{}%
Kritzler, S., Rakhshani, A., Terwiel, S., Fassbender, I., Donnellan, B., Lucas, R. E., \& Luhmann, M. (2021). How Are Common Major Life Events Perceived? Exploring Differences Between and Variability of Different Typical Event Profiles and Raters. \emph{PsyArXiv}. \url{https://doi.org/10.31234/osf.io/fncz3}

\leavevmode\hypertarget{ref-R-lmerTest}{}%
Kuznetsova, A., Brockhoff, P. B., \& Christensen, R. H. B. (2017). lmerTest package: Tests in linear mixed effects models. \emph{Journal of Statistical Software}, \emph{82}(13), 1--26. \url{https://doi.org/10.18637/jss.v082.i13}

\leavevmode\hypertarget{ref-lachmanMidlifeDevelopmentInventory1997}{}%
Lachman, M. E., \& Weaver, S. L. (1997). \emph{The Midlife Development Inventory (MIDI) personality scales: Scale construction and scoring}. Brandeis University.

\leavevmode\hypertarget{ref-leopoldDemographyGrandparenthoodInternational2015}{}%
Leopold, T., \& Skopek, J. (2015). The Demography of Grandparenthood: An International Profile. \emph{Social Forces}, \emph{94}(2), 801--832. \url{https://doi.org/10.1093/sf/sov066}

\leavevmode\hypertarget{ref-liUsingPropensityScore2013}{}%
Li, M. (2013). Using the Propensity Score Method to Estimate Causal Effects: A Review and Practical Guide. \emph{Organizational Research Methods}, \emph{16}(2), 188--226. \url{https://doi.org/10.1177/1094428112447816}

\leavevmode\hypertarget{ref-lodi-smithSocialInvestmentPersonality2007}{}%
Lodi-Smith, J., \& Roberts, B. W. (2007). Social Investment and Personality: A Meta-Analysis of the Relationship of Personality Traits to Investment in Work, Family, Religion, and Volunteerism. \emph{Personality and Social Psychology Review}, \emph{11}(1), 68--86. \url{https://doi.org/10.1177/1088868306294590}

\leavevmode\hypertarget{ref-luanYouSeeMy2017}{}%
Luan, Z., Hutteman, R., Denissen, J. J. A., Asendorpf, J. B., \& van Aken, M. A. G. (2017). Do you see my growth? Two longitudinal studies on personality development from childhood to young adulthood from multiple perspectives. \emph{Journal of Research in Personality}, \emph{67}, 44--60. \url{https://doi.org/10.1016/j.jrp.2016.03.004}

\leavevmode\hypertarget{ref-lucasPersonalityDevelopmentLife2011}{}%
Lucas, R. E., \& Donnellan, M. B. (2011). Personality development across the life span: Longitudinal analyses with a national sample from Germany. \emph{Journal of Personality and Social Psychology}, \emph{101}(4), 847--861. \url{https://doi.org/10.1037/a0024298}

\leavevmode\hypertarget{ref-luhmannDimensionalTaxonomyPerceived2020}{}%
Luhmann, M., Fassbender, I., Alcock, M., \& Haehner, P. (2020). A dimensional taxonomy of perceived characteristics of major life events. \emph{Journal of Personality and Social Psychology}, Advance Online Publication. \url{https://doi.org/10.1037/pspp0000291}

\leavevmode\hypertarget{ref-luhmannSubjectiveWellbeingAdaptation2012}{}%
Luhmann, M., Hofmann, W., Eid, M., \& Lucas, R. E. (2012). Subjective well-being and adaptation to life events: A meta-analysis. \emph{Journal of Personality and Social Psychology}, \emph{102}(3), 592--615. \url{https://doi.org/10.1037/a0025948}

\leavevmode\hypertarget{ref-luhmannStudyingChangesLife2014}{}%
Luhmann, M., Orth, U., Specht, J., Kandler, C., \& Lucas, R. E. (2014). Studying changes in life circumstances and personality: It's about time. \emph{European Journal of Personality}, \emph{28}(3), 256--266. \url{https://doi.org/10.1002/per.1951}

\leavevmode\hypertarget{ref-lumsdaineRetirementTimingWomen2015}{}%
Lumsdaine, R. L., \& Vermeer, S. J. C. (2015). Retirement timing of women and the role of care responsibilities for grandchildren. \emph{Demography}, \emph{52}(2), 433--454. \url{https://doi.org/10.1007/s13524-015-0382-5}

\leavevmode\hypertarget{ref-ludtkeRandomWalkUniversity2011}{}%
Lüdtke, O., Roberts, B. W., Trautwein, U., \& Nagy, G. (2011). A random walk down university avenue: Life paths, life events, and personality trait change at the transition to university life. \emph{Journal of Personality and Social Psychology}, \emph{101}(3), 620--637. \url{https://doi.org/10.1037/a0023743}

\leavevmode\hypertarget{ref-maccallumPracticeDichotomizationQuantitative2002}{}%
MacCallum, R. C., Zhang, S., Preacher, K. J., \& Rucker, D. D. (2002). On the practice of dichotomization of quantitative variables. \emph{Psychological Methods}, \emph{7}(1), 19--40. \url{https://doi.org/10.1037/1082-989X.7.1.19}

\leavevmode\hypertarget{ref-mahneGrandparenthoodSubjectiveWellBeing2014}{}%
Mahne, K., \& Huxhold, O. (2014). Grandparenthood and Subjective Well-Being: Moderating Effects of Educational Level. \emph{The Journals of Gerontology: Series B}, \emph{70}(5), 782--792. \url{https://doi.org/10.1093/geronb/gbu147}

\leavevmode\hypertarget{ref-mannOutShadowsGrandfatherhood2007}{}%
Mann, R. (2007). Out of the shadows?: Grandfatherhood, age and masculinities. \emph{Masculinity and Aging}, \emph{21}(4), 281--291. \url{https://doi.org/10.1016/j.jaging.2007.05.008}

\leavevmode\hypertarget{ref-mannGrandfathersContemporaryFamilies2010}{}%
Mann, R., \& Leeson, G. (2010). Grandfathers in Contemporary Families in Britain: Evidence from Qualitative Research. \emph{Journal of Intergenerational Relationships}, \emph{8}(3), 234--248. \url{https://doi.org/10.1080/15350770.2010.498774}

\leavevmode\hypertarget{ref-margolisCohortPerspectiveDemography2019}{}%
Margolis, R., \& Verdery, A. M. (2019). A Cohort Perspective on the Demography of Grandparenthood: Past, Present, and Future Changes in Race and Sex Disparities in the United States. \emph{Demography}, \emph{56}(4), 1495--1518. \url{https://doi.org/10.1007/s13524-019-00795-1}

\leavevmode\hypertarget{ref-margolisHealthyGrandparenthoodHow2017}{}%
Margolis, R., \& Wright, L. (2017). Healthy Grandparenthood: How Long Is It, and How Has It Changed? \emph{Demography}, \emph{54}(6), 2073--2099. \url{https://doi.org/10.1007/s13524-017-0620-0}

\leavevmode\hypertarget{ref-marshMeasurementInvarianceBigfive2013}{}%
Marsh, H. W., Nagengast, B., \& Morin, A. J. S. (2013). Measurement invariance of big-five factors over the life span: ESEM tests of gender, age, plasticity, maturity, and la dolce vita effects. \emph{Developmental Psychology}, \emph{49}(6), 1194--1218. \url{https://doi.org/10.1037/a0026913}

\leavevmode\hypertarget{ref-mccraeModeratedAnalysesLongitudinal1993}{}%
McCrae, R. R. (1993). Moderated analyses of longitudinal personality stability. \emph{Journal of Personality and Social Psychology}, \emph{65}(3), 577--585. \url{https://doi.org/10.1037/0022-3514.65.3.577}

\leavevmode\hypertarget{ref-mccraeMethodBiasesSinglesource2018}{}%
McCrae, R. R. (2018). Method biases in single-source personality assessments. \emph{Psychological Assessment}, \emph{30}(9), 1160--1173. \url{https://doi.org/10.1037/pas0000566}

\leavevmode\hypertarget{ref-mccraeWhatPersonalityScales2019}{}%
McCrae, R. R., \& Mõttus, R. (2019). What personality scales measure: A new psychometrics and its implications for theory and assessment. \emph{Current Directions in Psychological Science}, \emph{28}(4), 415--420. \url{https://doi.org/10.1177/0963721419849559}

\leavevmode\hypertarget{ref-mcneishThanksCoefficientAlpha2018}{}%
McNeish, D. (2018). Thanks coefficient alpha, we'll take it from here. \emph{Psychological Methods}, \emph{23}(3), 412--433. \url{https://doi.org/10.1037/met0000144}

\leavevmode\hypertarget{ref-mcneishFixedEffectsModels2019}{}%
McNeish, D., \& Kelley, K. (2019). Fixed effects models versus mixed effects models for clustered data: Reviewing the approaches, disentangling the differences, and making recommendations. \emph{Psychological Methods}, \emph{24}(1), 20--35. \url{https://doi.org/10.1037/met0000182}

\leavevmode\hypertarget{ref-meyerGrandparentingUnitedStates2017}{}%
Meyer, M. H., \& Kandic, A. (2017). Grandparenting in the United States. \emph{Innovation in Aging}, \emph{1}(2), 1--10. \url{https://doi.org/10.1093/geroni/igx023}

\leavevmode\hypertarget{ref-mitraComparisonTwoMethods2016}{}%
Mitra, R., \& Reiter, J. P. (2016). A comparison of two methods of estimating propensity scores after multiple imputation. \emph{Statistical Methods in Medical Research}, \emph{25}(1), 188--204. \url{https://doi.org/10.1177/0962280212445945}

\leavevmode\hypertarget{ref-mottusSelfReportsInformantRatingsMeasure2019}{}%
Mõttus, R., Allik, J., \& Realo, A. (2019). Do Self-Reports and Informant-Ratings Measure the Same Personality Constructs? \emph{European Journal of Psychological Assessment}, 1--7. \url{https://doi.org/10.1027/1015-5759/a000516}

\leavevmode\hypertarget{ref-mottusPersonalityTraitsOld2012}{}%
Mõttus, R., Johnson, W., \& Deary, I. J. (2012). Personality traits in old age: Measurement and rank-order stability and some mean-level change. \emph{Psychology and Aging}, \emph{27}(1), 243--249. \url{https://doi.org/10.1037/a0023690}

\leavevmode\hypertarget{ref-mottusDevelopmentDetailsAge2021}{}%
Mõttus, R., \& Rozgonjuk, D. (2021). Development is in the details: Age differences in the Big Five domains, facets, and nuances. \emph{Journal of Personality and Social Psychology}, \emph{120}(4), 1035--1048. \url{https://doi.org/10.1037/pspp0000276}

\leavevmode\hypertarget{ref-muellerPersonalityDevelopmentOld2016}{}%
Mueller, S., Wagner, J., Drewelies, J., Duezel, S., Eibich, P., Specht, J., Demuth, I., Steinhagen-Thiessen, E., Wagner, G. G., \& Gerstorf, D. (2016). Personality development in old age relates to physical health and cognitive performance: Evidence from the Berlin Aging Study II. \emph{Journal of Research in Personality}, \emph{65}, 94--108. \url{https://doi.org/10.1016/j.jrp.2016.08.007}

\leavevmode\hypertarget{ref-mullerGrandparentingWellbeingHow2011}{}%
Muller, Z., \& Litwin, H. (2011). Grandparenting and well-being: How important is grandparent-role centrality? \emph{European Journal of Ageing}, \emph{8}, 109--118. \url{https://doi.org/10.1007/s10433-011-0185-5}

\leavevmode\hypertarget{ref-oecdChildcareAffordablePolicy2020}{}%
OECD. (2020). \emph{Is Childcare Affordable? Policy Brief On Employment, Labour And Social Affairs}.

\leavevmode\hypertarget{ref-oltmannsPersonalityChangeLongitudinal2020}{}%
Oltmanns, J. R., Jackson, J. J., \& Oltmanns, T. F. (2020). Personality change: Longitudinal self-other agreement and convergence with retrospective-reports. \emph{Journal of Personality and Social Psychology}, \emph{118}(5), 1065--1079. \url{https://doi.org/10.1037/pspp0000238}

\leavevmode\hypertarget{ref-pearlCausalInferenceStatistics2009}{}%
Pearl, J. (2009). Causal inference in statistics: An overview. \emph{Statistics Surveys}, \emph{3}, 96--146. \url{https://doi.org/10.1214/09-SS057}

\leavevmode\hypertarget{ref-pilkauskasHistoricalTrendsChildren2020}{}%
Pilkauskas, N. V., Amorim, M., \& Dunifon, R. E. (2020). Historical Trends in Children Living in Multigenerational Households in the United States: 18702018. \emph{Demography}, \emph{57}(6), 2269--2296. \url{https://doi.org/10.1007/s13524-020-00920-5}

\leavevmode\hypertarget{ref-R-nlme}{}%
Pinheiro, J., Bates, D., \& R-core. (2021). \emph{Nlme: Linear and nonlinear mixed effects models} {[}Manual{]}.

\leavevmode\hypertarget{ref-podsakoffCommonMethodBiases2003}{}%
Podsakoff, P. M., MacKenzie, S. B., Jeong-Yeon, L., \& Podsakoff, N. P. (2003). Common method biases in behavioral research: A critical review of the literature and recommended remedies. \emph{Journal of Applied Psychology}, \emph{88}(5), 879--903. \url{https://doi.org/10.1037/0021-9010.88.5.879}

\leavevmode\hypertarget{ref-puschPersonalityDevelopmentEmerging2019}{}%
Pusch, S., Mund, M., Hagemeyer, B., \& Finn, C. (2019). Personality Development in Emerging and Young Adulthood: A Study of Age Differences. \emph{European Journal of Personality}, \emph{33}(3), 245--263. \url{https://doi.org/10.1002/per.2181}

\leavevmode\hypertarget{ref-ramMethodsMeasuresGrowth2009}{}%
Ram, N., \& Grimm, K. J. (2009). Methods and Measures: Growth mixture modeling: A method for identifying differences in longitudinal change among unobserved groups. \emph{International Journal of Behavioral Development}, \emph{33}(6), 565--576. \url{https://doi.org/10.1177/0165025409343765}

\leavevmode\hypertarget{ref-R-base}{}%
R Core Team. (2021). \emph{R: A language and environment for statistical computing}. R Foundation for Statistical Computing. \url{https://www.R-project.org/}

\leavevmode\hypertarget{ref-robertsYoungAdulthoodCrucible2016}{}%
Roberts, B. W., \& Davis, J. P. (2016). Young Adulthood Is the Crucible of Personality Development. \emph{Emerging Adulthood}, \emph{4}(5), 318--326. \url{https://doi.org/10.1177/2167696816653052}

\leavevmode\hypertarget{ref-robertsRankorderConsistencyPersonality2000}{}%
Roberts, B. W., \& DelVecchio, W. F. (2000). The rank-order consistency of personality traits from childhood to old age: A quantitative review of longitudinal studies. \emph{Psychological Bulletin}, \emph{126}(1), 3--25. \url{https://doi.org/10.1037/0033-2909.126.1.3}

\leavevmode\hypertarget{ref-robertsPatternsMeanlevelChange2006a}{}%
Roberts, B. W., Walton, K. E., \& Viechtbauer, W. (2006). Patterns of mean-level change in personality traits across the life course: A meta-analysis of longitudinal studies. \emph{Psychological Bulletin}, \emph{132}, 1--25. \url{https://doi.org/10.1037/0033-2909.132.1.1}

\leavevmode\hypertarget{ref-robertsPersonalityDevelopmentContext2006}{}%
Roberts, B. W., \& Wood, D. (2006). Personality Development in the Context of the Neo-Socioanalytic Model of Personality. In D. K. Mroczek \& T. D. Little (Eds.), \emph{Handbook of Personality Development}. Routledge.

\leavevmode\hypertarget{ref-robertsEvaluatingFiveFactor2005}{}%
Roberts, B. W., Wood, D., \& Smith, J. L. (2005). Evaluating Five Factor Theory and social investment perspectives on personality trait development. \emph{Journal of Research in Personality}, \emph{39}(1), 166--184. \url{https://doi.org/10.1016/j.jrp.2004.08.002}

\leavevmode\hypertarget{ref-robertsPersonalityPsychology2021}{}%
Roberts, B. W., \& Yoon, H. J. (2021). Personality Psychology. \emph{Annual Review of Psychology}, Advance Online Publication. \url{https://doi.org/10.1146/annurev-psych-020821-114927}

\leavevmode\hypertarget{ref-rohrerThinkingClearlyCorrelations2018}{}%
Rohrer, J. M. (2018). Thinking Clearly About Correlations and Causation: Graphical Causal Models for Observational Data. \emph{Advances in Methods and Practices in Psychological Science}, \emph{1}(1), 27--42. \url{https://doi.org/10.1177/2515245917745629}

\leavevmode\hypertarget{ref-rohrerThatLotPROCESS2021}{}%
Rohrer, J. M., Hünermund, P., Arslan, R. C., \& Elson, M. (2021). That's a lot to PROCESS! Pitfalls of Popular Path Models. \emph{PsyArXiv}. \url{https://doi.org/10.31234/osf.io/paeb7}

\leavevmode\hypertarget{ref-rosenbaumConsquencesAdjustmentConcomitant1984}{}%
Rosenbaum, P. (1984). The consquences of adjustment for a concomitant variable that has been affected by the treatment. \emph{Journal of the Royal Statistical Society. Series A (General)}, \emph{147}(5), 656--666. \url{https://doi.org/10.2307/2981697}

\leavevmode\hypertarget{ref-scherpenzeelDataCollectionProbabilityBased2011}{}%
Scherpenzeel, A. (2011). Data Collection in a Probability-Based Internet Panel: How the LISS Panel Was Built and How It Can Be Used. \emph{Bulletin of Sociological Methodology/Bulletin de Méthodologie Sociologique}, \emph{109}(1), 56--61. \url{https://doi.org/10.1177/0759106310387713}

\leavevmode\hypertarget{ref-scherpenzeelTrueLongitudinalProbabilitybased2010}{}%
Scherpenzeel, A. C., \& Das, M. (2010). True'' longitudinal and probability-based internet panels: Evidence from the Netherlands. In M. Das, P. Ester, \& L. Kaczmirek (Eds.), \emph{Social and behavioral research and the internet: Advances in applied methods and research strategies} (pp. 77--104). Taylor \& Francis.

\leavevmode\hypertarget{ref-schwabaPersonalityTraitDevelopment2019}{}%
Schwaba, T., \& Bleidorn, W. (2019). Personality trait development across the transition to retirement. \emph{Journal of Personality and Social Psychology}, \emph{116}(4), 651--665. \url{https://doi.org/10.1037/pspp0000179}

\leavevmode\hypertarget{ref-schwabaIndividualDifferencesPersonality2018}{}%
Schwaba, T., \& Bleidorn, W. (2018). Individual differences in personality change across the adult life span. \emph{Journal of Personality}, \emph{86}(3), 450--464. \url{https://doi.org/10.1111/jopy.12327}

\leavevmode\hypertarget{ref-seifertDevelopmentRankOrderStability2021}{}%
Seifert, I. S., Rohrer, J. M., Egloff, B., \& Schmukle, S. C. (2021). The Development of the Rank-Order Stability of the Big Five Across the Life Span. \emph{Journal of Personality and Social Psychology}. \url{https://doi.org/10.1037/pspp0000398}

\leavevmode\hypertarget{ref-shadishExperimentalQuasiexperimentalDesigns2002}{}%
Shadish, W. R., Cook, T. D., \& Campbell, D. T. (2002). \emph{Experimental and quasi-experimental designs for generalized causal inference}. Houghton, Mifflin and Company.

\leavevmode\hypertarget{ref-sheppardBecomingFirstTimeGrandparent2019}{}%
Sheppard, P., \& Monden, C. (2019). Becoming a First-Time Grandparent and Subjective Well-Being: A Fixed Effects Approach. \emph{Journal of Marriage and Family}, \emph{81}(4), 1016--1026. \url{https://doi.org/10.1111/jomf.12584}

\leavevmode\hypertarget{ref-silversteinHowAmericansEnact2001}{}%
Silverstein, M., \& Marenco, A. (2001). How Americans Enact the Grandparent Role Across the Family Life Course. \emph{Journal of Family Issues}, \emph{22}(4), 493--522. \url{https://doi.org/10.1177/019251301022004006}

\leavevmode\hypertarget{ref-skopekWhoBecomesGrandparent2017}{}%
Skopek, J., \& Leopold, T. (2017). Who becomes a grandparent and when? Educational differences in the chances and timing of grandparenthood. \emph{Demographic Research}, \emph{37}(29), 917--928. \url{https://doi.org/10.4054/DemRes.2017.37.29}

\leavevmode\hypertarget{ref-sonnegaCohortProfileHealth2014}{}%
Sonnega, A., Faul, J. D., Ofstedal, M. B., Langa, K. M., Phillips, J. W., \& Weir, D. R. (2014). Cohort Profile: The Health and Retirement Study (HRS). \emph{International Journal of Epidemiology}, \emph{43}(2), 576--585. \url{https://doi.org/10.1093/ije/dyu067}

\leavevmode\hypertarget{ref-spechtPersonalityDevelopmentAdulthood2017}{}%
Specht, J. (2017). Personality development in adulthood and old age. In J. Specht (Ed.), \emph{Personality Development Across the Lifespan} (pp. 53--67). Academic Press. \url{https://doi.org/10.1016/B978-0-12-804674-6.00005-3}

\leavevmode\hypertarget{ref-spechtWhatDrivesAdult2014}{}%
Specht, J., Bleidorn, W., Denissen, J. J. A., Hennecke, M., Hutteman, R., Kandler, C., Luhmann, M., Orth, U., Reitz, A. K., \& Zimmermann, J. (2014). What Drives Adult Personality Development? A Comparison of Theoretical Perspectives and Empirical Evidence. \emph{European Journal of Personality}, \emph{28}(3), 216--230. \url{https://doi.org/10.1002/per.1966}

\leavevmode\hypertarget{ref-spechtStabilityChangePersonality2011}{}%
Specht, J., Egloff, B., \& Schmukle, S. C. (2011). Stability and change of personality across the life course: The impact of age and major life events on mean-level and rank-order stability of the Big Five. \emph{Journal of Personality and Social Psychology}, \emph{101}(4), 862--882. \url{https://doi.org/10.1037/a0024950}

\leavevmode\hypertarget{ref-spikicDoesDivorceChange2021}{}%
Spikic, S., Mortelmans, D., \& Pasteels, I. (2021). Does divorce change your personality? Examining the effect of divorce occurrence on the Big Five personality traits using panel surveys from three countries. \emph{Personality and Individual Differences}, \emph{171}, 110428. \url{https://doi.org/10.1016/j.paid.2020.110428}

\leavevmode\hypertarget{ref-steinerImportanceCovariateSelection2010}{}%
Steiner, P., Cook, T., Shadish, W., \& Clark, M. (2010). The Importance of Covariate Selection in Controlling for Selection Bias in Observational Studies. \emph{Psychological Methods}, \emph{15}, 250--267. \url{https://doi.org/10.1037/a0018719}

\leavevmode\hypertarget{ref-stephanPhysicalActivityPersonality2014}{}%
Stephan, Y., Sutin, A. R., \& Terracciano, A. (2014). Physical activity and personality development across adulthood and old age: Evidence from two longitudinal studies. \emph{Journal of Research in Personality}, \emph{49}, 1--7. \url{https://doi.org/10.1016/j.jrp.2013.12.003}

\leavevmode\hypertarget{ref-stgeorgeMenExperiencesGrandfatherhood2014}{}%
StGeorge, J. M., \& Fletcher, R. J. (2014). Men's experiences of grandfatherhood: A welcome surprise. \emph{The International Journal of Aging \& Human Development}, \emph{78}(4), 351--378. \url{https://doi.org/10.2190/AG.78.4.c}

\leavevmode\hypertarget{ref-stuartMatchingMethodsCausal2010}{}%
Stuart, E. A. (2010). Matching methods for causal inference: A review and a look forward. \emph{Statistical Science: A Review Journal of the Institute of Mathematical Statistics}, \emph{25}(1), 1--21. \url{https://doi.org/10.1214/09-STS313}

\leavevmode\hypertarget{ref-tanskanenDoesTransitionRetirement2021}{}%
Tanskanen, A., Danielsbacka, M., Hämäläinen, H., \& Sole-Auro, A. (2021). Does Transition to Retirement Promote Grandchild Care? Results from the Survey of Health, Ageing and Retirement in Europe. \emph{PsyArXiv}. \url{https://doi.org/10.31235/osf.io/akme6}

\leavevmode\hypertarget{ref-tanskanenTransitionGrandparenthoodSubjective2019}{}%
Tanskanen, A. O., Danielsbacka, M., Coall, D. A., \& Jokela, M. (2019). Transition to Grandparenthood and Subjective Well-Being in Older Europeans: A Within-Person Investigation Using Longitudinal Data. \emph{Evolutionary Psychology}, \emph{17}(3), 1474704919875948. \url{https://doi.org/10.1177/1474704919875948}

\leavevmode\hypertarget{ref-thieleNatureDimensionsGrandparent2006a}{}%
Thiele, D. M., \& Whelan, T. A. (2006). The Nature and Dimensions of the Grandparent Role. \emph{Marriage \& Family Review}, \emph{40}(1), 93--108. \url{https://doi.org/10.1300/J002v40n01_06}

\leavevmode\hypertarget{ref-thoemmesSystematicReviewPropensity2011}{}%
Thoemmes, F. J., \& Kim, E. S. (2011). A Systematic Review of Propensity Score Methods in the Social Sciences. \emph{Multivariate Behavioral Research}, \emph{46}(1), 90--118. \url{https://doi.org/10.1080/00273171.2011.540475}

\leavevmode\hypertarget{ref-triadoGrandparentsWhoProvide2014}{}%
Triadó, C., Villar, F., Celdrán, M., \& Solé, C. (2014). Grandparents Who Provide Auxiliary Care for Their Grandchildren: Satisfaction, Difficulties, and Impact on Their Health and Well-being. \emph{Journal of Intergenerational Relationships}, \emph{12}(2), 113--127. \url{https://doi.org/10.1080/15350770.2014.901102}

\leavevmode\hypertarget{ref-mice2011}{}%
van Buuren, S., \& Groothuis-Oudshoorn, K. (2011). mice: Multivariate imputation by chained equations in r. \emph{Journal of Statistical Software}, \emph{45}(3), 1--67.

\leavevmode\hypertarget{ref-vanderlaanRepresentativityLISSPanel2009}{}%
van der Laan, J. (2009). \emph{Representativity of the LISS panel (Discussion Paper 09041)}. Statistics Netherlands.

\leavevmode\hypertarget{ref-vanderweelePrinciplesConfounderSelection2019}{}%
VanderWeele, T. J. (2019). Principles of confounder selection. \emph{European Journal of Epidemiology}, \emph{34}(3), 211--219. \url{https://doi.org/10.1007/s10654-019-00494-6}

\leavevmode\hypertarget{ref-vanderweeleOutcomeWideLongitudinalDesigns2020}{}%
VanderWeele, T. J., Mathur, M. B., \& Chen, Y. (2020). Outcome-Wide Longitudinal Designs for Causal Inference: A New Template for Empirical Studies. \emph{Statistical Science}, \emph{35}(3), 437--466. \url{https://doi.org/10.1214/19-STS728}

\leavevmode\hypertarget{ref-vanscheppingenPersonalityTraitDevelopment2016}{}%
van Scheppingen, M. A., Jackson, J. J., Specht, J., Hutteman, R., Denissen, J. J. A., \& Bleidorn, W. (2016). Personality Trait Development During the Transition to Parenthood: A Test of Social Investment Theory. \emph{Social Psychological and Personality Science}, \emph{7}(5), 452--462. \url{https://doi.org/10.1177/1948550616630032}

\leavevmode\hypertarget{ref-vanscheppingenTrajectoriesLifeSatisfaction2020}{}%
van Scheppingen, M. A., \& Leopold, T. (2020). Trajectories of life satisfaction before, upon, and after divorce: Evidence from a new matching approach. \emph{Journal of Personality and Social Psychology}, \emph{119}(6), 1444--1458. \url{https://doi.org/10.1037/pspp0000270}

\leavevmode\hypertarget{ref-vermoteImpactNonresidentialGrandchild2021a}{}%
Vermote, M., Deliens, T., Deforche, B., \& D'Hondt, E. (2021). The impact of non-residential grandchild care on physical activity and sedentary behavior in people aged 50 years and over: Study protocol of the Healthy Grandparenting Project. \emph{BMC Public Health}, \emph{21}. \url{https://doi.org/10.1186/s12889-020-10024-9}

\leavevmode\hypertarget{ref-wagnerFirstPartnershipExperience2015}{}%
Wagner, J., Becker, M., Lüdtke, O., \& Trautwein, U. (2015). The First Partnership Experience and Personality Development: A Propensity Score Matching Study in Young Adulthood. \emph{Social Psychological and Personality Science}, \emph{6}(4), 455--463. \url{https://doi.org/10.1177/1948550614566092}

\leavevmode\hypertarget{ref-wagnerDoesPersonalityBecome2019}{}%
Wagner, J., Lüdtke, O., \& Robitzsch, A. (2019). Does personality become more stable with age? Disentangling state and trait effects for the big five across the life span using local structural equation modeling. \emph{Journal of Personality and Social Psychology}, \emph{116}(4), 666--680. \url{https://doi.org/10.1037/pspp0000203}

\leavevmode\hypertarget{ref-wagnerIntegrativeModelSources2020}{}%
Wagner, J., Orth, U., Bleidorn, W., Hopwood, C. J., \& Kandler, C. (2020). Toward an Integrative Model of Sources of Personality Stability and Change. \emph{Current Directions in Psychological Science}, \emph{29}(5), 438--444. \url{https://doi.org/10.1177/0963721420924751}

\leavevmode\hypertarget{ref-wagnerPersonalityTraitDevelopment2016}{}%
Wagner, J., Ram, N., Smith, J., \& Gerstorf, D. (2016). Personality trait development at the end of life: Antecedents and correlates of mean-level trajectories. \emph{Journal of Personality and Social Psychology}, \emph{111}(3), 411--429. \url{https://doi.org/10.1037/pspp0000071}

\leavevmode\hypertarget{ref-tidyverse2019}{}%
Wickham, H., Averick, M., Bryan, J., Chang, W., McGowan, L. D., François, R., Grolemund, G., Hayes, A., Henry, L., Hester, J., Kuhn, M., Pedersen, T. L., Miller, E., Bache, S. M., Müller, K., Ooms, J., Robinson, D., Seidel, D. P., Spinu, V., \ldots{} Yutani, H. (2019). Welcome to the tidyverse. \emph{Journal of Open Source Software}, \emph{4}(43), 1686. \url{https://doi.org/10.21105/joss.01686}

\leavevmode\hypertarget{ref-wortmanStabilityChangeBig2012}{}%
Wortman, J., Lucas, R. E., \& Donnellan, M. B. (2012). Stability and change in the Big Five personality domains: Evidence from a longitudinal study of Australians. \emph{Psychology and Aging}, \emph{27}(4), 867--874. \url{https://doi.org/10.1037/a0029322}

\leavevmode\hypertarget{ref-wrzusProcessesPersonalityDevelopment2017}{}%
Wrzus, C., \& Roberts, B. W. (2017). Processes of personality development in adulthood: The TESSERA framework. \emph{Personality and Social Psychology Review}, \emph{21}(3), 253--277. \url{https://doi.org/10.1177/1088868316652279}

\leavevmode\hypertarget{ref-yapDoesPersonalityModerate2012}{}%
Yap, S., Anusic, I., \& Lucas, R. E. (2012). Does personality moderate reaction and adaptation to major life events? Evidence from the British Household Panel Survey. \emph{Journal of Research in Personality}, \emph{46}(5), 477--488. \url{https://doi.org/10.1016/j.jrp.2012.05.005}

\endgroup


\clearpage
\makeatletter
\efloat@restorefloats
\makeatother


\begin{appendix}
\renewcommand{\appendixname}{\textcolor{white}{.}}

\hypertarget{supplemental-material}{%
\section{Supplemental Material}\label{supplemental-material}}

\renewcommand{\thefigure}{S\arabic{figure}} \setcounter{figure}{0}
\renewcommand{\thetable}{S\arabic{table}} \setcounter{table}{0}

\setcounter{page}{1}

\noindent \textbf{Model Equations}

Model equation for the basic (i.e., unmoderated) models (ignoring the
additional nesting in households applied to the majority of models):
\begin{equation}
\begin{split}
y_{ti} =& \beta_{0i} + \beta_{1i}before_{ti} + \beta_{2i}after_{ti} + \beta_{3i}shift_{ti} + e_{ti} \\
& \beta_{0i} = \gamma_{00} + \gamma_{01}grandparent_i + \gamma_{02}pscore_i + \upsilon_{0i} \\
& \beta_{1i} = \gamma_{10} + \gamma_{11}grandparent_i \\
& \beta_{2i} = \gamma_{20} + \gamma_{21}grandparent_i \\
& \beta_{3i} = \gamma_{30} + \gamma_{31}grandparent_i\ ,
\end{split}
\label{eq:mlm1}
\end{equation} where at time \(t\) for person \(i\)
\(e_{ti} \sim N(0, \sigma_e^2)\) and
\(\upsilon_{0i} \sim N(0, \tau_{00})\). \(y_{ti}\) represented one of
the Big Five or life satisfaction. Separate models were computed for
LISS and HRS samples, and for parent and nonparent matched controls.\\
Model equation for the models including the gender interaction
(moderator variable \(female_i\)): \begin{equation}
\begin{split}
y_{ti} =& \beta_{0i} + \beta_{1i}before_{ti} + \beta_{2i}after_{ti} + \beta_{3i}shift_{ti} + e_{ti} \\
& \beta_{0i} = \gamma_{00} + \gamma_{01}grandparent_i + \gamma_{02}female_i + \gamma_{03}grandparent_{i}female_{i} \\
& \hphantom{{}=\beta_{0i}}+ \gamma_{04}pscore_i + \upsilon_{0i} \\
& \beta_{1i} = \gamma_{10} + \gamma_{11}grandparent_i + \gamma_{12}female_i + \gamma_{13}grandparent_{i}female_{i} \\
& \beta_{2i} = \gamma_{20} + \gamma_{21}grandparent_i + \gamma_{22}female_i + \gamma_{23}grandparent_{i}female_{i} \\
& \beta_{3i} = \gamma_{30} + \gamma_{31}grandparent_i + \gamma_{32}female_i + \gamma_{33}grandparent_{i}female_{i}\ ,
\end{split}
\label{eq:mlm2}
\end{equation} where \(e_{ti} \sim N(0, \sigma_e^2)\) and
\(\upsilon_{0i} \sim N(0, \tau_{00})\). Again, we estimated separate
models for each sample (LISS, HRS) and each comparison group (parents,
nonparents).\\
Model equation for the models including the interaction by paid work
(moderator variable \(working_{ti}\)): \begin{equation}
\begin{split}
y_{ti} =& \beta_{0i} + \beta_{1i}working_{ti} + \beta_{2i}before_{ti} + \beta_{3i}before_{ti}working_{ti} + \beta_{4i}after_{ti} \\
& + \beta_{5i}after_{ti}working_{ti} + \beta_{6i}shift_{ti} + \beta_{7i}shift_{ti}working_{ti} + e_{ti} \\
& \beta_{0i} = \gamma_{00} + \gamma_{01}grandparent_i + \gamma_{02}pscore_i + \upsilon_{0i} \\
& \beta_{1i} = \gamma_{10} + \gamma_{11}grandparent_i \\
& \beta_{2i} = \gamma_{20} + \gamma_{21}grandparent_i \\
& \beta_{3i} = \gamma_{30} + \gamma_{31}grandparent_i \\
& \beta_{4i} = \gamma_{40} + \gamma_{41}grandparent_i \\
& \beta_{5i} = \gamma_{50} + \gamma_{51}grandparent_i \\
& \beta_{6i} = \gamma_{60} + \gamma_{61}grandparent_i \\
& \beta_{7i} = \gamma_{70} + \gamma_{71}grandparent_i\ ,
\end{split}
\label{eq:mlm3}
\end{equation} where \(e_{ti} \sim N(0, \sigma_e^2)\) and
\(\upsilon_{0i} \sim N(0, \tau_{00})\). We estimated separate models for
each comparison group (parents, nonparents) in the HRS.\\
Model equation for the models including the interaction by grandchild
care (moderator variable \(caring_{ti}\)): \begin{equation}
\begin{split}
y_{ti} =& \beta_{0i} + \beta_{1i}caring_{ti} + \beta_{2i}after_{ti} + \beta_{3i}after_{ti}caring_{ti} + e_{ti} \\
& \beta_{0i} = \gamma_{00} + \gamma_{01}grandparent_i + \gamma_{02}pscore_i + \upsilon_{0i} \\
& \beta_{1i} = \gamma_{10} + \gamma_{11}grandparent_i \\
& \beta_{2i} = \gamma_{20} + \gamma_{21}grandparent_i \\
& \beta_{3i} = \gamma_{30} + \gamma_{31}grandparent_i\ ,
\end{split}
\label{eq:mlm4}
\end{equation} where \(e_{ti} \sim N(0, \sigma_e^2)\) and
\(\upsilon_{0i} \sim N(0, \tau_{00})\). Restricted to the HRS
post-transition period, we estimated separate models for each comparison
group (parents, nonparents).

\newpage

\noindent \textbf{Supplemental Tables}

\begin{table}[h]

\begin{center}
\begin{threeparttable}

\caption{\label{tab:icc-table}Intra-Class Correlations of Grandparents and Matched Controls in the Four Analysis Samples.}

\begin{tabular}{lcccccc}
\toprule
& \multicolumn{1}{c}{\textcolor{white}{xxx}A\textcolor{white}{xxx}} & \multicolumn{1}{c}{\textcolor{white}{xxx}C\textcolor{white}{xxx}} & \multicolumn{1}{c}{\textcolor{white}{xxx}E\textcolor{white}{xxx}} & \multicolumn{1}{c}{\textcolor{white}{xxx}N\textcolor{white}{xxx}} & \multicolumn{1}{c}{\textcolor{white}{xxx}O\textcolor{white}{xxx}} & \multicolumn{1}{c}{\textcolor{white}{xxx}LS\textcolor{white}{xxx}}\\
\midrule
LISS: Parent controls &  &  &  &  &  & \\
\ \ \ $ICC_{pid}$ \textcolor{white}{LP} & 0.74 & 0.77 & 0.81 & 0.71 & 0.78 & 0.35\\
\ \ \ $ICC_{hid}$ \textcolor{white}{LP} & 0.05 & 0.01 & 0.02 & 0.07 & 0.00 & 0.37\\
\ \ \ $ICC_{pid/hid}$ \textcolor{white}{LP} & 0.79 & 0.78 & 0.83 & 0.78 & 0.78 & 0.71\\
LISS: Nonparent controls &  &  &  &  &  & \\
\ \ \ $ICC_{pid}$ \textcolor{white}{LN} & 0.76 & 0.76 & 0.64 & 0.67 & 0.79 & 0.32\\
\ \ \ $ICC_{hid}$ \textcolor{white}{LN} & 0.00 & 0.00 & 0.22 & 0.10 & 0.02 & 0.36\\
\ \ \ $ICC_{pid/hid}$ \textcolor{white}{LN} & 0.76 & 0.77 & 0.85 & 0.77 & 0.81 & 0.67\\
HRS: Parent controls &  &  &  &  &  & \\
\ \ \ $ICC_{pid}$ \textcolor{white}{HP} & 0.76 & 0.69 & 0.79 & 0.73 & 0.57 & 0.31\\
\ \ \ $ICC_{hid}$ \textcolor{white}{HP} & 0.00 & 0.07 & 0.00 & 0.01 & 0.21 & 0.35\\
\ \ \ $ICC_{pid/hid}$ \textcolor{white}{HP} & 0.76 & 0.76 & 0.79 & 0.74 & 0.78 & 0.67\\
HRS: Nonparent controls &  &  &  &  &  & \\
\ \ \ $ICC_{pid}$ \textcolor{white}{HN} & 0.71 & 0.73 & 0.77 & 0.76 & 0.59 & 0.33\\
\ \ \ $ICC_{hid}$ \textcolor{white}{HN} & 0.07 & 0.06 & 0.04 & 0.00 & 0.23 & 0.38\\
\ \ \ $ICC_{pid/hid}$ \textcolor{white}{HN} & 0.78 & 0.79 & 0.80 & 0.76 & 0.82 & 0.71\\
\bottomrule
\addlinespace
\end{tabular}

\begin{tablenotes}[para]
\normalsize{\textit{Note.} A = agreeableness, C = conscientiousness, E = extraversion, N = neuroticism, O = openness, LS = life satisfaction. Intra-class correlations are the proportion of total variation that is explained by the respective nesting factor. $ICC_{pid}$ is the proportion of total variance explained by nesting in respondents which corresponds to the correlation between two randomly selected observations from the same respondent. $ICC_{hid}$ is the proportion of total variance explained by nesting in households which corresponds to the correlation between two randomly selected observations from the same household. $ICC_{pid/hid}$ is the proportion of total variance explained by nesting in respondents and in households which corresponds to the correlation between two randomly selected observations from the same respondent and the same household.}
\end{tablenotes}

\end{threeparttable}
\end{center}

\end{table}










\begin{lltable}

\begin{TableNotes}[para]
\normalsize{\textit{Note.} obs. = observations. \(time=0\) marks
the first year where the transition to grandparenthood has been
reported. The number of grandparent respondents included in the final
samples is \(N_{LISS}=\) 250 and \(N_{HRS}=\) 846.}
\end{TableNotes}

\small{

\begin{longtable}{lccccccccccccc}\noalign{\getlongtablewidth\global\LTcapwidth=\longtablewidth}
\caption{\label{tab:piecewise-coding-scheme}Longitudinal Sample Size in the
Analysis Samples and Coding Scheme for the Piecewise Regression
Coefficients.}\\
\toprule
& \multicolumn{6}{c}{Pre-transition years} & \multicolumn{7}{c}{Post-transition years} \\
\cmidrule(r){2-7} \cmidrule(r){8-14}
& -6 & -5 & -4 & -3 & -2 & -1 & 0 & 1 & 2 & 3 & 4 & 5 & 6\\
\midrule
\endfirsthead
\caption*{\normalfont{Table \ref{tab:piecewise-coding-scheme} continued}}\\
\toprule
& \multicolumn{6}{c}{Pre-transition years} & \multicolumn{7}{c}{Post-transition years} \\
\cmidrule(r){2-7} \cmidrule(r){8-14}
& -6 & -5 & -4 & -3 & -2 & -1 & 0 & 1 & 2 & 3 & 4 & 5 & 6\\
\midrule
\endhead
LISS: Analysis samples &  &  &  &  &  &  &  &  &  &  &  &  & \\
\ \ \ Grandparents: obs. \textcolor{white}{L} & 92 & 105 & 108 & 121 & 156 & 116 & 133 & 138 & 108 & 108 & 69 & 62 & 52\\
\ \ \ Grandparents: \% women \textcolor{white}{L} & 51.09 & 48.57 & 52.78 & 51.24 & 56.41 & 62.93 & 47.37 & 52.90 & 51.85 & 50.00 & 56.52 & 66.13 & 53.85\\
\ \ \ Parent controls: obs. \textcolor{white}{L} & 335 & 425 & 381 & 540 & 740 & 351 & 450 & 488 & 333 & 394 & 365 & 164 & 201\\
\ \ \ Parent controls: \% women \textcolor{white}{L} & 57.61 & 51.06 & 55.12 & 51.48 & 55.00 & 56.13 & 53.11 & 54.10 & 56.76 & 51.27 & 56.99 & 59.76 & 48.76\\
\ \ \ Nonparent controls: obs. \textcolor{white}{L} & 331 & 399 & 407 & 554 & 739 & 354 & 473 & 516 & 367 & 477 & 375 & 146 & 202\\
\ \ \ Nonparent controls: \% women \textcolor{white}{L} & 52.57 & 54.89 & 57.99 & 52.71 & 55.21 & 54.52 & 49.26 & 54.46 & 52.86 & 52.83 & 54.67 & 48.63 & 51.49\\
LISS: Coding scheme &  &  &  &  &  &  &  &  &  &  &  &  & \\
\ \ \ Before-slope \textcolor{white}{L} & 0 & 1 & 2 & 3 & 4 & 5 & 5 & 5 & 5 & 5 & 5 & 5 & 5\\
\ \ \ After-slope \textcolor{white}{L} & 0 & 0 & 0 & 0 & 0 & 0 & 1 & 2 & 3 & 4 & 5 & 6 & 7\\
\ \ \ Shift \textcolor{white}{L} & 0 & 0 & 0 & 0 & 0 & 0 & 1 & 1 & 1 & 1 & 1 & 1 & 1\\
HRS: Analysis samples &  &  &  &  &  &  &  &  &  &  &  &  & \\
\ \ \ Grandparents: obs. \textcolor{white}{H} & 162 &  & 388 &  & 461 &  & 380 &  & 444 &  & 195 &  & 232\\
\ \ \ Grandparents: \% women \textcolor{white}{H} & 57.41 &  & 54.12 &  & 55.53 &  & 53.95 &  & 55.41 &  & 56.41 &  & 53.45\\
\ \ \ Parent controls: obs. \textcolor{white}{H} & 619 &  & 1540 &  & 1844 &  & 1228 &  & 1504 &  & 658 &  & 864\\
\ \ \ Parent controls: \% women \textcolor{white}{H} & 55.41 &  & 54.03 &  & 55.53 &  & 54.64 &  & 56.45 &  & 56.08 &  & 57.64\\
\ \ \ Nonparent controls: obs. \textcolor{white}{H} & 620 &  & 1541 &  & 1844 &  & 1205 &  & 1448 &  & 688 &  & 821\\
\ \ \ Nonparent controls: \% women \textcolor{white}{H} & 56.45 &  & 54.06 &  & 55.53 &  & 56.10 &  & 58.91 &  & 57.56 &  & 60.54\\
HRS: Coding scheme &  &  &  &  &  &  &  &  &  &  &  &  & \\
\ \ \ Before-slope \textcolor{white}{H} & 0 &  & 1 &  & 2 &  & 2 &  & 2 &  & 2 &  & 2\\
\ \ \ After-slope \textcolor{white}{H} & 0 &  & 0 &  & 0 &  & 1 &  & 2 &  & 3 &  & 4\\
\ \ \ Shift \textcolor{white}{H} & 0 &  & 0 &  & 0 &  & 1 &  & 1 &  & 1 &  & 1\\
\bottomrule
\addlinespace
\insertTableNotes
\end{longtable}

}

\end{lltable}








\begin{lltable}

\begin{TableNotes}[para]
\normalsize{\textit{Note.} Standard deviations shown in parentheses;
\(time=0\) marks the first year where the transition to grandparenthood
was reported.}
\end{TableNotes}

\small{

\begin{longtable}{lccccccccccccc}\noalign{\getlongtablewidth\global\LTcapwidth=\longtablewidth}
\caption{\label{tab:descriptives-liss}Means and Standard Deviations of the Big
Five and Life Satisfaction over Time in the LISS Panel.}\\
\toprule
& \multicolumn{6}{c}{Pre-transition years} & \multicolumn{7}{c}{Post-transition years} \\
\cmidrule(r){2-7} \cmidrule(r){8-14}
& -6 & -5 & -4 & -3 & -2 & -1 & 0 & 1 & 2 & 3 & 4 & 5 & 6\\
\midrule
\endfirsthead
\caption*{\normalfont{Table \ref{tab:descriptives-liss} continued}}\\
\toprule
& \multicolumn{6}{c}{Pre-transition years} & \multicolumn{7}{c}{Post-transition years} \\
\cmidrule(r){2-7} \cmidrule(r){8-14}
& -6 & -5 & -4 & -3 & -2 & -1 & 0 & 1 & 2 & 3 & 4 & 5 & 6\\
\midrule
\endhead
Agreeableness &  &  &  &  &  &  &  &  &  &  &  &  & \\
\ \ \ Grandparents \textcolor{white}{A} & 3.85 & 3.87 & 3.93 & 3.87 & 3.90 & 3.93 & 3.87 & 3.92 & 3.91 & 3.91 & 3.89 & 4.01 & 3.98\\
\ \ \ \textcolor{white}{Ag} & (0.52) & (0.50) & (0.46) & (0.49) & (0.54) & (0.47) & (0.49) & (0.52) & (0.52) & (0.51) & (0.52) & (0.49) & (0.37)\\
\ \ \ Parent controls \textcolor{white}{A} & 3.93 & 3.89 & 3.90 & 3.87 & 3.91 & 3.95 & 3.91 & 3.89 & 3.90 & 3.92 & 3.86 & 3.86 & 3.81\\
\ \ \ \textcolor{white}{Ap} & (0.52) & (0.51) & (0.47) & (0.50) & (0.48) & (0.48) & (0.47) & (0.51) & (0.53) & (0.48) & (0.50) & (0.43) & (0.43)\\
\ \ \ Nonparent controls \textcolor{white}{A} & 3.95 & 3.94 & 3.98 & 3.98 & 3.94 & 3.91 & 3.94 & 3.95 & 3.94 & 3.94 & 3.92 & 3.92 & 3.88\\
\ \ \ \textcolor{white}{An} & (0.47) & (0.50) & (0.45) & (0.50) & (0.49) & (0.47) & (0.44) & (0.45) & (0.46) & (0.47) & (0.41) & (0.44) & (0.42)\\
Conscientiousness &  &  &  &  &  &  &  &  &  &  &  &  & \\
\ \ \ Grandparents \textcolor{white}{C} & 3.76 & 3.84 & 3.74 & 3.75 & 3.77 & 3.79 & 3.77 & 3.78 & 3.75 & 3.79 & 3.84 & 3.74 & 3.76\\
\ \ \ \textcolor{white}{Cg} & (0.50) & (0.45) & (0.49) & (0.46) & (0.53) & (0.48) & (0.49) & (0.51) & (0.49) & (0.51) & (0.44) & (0.48) & (0.43)\\
\ \ \ Parent controls \textcolor{white}{C} & 3.80 & 3.78 & 3.80 & 3.77 & 3.79 & 3.83 & 3.82 & 3.79 & 3.80 & 3.79 & 3.78 & 3.76 & 3.77\\
\ \ \ \textcolor{white}{Cp} & (0.52) & (0.50) & (0.52) & (0.49) & (0.49) & (0.50) & (0.49) & (0.47) & (0.47) & (0.46) & (0.43) & (0.44) & (0.45)\\
\ \ \ Nonparent controls \textcolor{white}{C} & 3.77 & 3.79 & 3.76 & 3.80 & 3.74 & 3.75 & 3.77 & 3.72 & 3.82 & 3.81 & 3.78 & 3.84 & 3.80\\
\ \ \ \textcolor{white}{Cn} & (0.53) & (0.50) & (0.51) & (0.50) & (0.51) & (0.53) & (0.50) & (0.50) & (0.50) & (0.51) & (0.48) & (0.46) & (0.50)\\
Extraversion &  &  &  &  &  &  &  &  &  &  &  &  & \\
\ \ \ Grandparents \textcolor{white}{E} & 3.23 & 3.20 & 3.31 & 3.32 & 3.28 & 3.30 & 3.19 & 3.24 & 3.22 & 3.19 & 3.33 & 3.34 & 3.19\\
\ \ \ \textcolor{white}{Eg} & (0.66) & (0.74) & (0.54) & (0.58) & (0.64) & (0.57) & (0.61) & (0.69) & (0.65) & (0.60) & (0.60) & (0.58) & (0.55)\\
\ \ \ Parent controls \textcolor{white}{E} & 3.32 & 3.30 & 3.28 & 3.27 & 3.26 & 3.30 & 3.25 & 3.20 & 3.22 & 3.28 & 3.19 & 3.19 & 3.14\\
\ \ \ \textcolor{white}{Ep} & (0.58) & (0.59) & (0.58) & (0.59) & (0.59) & (0.59) & (0.64) & (0.62) & (0.59) & (0.61) & (0.58) & (0.53) & (0.56)\\
\ \ \ Nonparent controls \textcolor{white}{E} & 3.31 & 3.27 & 3.21 & 3.32 & 3.32 & 3.28 & 3.30 & 3.27 & 3.31 & 3.31 & 3.28 & 3.13 & 3.26\\
\ \ \ \textcolor{white}{En} & (0.74) & (0.70) & (0.79) & (0.75) & (0.69) & (0.70) & (0.72) & (0.73) & (0.77) & (0.78) & (0.73) & (0.75) & (0.74)\\
Neuroticism &  &  &  &  &  &  &  &  &  &  &  &  & \\
\ \ \ Grandparents \textcolor{white}{N} & 2.39 & 2.31 & 2.33 & 2.41 & 2.45 & 2.47 & 2.30 & 2.39 & 2.30 & 2.36 & 2.33 & 2.44 & 2.53\\
\ \ \ \textcolor{white}{Ng} & (0.71) & (0.64) & (0.60) & (0.64) & (0.65) & (0.71) & (0.67) & (0.76) & (0.68) & (0.66) & (0.67) & (0.80) & (0.67)\\
\ \ \ Parent controls \textcolor{white}{N} & 2.43 & 2.42 & 2.42 & 2.38 & 2.40 & 2.37 & 2.35 & 2.35 & 2.30 & 2.28 & 2.35 & 2.31 & 2.33\\
\ \ \ \textcolor{white}{Np} & (0.59) & (0.63) & (0.56) & (0.58) & (0.58) & (0.60) & (0.63) & (0.65) & (0.56) & (0.56) & (0.60) & (0.55) & (0.56)\\
\ \ \ Nonparent controls \textcolor{white}{N} & 2.41 & 2.44 & 2.47 & 2.36 & 2.43 & 2.37 & 2.33 & 2.37 & 2.34 & 2.33 & 2.35 & 2.48 & 2.35\\
\ \ \ \textcolor{white}{Nn} & (0.64) & (0.63) & (0.69) & (0.70) & (0.69) & (0.63) & (0.69) & (0.71) & (0.74) & (0.68) & (0.70) & (0.82) & (0.83)\\
Openness &  &  &  &  &  &  &  &  &  &  &  &  & \\
\ \ \ Grandparents \textcolor{white}{O} & 3.43 & 3.50 & 3.54 & 3.49 & 3.49 & 3.50 & 3.48 & 3.48 & 3.50 & 3.45 & 3.50 & 3.43 & 3.36\\
\ \ \ \textcolor{white}{Og} & (0.51) & (0.50) & (0.49) & (0.45) & (0.49) & (0.50) & (0.48) & (0.54) & (0.43) & (0.46) & (0.50) & (0.53) & (0.56)\\
\ \ \ Parent controls \textcolor{white}{O} & 3.53 & 3.46 & 3.43 & 3.48 & 3.48 & 3.48 & 3.50 & 3.49 & 3.44 & 3.51 & 3.42 & 3.37 & 3.42\\
\ \ \ \textcolor{white}{Op} & (0.52) & (0.52) & (0.50) & (0.53) & (0.51) & (0.51) & (0.52) & (0.50) & (0.48) & (0.48) & (0.49) & (0.48) & (0.42)\\
\ \ \ Nonparent controls \textcolor{white}{O} & 3.53 & 3.57 & 3.53 & 3.58 & 3.52 & 3.51 & 3.52 & 3.55 & 3.54 & 3.59 & 3.53 & 3.51 & 3.51\\
\ \ \ \textcolor{white}{On} & (0.52) & (0.51) & (0.51) & (0.52) & (0.52) & (0.51) & (0.51) & (0.51) & (0.52) & (0.51) & (0.50) & (0.47) & (0.53)\\
Life satisfaction &  &  &  &  &  &  &  &  &  &  &  &  & \\
\ \ \ Grandparents \textcolor{white}{L} & 5.18 & 5.29 & 5.23 & 5.16 & 5.28 & 5.24 & 5.31 & 5.24 & 5.37 & 5.38 & 5.39 & 5.25 & 5.15\\
\ \ \ \textcolor{white}{Lg} & (1.06) & (0.93) & (1.13) & (0.95) & (0.93) & (1.10) & (0.93) & (1.03) & (1.09) & (0.90) & (1.10) & (1.10) & (1.00)\\
\ \ \ Parent controls \textcolor{white}{L} & 5.21 & 5.30 & 5.26 & 5.23 & 5.28 & 5.29 & 5.36 & 5.25 & 5.26 & 5.45 & 5.33 & 5.40 & 5.41\\
\ \ \ \textcolor{white}{Lp} & (1.11) & (1.03) & (1.01) & (0.97) & (1.01) & (1.07) & (0.99) & (1.03) & (1.04) & (0.93) & (1.04) & (1.05) & (1.05)\\
\ \ \ Nonparent controls \textcolor{white}{L} & 5.27 & 5.19 & 5.10 & 5.21 & 5.26 & 5.18 & 5.24 & 5.09 & 5.10 & 5.07 & 5.23 & 4.98 & 5.19\\
\ \ \ \textcolor{white}{Ln} & (0.92) & (0.87) & (0.90) & (0.92) & (0.95) & (0.90) & (0.96) & (1.04) & (1.12) & (1.13) & (1.08) & (1.30) & (1.18)\\
\bottomrule
\addlinespace
\insertTableNotes
\end{longtable}

}

\end{lltable}










\begin{lltable}

\begin{TableNotes}[para]
\normalsize{\textit{Note.} Standard deviations shown in parentheses;
\(time=0\) marks the first year where the transition to grandparenthood
was reported. To aid comparability with the LISS panel measures, we
reverse scored all items so that higher values corresponded to higher
trait levels.}
\end{TableNotes}

\small{

\begin{longtable}{lccccccccccccc}\noalign{\getlongtablewidth\global\LTcapwidth=\longtablewidth}
\caption{\label{tab:descriptives-hrs}Means and Standard Deviations of the Big Five
and Life Satisfaction over Time in the HRS.}\\
\toprule
& \multicolumn{6}{c}{Pre-transition years} & \multicolumn{7}{c}{Post-transition years} \\
\cmidrule(r){2-7} \cmidrule(r){8-14}
& -6 & -5 & -4 & -3 & -2 & -1 & 0 & 1 & 2 & 3 & 4 & 5 & 6\\
\midrule
\endfirsthead
\caption*{\normalfont{Table \ref{tab:descriptives-hrs} continued}}\\
\toprule
& \multicolumn{6}{c}{Pre-transition years} & \multicolumn{7}{c}{Post-transition years} \\
\cmidrule(r){2-7} \cmidrule(r){8-14}
& -6 & -5 & -4 & -3 & -2 & -1 & 0 & 1 & 2 & 3 & 4 & 5 & 6\\
\midrule
\endhead
Agreeableness &  &  &  &  &  &  &  &  &  &  &  &  & \\
\ \ \ Grandparents \textcolor{white}{A} & 3.46 &  & 3.51 &  & 3.51 &  & 3.52 &  & 3.52 &  & 3.50 &  & 3.56\\
\ \ \ \textcolor{white}{Ag} & (0.47) &  & (0.48) &  & (0.49) &  & (0.49) &  & (0.48) &  & (0.53) &  & (0.44)\\
\ \ \ Parent controls \textcolor{white}{A} & 3.50 &  & 3.48 &  & 3.50 &  & 3.49 &  & 3.49 &  & 3.44 &  & 3.47\\
\ \ \ \textcolor{white}{Ap} & (0.48) &  & (0.49) &  & (0.46) &  & (0.50) &  & (0.48) &  & (0.52) &  & (0.51)\\
\ \ \ Nonparent controls \textcolor{white}{A} & 3.50 &  & 3.50 &  & 3.50 &  & 3.52 &  & 3.52 &  & 3.44 &  & 3.48\\
\ \ \ \textcolor{white}{An} & (0.50) &  & (0.50) &  & (0.51) &  & (0.50) &  & (0.50) &  & (0.53) &  & (0.53)\\
Conscientiousness &  &  &  &  &  &  &  &  &  &  &  &  & \\
\ \ \ Grandparents \textcolor{white}{C} & 3.47 &  & 3.46 &  & 3.47 &  & 3.46 &  & 3.45 &  & 3.44 &  & 3.49\\
\ \ \ \textcolor{white}{Cg} & (0.46) &  & (0.45) &  & (0.44) &  & (0.45) &  & (0.44) &  & (0.43) &  & (0.44)\\
\ \ \ Parent controls \textcolor{white}{C} & 3.45 &  & 3.45 &  & 3.45 &  & 3.47 &  & 3.46 &  & 3.43 &  & 3.44\\
\ \ \ \textcolor{white}{Cp} & (0.45) &  & (0.45) &  & (0.45) &  & (0.45) &  & (0.46) &  & (0.50) &  & (0.50)\\
\ \ \ Nonparent controls \textcolor{white}{C} & 3.50 &  & 3.48 &  & 3.49 &  & 3.50 &  & 3.48 &  & 3.46 &  & 3.49\\
\ \ \ \textcolor{white}{Cn} & (0.44) &  & (0.44) &  & (0.44) &  & (0.42) &  & (0.45) &  & (0.45) &  & (0.43)\\
Extraversion &  &  &  &  &  &  &  &  &  &  &  &  & \\
\ \ \ Grandparents \textcolor{white}{E} & 3.15 &  & 3.22 &  & 3.20 &  & 3.21 &  & 3.19 &  & 3.22 &  & 3.22\\
\ \ \ \textcolor{white}{Eg} & (0.56) &  & (0.56) &  & (0.54) &  & (0.56) &  & (0.58) &  & (0.59) &  & (0.58)\\
\ \ \ Parent controls \textcolor{white}{E} & 3.20 &  & 3.18 &  & 3.19 &  & 3.21 &  & 3.21 &  & 3.17 &  & 3.19\\
\ \ \ \textcolor{white}{Ep} & (0.51) &  & (0.56) &  & (0.54) &  & (0.54) &  & (0.54) &  & (0.55) &  & (0.56)\\
\ \ \ Nonparent controls \textcolor{white}{E} & 3.19 &  & 3.20 &  & 3.20 &  & 3.23 &  & 3.22 &  & 3.23 &  & 3.24\\
\ \ \ \textcolor{white}{En} & (0.55) &  & (0.54) &  & (0.56) &  & (0.54) &  & (0.54) &  & (0.56) &  & (0.57)\\
Neuroticism &  &  &  &  &  &  &  &  &  &  &  &  & \\
\ \ \ Grandparents \textcolor{white}{N} & 2.00 &  & 1.97 &  & 2.06 &  & 1.91 &  & 1.96 &  & 1.91 &  & 1.91\\
\ \ \ \textcolor{white}{Ng} & (0.56) &  & (0.63) &  & (0.62) &  & (0.60) &  & (0.58) &  & (0.59) &  & (0.61)\\
\ \ \ Parent controls \textcolor{white}{N} & 2.01 &  & 2.05 &  & 2.01 &  & 2.03 &  & 2.00 &  & 2.01 &  & 1.95\\
\ \ \ \textcolor{white}{Np} & (0.59) &  & (0.60) &  & (0.59) &  & (0.61) &  & (0.61) &  & (0.61) &  & (0.60)\\
\ \ \ Nonparent controls \textcolor{white}{N} & 2.05 &  & 2.00 &  & 2.02 &  & 1.92 &  & 1.97 &  & 1.84 &  & 1.90\\
\ \ \ \textcolor{white}{Nn} & (0.56) &  & (0.58) &  & (0.60) &  & (0.57) &  & (0.59) &  & (0.55) &  & (0.58)\\
Openness &  &  &  &  &  &  &  &  &  &  &  &  & \\
\ \ \ Grandparents \textcolor{white}{O} & 3.00 &  & 3.02 &  & 3.04 &  & 3.01 &  & 3.00 &  & 2.96 &  & 3.04\\
\ \ \ \textcolor{white}{Og} & (0.51) &  & (0.53) &  & (0.51) &  & (0.52) &  & (0.52) &  & (0.59) &  & (0.51)\\
\ \ \ Parent controls \textcolor{white}{O} & 3.03 &  & 3.00 &  & 2.98 &  & 3.03 &  & 3.00 &  & 2.96 &  & 2.96\\
\ \ \ \textcolor{white}{Op} & (0.51) &  & (0.56) &  & (0.54) &  & (0.54) &  & (0.52) &  & (0.58) &  & (0.56)\\
\ \ \ Nonparent controls \textcolor{white}{O} & 3.06 &  & 3.05 &  & 3.05 &  & 3.07 &  & 3.06 &  & 3.02 &  & 3.04\\
\ \ \ \textcolor{white}{On} & (0.54) &  & (0.53) &  & (0.55) &  & (0.54) &  & (0.55) &  & (0.57) &  & (0.57)\\
Life satisfaction &  &  &  &  &  &  &  &  &  &  &  &  & \\
\ \ \ Grandparents \textcolor{white}{L} & 5.14 &  & 5.08 &  & 5.15 &  & 5.17 &  & 5.16 &  & 5.29 &  & 5.28\\
\ \ \ \textcolor{white}{Lg} & (1.44) &  & (1.45) &  & (1.46) &  & (1.40) &  & (1.44) &  & (1.38) &  & (1.50)\\
\ \ \ Parent controls \textcolor{white}{L} & 5.14 &  & 4.98 &  & 5.01 &  & 5.11 &  & 5.10 &  & 5.06 &  & 5.12\\
\ \ \ \textcolor{white}{Lp} & (1.52) &  & (1.57) &  & (1.57) &  & (1.52) &  & (1.53) &  & (1.47) &  & (1.47)\\
\ \ \ Nonparent controls \textcolor{white}{L} & 5.10 &  & 5.14 &  & 5.09 &  & 5.26 &  & 5.21 &  & 5.40 &  & 5.40\\
\ \ \ \textcolor{white}{Ln} & (1.49) &  & (1.50) &  & (1.52) &  & (1.44) &  & (1.51) &  & (1.30) &  & (1.36)\\
\bottomrule
\addlinespace
\insertTableNotes
\end{longtable}

}

\end{lltable}













\begin{lltable}

\begin{TableNotes}[para]
\normalsize{\textit{Note.} PSM = propensity score matching, ref. =
reference category, f.~= female, m. = male, NA = covariate not used in
this sample. The standardized difference in means between the
grandparent and the two control groups (parent and nonparent) was
computed by \((\bar{x}_{gp}-\bar{x}_{c})/ (\hat\sigma_{gp})\). Rules of
thumb say that this measure should ideally be below \(.25\) (Stuart,
2010) or below \(.10\) (Austin, 2011).}
\end{TableNotes}

\footnotesize{

\begin{longtable}{lllrrrr}\noalign{\getlongtablewidth\global\LTcapwidth=\longtablewidth}
\caption{\label{tab:stddiffmeans-balance-liss}Standardized Difference in Means for
Covariates Used in Propensity Score Matching and the Propensity Score in
the LISS.}\\
\toprule
&  &  & \multicolumn{2}{c}{Parent control group} & \multicolumn{2}{c}{Nonparent control group} \\
\cmidrule(r){4-5} \cmidrule(r){6-7}
Covariate & Description & Raw variables & Before PSM & After PSM & Before PSM & After PSM\\
\midrule
\endfirsthead
\caption*{\normalfont{Table \ref{tab:stddiffmeans-balance-liss} continued}}\\
\toprule
&  &  & \multicolumn{2}{c}{Parent control group} & \multicolumn{2}{c}{Nonparent control group} \\
\cmidrule(r){4-5} \cmidrule(r){6-7}
Covariate & Description & Raw variables & Before PSM & After PSM & Before PSM & After PSM\\
\midrule
\endhead
pscore & Propensity score & / & 1.14 & 0.02 & 1.34 & 0.04\\
female & Gender (f.=1, m.=0) & geslacht & 0.05 & 0.00 & 0.05 & 0.00\\
age & Age & gebjaar & 0.85 & -0.10 & 4.05 & -0.01\\
degreehighersec & Higher secondary/preparatory university education & oplmet & 0.07 & -0.06 & -0.07 & 0.12\\
degreevocational & Intermediate vocational education & oplmet & -0.20 & -0.06 & -0.02 & 0.00\\
degreecollege & Higher vocational education & oplmet & 0.00 & 0.05 & 0.02 & -0.09\\
degreeuniversity & University degree & oplmet & -0.08 & 0.14 & -0.15 & -0.05\\
religion & Member of religion/church & cr*012 & 0.10 & 0.08 & 0.33 & 0.07\\
speakdutch & Dutch spoken at home (primarily) & cr*089 & -0.02 & -0.06 & 0.00 & -0.02\\
divorced & Divorced (marital status) & burgstat & 0.02 & -0.03 & 0.29 & -0.02\\
widowed & Widowed (marital status) & burgstat & 0.09 & -0.12 & 0.13 & -0.07\\
livetogether & Live together with partner & cf*025 & -0.08 & 0.04 & 1.05 & -0.02\\
rooms & Rooms in dwelling & cd*034 & -0.03 & 0.05 & 0.63 & -0.11\\
logincome & Personal net monthly income in Euros (logarithm) & nettoink & -0.01 & 0.04 & 0.59 & -0.14\\
rental & Live for rent (vs. self-owned dwelling) & woning & -0.08 & -0.09 & -0.47 & -0.03\\
financialsit & Financial situation of household (scale from 1-5) & ci*252 & 0.08 & 0.00 & -0.03 & 0.00\\
jobhours & Average work hours per week & cw*127 & 0.02 & 0.08 & 0.11 & -0.04\\
mobility & Mobility problems (walking, staircase, shopping) & ch*023/027/041 & 0.07 & 0.04 & 0.09 & -0.02\\
dep & Depression items from Mental Health Inventory & ch*011 - ch*015 & -0.01 & 0.08 & -0.22 & -0.08\\
betterhealth & Poor/moderate health status (ref.: good) & ch*004 & 0.00 & -0.01 & -0.26 & 0.07\\
worsehealth & Very good/excellent health status (ref.: good) & ch*004 & 0.04 & -0.02 & 0.11 & -0.04\\
totalchildren & Number living children & cf*455 / cf*036 & 0.25 & 0.02 & NA & NA\\
totalresidentkids & Number of living-at-home children in household & aantalki & -0.71 & 0.02 & NA & NA\\
secondkid & Has two or more children & cf*455 / cf*036 & 0.20 & 0.04 & NA & NA\\
thirdkid & Has three or more children & cf*455 / cf*036 & 0.26 & 0.01 & NA & NA\\
kid1female & Gender of first child (f.=1, m.=0) & cf*068 & 0.04 & 0.04 & NA & NA\\
kid2female & Gender of second child (f.=1, m.=0) & cf*069 & 0.01 & -0.06 & NA & NA\\
kid3female & Gender of third child (f.=1, m.=0) & cf*070 & 0.17 & 0.02 & NA & NA\\
kid1age & Age of first child & cf*456 / cf*037 & 1.70 & -0.17 & NA & NA\\
kid2age & Age of second child & cf*457 / cf*038 & 0.87 & -0.01 & NA & NA\\
kid3age & Age of third child & cf*458 / cf*039 & 0.40 & 0.01 & NA & NA\\
kid1home & First child living at home & cf*083 & -1.56 & 0.05 & NA & NA\\
kid2home & Second child living at home & cf*084 & -1.05 & 0.04 & NA & NA\\
kid3home & Third child living at home & cf*085 & -0.05 & 0.00 & NA & NA\\
swls & Satisfaction with Life Scale & cp*014 - cp*018 & 0.10 & -0.03 & 0.25 & -0.06\\
agree & Agreeableness & cp*021 - cp*066 & 0.05 & -0.01 & 0.13 & -0.13\\
con & Conscientiousness & cp*022 - cp*067 & -0.06 & -0.05 & 0.16 & 0.00\\
extra & Extraversion & cp*020 - cp*065 & 0.05 & 0.02 & 0.02 & -0.07\\
neur & Neuroticism & cp*023 - cp*068 & -0.02 & 0.02 & -0.26 & 0.03\\
open & Openness & cp*024 - cp*069 & 0.06 & 0.05 & -0.16 & -0.08\\
participation & Waves participated & / & -0.27 & -0.09 & 0.09 & -0.03\\
year & Year of assessment & wave & -0.23 & -0.07 & 0.08 & -0.06\\
\bottomrule
\addlinespace
\insertTableNotes
\end{longtable}

}

\end{lltable}





\begin{lltable}

\begin{TableNotes}[para]
\normalsize{\textit{Note.} PSM = propensity score matching, ref. =
reference category, f.~= female, m. = male, NA = covariate not used in
this sample. The standardized difference in means between the
grandparent and the two control groups (parent and nonparent) was
computed by \((\bar{x}_{gp}-\bar{x}_{c})/ (\hat\sigma_{gp})\). Rules of
thumb say that this measure should ideally be below \(.25\) (Stuart,
2010) or below \(.10\) (Austin, 2011).}
\end{TableNotes}

\footnotesize{

\begin{longtable}{lllrrrr}\noalign{\getlongtablewidth\global\LTcapwidth=\longtablewidth}
\caption{\label{tab:stddiffmeans-balance-hrs}Standardized Difference in Means for
Covariates Used in Propensity Score Matching and the Propensity Score in
the HRS.}\\
\toprule
&  &  & \multicolumn{2}{c}{Parent control group} & \multicolumn{2}{c}{Nonparent control group} \\
\cmidrule(r){4-5} \cmidrule(r){6-7}
Covariate & Description & Raw variables & Before PSM & After PSM & Before PSM & After PSM\\
\midrule
\endfirsthead
\caption*{\normalfont{Table \ref{tab:stddiffmeans-balance-hrs} continued}}\\
\toprule
&  &  & \multicolumn{2}{c}{Parent control group} & \multicolumn{2}{c}{Nonparent control group} \\
\cmidrule(r){4-5} \cmidrule(r){6-7}
Covariate & Description & Raw variables & Before PSM & After PSM & Before PSM & After PSM\\
\midrule
\endhead
pscore & Propensity score & / & 0.92 & 0.01 & 1.45 & 0.00\\
female & Gender (f.=1, m.=0) & RAGENDER & -0.07 & 0.00 & 0.01 & 0.00\\
age & Age & RABYEAR & -0.46 & -0.01 & -1.02 & 0.11\\
schlyrs & Years of education & RAEDYRS & 0.11 & 0.03 & 0.25 & -0.04\\
religyear & Religious attendance: yearly & *B082 & 0.04 & 0.01 & 0.13 & 0.00\\
religmonth & Religious attendance: monthly & *B082 & 0.01 & -0.02 & 0.10 & 0.05\\
religweek & Religious attendance: weekly & *B082 & 0.06 & 0.02 & 0.04 & 0.03\\
religmore & Religious attendance: more & *B082 & 0.09 & -0.04 & 0.06 & -0.01\\
notusaborn & Not born in the US & *Z230 & -0.05 & 0.03 & 0.13 & -0.02\\
black & Race: black/african american (ref.: white) & RARACEM & -0.13 & -0.08 & -0.22 & 0.01\\
raceother & Race: other (ref.: white) & RARACEM & -0.09 & -0.06 & 0.01 & -0.05\\
divorced & Divorced (marital status) & R*MSTAT & -0.06 & 0.01 & 0.01 & 0.03\\
widowed & Widowed (marital status) & R*MSTAT & -0.31 & 0.02 & -0.41 & 0.04\\
livetogether & Live together with partner & *A030 / *XF065\_R & 0.25 & -0.02 & 1.05 & -0.04\\
roomslessthree & Number of rooms (in housing unit) & *H147 / *066 & -0.15 & -0.05 & -0.59 & -0.01\\
roomsfourfive & Number of rooms (in housing unit) & *H147 / *066 & 0.00 & -0.02 & -0.25 & -0.03\\
roomsmoreeight & Number of rooms (in housing unit) & *H147 / *066 & 0.07 & -0.03 & 0.28 & 0.00\\
loghhincome & Household income (logarithm) & *ITOT & 0.03 & 0.03 & 0.41 & 0.00\\
loghhwealth & Household wealth (logarithm) & *ATOTB & 0.07 & 0.05 & 0.34 & -0.02\\
renter & Live for rent (vs. self-owned dwelling) & *H004 & -0.10 & -0.08 & -0.51 & -0.02\\
jobhours & Hours worked/week main job & R*JHOURS & 0.25 & 0.08 & 0.59 & 0.00\\
paidwork & Working for pay & *J020 & 0.28 & 0.07 & 0.62 & -0.04\\
mobilitydiff & Difficulty in mobility rated from 0-5 & R*MOBILA & -0.16 & -0.04 & -0.52 & 0.00\\
cesd & CESD score (depression) & R*CESD & -0.13 & -0.04 & -0.26 & -0.04\\
conde & Sum of health conditions & R*CONDE & -0.22 & -0.03 & -0.51 & 0.04\\
healthexcellent & Self-report of health - excellent (ref: good) & R*SHLT & 0.05 & 0.02 & 0.15 & -0.03\\
healthverygood & Self-report of health - very good (ref: good) & R*SHLT & 0.23 & 0.02 & 0.31 & -0.02\\
healthfair & Self-report of health - fair (ref: good) & R*SHLT & -0.16 & -0.02 & -0.29 & 0.00\\
healthpoor & Self-report of health - poor (ref: good) & R*SHLT & -0.07 & -0.03 & -0.24 & 0.02\\
totalnonresidentkids & Number of nonresident kids & *A100 & 0.66 & -0.05 & NA & NA\\
totalresidentkids & Number of resident children & *A099 & -0.22 & 0.00 & NA & NA\\
secondkid & Has two or more children & KIDID & 0.52 & -0.03 & NA & NA\\
thirdkid & Has three or more children & KIDID & 0.38 & -0.03 & NA & NA\\
kid1female & Gender of first child (f.=1, m.=0) & KAGENDERBG & 0.11 & 0.03 & NA & NA\\
kid2female & Gender of second child (f.=1, m.=0) & KAGENDERBG & 0.17 & -0.01 & NA & NA\\
kid3female & Gender of third child (f.=1, m.=0) & KAGENDERBG & 0.24 & 0.02 & NA & NA\\
kid1age & Age of first child & KABYEARBG & -0.35 & -0.02 & NA & NA\\
kid2age & Age of second child & KABYEARBG & 0.36 & -0.03 & NA & NA\\
kid3age & Age of third child & KABYEARBG & 0.35 & -0.01 & NA & NA\\
kid1educ & Education of first child (years) & KAEDUC & 0.30 & 0.02 & NA & NA\\
kid2educ & Education of second child (years) & KAEDUC & 0.57 & 0.00 & NA & NA\\
kid3educ & Education of third child (years) & KAEDUC & 0.40 & -0.02 & NA & NA\\
childrenclose & Children live within 10 miles & *E012 & 0.14 & 0.01 & NA & NA\\
siblings & Number of living siblings & R*LIVSIB & 0.05 & -0.04 & 0.21 & 0.03\\
swls & Satisfaction with Life Scale & *LB003* & 0.17 & 0.08 & 0.30 & 0.00\\
agree & Agreeableness & *LB033* & 0.06 & 0.04 & 0.11 & 0.02\\
con & Conscientiousness & *LB033* & 0.14 & 0.04 & 0.26 & -0.04\\
extra & Extraversion & *LB033* & 0.04 & 0.04 & 0.18 & 0.01\\
neur & Neuroticism & *LB033* & -0.06 & 0.00 & -0.04 & 0.01\\
open & Openness & *LB033* & 0.04 & 0.07 & 0.05 & -0.04\\
participation & Waves participated (2006-2018) & / & -0.36 & -0.01 & -0.26 & -0.04\\
interviewyear & Date of interview - year & *A501 & -0.33 & -0.05 & -0.18 & -0.05\\
\bottomrule
\addlinespace
\insertTableNotes
\end{longtable}

}

\end{lltable}








\begin{lltable}

\begin{TableNotes}[para]
\normalsize{\textit{Note.} Two models were computed for each of the two
samples (LISS, HRS): grandparents matched with parent controls and with
nonparent controls. CI = confidence interval.}
\end{TableNotes}

\footnotesize{

\begin{longtable}{lrcrrrcrr}\noalign{\getlongtablewidth\global\LTcapwidth=\longtablewidth}
\caption{\label{tab:H1-agree-tab}Fixed Effects of Agreeableness Over the
Transition to Grandparenthood.}\\
\toprule
& \multicolumn{4}{c}{Parent controls} & \multicolumn{4}{c}{Nonparent controls} \\
\cmidrule(r){2-5} \cmidrule(r){6-9}
Parameter & $\hat{\gamma}$ & 95\% CI & $t$ & $p$ & $\hat{\gamma}$ & 95\% CI & $t$ & $p$\\
\midrule
\endfirsthead
\caption*{\normalfont{Table \ref{tab:H1-agree-tab} continued}}\\
\toprule
& \multicolumn{4}{c}{Parent controls} & \multicolumn{4}{c}{Nonparent controls} \\
\cmidrule(r){2-5} \cmidrule(r){6-9}
Parameter & $\hat{\gamma}$ & 95\% CI & $t$ & $p$ & $\hat{\gamma}$ & 95\% CI & $t$ & $p$\\
\midrule
\endhead
LISS &  &  &  &  &  &  &  & \\
\ \ \ Intercept, $\hat{\gamma}_{00}$ \textcolor{white}{L} & 3.86 & [3.80, 3.92] & 131.70 & < .001 & 3.90 & [3.83, 3.97] & 112.97 & < .001\\
\ \ \ Propensity score, $\hat{\gamma}_{02}$ \textcolor{white}{L} & -0.02 & [-0.10, 0.05] & -0.56 & .572 & -0.01 & [-0.08, 0.06] & -0.20 & .838\\
\ \ \ Before-slope, $\hat{\gamma}_{10}$ \textcolor{white}{L} & 0.00 & [-0.01, 0.00] & -0.25 & .802 & -0.01 & [-0.01, 0.00] & -1.81 & .070\\
\ \ \ After-slope, $\hat{\gamma}_{20}$ \textcolor{white}{L} & -0.02 & [-0.02, -0.01] & -6.76 & < .001 & -0.01 & [-0.01, 0.00] & -3.32 & .001\\
\ \ \ Shift, $\hat{\gamma}_{30}$ \textcolor{white}{L} & 0.04 & [0.01, 0.06] & 3.12 & .002 & 0.03 & [0.00, 0.05] & 1.98 & .048\\
\ \ \ Grandparent, $\hat{\gamma}_{01}$ \textcolor{white}{L} & 0.06 & [-0.03, 0.15] & 1.33 & .183 & 0.01 & [-0.08, 0.11] & 0.30 & .768\\
\ \ \ Before-slope * Grandparent, $\hat{\gamma}_{11}$ \textcolor{white}{L} & -0.01 & [-0.02, 0.01] & -1.06 & .289 & 0.00 & [-0.01, 0.01] & -0.26 & .791\\
\ \ \ After-slope * Grandparent, $\hat{\gamma}_{21}$ \textcolor{white}{L} & 0.02 & [0.01, 0.03] & 2.99 & .003 & 0.01 & [0.00, 0.02] & 1.44 & .149\\
\ \ \ Shift * Grandparent, $\hat{\gamma}_{31}$ \textcolor{white}{L} & -0.01 & [-0.06, 0.04] & -0.37 & .714 & 0.00 & [-0.06, 0.06] & 0.08 & .937\\
HRS &  &  &  &  &  &  &  & \\
\ \ \ Intercept, $\hat{\gamma}_{00}$ \textcolor{white}{H} & 3.46 & [3.43, 3.50] & 196.32 & < .001 & 3.48 & [3.44, 3.52] & 166.19 & < .001\\
\ \ \ Propensity score, $\hat{\gamma}_{02}$ \textcolor{white}{H} & 0.08 & [0.02, 0.14] & 2.51 & .012 & 0.05 & [-0.01, 0.11] & 1.51 & .131\\
\ \ \ Before-slope, $\hat{\gamma}_{10}$ \textcolor{white}{H} & 0.01 & [0.00, 0.02] & 1.37 & .169 & -0.01 & [-0.02, 0.00] & -1.33 & .184\\
\ \ \ After-slope, $\hat{\gamma}_{20}$ \textcolor{white}{H} & -0.01 & [-0.02, 0.00] & -2.87 & .004 & -0.02 & [-0.02, -0.01] & -5.16 & < .001\\
\ \ \ Shift, $\hat{\gamma}_{30}$ \textcolor{white}{H} & 0.01 & [-0.01, 0.03] & 0.71 & .476 & 0.04 & [0.02, 0.06] & 4.30 & < .001\\
\ \ \ Grandparent, $\hat{\gamma}_{01}$ \textcolor{white}{H} & 0.02 & [-0.03, 0.08] & 0.88 & .378 & 0.01 & [-0.04, 0.07] & 0.44 & .662\\
\ \ \ Before-slope * Grandparent, $\hat{\gamma}_{11}$ \textcolor{white}{H} & -0.01 & [-0.04, 0.01] & -0.87 & .384 & 0.00 & [-0.02, 0.03] & 0.28 & .781\\
\ \ \ After-slope * Grandparent, $\hat{\gamma}_{21}$ \textcolor{white}{H} & 0.01 & [0.00, 0.03] & 1.71 & .088 & 0.02 & [0.01, 0.04] & 2.78 & .006\\
\ \ \ Shift * Grandparent, $\hat{\gamma}_{31}$ \textcolor{white}{H} & -0.01 & [-0.05, 0.04] & -0.35 & .729 & -0.04 & [-0.09, 0.00] & -1.97 & .049\\
\bottomrule
\addlinespace
\insertTableNotes
\end{longtable}

}

\end{lltable}









\begin{lltable}

\begin{TableNotes}[para]
\normalsize{\textit{Note.} The linear contrasts are needed in cases
where estimates of interest are represented by multiple fixed-effects
coefficients and are computed using the \emph{linearHypothesis} function
from the \emph{car} R package (Fox \& Weisberg, 2019) based on the
models from Table \ref{tab:H1-agree-tab}. \(\hat{\gamma}_{c}\) =
combined fixed-effects estimate.}
\end{TableNotes}

\footnotesize{

\begin{longtable}{lrrrrrr}\noalign{\getlongtablewidth\global\LTcapwidth=\longtablewidth}
\caption{\label{tab:H1-agree-contrasts}Linear Contrasts for Agreeableness.}\\
\toprule
& \multicolumn{3}{c}{Parent controls} & \multicolumn{3}{c}{Nonparent controls} \\
\cmidrule(r){2-4} \cmidrule(r){5-7}
Linear Contrast & $\hat{\gamma}_{c}$ & $\chi^2$ & $p$ & $\hat{\gamma}_{c}$ & $\chi^2$ & $p$\\
\midrule
\endfirsthead
\caption*{\normalfont{Table \ref{tab:H1-agree-contrasts} continued}}\\
\toprule
& \multicolumn{3}{c}{Parent controls} & \multicolumn{3}{c}{Nonparent controls} \\
\cmidrule(r){2-4} \cmidrule(r){5-7}
Linear Contrast & $\hat{\gamma}_{c}$ & $\chi^2$ & $p$ & $\hat{\gamma}_{c}$ & $\chi^2$ & $p$\\
\midrule
\endhead
LISS &  &  &  &  &  & \\
\ \ \ Shift of the controls vs. 0 ($\hat{\gamma}_{20}$ + 
$\hat{\gamma}_{30}$) \textcolor{white}{L} & 0.02 & 4.00 & .046 & 0.02 & 2.22 & .136\\
\ \ \ Shift of the grandparents vs. 0 ($\hat{\gamma}_{20}$ + 
$\hat{\gamma}_{30}$ + $\hat{\gamma}_{21}$ + 
$\hat{\gamma}_{31}$) \textcolor{white}{L} & 0.03 & 1.79 & .181 & 0.03 & 1.51 & .219\\
\ \ \ Shift of the controls vs. shift of the grandparents 
($\hat{\gamma}_{21}$ + $\hat{\gamma}_{31}$) \textcolor{white}{L} & 0.01 & 0.08 & .779 & 0.01 & 0.18 & .668\\
\ \ \ Before-slope of the grandparents vs. 0 ($\hat{\gamma}_{10}$ + 
$\hat{\gamma}_{11}$) \textcolor{white}{L} & -0.01 & 1.72 & .189 & -0.01 & 1.45 & .228\\
\ \ \ After-slope of the grandparents vs. 0 ($\hat{\gamma}_{20}$ + 
$\hat{\gamma}_{21}$) \textcolor{white}{L} & 0.00 & 0.01 & .934 & 0.00 & 0.00 & .958\\
HRS &  &  &  &  &  & \\
\ \ \ Shift of the controls vs. 0 ($\hat{\gamma}_{20}$ + 
$\hat{\gamma}_{30}$) \textcolor{white}{H} & 0.00 & 0.12 & .725 & 0.03 & 10.76 & .001\\
\ \ \ Shift of the grandparents vs. 0 ($\hat{\gamma}_{20}$ + 
$\hat{\gamma}_{30}$ + $\hat{\gamma}_{21}$ + 
$\hat{\gamma}_{31}$) \textcolor{white}{H} & 0.00 & 0.03 & .859 & 0.00 & 0.03 & .862\\
\ \ \ Shift of the controls vs. shift of the grandparents 
($\hat{\gamma}_{21}$ + $\hat{\gamma}_{31}$) \textcolor{white}{H} & 0.01 & 0.10 & .751 & -0.02 & 1.77 & .183\\
\ \ \ Before-slope of the grandparents vs. 0 ($\hat{\gamma}_{10}$ + 
$\hat{\gamma}_{11}$) \textcolor{white}{H} & 0.00 & 0.09 & .762 & 0.00 & 0.11 & .743\\
\ \ \ After-slope of the grandparents vs. 0 ($\hat{\gamma}_{20}$ + 
$\hat{\gamma}_{21}$) \textcolor{white}{H} & 0.00 & 0.23 & .633 & 0.00 & 0.28 & .596\\
\bottomrule
\addlinespace
\insertTableNotes
\end{longtable}

}

\end{lltable}







\begin{lltable}

\begin{TableNotes}[para]
\normalsize{\textit{Note.} Two models were computed for each of the
two samples (LISS, HRS): grandparents matched with parent controls and
with nonparent controls. CI = confidence interval.}
\end{TableNotes}

\footnotesize{

\begin{longtable}{lrcrrrcrr}\noalign{\getlongtablewidth\global\LTcapwidth=\longtablewidth}
\caption{\label{tab:H1-agree-gender-tab}Fixed Effects of Agreeableness Over the
Transition to Grandparenthood Moderated by Gender.}\\
\toprule
& \multicolumn{4}{c}{Parent controls} & \multicolumn{4}{c}{Nonparent controls} \\
\cmidrule(r){2-5} \cmidrule(r){6-9}
Parameter & $\hat{\gamma}$ & 95\% CI & $t$ & $p$ & $\hat{\gamma}$ & 95\% CI & $t$ & $p$\\
\midrule
\endfirsthead
\caption*{\normalfont{Table \ref{tab:H1-agree-gender-tab} continued}}\\
\toprule
& \multicolumn{4}{c}{Parent controls} & \multicolumn{4}{c}{Nonparent controls} \\
\cmidrule(r){2-5} \cmidrule(r){6-9}
Parameter & $\hat{\gamma}$ & 95\% CI & $t$ & $p$ & $\hat{\gamma}$ & 95\% CI & $t$ & $p$\\
\midrule
\endhead
LISS &  &  &  &  &  &  &  & \\
\ \ \ Intercept, $\hat{\gamma}_{00}$ \textcolor{white}{L} & 3.65 & [3.58, 3.73] & 93.02 & < .001 & 3.66 & [3.57, 3.75] & 79.73 & < .001\\
\ \ \ Propensity score, $\hat{\gamma}_{04}$ \textcolor{white}{L} & -0.01 & [-0.08, 0.07] & -0.21 & .833 & 0.02 & [-0.05, 0.08] & 0.45 & .653\\
\ \ \ Before-slope, $\hat{\gamma}_{10}$ \textcolor{white}{L} & 0.00 & [-0.01, 0.01] & 0.02 & .984 & 0.00 & [-0.01, 0.01] & -0.37 & .712\\
\ \ \ After-slope, $\hat{\gamma}_{20}$ \textcolor{white}{L} & -0.02 & [-0.03, -0.02] & -6.37 & < .001 & -0.01 & [-0.02, 0.00] & -2.49 & .013\\
\ \ \ Shift, $\hat{\gamma}_{30}$ \textcolor{white}{L} & 0.03 & [-0.01, 0.07] & 1.66 & .097 & 0.07 & [0.03, 0.11] & 3.66 & < .001\\
\ \ \ Grandparent, $\hat{\gamma}_{01}$ \textcolor{white}{L} & 0.06 & [-0.06, 0.17] & 0.92 & .356 & 0.04 & [-0.09, 0.17] & 0.60 & .550\\
\ \ \ Female, $\hat{\gamma}_{02}$ \textcolor{white}{L} & 0.38 & [0.27, 0.48] & 7.16 & < .001 & 0.44 & [0.32, 0.56] & 7.11 & < .001\\
\ \ \ Before-slope * Grandparent, $\hat{\gamma}_{11}$ \textcolor{white}{L} & -0.01 & [-0.03, 0.01] & -0.73 & .466 & 0.00 & [-0.02, 0.01] & -0.50 & .615\\
\ \ \ After-slope * Grandparent, $\hat{\gamma}_{21}$ \textcolor{white}{L} & 0.03 & [0.01, 0.05] & 3.43 & .001 & 0.01 & [0.00, 0.03] & 1.64 & .101\\
\ \ \ Shift * Grandparent, $\hat{\gamma}_{31}$ \textcolor{white}{L} & -0.01 & [-0.09, 0.07] & -0.33 & .739 & -0.05 & [-0.14, 0.03] & -1.23 & .217\\
\ \ \ Before-slope * Female, $\hat{\gamma}_{12}$ \textcolor{white}{L} & 0.00 & [-0.01, 0.01] & -0.26 & .799 & -0.01 & [-0.02, 0.00] & -1.14 & .254\\
\ \ \ After-slope * Female, $\hat{\gamma}_{22}$ \textcolor{white}{L} & 0.01 & [0.00, 0.02] & 2.34 & .019 & 0.00 & [-0.01, 0.01] & 0.28 & .781\\
\ \ \ Shift * Female, $\hat{\gamma}_{32}$ \textcolor{white}{L} & 0.02 & [-0.03, 0.06] & 0.60 & .550 & -0.08 & [-0.14, -0.03] & -3.18 & .001\\
\ \ \ Grandparent * Female, $\hat{\gamma}_{03}$ \textcolor{white}{L} & 0.01 & [-0.15, 0.17] & 0.15 & .883 & -0.05 & [-0.22, 0.12] & -0.57 & .568\\
\ \ \ Before-slope * Grandparent * Female, $\hat{\gamma}_{13}$ \textcolor{white}{L} & 0.00 & [-0.03, 0.02] & -0.05 & .959 & 0.00 & [-0.02, 0.03] & 0.35 & .728\\
\ \ \ After-slope * Grandparent * Female, $\hat{\gamma}_{23}$ \textcolor{white}{L} & -0.02 & [-0.04, 0.00] & -1.92 & .056 & -0.01 & [-0.03, 0.01] & -0.93 & .351\\
\ \ \ Shift * Grandparent * Female, $\hat{\gamma}_{33}$ \textcolor{white}{L} & 0.01 & [-0.10, 0.12] & 0.21 & .836 & 0.11 & [-0.01, 0.23] & 1.87 & .061\\
HRS &  &  &  &  &  &  &  & \\
\ \ \ Intercept, $\hat{\gamma}_{00}$ \textcolor{white}{H} & 3.27 & [3.23, 3.32] & 132.82 & < .001 & 3.38 & [3.33, 3.43] & 122.35 & < .001\\
\ \ \ Propensity score, $\hat{\gamma}_{04}$ \textcolor{white}{H} & 0.09 & [0.03, 0.15] & 2.91 & .004 & 0.04 & [-0.03, 0.10] & 1.12 & .261\\
\ \ \ Before-slope, $\hat{\gamma}_{10}$ \textcolor{white}{H} & 0.02 & [0.01, 0.04] & 2.98 & .003 & -0.01 & [-0.02, 0.01] & -1.12 & .262\\
\ \ \ After-slope, $\hat{\gamma}_{20}$ \textcolor{white}{H} & -0.02 & [-0.03, -0.01] & -3.95 & < .001 & -0.02 & [-0.03, -0.01] & -3.43 & .001\\
\ \ \ Shift, $\hat{\gamma}_{30}$ \textcolor{white}{H} & 0.04 & [0.01, 0.07] & 2.77 & .006 & 0.03 & [0.00, 0.06] & 1.68 & .093\\
\ \ \ Grandparent, $\hat{\gamma}_{01}$ \textcolor{white}{H} & 0.08 & [0.00, 0.16] & 1.97 & .048 & -0.01 & [-0.09, 0.08] & -0.16 & .877\\
\ \ \ Female, $\hat{\gamma}_{02}$ \textcolor{white}{H} & 0.33 & [0.27, 0.39] & 10.55 & < .001 & 0.20 & [0.13, 0.26] & 5.76 & < .001\\
\ \ \ Before-slope * Grandparent, $\hat{\gamma}_{11}$ \textcolor{white}{H} & -0.04 & [-0.08, 0.00] & -2.18 & .030 & -0.01 & [-0.04, 0.03] & -0.47 & .640\\
\ \ \ After-slope * Grandparent, $\hat{\gamma}_{21}$ \textcolor{white}{H} & 0.04 & [0.01, 0.06] & 3.00 & .003 & 0.03 & [0.01, 0.05] & 2.85 & .004\\
\ \ \ Shift * Grandparent, $\hat{\gamma}_{31}$ \textcolor{white}{H} & -0.05 & [-0.12, 0.02] & -1.50 & .133 & -0.03 & [-0.10, 0.03] & -1.04 & .298\\
\ \ \ Before-slope * Female, $\hat{\gamma}_{12}$ \textcolor{white}{H} & -0.03 & [-0.05, -0.01] & -2.84 & .004 & 0.00 & [-0.02, 0.02] & 0.38 & .702\\
\ \ \ After-slope * Female, $\hat{\gamma}_{22}$ \textcolor{white}{H} & 0.02 & [0.01, 0.03] & 2.74 & .006 & 0.00 & [-0.01, 0.01] & 0.08 & .937\\
\ \ \ Shift * Female, $\hat{\gamma}_{32}$ \textcolor{white}{H} & -0.06 & [-0.11, -0.02] & -3.07 & .002 & 0.03 & [-0.01, 0.07] & 1.50 & .134\\
\ \ \ Grandparent * Female, $\hat{\gamma}_{03}$ \textcolor{white}{H} & -0.10 & [-0.20, 0.01] & -1.77 & .077 & 0.03 & [-0.07, 0.14] & 0.64 & .521\\
\ \ \ Before-slope * Grandparent * Female, $\hat{\gamma}_{13}$ \textcolor{white}{H} & 0.06 & [0.01, 0.11] & 2.20 & .028 & 0.02 & [-0.03, 0.07] & 0.86 & .392\\
\ \ \ After-slope * Grandparent * Female, $\hat{\gamma}_{23}$ \textcolor{white}{H} & -0.04 & [-0.07, -0.01] & -2.48 & .013 & -0.02 & [-0.05, 0.01] & -1.34 & .180\\
\ \ \ Shift * Grandparent * Female, $\hat{\gamma}_{33}$ \textcolor{white}{H} & 0.08 & [-0.01, 0.17] & 1.73 & .084 & -0.01 & [-0.10, 0.07] & -0.31 & .758\\
\bottomrule
\addlinespace
\insertTableNotes
\end{longtable}

}

\end{lltable}








\begin{lltable}

\begin{TableNotes}[para]
\normalsize{\textit{Note.} The linear contrasts are based on
the models from Table \ref{tab:H1-agree-gender-tab}.
\(\hat{\gamma}_{c}\) = combined fixed-effects estimate.}
\end{TableNotes}

\footnotesize{

\begin{longtable}{lrrrrrr}\noalign{\getlongtablewidth\global\LTcapwidth=\longtablewidth}
\caption{\label{tab:H1-agree-gender-contrasts}Linear Contrasts for Agreeableness
(Moderated by Gender).}\\
\toprule
& \multicolumn{3}{c}{Parent controls} & \multicolumn{3}{c}{Nonparent controls} \\
\cmidrule(r){2-4} \cmidrule(r){5-7}
Linear Contrast & $\hat{\gamma}_{c}$ & $\chi^2$ & $p$ & $\hat{\gamma}_{c}$ & $\chi^2$ & $p$\\
\midrule
\endfirsthead
\caption*{\normalfont{Table \ref{tab:H1-agree-gender-contrasts} continued}}\\
\toprule
& \multicolumn{3}{c}{Parent controls} & \multicolumn{3}{c}{Nonparent controls} \\
\cmidrule(r){2-4} \cmidrule(r){5-7}
Linear Contrast & $\hat{\gamma}_{c}$ & $\chi^2$ & $p$ & $\hat{\gamma}_{c}$ & $\chi^2$ & $p$\\
\midrule
\endhead
LISS &  &  &  &  &  & \\
\ \ \ Shift of male controls vs. 0 ($\hat{\gamma}_{20}$ + 
$\hat{\gamma}_{30}$) \textcolor{white}{L} & 0.01 & 0.19 & .665 & 0.06 & 13.04 & < .001\\
\ \ \ Shift of female controls vs. 0 ($\hat{\gamma}_{20}$ + 
$\hat{\gamma}_{30}$ + $\hat{\gamma}_{22}$ + 
$\hat{\gamma}_{32}$) \textcolor{white}{L} & 0.03 & 5.25 & .022 & -0.02 & 1.90 & .168\\
\ \ \ Shift of grandfathers vs. 0 ($\hat{\gamma}_{20}$ + 
$\hat{\gamma}_{30}$ + $\hat{\gamma}_{21}$ + 
$\hat{\gamma}_{31}$) \textcolor{white}{L} & 0.02 & 0.47 & .493 & 0.02 & 0.40 & .525\\
\ \ \ Shift of grandmothers vs. 0 ($\hat{\gamma}_{20}$ + 
$\hat{\gamma}_{30}$ + $\hat{\gamma}_{21}$ + 
$\hat{\gamma}_{31}$ + $\hat{\gamma}_{22}$ + 
$\hat{\gamma}_{32}$ + $\hat{\gamma}_{23}$ +
$\hat{\gamma}_{33}$) \textcolor{white}{L} & 0.04 & 1.79 & .181 & 0.04 & 1.56 & .212\\
\ \ \ Shift of male controls vs. grandfathers 
($\hat{\gamma}_{21}$ + $\hat{\gamma}_{31}$) \textcolor{white}{L} & 0.01 & 0.17 & .678 & -0.04 & 1.05 & .305\\
\ \ \ Before-slope of female controls vs. grandmothers 
($\hat{\gamma}_{11}$ + $\hat{\gamma}_{13}$) \textcolor{white}{L} & -0.01 & 0.78 & .376 & 0.00 & 0.00 & .971\\
\ \ \ After-slope of female controls vs. grandmothers 
($\hat{\gamma}_{21}$ + $\hat{\gamma}_{23}$) \textcolor{white}{L} & 0.01 & 0.78 & .377 & 0.00 & 0.15 & .695\\
\ \ \ Shift of female controls vs. grandmothers 
($\hat{\gamma}_{21}$ + $\hat{\gamma}_{31}$ + 
$\hat{\gamma}_{23}$ + $\hat{\gamma}_{33}$) \textcolor{white}{L} & 0.00 & 0.02 & .886 & 0.06 & 3.02 & .082\\
\ \ \ Shift of male vs. female controls 
($\hat{\gamma}_{22}$ + $\hat{\gamma}_{32}$) \textcolor{white}{L} & 0.03 & 1.51 & .219 & -0.08 & 12.80 & < .001\\
\ \ \ Before-slope of grandfathers vs. grandmothers 
($\hat{\gamma}_{12}$ + $\hat{\gamma}_{13}$) \textcolor{white}{L} & 0.00 & 0.03 & .853 & 0.00 & 0.03 & .857\\
\ \ \ After-slope of grandfathers vs. grandmothers 
($\hat{\gamma}_{22}$ + $\hat{\gamma}_{23}$) \textcolor{white}{L} & -0.01 & 0.92 & .337 & -0.01 & 0.82 & .366\\
\ \ \ Shift of grandfathers vs. grandmothers 
($\hat{\gamma}_{22}$ + $\hat{\gamma}_{32}$ + 
$\hat{\gamma}_{23}$ + $\hat{\gamma}_{33}$) \textcolor{white}{L} & 0.02 & 0.15 & .695 & 0.02 & 0.14 & .712\\
HRS &  &  &  &  &  & \\
\ \ \ Shift of male controls vs. 0 ($\hat{\gamma}_{20}$ + 
$\hat{\gamma}_{30}$) \textcolor{white}{H} & 0.02 & 3.34 & .067 & 0.01 & 0.41 & .520\\
\ \ \ Shift of female controls vs. 0 ($\hat{\gamma}_{20}$ + 
$\hat{\gamma}_{30}$ + $\hat{\gamma}_{22}$ + 
$\hat{\gamma}_{32}$) \textcolor{white}{H} & -0.02 & 4.49 & .034 & 0.04 & 14.19 & < .001\\
\ \ \ Shift of grandfathers vs. 0 ($\hat{\gamma}_{20}$ + 
$\hat{\gamma}_{30}$ + $\hat{\gamma}_{21}$ + 
$\hat{\gamma}_{31}$) \textcolor{white}{H} & 0.01 & 0.05 & .818 & 0.01 & 0.05 & .815\\
\ \ \ Shift of grandmothers vs. 0 ($\hat{\gamma}_{20}$ + 
$\hat{\gamma}_{30}$ + $\hat{\gamma}_{21}$ + 
$\hat{\gamma}_{31}$ + $\hat{\gamma}_{22}$ + 
$\hat{\gamma}_{32}$ + $\hat{\gamma}_{23}$ +
$\hat{\gamma}_{33}$) \textcolor{white}{H} & 0.00 & 0.01 & .927 & 0.00 & 0.01 & .936\\
\ \ \ Shift of male controls vs. grandfathers 
($\hat{\gamma}_{21}$ + $\hat{\gamma}_{31}$) \textcolor{white}{H} & -0.02 & 0.39 & .531 & 0.00 & 0.01 & .929\\
\ \ \ Before-slope of female controls vs. grandmothers 
($\hat{\gamma}_{11}$ + $\hat{\gamma}_{13}$) \textcolor{white}{H} & 0.01 & 0.74 & .390 & 0.01 & 0.58 & .445\\
\ \ \ After-slope of female controls vs. grandmothers 
($\hat{\gamma}_{21}$ + $\hat{\gamma}_{23}$) \textcolor{white}{H} & 0.00 & 0.15 & .701 & 0.01 & 1.32 & .250\\
\ \ \ Shift of female controls vs. grandmothers 
($\hat{\gamma}_{21}$ + $\hat{\gamma}_{31}$ + 
$\hat{\gamma}_{23}$ + $\hat{\gamma}_{33}$) \textcolor{white}{H} & 0.02 & 1.07 & .301 & -0.04 & 2.61 & .106\\
\ \ \ Shift of male vs. female controls 
($\hat{\gamma}_{22}$ + $\hat{\gamma}_{32}$) \textcolor{white}{H} & -0.04 & 7.70 & .006 & 0.03 & 3.92 & .048\\
\ \ \ Before-slope of grandfathers vs. grandmothers 
($\hat{\gamma}_{12}$ + $\hat{\gamma}_{13}$) \textcolor{white}{H} & 0.02 & 1.17 & .279 & 0.02 & 1.28 & .258\\
\ \ \ After-slope of grandfathers vs. grandmothers 
($\hat{\gamma}_{22}$ + $\hat{\gamma}_{23}$) \textcolor{white}{H} & -0.02 & 1.94 & .163 & -0.02 & 2.13 & .144\\
\ \ \ Shift of grandfathers vs. grandmothers 
($\hat{\gamma}_{22}$ + $\hat{\gamma}_{32}$ + 
$\hat{\gamma}_{23}$ + $\hat{\gamma}_{33}$) \textcolor{white}{H} & 0.00 & 0.01 & .912 & 0.00 & 0.01 & .904\\
\bottomrule
\addlinespace
\insertTableNotes
\end{longtable}

}

\end{lltable}








\begin{lltable}

\begin{TableNotes}[para]
\normalsize{\textit{Note.} Two models were computed (only HRS):
grandparents matched with parent controls and with nonparent controls.
CI = confidence interval. \(working=1\) indicates being employed in paid
work.}
\end{TableNotes}

\footnotesize{

\begin{longtable}{lrcrrrcrr}\noalign{\getlongtablewidth\global\LTcapwidth=\longtablewidth}
\caption{\label{tab:H1-agree-work-tab}Fixed Effects of Agreeableness Over the
Transition to Grandparenthood Moderated by Performing Paid Work.}\\
\toprule
& \multicolumn{4}{c}{Parent controls} & \multicolumn{4}{c}{Nonparent controls} \\
\cmidrule(r){2-5} \cmidrule(r){6-9}
Parameter & $\hat{\gamma}$ & 95\% CI & $t$ & $p$ & $\hat{\gamma}$ & 95\% CI & $t$ & $p$\\
\midrule
\endfirsthead
\caption*{\normalfont{Table \ref{tab:H1-agree-work-tab} continued}}\\
\toprule
& \multicolumn{4}{c}{Parent controls} & \multicolumn{4}{c}{Nonparent controls} \\
\cmidrule(r){2-5} \cmidrule(r){6-9}
Parameter & $\hat{\gamma}$ & 95\% CI & $t$ & $p$ & $\hat{\gamma}$ & 95\% CI & $t$ & $p$\\
\midrule
\endhead
Intercept, $\hat{\gamma}_{00}$ & 3.50 & [3.45, 3.54] & 157.26 & < .001 & 3.48 & [3.43, 3.52] & 138.40 & < .001\\
Propensity score, $\hat{\gamma}_{02}$ & 0.09 & [0.03, 0.15] & 2.93 & .003 & 0.04 & [-0.03, 0.10] & 1.14 & .253\\
Before-slope, $\hat{\gamma}_{20}$ & 0.01 & [-0.01, 0.03] & 0.91 & .363 & 0.00 & [-0.02, 0.02] & -0.23 & .819\\
After-slope, $\hat{\gamma}_{40}$ & -0.02 & [-0.03, -0.01] & -4.07 & < .001 & -0.03 & [-0.04, -0.02] & -5.38 & < .001\\
Shift, $\hat{\gamma}_{60}$ & -0.01 & [-0.04, 0.02] & -0.53 & .594 & 0.07 & [0.03, 0.10] & 3.93 & < .001\\
Grandparent, $\hat{\gamma}_{01}$ & -0.11 & [-0.20, -0.02] & -2.33 & .020 & -0.07 & [-0.16, 0.02] & -1.49 & .137\\
Working, $\hat{\gamma}_{10}$ & -0.06 & [-0.10, -0.02] & -2.77 & .006 & 0.01 & [-0.03, 0.05] & 0.61 & .540\\
Before-slope * Grandparent, $\hat{\gamma}_{21}$ & 0.04 & [-0.01, 0.09] & 1.55 & .121 & 0.05 & [0.00, 0.10] & 2.09 & .037\\
After-slope * Grandparent, $\hat{\gamma}_{41}$ & 0.02 & [0.00, 0.05] & 1.96 & .050 & 0.03 & [0.01, 0.05] & 2.68 & .007\\
Shift * Grandparent, $\hat{\gamma}_{61}$ & 0.00 & [-0.08, 0.07] & -0.07 & .947 & -0.08 & [-0.15, -0.01] & -2.17 & .030\\
Before-slope * Working, $\hat{\gamma}_{30}$ & 0.00 & [-0.03, 0.02] & -0.30 & .767 & 0.00 & [-0.03, 0.02] & -0.37 & .712\\
After-slope * Working, $\hat{\gamma}_{50}$ & 0.02 & [0.01, 0.04] & 2.87 & .004 & 0.02 & [0.01, 0.03] & 2.83 & .005\\
Shift * Working, $\hat{\gamma}_{70}$ & 0.02 & [-0.03, 0.06] & 0.77 & .441 & -0.04 & [-0.08, 0.00] & -1.87 & .061\\
Grandparent * Working, $\hat{\gamma}_{11}$ & 0.18 & [0.08, 0.28] & 3.68 & < .001 & 0.11 & [0.02, 0.20] & 2.40 & .017\\
Before-slope * Grandparent * Working, $\hat{\gamma}_{31}$ & -0.06 & [-0.12, -0.01] & -2.15 & .032 & -0.06 & [-0.12, -0.01] & -2.22 & .026\\
After-slope * Grandparent * Working, $\hat{\gamma}_{51}$ & -0.02 & [-0.05, 0.02] & -0.97 & .333 & -0.01 & [-0.05, 0.02] & -0.94 & .347\\
Shift * Grandparent * Working, $\hat{\gamma}_{71}$ & -0.01 & [-0.10, 0.09] & -0.11 & .914 & 0.05 & [-0.04, 0.14] & 1.08 & .282\\
\bottomrule
\addlinespace
\insertTableNotes
\end{longtable}

}

\end{lltable}








\begin{lltable}

\begin{TableNotes}[para]
\normalsize{\textit{Note.} The linear contrasts are based on the
models from Table \ref{tab:H1-agree-work-tab}. \(\hat{\gamma}_{c}\) =
combined fixed-effects estimate.}
\end{TableNotes}

\footnotesize{

\begin{longtable}{lrrrrrr}\noalign{\getlongtablewidth\global\LTcapwidth=\longtablewidth}
\caption{\label{tab:H1-agree-work-contrasts}Linear Contrasts for Agreeableness
(Moderated by Paid Work; only HRS).}\\
\toprule
& \multicolumn{3}{c}{Parent controls} & \multicolumn{3}{c}{Nonparent controls} \\
\cmidrule(r){2-4} \cmidrule(r){5-7}
Linear Contrast & $\hat{\gamma}_{c}$ & $\chi^2$ & $p$ & $\hat{\gamma}_{c}$ & $\chi^2$ & $p$\\
\midrule
\endfirsthead
\caption*{\normalfont{Table \ref{tab:H1-agree-work-contrasts} continued}}\\
\toprule
& \multicolumn{3}{c}{Parent controls} & \multicolumn{3}{c}{Nonparent controls} \\
\cmidrule(r){2-4} \cmidrule(r){5-7}
Linear Contrast & $\hat{\gamma}_{c}$ & $\chi^2$ & $p$ & $\hat{\gamma}_{c}$ & $\chi^2$ & $p$\\
\midrule
\endhead
Shift of not-working controls vs. 0 ($\hat{\gamma}_{40}$ + 
$\hat{\gamma}_{60}$) & -0.03 & 5.08 & .024 & 0.04 & 7.79 & .005\\
Shift of working controls vs. 0 ($\hat{\gamma}_{40}$ + 
$\hat{\gamma}_{60}$ + $\hat{\gamma}_{50}$ + 
$\hat{\gamma}_{70}$) & 0.01 & 0.52 & .472 & 0.02 & 3.86 & .049\\
Shift of not-working grandparents vs. 0 ($\hat{\gamma}_{40}$ + 
$\hat{\gamma}_{60}$ + $\hat{\gamma}_{41}$ + 
$\hat{\gamma}_{61}$) & -0.01 & 0.14 & .713 & -0.01 & 0.15 & .699\\
Shift of working grandparents vs. 0 ($\hat{\gamma}_{40}$ + 
$\hat{\gamma}_{60}$ + $\hat{\gamma}_{41}$ + 
$\hat{\gamma}_{61}$ + $\hat{\gamma}_{50}$ + 
$\hat{\gamma}_{70}$ + $\hat{\gamma}_{51}$ +
$\hat{\gamma}_{71}$) & 0.01 & 0.10 & .755 & 0.01 & 0.09 & .768\\
Shift of not-working controls vs. not-working grandparents 
($\hat{\gamma}_{41}$ + $\hat{\gamma}_{61}$) & 0.02 & 0.44 & .505 & -0.05 & 2.76 & .097\\
Before-slope of working controls vs. working grandparents 
($\hat{\gamma}_{21}$ + $\hat{\gamma}_{31}$) & -0.02 & 2.73 & .099 & -0.01 & 0.76 & .383\\
After-slope of working controls vs. working grandparents 
($\hat{\gamma}_{41}$ + $\hat{\gamma}_{51}$) & 0.01 & 0.36 & .548 & 0.02 & 2.00 & .157\\
Shift of working controls vs. working grandparents 
($\hat{\gamma}_{41}$ + $\hat{\gamma}_{61}$ + 
$\hat{\gamma}_{51}$ + $\hat{\gamma}_{71}$) & 0.00 & 0.00 & .966 & -0.01 & 0.35 & .553\\
Shift of not-working controls vs. working controls 
($\hat{\gamma}_{50}$ + $\hat{\gamma}_{70}$) & 0.04 & 4.89 & .027 & -0.02 & 1.43 & .232\\
Before-slope of not-working grandparents vs. working grandparents 
($\hat{\gamma}_{30}$ + $\hat{\gamma}_{31}$) & -0.07 & 6.12 & .013 & -0.07 & 6.87 & .009\\
After-slope of not-working grandparents vs. working grandparents 
($\hat{\gamma}_{50}$ + $\hat{\gamma}_{51}$) & 0.01 & 0.12 & .734 & 0.01 & 0.13 & .714\\
Shift of not-working grandparents vs. working grandparents 
($\hat{\gamma}_{50}$ + $\hat{\gamma}_{70}$ + 
$\hat{\gamma}_{51}$ + $\hat{\gamma}_{71}$) & 0.02 & 0.22 & .637 & 0.02 & 0.23 & .633\\
\bottomrule
\addlinespace
\insertTableNotes
\end{longtable}

}

\end{lltable}








\begin{lltable}

\begin{TableNotes}[para]
\normalsize{\textit{Note.} Two models were computed (only HRS):
grandparents matched with parent controls and with nonparent controls.
CI = confidence interval. \(caring=1\) indicates more than 100 hours of
grandchild care since the last assessment.}
\end{TableNotes}

\footnotesize{

\begin{longtable}{lrcrrrcrr}\noalign{\getlongtablewidth\global\LTcapwidth=\longtablewidth}
\caption{\label{tab:H1-agree-care-tab}Fixed Effects of Agreeableness Over the
Transition to Grandparenthood Moderated by Grandchild Care.}\\
\toprule
& \multicolumn{4}{c}{Parent controls} & \multicolumn{4}{c}{Nonparent controls} \\
\cmidrule(r){2-5} \cmidrule(r){6-9}
Parameter & $\hat{\gamma}$ & 95\% CI & $t$ & $p$ & $\hat{\gamma}$ & 95\% CI & $t$ & $p$\\
\midrule
\endfirsthead
\caption*{\normalfont{Table \ref{tab:H1-agree-care-tab} continued}}\\
\toprule
& \multicolumn{4}{c}{Parent controls} & \multicolumn{4}{c}{Nonparent controls} \\
\cmidrule(r){2-5} \cmidrule(r){6-9}
Parameter & $\hat{\gamma}$ & 95\% CI & $t$ & $p$ & $\hat{\gamma}$ & 95\% CI & $t$ & $p$\\
\midrule
\endhead
Intercept, $\hat{\gamma}_{00}$ & 3.47 & [3.43, 3.52] & 155.84 & < .001 & 3.47 & [3.42, 3.53] & 130.92 & < .001\\
Propensity score, $\hat{\gamma}_{02}$ & 0.16 & [0.08, 0.24] & 3.91 & < .001 & 0.15 & [0.07, 0.23] & 3.67 & < .001\\
After-slope, $\hat{\gamma}_{20}$ & -0.02 & [-0.03, -0.01] & -4.36 & < .001 & -0.02 & [-0.03, -0.01] & -3.63 & < .001\\
Grandparent, $\hat{\gamma}_{01}$ & -0.04 & [-0.11, 0.03] & -1.16 & .246 & -0.05 & [-0.12, 0.02] & -1.49 & .137\\
Caring, $\hat{\gamma}_{10}$ & 0.00 & [-0.04, 0.03] & -0.27 & .784 & 0.02 & [-0.01, 0.05] & 1.09 & .276\\
After-slope * Grandparent, $\hat{\gamma}_{21}$ & 0.03 & [0.00, 0.05] & 2.36 & .018 & 0.02 & [0.00, 0.04] & 2.02 & .044\\
After-slope * Caring, $\hat{\gamma}_{30}$ & 0.00 & [-0.01, 0.02] & 0.29 & .773 & 0.00 & [-0.02, 0.01] & -0.60 & .550\\
Grandparent * Caring, $\hat{\gamma}_{11}$ & 0.02 & [-0.07, 0.11] & 0.46 & .645 & 0.00 & [-0.09, 0.08] & -0.09 & .925\\
After-slope * Grandparent * Caring, $\hat{\gamma}_{31}$ & 0.01 & [-0.02, 0.04] & 0.57 & .572 & 0.02 & [-0.02, 0.05] & 1.00 & .319\\
\bottomrule
\addlinespace
\insertTableNotes
\end{longtable}

}

\end{lltable}







\begin{lltable}

\begin{TableNotes}[para]
\normalsize{\textit{Note.} The linear contrasts are based on the
models from Table \ref{tab:H1-agree-care-tab}. \(\hat{\gamma}_{c}\) =
combined fixed-effects estimate.}
\end{TableNotes}

\footnotesize{

\begin{longtable}{lrrrrrr}\noalign{\getlongtablewidth\global\LTcapwidth=\longtablewidth}
\caption{\label{tab:H1-agree-care-contrasts}Linear Contrasts for Agreeableness
(Moderated by Grandchild Care; only HRS).}\\
\toprule
& \multicolumn{3}{c}{Parent controls} & \multicolumn{3}{c}{Nonparent controls} \\
\cmidrule(r){2-4} \cmidrule(r){5-7}
Linear Contrast & $\hat{\gamma}_{c}$ & $\chi^2$ & $p$ & $\hat{\gamma}_{c}$ & $\chi^2$ & $p$\\
\midrule
\endfirsthead
\caption*{\normalfont{Table \ref{tab:H1-agree-care-contrasts} continued}}\\
\toprule
& \multicolumn{3}{c}{Parent controls} & \multicolumn{3}{c}{Nonparent controls} \\
\cmidrule(r){2-4} \cmidrule(r){5-7}
Linear Contrast & $\hat{\gamma}_{c}$ & $\chi^2$ & $p$ & $\hat{\gamma}_{c}$ & $\chi^2$ & $p$\\
\midrule
\endhead
After-slope of caring controls vs. caring grandparents 
($\hat{\gamma}_{21}$ + $\hat{\gamma}_{31}$) & 0.04 & 7.62 & .006 & 0.04 & 9.15 & .002\\
After-slope of not-caring grandparents vs. caring grandparents 
($\hat{\gamma}_{30}$ + $\hat{\gamma}_{31}$) & 0.01 & 0.61 & .434 & 0.01 & 0.66 & .415\\
\bottomrule
\addlinespace
\insertTableNotes
\end{longtable}

}

\end{lltable}








\begin{lltable}

\begin{TableNotes}[para]
\normalsize{\textit{Note.} Two models were computed for each of the two
samples (LISS, HRS): grandparents matched with parent controls and with
nonparent controls. CI = confidence interval.}
\end{TableNotes}

\footnotesize{

\begin{longtable}{lrcrrrcrr}\noalign{\getlongtablewidth\global\LTcapwidth=\longtablewidth}
\caption{\label{tab:H1-con-tab}Fixed Effects of Conscientiousness Over the
Transition to Grandparenthood.}\\
\toprule
& \multicolumn{4}{c}{Parent controls} & \multicolumn{4}{c}{Nonparent controls} \\
\cmidrule(r){2-5} \cmidrule(r){6-9}
Parameter & $\hat{\gamma}$ & 95\% CI & $t$ & $p$ & $\hat{\gamma}$ & 95\% CI & $t$ & $p$\\
\midrule
\endfirsthead
\caption*{\normalfont{Table \ref{tab:H1-con-tab} continued}}\\
\toprule
& \multicolumn{4}{c}{Parent controls} & \multicolumn{4}{c}{Nonparent controls} \\
\cmidrule(r){2-5} \cmidrule(r){6-9}
Parameter & $\hat{\gamma}$ & 95\% CI & $t$ & $p$ & $\hat{\gamma}$ & 95\% CI & $t$ & $p$\\
\midrule
\endhead
LISS &  &  &  &  &  &  &  & \\
\ \ \ Intercept, $\hat{\gamma}_{00}$ \textcolor{white}{L} & 3.77 & [3.72, 3.83] & 130.27 & < .001 & 3.82 & [3.75, 3.88] & 112.10 & < .001\\
\ \ \ Propensity score, $\hat{\gamma}_{02}$ \textcolor{white}{L} & 0.00 & [-0.08, 0.08] & -0.02 & .987 & 0.01 & [-0.06, 0.08] & 0.24 & .813\\
\ \ \ Before-slope, $\hat{\gamma}_{10}$ \textcolor{white}{L} & 0.00 & [-0.01, 0.00] & -0.84 & .402 & 0.00 & [-0.01, 0.01] & -0.26 & .796\\
\ \ \ After-slope, $\hat{\gamma}_{20}$ \textcolor{white}{L} & -0.02 & [-0.02, -0.01] & -6.17 & < .001 & 0.01 & [0.00, 0.01] & 3.45 & .001\\
\ \ \ Shift, $\hat{\gamma}_{30}$ \textcolor{white}{L} & 0.04 & [0.02, 0.07] & 3.14 & .002 & 0.00 & [-0.03, 0.02] & -0.15 & .881\\
\ \ \ Grandparent, $\hat{\gamma}_{01}$ \textcolor{white}{L} & -0.01 & [-0.10, 0.08] & -0.24 & .813 & -0.06 & [-0.15, 0.04] & -1.22 & .225\\
\ \ \ Before-slope * Grandparent, $\hat{\gamma}_{11}$ \textcolor{white}{L} & 0.00 & [-0.01, 0.02] & 0.77 & .439 & 0.00 & [-0.01, 0.02] & 0.50 & .617\\
\ \ \ After-slope * Grandparent, $\hat{\gamma}_{21}$ \textcolor{white}{L} & 0.02 & [0.00, 0.03] & 2.73 & .006 & -0.01 & [-0.02, 0.00] & -1.61 & .107\\
\ \ \ Shift * Grandparent, $\hat{\gamma}_{31}$ \textcolor{white}{L} & -0.04 & [-0.10, 0.01] & -1.49 & .137 & 0.00 & [-0.06, 0.06] & 0.01 & .989\\
HRS &  &  &  &  &  &  &  & \\
\ \ \ Intercept, $\hat{\gamma}_{00}$ \textcolor{white}{H} & 3.41 & [3.38, 3.44] & 206.26 & < .001 & 3.35 & [3.31, 3.38] & 172.70 & < .001\\
\ \ \ Propensity score, $\hat{\gamma}_{02}$ \textcolor{white}{H} & 0.08 & [0.03, 0.14] & 2.86 & .004 & 0.17 & [0.11, 0.23] & 5.74 & < .001\\
\ \ \ Before-slope, $\hat{\gamma}_{10}$ \textcolor{white}{H} & 0.00 & [-0.01, 0.01] & 0.31 & .754 & 0.00 & [-0.01, 0.01] & 0.72 & .473\\
\ \ \ After-slope, $\hat{\gamma}_{20}$ \textcolor{white}{H} & -0.01 & [-0.02, -0.01] & -4.11 & < .001 & -0.01 & [-0.02, -0.01] & -3.84 & < .001\\
\ \ \ Shift, $\hat{\gamma}_{30}$ \textcolor{white}{H} & 0.02 & [0.00, 0.04] & 1.93 & .053 & 0.00 & [-0.02, 0.02] & 0.01 & .991\\
\ \ \ Grandparent, $\hat{\gamma}_{01}$ \textcolor{white}{H} & 0.02 & [-0.04, 0.07] & 0.60 & .547 & 0.03 & [-0.02, 0.08] & 1.08 & .280\\
\ \ \ Before-slope * Grandparent, $\hat{\gamma}_{11}$ \textcolor{white}{H} & 0.01 & [-0.02, 0.03] & 0.55 & .580 & 0.00 & [-0.02, 0.03] & 0.43 & .664\\
\ \ \ After-slope * Grandparent, $\hat{\gamma}_{21}$ \textcolor{white}{H} & 0.02 & [0.01, 0.04] & 3.06 & .002 & 0.02 & [0.01, 0.04] & 3.01 & .003\\
\ \ \ Shift * Grandparent, $\hat{\gamma}_{31}$ \textcolor{white}{H} & -0.05 & [-0.09, -0.01] & -2.36 & .018 & -0.03 & [-0.07, 0.01] & -1.59 & .111\\
\bottomrule
\addlinespace
\insertTableNotes
\end{longtable}

}

\end{lltable}









\begin{lltable}

\begin{TableNotes}[para]
\normalsize{\textit{Note.} The linear contrasts are needed in cases
where estimates of interest are represented by multiple fixed-effects
coefficients and are computed using the \emph{linearHypothesis} function
from the \emph{car} R package (Fox \& Weisberg, 2019) based on the
models from Table \ref{tab:H1-con-tab}. \(\hat{\gamma}_{c}\) = combined
fixed-effects estimate.}
\end{TableNotes}

\footnotesize{

\begin{longtable}{lrrrrrr}\noalign{\getlongtablewidth\global\LTcapwidth=\longtablewidth}
\caption{\label{tab:H1-con-contrasts}Linear Contrasts for Conscientiousness.}\\
\toprule
& \multicolumn{3}{c}{Parent controls} & \multicolumn{3}{c}{Nonparent controls} \\
\cmidrule(r){2-4} \cmidrule(r){5-7}
Linear Contrast & $\hat{\gamma}_{c}$ & $\chi^2$ & $p$ & $\hat{\gamma}_{c}$ & $\chi^2$ & $p$\\
\midrule
\endfirsthead
\caption*{\normalfont{Table \ref{tab:H1-con-contrasts} continued}}\\
\toprule
& \multicolumn{3}{c}{Parent controls} & \multicolumn{3}{c}{Nonparent controls} \\
\cmidrule(r){2-4} \cmidrule(r){5-7}
Linear Contrast & $\hat{\gamma}_{c}$ & $\chi^2$ & $p$ & $\hat{\gamma}_{c}$ & $\chi^2$ & $p$\\
\midrule
\endhead
LISS &  &  &  &  &  & \\
\ \ \ Shift of the controls vs. 0 ($\hat{\gamma}_{20}$ + 
$\hat{\gamma}_{30}$) \textcolor{white}{L} & 0.02 & 4.71 & .030 & 0.01 & 0.40 & .525\\
\ \ \ Shift of the grandparents vs. 0 ($\hat{\gamma}_{20}$ + 
$\hat{\gamma}_{30}$ + $\hat{\gamma}_{21}$ + 
$\hat{\gamma}_{31}$) \textcolor{white}{L} & 0.00 & 0.01 & .928 & 0.00 & 0.01 & .932\\
\ \ \ Shift of the controls vs. shift of the grandparents 
($\hat{\gamma}_{21}$ + $\hat{\gamma}_{31}$) \textcolor{white}{L} & -0.03 & 1.14 & .286 & -0.01 & 0.13 & .718\\
\ \ \ Before-slope of the grandparents vs. 0 ($\hat{\gamma}_{10}$ + 
$\hat{\gamma}_{11}$) \textcolor{white}{L} & 0.00 & 0.20 & .655 & 0.00 & 0.18 & .667\\
\ \ \ After-slope of the grandparents vs. 0 ($\hat{\gamma}_{20}$ + 
$\hat{\gamma}_{21}$) \textcolor{white}{L} & 0.00 & 0.01 & .942 & 0.00 & 0.01 & .943\\
HRS &  &  &  &  &  & \\
\ \ \ Shift of the controls vs. 0 ($\hat{\gamma}_{20}$ + 
$\hat{\gamma}_{30}$) \textcolor{white}{H} & 0.01 & 0.47 & .491 & -0.01 & 2.83 & .092\\
\ \ \ Shift of the grandparents vs. 0 ($\hat{\gamma}_{20}$ + 
$\hat{\gamma}_{30}$ + $\hat{\gamma}_{21}$ + 
$\hat{\gamma}_{31}$) \textcolor{white}{H} & -0.02 & 2.49 & .114 & -0.02 & 2.82 & .093\\
\ \ \ Shift of the controls vs. shift of the grandparents 
($\hat{\gamma}_{21}$ + $\hat{\gamma}_{31}$) \textcolor{white}{H} & -0.03 & 2.96 & .085 & -0.01 & 0.54 & .462\\
\ \ \ Before-slope of the grandparents vs. 0 ($\hat{\gamma}_{10}$ + 
$\hat{\gamma}_{11}$) \textcolor{white}{H} & 0.01 & 0.59 & .444 & 0.01 & 0.68 & .409\\
\ \ \ After-slope of the grandparents vs. 0 ($\hat{\gamma}_{20}$ + 
$\hat{\gamma}_{21}$) \textcolor{white}{H} & 0.01 & 1.88 & .170 & 0.01 & 2.13 & .145\\
\bottomrule
\addlinespace
\insertTableNotes
\end{longtable}

}

\end{lltable}







\begin{lltable}

\begin{TableNotes}[para]
\normalsize{\textit{Note.} Two models were computed for each of the
two samples (LISS, HRS): grandparents matched with parent controls and
with nonparent controls. CI = confidence interval.}
\end{TableNotes}

\footnotesize{

\begin{longtable}{lrcrrrcrr}\noalign{\getlongtablewidth\global\LTcapwidth=\longtablewidth}
\caption{\label{tab:H1-con-gender-tab}Fixed Effects of Conscientiousness Over the
Transition to Grandparenthood Moderated by Gender.}\\
\toprule
& \multicolumn{4}{c}{Parent controls} & \multicolumn{4}{c}{Nonparent controls} \\
\cmidrule(r){2-5} \cmidrule(r){6-9}
Parameter & $\hat{\gamma}$ & 95\% CI & $t$ & $p$ & $\hat{\gamma}$ & 95\% CI & $t$ & $p$\\
\midrule
\endfirsthead
\caption*{\normalfont{Table \ref{tab:H1-con-gender-tab} continued}}\\
\toprule
& \multicolumn{4}{c}{Parent controls} & \multicolumn{4}{c}{Nonparent controls} \\
\cmidrule(r){2-5} \cmidrule(r){6-9}
Parameter & $\hat{\gamma}$ & 95\% CI & $t$ & $p$ & $\hat{\gamma}$ & 95\% CI & $t$ & $p$\\
\midrule
\endhead
LISS &  &  &  &  &  &  &  & \\
\ \ \ Intercept, $\hat{\gamma}_{00}$ \textcolor{white}{L} & 3.69 & [3.60, 3.77] & 87.30 & < .001 & 3.70 & [3.61, 3.80] & 75.84 & < .001\\
\ \ \ Propensity score, $\hat{\gamma}_{04}$ \textcolor{white}{L} & 0.00 & [-0.08, 0.07] & -0.03 & .976 & 0.01 & [-0.06, 0.08] & 0.34 & .732\\
\ \ \ Before-slope, $\hat{\gamma}_{10}$ \textcolor{white}{L} & 0.00 & [-0.01, 0.01] & 0.64 & .524 & 0.00 & [-0.01, 0.01] & 0.75 & .455\\
\ \ \ After-slope, $\hat{\gamma}_{20}$ \textcolor{white}{L} & -0.01 & [-0.02, -0.01] & -3.43 & .001 & 0.00 & [0.00, 0.01] & 0.71 & .477\\
\ \ \ Shift, $\hat{\gamma}_{30}$ \textcolor{white}{L} & 0.04 & [0.00, 0.08] & 2.16 & .031 & 0.00 & [-0.03, 0.04] & 0.14 & .892\\
\ \ \ Grandparent, $\hat{\gamma}_{01}$ \textcolor{white}{L} & 0.03 & [-0.09, 0.16] & 0.48 & .634 & 0.01 & [-0.13, 0.14] & 0.12 & .907\\
\ \ \ Female, $\hat{\gamma}_{02}$ \textcolor{white}{L} & 0.16 & [0.05, 0.27] & 2.88 & .004 & 0.22 & [0.09, 0.34] & 3.26 & .001\\
\ \ \ Before-slope * Grandparent, $\hat{\gamma}_{11}$ \textcolor{white}{L} & 0.00 & [-0.02, 0.02] & -0.01 & .994 & 0.00 & [-0.02, 0.02] & -0.06 & .953\\
\ \ \ After-slope * Grandparent, $\hat{\gamma}_{21}$ \textcolor{white}{L} & 0.02 & [0.00, 0.04] & 2.53 & .011 & 0.01 & [-0.01, 0.02] & 0.65 & .513\\
\ \ \ Shift * Grandparent, $\hat{\gamma}_{31}$ \textcolor{white}{L} & -0.04 & [-0.13, 0.04] & -1.07 & .286 & -0.01 & [-0.09, 0.08] & -0.14 & .886\\
\ \ \ Before-slope * Female, $\hat{\gamma}_{12}$ \textcolor{white}{L} & -0.01 & [-0.02, 0.00] & -1.61 & .108 & -0.01 & [-0.02, 0.00] & -1.23 & .218\\
\ \ \ After-slope * Female, $\hat{\gamma}_{22}$ \textcolor{white}{L} & -0.01 & [-0.02, 0.00] & -1.11 & .268 & 0.01 & [0.00, 0.02] & 2.38 & .017\\
\ \ \ Shift * Female, $\hat{\gamma}_{32}$ \textcolor{white}{L} & 0.00 & [-0.05, 0.05] & -0.04 & .970 & -0.01 & [-0.06, 0.04] & -0.41 & .683\\
\ \ \ Grandparent * Female, $\hat{\gamma}_{03}$ \textcolor{white}{L} & -0.07 & [-0.24, 0.10] & -0.81 & .418 & -0.12 & [-0.30, 0.06] & -1.30 & .193\\
\ \ \ Before-slope * Grandparent * Female, $\hat{\gamma}_{13}$ \textcolor{white}{L} & 0.01 & [-0.02, 0.03] & 0.61 & .542 & 0.01 & [-0.02, 0.03] & 0.44 & .663\\
\ \ \ After-slope * Grandparent * Female, $\hat{\gamma}_{23}$ \textcolor{white}{L} & -0.01 & [-0.03, 0.01] & -0.84 & .403 & -0.03 & [-0.05, 0.00] & -2.37 & .018\\
\ \ \ Shift * Grandparent * Female, $\hat{\gamma}_{33}$ \textcolor{white}{L} & 0.01 & [-0.11, 0.12] & 0.11 & .916 & 0.02 & [-0.10, 0.13] & 0.27 & .787\\
HRS &  &  &  &  &  &  &  & \\
\ \ \ Intercept, $\hat{\gamma}_{00}$ \textcolor{white}{H} & 3.35 & [3.30, 3.39] & 143.72 & < .001 & 3.26 & [3.21, 3.31] & 124.79 & < .001\\
\ \ \ Propensity score, $\hat{\gamma}_{04}$ \textcolor{white}{H} & 0.09 & [0.03, 0.14] & 3.00 & .003 & 0.17 & [0.11, 0.23] & 5.65 & < .001\\
\ \ \ Before-slope, $\hat{\gamma}_{10}$ \textcolor{white}{H} & 0.01 & [-0.01, 0.02] & 1.19 & .234 & 0.01 & [0.00, 0.03] & 2.08 & .037\\
\ \ \ After-slope, $\hat{\gamma}_{20}$ \textcolor{white}{H} & -0.01 & [-0.02, 0.00] & -2.42 & .016 & 0.00 & [-0.01, 0.01] & -0.10 & .920\\
\ \ \ Shift, $\hat{\gamma}_{30}$ \textcolor{white}{H} & 0.02 & [-0.01, 0.05] & 1.18 & .237 & -0.01 & [-0.04, 0.02] & -0.74 & .462\\
\ \ \ Grandparent, $\hat{\gamma}_{01}$ \textcolor{white}{H} & -0.03 & [-0.10, 0.05] & -0.74 & .461 & 0.01 & [-0.07, 0.09] & 0.28 & .780\\
\ \ \ Female, $\hat{\gamma}_{02}$ \textcolor{white}{H} & 0.11 & [0.05, 0.17] & 3.81 & < .001 & 0.15 & [0.09, 0.22] & 4.67 & < .001\\
\ \ \ Before-slope * Grandparent, $\hat{\gamma}_{11}$ \textcolor{white}{H} & 0.01 & [-0.02, 0.05] & 0.74 & .460 & 0.01 & [-0.03, 0.04] & 0.45 & .651\\
\ \ \ After-slope * Grandparent, $\hat{\gamma}_{21}$ \textcolor{white}{H} & 0.03 & [0.01, 0.05] & 2.64 & .008 & 0.02 & [0.00, 0.04] & 1.71 & .088\\
\ \ \ Shift * Grandparent, $\hat{\gamma}_{31}$ \textcolor{white}{H} & -0.08 & [-0.15, -0.02] & -2.57 & .010 & -0.06 & [-0.12, 0.00] & -1.85 & .064\\
\ \ \ Before-slope * Female, $\hat{\gamma}_{12}$ \textcolor{white}{H} & -0.01 & [-0.03, 0.01] & -1.34 & .180 & -0.02 & [-0.04, 0.00] & -2.16 & .031\\
\ \ \ After-slope * Female, $\hat{\gamma}_{22}$ \textcolor{white}{H} & 0.00 & [-0.02, 0.01] & -0.39 & .695 & -0.02 & [-0.03, -0.01] & -3.05 & .002\\
\ \ \ Shift * Female, $\hat{\gamma}_{32}$ \textcolor{white}{H} & 0.00 & [-0.04, 0.04] & 0.13 & .895 & 0.02 & [-0.02, 0.05] & 0.92 & .356\\
\ \ \ Grandparent * Female, $\hat{\gamma}_{03}$ \textcolor{white}{H} & 0.08 & [-0.02, 0.18] & 1.64 & .101 & 0.03 & [-0.07, 0.13] & 0.62 & .538\\
\ \ \ Before-slope * Grandparent * Female, $\hat{\gamma}_{13}$ \textcolor{white}{H} & -0.01 & [-0.06, 0.03] & -0.47 & .637 & 0.00 & [-0.05, 0.04] & -0.21 & .836\\
\ \ \ After-slope * Grandparent * Female, $\hat{\gamma}_{23}$ \textcolor{white}{H} & -0.01 & [-0.04, 0.02] & -0.79 & .428 & 0.00 & [-0.02, 0.03] & 0.29 & .770\\
\ \ \ Shift * Grandparent * Female, $\hat{\gamma}_{33}$ \textcolor{white}{H} & 0.06 & [-0.03, 0.14] & 1.34 & .181 & 0.05 & [-0.04, 0.13] & 1.11 & .269\\
\bottomrule
\addlinespace
\insertTableNotes
\end{longtable}

}

\end{lltable}








\begin{lltable}

\begin{TableNotes}[para]
\normalsize{\textit{Note.} The linear contrasts are based on the
models from Table \ref{tab:H1-con-gender-tab}. \(\hat{\gamma}_{c}\) =
combined fixed-effects estimate.}
\end{TableNotes}

\footnotesize{

\begin{longtable}{lrrrrrr}\noalign{\getlongtablewidth\global\LTcapwidth=\longtablewidth}
\caption{\label{tab:H1-con-gender-contrasts}Linear Contrasts for Conscientiousness
(Moderated by Gender).}\\
\toprule
& \multicolumn{3}{c}{Parent controls} & \multicolumn{3}{c}{Nonparent controls} \\
\cmidrule(r){2-4} \cmidrule(r){5-7}
Linear Contrast & $\hat{\gamma}_{c}$ & $\chi^2$ & $p$ & $\hat{\gamma}_{c}$ & $\chi^2$ & $p$\\
\midrule
\endfirsthead
\caption*{\normalfont{Table \ref{tab:H1-con-gender-contrasts} continued}}\\
\toprule
& \multicolumn{3}{c}{Parent controls} & \multicolumn{3}{c}{Nonparent controls} \\
\cmidrule(r){2-4} \cmidrule(r){5-7}
Linear Contrast & $\hat{\gamma}_{c}$ & $\chi^2$ & $p$ & $\hat{\gamma}_{c}$ & $\chi^2$ & $p$\\
\midrule
\endhead
LISS &  &  &  &  &  & \\
\ \ \ Shift of male controls vs. 0 ($\hat{\gamma}_{20}$ + 
$\hat{\gamma}_{30}$) \textcolor{white}{L} & 0.03 & 2.83 & .092 & 0.01 & 0.10 & .750\\
\ \ \ Shift of female controls vs. 0 ($\hat{\gamma}_{20}$ + 
$\hat{\gamma}_{30}$ + $\hat{\gamma}_{22}$ + 
$\hat{\gamma}_{32}$) \textcolor{white}{L} & 0.02 & 1.93 & .165 & 0.01 & 0.22 & .640\\
\ \ \ Shift of grandfathers vs. 0 ($\hat{\gamma}_{20}$ + 
$\hat{\gamma}_{30}$ + $\hat{\gamma}_{21}$ + 
$\hat{\gamma}_{31}$) \textcolor{white}{L} & 0.00 & 0.02 & .883 & 0.00 & 0.02 & .886\\
\ \ \ Shift of grandmothers vs. 0 ($\hat{\gamma}_{20}$ + 
$\hat{\gamma}_{30}$ + $\hat{\gamma}_{21}$ + 
$\hat{\gamma}_{31}$ + $\hat{\gamma}_{22}$ + 
$\hat{\gamma}_{32}$ + $\hat{\gamma}_{23}$ +
$\hat{\gamma}_{33}$) \textcolor{white}{L} & -0.01 & 0.04 & .849 & -0.01 & 0.03 & .857\\
\ \ \ Shift of male controls vs. grandfathers 
($\hat{\gamma}_{21}$ + $\hat{\gamma}_{31}$) \textcolor{white}{L} & -0.02 & 0.40 & .528 & 0.00 & 0.00 & .991\\
\ \ \ Before-slope of female controls vs. grandmothers 
($\hat{\gamma}_{11}$ + $\hat{\gamma}_{13}$) \textcolor{white}{L} & 0.01 & 0.81 & .368 & 0.01 & 0.34 & .560\\
\ \ \ After-slope of female controls vs. grandmothers 
($\hat{\gamma}_{21}$ + $\hat{\gamma}_{23}$) \textcolor{white}{L} & 0.01 & 2.25 & .133 & -0.02 & 7.67 & .006\\
\ \ \ Shift of female controls vs. grandmothers 
($\hat{\gamma}_{21}$ + $\hat{\gamma}_{31}$ + 
$\hat{\gamma}_{23}$ + $\hat{\gamma}_{33}$) \textcolor{white}{L} & -0.03 & 0.64 & .422 & -0.01 & 0.14 & .709\\
\ \ \ Shift of male vs. female controls 
($\hat{\gamma}_{22}$ + $\hat{\gamma}_{32}$) \textcolor{white}{L} & -0.01 & 0.09 & .763 & 0.00 & 0.01 & .930\\
\ \ \ Before-slope of grandfathers vs. grandmothers 
($\hat{\gamma}_{12}$ + $\hat{\gamma}_{13}$) \textcolor{white}{L} & 0.00 & 0.02 & .901 & 0.00 & 0.02 & .899\\
\ \ \ After-slope of grandfathers vs. grandmothers 
($\hat{\gamma}_{22}$ + $\hat{\gamma}_{23}$) \textcolor{white}{L} & -0.02 & 2.25 & .134 & -0.02 & 2.12 & .146\\
\ \ \ Shift of grandfathers vs. grandmothers 
($\hat{\gamma}_{22}$ + $\hat{\gamma}_{32}$ + 
$\hat{\gamma}_{23}$ + $\hat{\gamma}_{33}$) \textcolor{white}{L} & -0.01 & 0.06 & .812 & -0.01 & 0.05 & .820\\
HRS &  &  &  &  &  & \\
\ \ \ Shift of male controls vs. 0 ($\hat{\gamma}_{20}$ + 
$\hat{\gamma}_{30}$) \textcolor{white}{H} & 0.01 & 0.21 & .648 & -0.01 & 1.00 & .317\\
\ \ \ Shift of female controls vs. 0 ($\hat{\gamma}_{20}$ + 
$\hat{\gamma}_{30}$ + $\hat{\gamma}_{22}$ + 
$\hat{\gamma}_{32}$) \textcolor{white}{H} & 0.01 & 0.26 & .609 & -0.01 & 1.95 & .163\\
\ \ \ Shift of grandfathers vs. 0 ($\hat{\gamma}_{20}$ + 
$\hat{\gamma}_{30}$ + $\hat{\gamma}_{21}$ + 
$\hat{\gamma}_{31}$) \textcolor{white}{H} & -0.05 & 4.94 & .026 & -0.05 & 5.72 & .017\\
\ \ \ Shift of grandmothers vs. 0 ($\hat{\gamma}_{20}$ + 
$\hat{\gamma}_{30}$ + $\hat{\gamma}_{21}$ + 
$\hat{\gamma}_{31}$ + $\hat{\gamma}_{22}$ + 
$\hat{\gamma}_{32}$ + $\hat{\gamma}_{23}$ +
$\hat{\gamma}_{33}$) \textcolor{white}{H} & 0.00 & 0.01 & .906 & 0.00 & 0.01 & .912\\
\ \ \ Shift of male controls vs. grandfathers 
($\hat{\gamma}_{21}$ + $\hat{\gamma}_{31}$) \textcolor{white}{H} & -0.05 & 4.78 & .029 & -0.04 & 2.75 & .097\\
\ \ \ Before-slope of female controls vs. grandmothers 
($\hat{\gamma}_{11}$ + $\hat{\gamma}_{13}$) \textcolor{white}{H} & 0.00 & 0.02 & .900 & 0.00 & 0.04 & .839\\
\ \ \ After-slope of female controls vs. grandmothers 
($\hat{\gamma}_{21}$ + $\hat{\gamma}_{23}$) \textcolor{white}{H} & 0.02 & 2.96 & .085 & 0.02 & 5.42 & .020\\
\ \ \ Shift of female controls vs. grandmothers 
($\hat{\gamma}_{21}$ + $\hat{\gamma}_{31}$ + 
$\hat{\gamma}_{23}$ + $\hat{\gamma}_{33}$) \textcolor{white}{H} & -0.01 & 0.11 & .737 & 0.01 & 0.27 & .600\\
\ \ \ Shift of male vs. female controls 
($\hat{\gamma}_{22}$ + $\hat{\gamma}_{32}$) \textcolor{white}{H} & 0.00 & 0.00 & .998 & 0.00 & 0.02 & .877\\
\ \ \ Before-slope of grandfathers vs. grandmothers 
($\hat{\gamma}_{12}$ + $\hat{\gamma}_{13}$) \textcolor{white}{H} & -0.02 & 1.36 & .244 & -0.03 & 1.58 & .208\\
\ \ \ After-slope of grandfathers vs. grandmothers 
($\hat{\gamma}_{22}$ + $\hat{\gamma}_{23}$) \textcolor{white}{H} & -0.01 & 1.17 & .279 & -0.02 & 1.43 & .232\\
\ \ \ Shift of grandfathers vs. grandmothers 
($\hat{\gamma}_{22}$ + $\hat{\gamma}_{32}$ + 
$\hat{\gamma}_{23}$ + $\hat{\gamma}_{33}$) \textcolor{white}{H} & 0.05 & 2.47 & .116 & 0.05 & 2.90 & .089\\
\bottomrule
\addlinespace
\insertTableNotes
\end{longtable}

}

\end{lltable}








\begin{lltable}

\begin{TableNotes}[para]
\normalsize{\textit{Note.} Two models were computed (only HRS):
grandparents matched with parent controls and with nonparent controls.
CI = confidence interval. \(working=1\) indicates being employed in paid
work.}
\end{TableNotes}

\footnotesize{

\begin{longtable}{lrcrrrcrr}\noalign{\getlongtablewidth\global\LTcapwidth=\longtablewidth}
\caption{\label{tab:H1-con-work-tab}Fixed Effects of Conscientiousness Over the
Transition to Grandparenthood Moderated by Performing Paid Work.}\\
\toprule
& \multicolumn{4}{c}{Parent controls} & \multicolumn{4}{c}{Nonparent controls} \\
\cmidrule(r){2-5} \cmidrule(r){6-9}
Parameter & $\hat{\gamma}$ & 95\% CI & $t$ & $p$ & $\hat{\gamma}$ & 95\% CI & $t$ & $p$\\
\midrule
\endfirsthead
\caption*{\normalfont{Table \ref{tab:H1-con-work-tab} continued}}\\
\toprule
& \multicolumn{4}{c}{Parent controls} & \multicolumn{4}{c}{Nonparent controls} \\
\cmidrule(r){2-5} \cmidrule(r){6-9}
Parameter & $\hat{\gamma}$ & 95\% CI & $t$ & $p$ & $\hat{\gamma}$ & 95\% CI & $t$ & $p$\\
\midrule
\endhead
Intercept, $\hat{\gamma}_{00}$ & 3.41 & [3.37, 3.45] & 165.13 & < .001 & 3.37 & [3.33, 3.42] & 146.02 & < .001\\
Propensity score, $\hat{\gamma}_{02}$ & 0.06 & [0.00, 0.12] & 2.13 & .033 & 0.14 & [0.09, 0.20] & 4.83 & < .001\\
Before-slope, $\hat{\gamma}_{20}$ & -0.01 & [-0.03, 0.00] & -1.55 & .121 & 0.00 & [-0.02, 0.02] & -0.28 & .779\\
After-slope, $\hat{\gamma}_{40}$ & -0.02 & [-0.03, -0.01] & -3.55 & < .001 & -0.02 & [-0.03, -0.01] & -4.10 & < .001\\
Shift, $\hat{\gamma}_{60}$ & 0.02 & [-0.01, 0.05] & 1.49 & .137 & -0.02 & [-0.05, 0.01] & -1.30 & .193\\
Grandparent, $\hat{\gamma}_{01}$ & -0.09 & [-0.18, -0.01] & -2.19 & .029 & -0.10 & [-0.18, -0.01] & -2.30 & .022\\
Working, $\hat{\gamma}_{10}$ & 0.01 & [-0.03, 0.05] & 0.45 & .650 & -0.03 & [-0.06, 0.01] & -1.60 & .109\\
Before-slope * Grandparent, $\hat{\gamma}_{21}$ & 0.08 & [0.04, 0.13] & 3.54 & < .001 & 0.07 & [0.03, 0.12] & 3.16 & .002\\
After-slope * Grandparent, $\hat{\gamma}_{41}$ & 0.03 & [0.01, 0.05] & 2.66 & .008 & 0.03 & [0.01, 0.05] & 2.96 & .003\\
Shift * Grandparent, $\hat{\gamma}_{61}$ & -0.09 & [-0.16, -0.02] & -2.64 & .008 & -0.05 & [-0.11, 0.02] & -1.46 & .145\\
Before-slope * Working, $\hat{\gamma}_{30}$ & 0.02 & [0.00, 0.05] & 2.21 & .027 & 0.01 & [-0.01, 0.03] & 0.91 & .362\\
After-slope * Working, $\hat{\gamma}_{50}$ & 0.01 & [0.00, 0.03] & 1.92 & .055 & 0.02 & [0.01, 0.03] & 2.96 & .003\\
Shift * Working, $\hat{\gamma}_{70}$ & -0.01 & [-0.05, 0.03] & -0.45 & .653 & 0.03 & [-0.01, 0.06] & 1.30 & .194\\
Grandparent * Working, $\hat{\gamma}_{11}$ & 0.14 & [0.05, 0.23] & 3.16 & .002 & 0.17 & [0.09, 0.26] & 4.05 & < .001\\
Before-slope * Grandparent * Working, $\hat{\gamma}_{31}$ & -0.10 & [-0.16, -0.05] & -3.69 & < .001 & -0.09 & [-0.14, -0.04] & -3.31 & .001\\
After-slope * Grandparent * Working, $\hat{\gamma}_{51}$ & -0.01 & [-0.04, 0.02] & -0.76 & .449 & -0.02 & [-0.05, 0.01] & -1.17 & .240\\
Shift * Grandparent * Working, $\hat{\gamma}_{71}$ & 0.06 & [-0.03, 0.15] & 1.31 & .191 & 0.02 & [-0.06, 0.11] & 0.56 & .578\\
\bottomrule
\addlinespace
\insertTableNotes
\end{longtable}

}

\end{lltable}








\begin{lltable}

\begin{TableNotes}[para]
\normalsize{\textit{Note.} The linear contrasts are based on the
models from Table \ref{tab:H1-con-work-tab}. \(\hat{\gamma}_{c}\) =
combined fixed-effects estimate.}
\end{TableNotes}

\footnotesize{

\begin{longtable}{lrrrrrr}\noalign{\getlongtablewidth\global\LTcapwidth=\longtablewidth}
\caption{\label{tab:H1-con-work-contrasts}Linear Contrasts for Conscientiousness
(Moderated by Paid Work; only HRS).}\\
\toprule
& \multicolumn{3}{c}{Parent controls} & \multicolumn{3}{c}{Nonparent controls} \\
\cmidrule(r){2-4} \cmidrule(r){5-7}
Linear Contrast & $\hat{\gamma}_{c}$ & $\chi^2$ & $p$ & $\hat{\gamma}_{c}$ & $\chi^2$ & $p$\\
\midrule
\endfirsthead
\caption*{\normalfont{Table \ref{tab:H1-con-work-contrasts} continued}}\\
\toprule
& \multicolumn{3}{c}{Parent controls} & \multicolumn{3}{c}{Nonparent controls} \\
\cmidrule(r){2-4} \cmidrule(r){5-7}
Linear Contrast & $\hat{\gamma}_{c}$ & $\chi^2$ & $p$ & $\hat{\gamma}_{c}$ & $\chi^2$ & $p$\\
\midrule
\endhead
Shift of not-working controls vs. 0 ($\hat{\gamma}_{40}$ + 
$\hat{\gamma}_{60}$) & 0.01 & 0.23 & .635 & -0.04 & 9.72 & .002\\
Shift of working controls vs. 0 ($\hat{\gamma}_{40}$ + 
$\hat{\gamma}_{60}$ + $\hat{\gamma}_{50}$ + 
$\hat{\gamma}_{70}$) & 0.01 & 1.06 & .304 & 0.00 & 0.28 & .598\\
Shift of not-working grandparents vs. 0 ($\hat{\gamma}_{40}$ + 
$\hat{\gamma}_{60}$ + $\hat{\gamma}_{41}$ + 
$\hat{\gamma}_{61}$) & -0.06 & 5.20 & .023 & -0.06 & 5.93 & .015\\
Shift of working grandparents vs. 0 ($\hat{\gamma}_{40}$ + 
$\hat{\gamma}_{60}$ + $\hat{\gamma}_{41}$ + 
$\hat{\gamma}_{61}$ + $\hat{\gamma}_{50}$ + 
$\hat{\gamma}_{70}$ + $\hat{\gamma}_{51}$ +
$\hat{\gamma}_{71}$) & -0.01 & 0.09 & .768 & -0.01 & 0.13 & .717\\
Shift of not-working controls vs. not-working grandparents 
($\hat{\gamma}_{41}$ + $\hat{\gamma}_{61}$) & -0.06 & 5.09 & .024 & -0.02 & 0.46 & .498\\
Before-slope of working controls vs. working grandparents 
($\hat{\gamma}_{21}$ + $\hat{\gamma}_{31}$) & -0.02 & 1.75 & .185 & -0.02 & 1.50 & .221\\
After-slope of working controls vs. working grandparents 
($\hat{\gamma}_{41}$ + $\hat{\gamma}_{51}$) & 0.02 & 2.59 & .107 & 0.01 & 1.83 & .176\\
Shift of working controls vs. working grandparents 
($\hat{\gamma}_{41}$ + $\hat{\gamma}_{61}$ + 
$\hat{\gamma}_{51}$ + $\hat{\gamma}_{71}$) & -0.02 & 0.52 & .469 & -0.01 & 0.31 & .578\\
Shift of not-working controls vs. working controls 
($\hat{\gamma}_{50}$ + $\hat{\gamma}_{70}$) & 0.00 & 0.06 & .809 & 0.04 & 8.10 & .004\\
Before-slope of not-working grandparents vs. working grandparents 
($\hat{\gamma}_{30}$ + $\hat{\gamma}_{31}$) & -0.08 & 9.38 & .002 & -0.08 & 10.44 & .001\\
After-slope of not-working grandparents vs. working grandparents 
($\hat{\gamma}_{50}$ + $\hat{\gamma}_{51}$) & 0.00 & 0.01 & .920 & 0.00 & 0.02 & .879\\
Shift of not-working grandparents vs. working grandparents 
($\hat{\gamma}_{50}$ + $\hat{\gamma}_{70}$ + 
$\hat{\gamma}_{51}$ + $\hat{\gamma}_{71}$) & 0.05 & 2.62 & .106 & 0.05 & 2.89 & .089\\
\bottomrule
\addlinespace
\insertTableNotes
\end{longtable}

}

\end{lltable}








\begin{lltable}

\begin{TableNotes}[para]
\normalsize{\textit{Note.} Two models were computed (only HRS):
grandparents matched with parent controls and with nonparent controls.
CI = confidence interval. \(caring=1\) indicates more than 100 hours of
grandchild care since the last assessment.}
\end{TableNotes}

\footnotesize{

\begin{longtable}{lrcrrrcrr}\noalign{\getlongtablewidth\global\LTcapwidth=\longtablewidth}
\caption{\label{tab:H1-con-care-tab}Fixed Effects of Conscientiousness Over the
Transition to Grandparenthood Moderated by Grandchild Care.}\\
\toprule
& \multicolumn{4}{c}{Parent controls} & \multicolumn{4}{c}{Nonparent controls} \\
\cmidrule(r){2-5} \cmidrule(r){6-9}
Parameter & $\hat{\gamma}$ & 95\% CI & $t$ & $p$ & $\hat{\gamma}$ & 95\% CI & $t$ & $p$\\
\midrule
\endfirsthead
\caption*{\normalfont{Table \ref{tab:H1-con-care-tab} continued}}\\
\toprule
& \multicolumn{4}{c}{Parent controls} & \multicolumn{4}{c}{Nonparent controls} \\
\cmidrule(r){2-5} \cmidrule(r){6-9}
Parameter & $\hat{\gamma}$ & 95\% CI & $t$ & $p$ & $\hat{\gamma}$ & 95\% CI & $t$ & $p$\\
\midrule
\endhead
Intercept, $\hat{\gamma}_{00}$ & 3.44 & [3.40, 3.48] & 168.69 & < .001 & 3.34 & [3.30, 3.39] & 138.33 & < .001\\
Propensity score, $\hat{\gamma}_{02}$ & 0.08 & [0.00, 0.15] & 2.03 & .042 & 0.29 & [0.22, 0.37] & 7.78 & < .001\\
After-slope, $\hat{\gamma}_{20}$ & -0.02 & [-0.03, -0.01] & -3.80 & < .001 & -0.01 & [-0.02, 0.00] & -2.74 & .006\\
Grandparent, $\hat{\gamma}_{01}$ & -0.02 & [-0.08, 0.05] & -0.51 & .610 & -0.02 & [-0.09, 0.04] & -0.74 & .462\\
Caring, $\hat{\gamma}_{10}$ & 0.00 & [-0.03, 0.03] & 0.03 & .972 & 0.02 & [0.00, 0.05] & 1.64 & .102\\
After-slope * Grandparent, $\hat{\gamma}_{21}$ & 0.01 & [-0.01, 0.03] & 1.37 & .170 & 0.01 & [-0.01, 0.02] & 0.73 & .468\\
After-slope * Caring, $\hat{\gamma}_{30}$ & 0.00 & [-0.01, 0.01] & 0.01 & .993 & -0.01 & [-0.02, 0.00] & -1.72 & .085\\
Grandparent * Caring, $\hat{\gamma}_{11}$ & -0.04 & [-0.12, 0.04] & -0.93 & .355 & -0.07 & [-0.14, 0.01] & -1.74 & .081\\
After-slope * Grandparent * Caring, $\hat{\gamma}_{31}$ & 0.03 & [0.00, 0.06] & 1.88 & .060 & 0.04 & [0.01, 0.07] & 2.82 & .005\\
\bottomrule
\addlinespace
\insertTableNotes
\end{longtable}

}

\end{lltable}







\begin{lltable}

\begin{TableNotes}[para]
\normalsize{\textit{Note.} The linear contrasts are based on the
models from Table \ref{tab:H1-con-care-tab}. \(\hat{\gamma}_{c}\) =
combined fixed-effects estimate.}
\end{TableNotes}

\footnotesize{

\begin{longtable}{lrrrrrr}\noalign{\getlongtablewidth\global\LTcapwidth=\longtablewidth}
\caption{\label{tab:H1-con-care-contrasts}Linear Contrasts for Conscientiousness
(Moderated by Grandchild Care; only HRS).}\\
\toprule
& \multicolumn{3}{c}{Parent controls} & \multicolumn{3}{c}{Nonparent controls} \\
\cmidrule(r){2-4} \cmidrule(r){5-7}
Linear Contrast & $\hat{\gamma}_{c}$ & $\chi^2$ & $p$ & $\hat{\gamma}_{c}$ & $\chi^2$ & $p$\\
\midrule
\endfirsthead
\caption*{\normalfont{Table \ref{tab:H1-con-care-contrasts} continued}}\\
\toprule
& \multicolumn{3}{c}{Parent controls} & \multicolumn{3}{c}{Nonparent controls} \\
\cmidrule(r){2-4} \cmidrule(r){5-7}
Linear Contrast & $\hat{\gamma}_{c}$ & $\chi^2$ & $p$ & $\hat{\gamma}_{c}$ & $\chi^2$ & $p$\\
\midrule
\endhead
After-slope of caring controls vs. caring grandparents 
($\hat{\gamma}_{21}$ + $\hat{\gamma}_{31}$) & 0.04 & 13.75 & < .001 & 0.05 & 19.49 & < .001\\
After-slope of not-caring grandparents vs. caring grandparents 
($\hat{\gamma}_{30}$ + $\hat{\gamma}_{31}$) & 0.03 & 4.48 & .034 & 0.03 & 5.28 & .022\\
\bottomrule
\addlinespace
\insertTableNotes
\end{longtable}

}

\end{lltable}








\begin{lltable}

\begin{TableNotes}[para]
\normalsize{\textit{Note.} Two models were computed for each of the two
samples (LISS, HRS): grandparents matched with parent controls and with
nonparent controls. CI = confidence interval.}
\end{TableNotes}

\footnotesize{

\begin{longtable}{lrcrrrcrr}\noalign{\getlongtablewidth\global\LTcapwidth=\longtablewidth}
\caption{\label{tab:H1-extra-tab}Fixed Effects of Extraversion Over the Transition
to Grandparenthood.}\\
\toprule
& \multicolumn{4}{c}{Parent controls} & \multicolumn{4}{c}{Nonparent controls} \\
\cmidrule(r){2-5} \cmidrule(r){6-9}
Parameter & $\hat{\gamma}$ & 95\% CI & $t$ & $p$ & $\hat{\gamma}$ & 95\% CI & $t$ & $p$\\
\midrule
\endfirsthead
\caption*{\normalfont{Table \ref{tab:H1-extra-tab} continued}}\\
\toprule
& \multicolumn{4}{c}{Parent controls} & \multicolumn{4}{c}{Nonparent controls} \\
\cmidrule(r){2-5} \cmidrule(r){6-9}
Parameter & $\hat{\gamma}$ & 95\% CI & $t$ & $p$ & $\hat{\gamma}$ & 95\% CI & $t$ & $p$\\
\midrule
\endhead
LISS &  &  &  &  &  &  &  & \\
\ \ \ Intercept, $\hat{\gamma}_{00}$ \textcolor{white}{L} & 3.25 & [3.18, 3.33] & 87.65 & < .001 & 3.29 & [3.20, 3.39] & 67.72 & < .001\\
\ \ \ Propensity score, $\hat{\gamma}_{02}$ \textcolor{white}{L} & -0.01 & [-0.10, 0.07] & -0.26 & .793 & 0.01 & [-0.07, 0.08] & 0.18 & .860\\
\ \ \ Before-slope, $\hat{\gamma}_{10}$ \textcolor{white}{L} & -0.01 & [-0.01, 0.00] & -1.77 & .077 & 0.00 & [0.00, 0.01] & 0.65 & .515\\
\ \ \ After-slope, $\hat{\gamma}_{20}$ \textcolor{white}{L} & 0.00 & [-0.01, 0.00] & -1.47 & .141 & -0.01 & [-0.02, 0.00] & -3.62 & < .001\\
\ \ \ Shift, $\hat{\gamma}_{30}$ \textcolor{white}{L} & -0.01 & [-0.04, 0.01] & -0.97 & .332 & -0.01 & [-0.03, 0.02] & -0.41 & .683\\
\ \ \ Grandparent, $\hat{\gamma}_{01}$ \textcolor{white}{L} & 0.06 & [-0.05, 0.17] & 1.03 & .306 & 0.01 & [-0.12, 0.14] & 0.19 & .849\\
\ \ \ Before-slope * Grandparent, $\hat{\gamma}_{11}$ \textcolor{white}{L} & 0.00 & [-0.02, 0.01] & -0.40 & .690 & -0.01 & [-0.02, 0.00] & -1.44 & .150\\
\ \ \ After-slope * Grandparent, $\hat{\gamma}_{21}$ \textcolor{white}{L} & 0.00 & [-0.01, 0.02] & 0.57 & .569 & 0.01 & [0.00, 0.02] & 1.45 & .146\\
\ \ \ Shift * Grandparent, $\hat{\gamma}_{31}$ \textcolor{white}{L} & -0.02 & [-0.08, 0.05] & -0.51 & .607 & -0.02 & [-0.08, 0.04] & -0.73 & .467\\
HRS &  &  &  &  &  &  &  & \\
\ \ \ Intercept, $\hat{\gamma}_{00}$ \textcolor{white}{H} & 3.20 & [3.16, 3.24] & 159.82 & < .001 & 3.11 & [3.07, 3.16] & 133.29 & < .001\\
\ \ \ Propensity score, $\hat{\gamma}_{02}$ \textcolor{white}{H} & 0.02 & [-0.05, 0.08] & 0.56 & .577 & 0.05 & [-0.02, 0.12] & 1.44 & .150\\
\ \ \ Before-slope, $\hat{\gamma}_{10}$ \textcolor{white}{H} & 0.00 & [-0.01, 0.01] & -0.52 & .604 & 0.01 & [-0.01, 0.02] & 0.99 & .321\\
\ \ \ After-slope, $\hat{\gamma}_{20}$ \textcolor{white}{H} & 0.00 & [-0.01, 0.01] & -0.64 & .520 & 0.00 & [-0.01, 0.01] & -0.35 & .729\\
\ \ \ Shift, $\hat{\gamma}_{30}$ \textcolor{white}{H} & 0.02 & [0.00, 0.04] & 1.68 & .093 & 0.01 & [-0.01, 0.03] & 1.07 & .285\\
\ \ \ Grandparent, $\hat{\gamma}_{01}$ \textcolor{white}{H} & 0.00 & [-0.06, 0.06] & 0.05 & .957 & 0.07 & [0.01, 0.14] & 2.20 & .028\\
\ \ \ Before-slope * Grandparent, $\hat{\gamma}_{11}$ \textcolor{white}{H} & 0.00 & [-0.02, 0.03] & 0.31 & .757 & 0.00 & [-0.03, 0.02] & -0.35 & .728\\
\ \ \ After-slope * Grandparent, $\hat{\gamma}_{21}$ \textcolor{white}{H} & 0.01 & [0.00, 0.03] & 1.46 & .143 & 0.01 & [0.00, 0.03] & 1.38 & .169\\
\ \ \ Shift * Grandparent, $\hat{\gamma}_{31}$ \textcolor{white}{H} & -0.04 & [-0.09, 0.01] & -1.55 & .121 & -0.03 & [-0.08, 0.02] & -1.30 & .193\\
\bottomrule
\addlinespace
\insertTableNotes
\end{longtable}

}

\end{lltable}









\begin{lltable}

\begin{TableNotes}[para]
\normalsize{\textit{Note.} The linear contrasts are needed in cases
where estimates of interest are represented by multiple fixed-effects
coefficients and are computed using the \emph{linearHypothesis} function
from the \emph{car} R package (Fox \& Weisberg, 2019) based on the
models from Table \ref{tab:H1-extra-tab}. \(\hat{\gamma}_{c}\) =
combined fixed-effects estimate.}
\end{TableNotes}

\footnotesize{

\begin{longtable}{lrrrrrr}\noalign{\getlongtablewidth\global\LTcapwidth=\longtablewidth}
\caption{\label{tab:H1-extra-contrasts}Linear Contrasts for Extraversion.}\\
\toprule
& \multicolumn{3}{c}{Parent controls} & \multicolumn{3}{c}{Nonparent controls} \\
\cmidrule(r){2-4} \cmidrule(r){5-7}
Linear Contrast & $\hat{\gamma}_{c}$ & $\chi^2$ & $p$ & $\hat{\gamma}_{c}$ & $\chi^2$ & $p$\\
\midrule
\endfirsthead
\caption*{\normalfont{Table \ref{tab:H1-extra-contrasts} continued}}\\
\toprule
& \multicolumn{3}{c}{Parent controls} & \multicolumn{3}{c}{Nonparent controls} \\
\cmidrule(r){2-4} \cmidrule(r){5-7}
Linear Contrast & $\hat{\gamma}_{c}$ & $\chi^2$ & $p$ & $\hat{\gamma}_{c}$ & $\chi^2$ & $p$\\
\midrule
\endhead
LISS &  &  &  &  &  & \\
\ \ \ Shift of the controls vs. 0 ($\hat{\gamma}_{20}$ + 
$\hat{\gamma}_{30}$) \textcolor{white}{L} & -0.02 & 2.12 & .145 & -0.02 & 1.73 & .188\\
\ \ \ Shift of the grandparents vs. 0 ($\hat{\gamma}_{20}$ + 
$\hat{\gamma}_{30}$ + $\hat{\gamma}_{21}$ + 
$\hat{\gamma}_{31}$) \textcolor{white}{L} & -0.03 & 1.58 & .208 & -0.03 & 1.47 & .225\\
\ \ \ Shift of the controls vs. shift of the grandparents 
($\hat{\gamma}_{21}$ + $\hat{\gamma}_{31}$) \textcolor{white}{L} & -0.01 & 0.21 & .647 & -0.01 & 0.25 & .620\\
\ \ \ Before-slope of the grandparents vs. 0 ($\hat{\gamma}_{10}$ + 
$\hat{\gamma}_{11}$) \textcolor{white}{L} & -0.01 & 1.77 & .183 & -0.01 & 1.65 & .200\\
\ \ \ After-slope of the grandparents vs. 0 ($\hat{\gamma}_{20}$ + 
$\hat{\gamma}_{21}$) \textcolor{white}{L} & 0.00 & 0.01 & .912 & 0.00 & 0.03 & .852\\
HRS &  &  &  &  &  & \\
\ \ \ Shift of the controls vs. 0 ($\hat{\gamma}_{20}$ + 
$\hat{\gamma}_{30}$) \textcolor{white}{H} & 0.02 & 3.63 & .057 & 0.01 & 1.51 & .219\\
\ \ \ Shift of the grandparents vs. 0 ($\hat{\gamma}_{20}$ + 
$\hat{\gamma}_{30}$ + $\hat{\gamma}_{21}$ + 
$\hat{\gamma}_{31}$) \textcolor{white}{H} & -0.01 & 0.34 & .561 & -0.01 & 0.36 & .548\\
\ \ \ Shift of the controls vs. shift of the grandparents 
($\hat{\gamma}_{21}$ + $\hat{\gamma}_{31}$) \textcolor{white}{H} & -0.03 & 1.90 & .168 & -0.02 & 1.19 & .275\\
\ \ \ Before-slope of the grandparents vs. 0 ($\hat{\gamma}_{10}$ + 
$\hat{\gamma}_{11}$) \textcolor{white}{H} & 0.00 & 0.01 & .925 & 0.00 & 0.01 & .929\\
\ \ \ After-slope of the grandparents vs. 0 ($\hat{\gamma}_{20}$ + 
$\hat{\gamma}_{21}$) \textcolor{white}{H} & 0.01 & 1.73 & .189 & 0.01 & 1.86 & .173\\
\bottomrule
\addlinespace
\insertTableNotes
\end{longtable}

}

\end{lltable}







\begin{lltable}

\begin{TableNotes}[para]
\normalsize{\textit{Note.} Two models were computed for each of the
two samples (LISS, HRS): grandparents matched with parent controls and
with nonparent controls. CI = confidence interval.}
\end{TableNotes}

\footnotesize{

\begin{longtable}{lrcrrrcrr}\noalign{\getlongtablewidth\global\LTcapwidth=\longtablewidth}
\caption{\label{tab:H1-extra-gender-tab}Fixed Effects of Extraversion Over the
Transition to Grandparenthood Moderated by Gender.}\\
\toprule
& \multicolumn{4}{c}{Parent controls} & \multicolumn{4}{c}{Nonparent controls} \\
\cmidrule(r){2-5} \cmidrule(r){6-9}
Parameter & $\hat{\gamma}$ & 95\% CI & $t$ & $p$ & $\hat{\gamma}$ & 95\% CI & $t$ & $p$\\
\midrule
\endfirsthead
\caption*{\normalfont{Table \ref{tab:H1-extra-gender-tab} continued}}\\
\toprule
& \multicolumn{4}{c}{Parent controls} & \multicolumn{4}{c}{Nonparent controls} \\
\cmidrule(r){2-5} \cmidrule(r){6-9}
Parameter & $\hat{\gamma}$ & 95\% CI & $t$ & $p$ & $\hat{\gamma}$ & 95\% CI & $t$ & $p$\\
\midrule
\endhead
LISS &  &  &  &  &  &  &  & \\
\ \ \ Intercept, $\hat{\gamma}_{00}$ \textcolor{white}{L} & 3.28 & [3.18, 3.39] & 60.26 & < .001 & 3.22 & [3.08, 3.35] & 46.79 & < .001\\
\ \ \ Propensity score, $\hat{\gamma}_{04}$ \textcolor{white}{L} & -0.01 & [-0.09, 0.08] & -0.15 & .881 & 0.01 & [-0.06, 0.09] & 0.30 & .765\\
\ \ \ Before-slope, $\hat{\gamma}_{10}$ \textcolor{white}{L} & -0.01 & [-0.02, 0.00] & -1.82 & .069 & 0.02 & [0.01, 0.03] & 4.00 & < .001\\
\ \ \ After-slope, $\hat{\gamma}_{20}$ \textcolor{white}{L} & -0.01 & [-0.02, 0.00] & -2.56 & .011 & 0.00 & [-0.01, 0.00] & -1.08 & .280\\
\ \ \ Shift, $\hat{\gamma}_{30}$ \textcolor{white}{L} & -0.04 & [-0.08, 0.01] & -1.68 & .094 & -0.05 & [-0.09, -0.01] & -2.43 & .015\\
\ \ \ Grandparent, $\hat{\gamma}_{01}$ \textcolor{white}{L} & 0.01 & [-0.15, 0.17] & 0.09 & .929 & 0.07 & [-0.11, 0.26] & 0.78 & .435\\
\ \ \ Female, $\hat{\gamma}_{02}$ \textcolor{white}{L} & -0.06 & [-0.20, 0.09] & -0.78 & .436 & 0.13 & [-0.05, 0.31] & 1.45 & .148\\
\ \ \ Before-slope * Grandparent, $\hat{\gamma}_{11}$ \textcolor{white}{L} & 0.00 & [-0.02, 0.02] & 0.14 & .893 & -0.03 & [-0.05, -0.01] & -2.49 & .013\\
\ \ \ After-slope * Grandparent, $\hat{\gamma}_{21}$ \textcolor{white}{L} & 0.01 & [-0.01, 0.03] & 1.19 & .236 & 0.00 & [-0.01, 0.02] & 0.48 & .628\\
\ \ \ Shift * Grandparent, $\hat{\gamma}_{31}$ \textcolor{white}{L} & -0.01 & [-0.10, 0.08] & -0.12 & .903 & 0.01 & [-0.08, 0.10] & 0.22 & .825\\
\ \ \ Before-slope * Female, $\hat{\gamma}_{12}$ \textcolor{white}{L} & 0.01 & [-0.01, 0.02] & 0.87 & .386 & -0.03 & [-0.04, -0.02] & -4.83 & < .001\\
\ \ \ After-slope * Female, $\hat{\gamma}_{22}$ \textcolor{white}{L} & 0.01 & [0.00, 0.02] & 2.10 & .035 & -0.01 & [-0.02, 0.00] & -2.03 & .043\\
\ \ \ Shift * Female, $\hat{\gamma}_{32}$ \textcolor{white}{L} & 0.04 & [-0.02, 0.09] & 1.36 & .174 & 0.08 & [0.03, 0.14] & 2.91 & .004\\
\ \ \ Grandparent * Female, $\hat{\gamma}_{03}$ \textcolor{white}{L} & 0.09 & [-0.13, 0.31] & 0.82 & .411 & -0.11 & [-0.36, 0.13] & -0.90 & .369\\
\ \ \ Before-slope * Grandparent * Female, $\hat{\gamma}_{13}$ \textcolor{white}{L} & -0.01 & [-0.04, 0.02] & -0.53 & .593 & 0.03 & [0.00, 0.06] & 2.09 & .037\\
\ \ \ After-slope * Grandparent * Female, $\hat{\gamma}_{23}$ \textcolor{white}{L} & -0.01 & [-0.04, 0.01] & -1.11 & .266 & 0.01 & [-0.02, 0.03] & 0.71 & .475\\
\ \ \ Shift * Grandparent * Female, $\hat{\gamma}_{33}$ \textcolor{white}{L} & -0.02 & [-0.14, 0.10] & -0.29 & .768 & -0.06 & [-0.18, 0.06] & -0.98 & .328\\
HRS &  &  &  &  &  &  &  & \\
\ \ \ Intercept, $\hat{\gamma}_{00}$ \textcolor{white}{H} & 3.15 & [3.09, 3.21] & 108.70 & < .001 & 3.11 & [3.04, 3.17] & 96.32 & < .001\\
\ \ \ Propensity score, $\hat{\gamma}_{04}$ \textcolor{white}{H} & 0.02 & [-0.04, 0.09] & 0.64 & .520 & 0.05 & [-0.02, 0.12] & 1.31 & .191\\
\ \ \ Before-slope, $\hat{\gamma}_{10}$ \textcolor{white}{H} & 0.01 & [-0.01, 0.02] & 0.70 & .482 & 0.00 & [-0.02, 0.01] & -0.37 & .709\\
\ \ \ After-slope, $\hat{\gamma}_{20}$ \textcolor{white}{H} & 0.01 & [0.00, 0.02] & 2.05 & .040 & 0.00 & [-0.01, 0.01] & 0.51 & .609\\
\ \ \ Shift, $\hat{\gamma}_{30}$ \textcolor{white}{H} & -0.01 & [-0.04, 0.02] & -0.52 & .601 & -0.01 & [-0.04, 0.03] & -0.41 & .685\\
\ \ \ Grandparent, $\hat{\gamma}_{01}$ \textcolor{white}{H} & -0.01 & [-0.10, 0.08] & -0.28 & .782 & 0.02 & [-0.08, 0.11] & 0.39 & .697\\
\ \ \ Female, $\hat{\gamma}_{02}$ \textcolor{white}{H} & 0.08 & [0.01, 0.16] & 2.24 & .025 & 0.01 & [-0.07, 0.09] & 0.30 & .767\\
\ \ \ Before-slope * Grandparent, $\hat{\gamma}_{11}$ \textcolor{white}{H} & -0.02 & [-0.06, 0.02] & -0.85 & .397 & -0.01 & [-0.05, 0.03] & -0.41 & .684\\
\ \ \ After-slope * Grandparent, $\hat{\gamma}_{21}$ \textcolor{white}{H} & 0.00 & [-0.02, 0.03] & 0.35 & .730 & 0.01 & [-0.01, 0.04] & 1.09 & .276\\
\ \ \ Shift * Grandparent, $\hat{\gamma}_{31}$ \textcolor{white}{H} & 0.00 & [-0.08, 0.07] & -0.12 & .905 & -0.01 & [-0.08, 0.06] & -0.19 & .853\\
\ \ \ Before-slope * Female, $\hat{\gamma}_{12}$ \textcolor{white}{H} & -0.02 & [-0.04, 0.01] & -1.44 & .150 & 0.02 & [-0.01, 0.04] & 1.40 & .161\\
\ \ \ After-slope * Female, $\hat{\gamma}_{22}$ \textcolor{white}{H} & -0.03 & [-0.04, -0.01] & -3.28 & .001 & -0.01 & [-0.02, 0.01] & -0.98 & .327\\
\ \ \ Shift * Female, $\hat{\gamma}_{32}$ \textcolor{white}{H} & 0.05 & [0.00, 0.09] & 2.17 & .030 & 0.03 & [-0.01, 0.07] & 1.45 & .146\\
\ \ \ Grandparent * Female, $\hat{\gamma}_{03}$ \textcolor{white}{H} & 0.03 & [-0.09, 0.15] & 0.45 & .649 & 0.10 & [-0.03, 0.22] & 1.51 & .131\\
\ \ \ Before-slope * Grandparent * Female, $\hat{\gamma}_{13}$ \textcolor{white}{H} & 0.04 & [-0.01, 0.09] & 1.42 & .155 & 0.01 & [-0.05, 0.06] & 0.23 & .817\\
\ \ \ After-slope * Grandparent * Female, $\hat{\gamma}_{23}$ \textcolor{white}{H} & 0.01 & [-0.02, 0.05] & 0.79 & .431 & 0.00 & [-0.04, 0.03] & -0.27 & .790\\
\ \ \ Shift * Grandparent * Female, $\hat{\gamma}_{33}$ \textcolor{white}{H} & -0.06 & [-0.16, 0.04] & -1.19 & .234 & -0.04 & [-0.14, 0.05] & -0.87 & .383\\
\bottomrule
\addlinespace
\insertTableNotes
\end{longtable}

}

\end{lltable}








\begin{lltable}

\begin{TableNotes}[para]
\normalsize{\textit{Note.} The linear contrasts are based on
the models from Table \ref{tab:H1-extra-gender-tab}.
\(\hat{\gamma}_{c}\) = combined fixed-effects estimate.}
\end{TableNotes}

\footnotesize{

\begin{longtable}{lrrrrrr}\noalign{\getlongtablewidth\global\LTcapwidth=\longtablewidth}
\caption{\label{tab:H1-extra-gender-contrasts}Linear Contrasts for Extraversion
(Moderated by Gender).}\\
\toprule
& \multicolumn{3}{c}{Parent controls} & \multicolumn{3}{c}{Nonparent controls} \\
\cmidrule(r){2-4} \cmidrule(r){5-7}
Linear Contrast & $\hat{\gamma}_{c}$ & $\chi^2$ & $p$ & $\hat{\gamma}_{c}$ & $\chi^2$ & $p$\\
\midrule
\endfirsthead
\caption*{\normalfont{Table \ref{tab:H1-extra-gender-contrasts} continued}}\\
\toprule
& \multicolumn{3}{c}{Parent controls} & \multicolumn{3}{c}{Nonparent controls} \\
\cmidrule(r){2-4} \cmidrule(r){5-7}
Linear Contrast & $\hat{\gamma}_{c}$ & $\chi^2$ & $p$ & $\hat{\gamma}_{c}$ & $\chi^2$ & $p$\\
\midrule
\endhead
LISS &  &  &  &  &  & \\
\ \ \ Shift of male controls vs. 0 ($\hat{\gamma}_{20}$ + 
$\hat{\gamma}_{30}$) \textcolor{white}{L} & -0.05 & 6.28 & .012 & -0.05 & 9.10 & .003\\
\ \ \ Shift of female controls vs. 0 ($\hat{\gamma}_{20}$ + 
$\hat{\gamma}_{30}$ + $\hat{\gamma}_{22}$ + 
$\hat{\gamma}_{32}$) \textcolor{white}{L} & 0.01 & 0.09 & .763 & 0.02 & 0.95 & .330\\
\ \ \ Shift of grandfathers vs. 0 ($\hat{\gamma}_{20}$ + 
$\hat{\gamma}_{30}$ + $\hat{\gamma}_{21}$ + 
$\hat{\gamma}_{31}$) \textcolor{white}{L} & -0.04 & 1.25 & .264 & -0.04 & 1.16 & .281\\
\ \ \ Shift of grandmothers vs. 0 ($\hat{\gamma}_{20}$ + 
$\hat{\gamma}_{30}$ + $\hat{\gamma}_{21}$ + 
$\hat{\gamma}_{31}$ + $\hat{\gamma}_{22}$ + 
$\hat{\gamma}_{32}$ + $\hat{\gamma}_{23}$ +
$\hat{\gamma}_{33}$) \textcolor{white}{L} & -0.02 & 0.45 & .500 & -0.02 & 0.41 & .520\\
\ \ \ Shift of male controls vs. grandfathers 
($\hat{\gamma}_{21}$ + $\hat{\gamma}_{31}$) \textcolor{white}{L} & 0.01 & 0.02 & .891 & 0.01 & 0.13 & .716\\
\ \ \ Before-slope of female controls vs. grandmothers 
($\hat{\gamma}_{11}$ + $\hat{\gamma}_{13}$) \textcolor{white}{L} & -0.01 & 0.42 & .518 & 0.00 & 0.13 & .720\\
\ \ \ After-slope of female controls vs. grandmothers 
($\hat{\gamma}_{21}$ + $\hat{\gamma}_{23}$) \textcolor{white}{L} & 0.00 & 0.13 & .722 & 0.01 & 2.45 & .117\\
\ \ \ Shift of female controls vs. grandmothers 
($\hat{\gamma}_{21}$ + $\hat{\gamma}_{31}$ + 
$\hat{\gamma}_{23}$ + $\hat{\gamma}_{33}$) \textcolor{white}{L} & -0.03 & 0.54 & .461 & -0.04 & 1.03 & .311\\
\ \ \ Shift of male vs. female controls 
($\hat{\gamma}_{22}$ + $\hat{\gamma}_{32}$) \textcolor{white}{L} & 0.05 & 4.20 & .040 & 0.07 & 8.22 & .004\\
\ \ \ Before-slope of grandfathers vs. grandmothers 
($\hat{\gamma}_{12}$ + $\hat{\gamma}_{13}$) \textcolor{white}{L} & 0.00 & 0.03 & .871 & 0.00 & 0.01 & .943\\
\ \ \ After-slope of grandfathers vs. grandmothers 
($\hat{\gamma}_{22}$ + $\hat{\gamma}_{23}$) \textcolor{white}{L} & 0.00 & 0.03 & .857 & 0.00 & 0.04 & .834\\
\ \ \ Shift of grandfathers vs. grandmothers 
($\hat{\gamma}_{22}$ + $\hat{\gamma}_{32}$ + 
$\hat{\gamma}_{23}$ + $\hat{\gamma}_{33}$) \textcolor{white}{L} & 0.02 & 0.14 & .709 & 0.02 & 0.13 & .717\\
HRS &  &  &  &  &  & \\
\ \ \ Shift of male controls vs. 0 ($\hat{\gamma}_{20}$ + 
$\hat{\gamma}_{30}$) \textcolor{white}{H} & 0.00 & 0.06 & .812 & 0.00 & 0.09 & .765\\
\ \ \ Shift of female controls vs. 0 ($\hat{\gamma}_{20}$ + 
$\hat{\gamma}_{30}$ + $\hat{\gamma}_{22}$ + 
$\hat{\gamma}_{32}$) \textcolor{white}{H} & 0.03 & 5.44 & .020 & 0.02 & 3.52 & .061\\
\ \ \ Shift of grandfathers vs. 0 ($\hat{\gamma}_{20}$ + 
$\hat{\gamma}_{30}$ + $\hat{\gamma}_{21}$ + 
$\hat{\gamma}_{31}$) \textcolor{white}{H} & 0.00 & 0.01 & .905 & 0.00 & 0.01 & .903\\
\ \ \ Shift of grandmothers vs. 0 ($\hat{\gamma}_{20}$ + 
$\hat{\gamma}_{30}$ + $\hat{\gamma}_{21}$ + 
$\hat{\gamma}_{31}$ + $\hat{\gamma}_{22}$ + 
$\hat{\gamma}_{32}$ + $\hat{\gamma}_{23}$ +
$\hat{\gamma}_{33}$) \textcolor{white}{H} & -0.02 & 0.73 & .393 & -0.02 & 0.78 & .377\\
\ \ \ Shift of male controls vs. grandfathers 
($\hat{\gamma}_{21}$ + $\hat{\gamma}_{31}$) \textcolor{white}{H} & 0.00 & 0.00 & .999 & 0.01 & 0.06 & .805\\
\ \ \ Before-slope of female controls vs. grandmothers 
($\hat{\gamma}_{11}$ + $\hat{\gamma}_{13}$) \textcolor{white}{H} & 0.02 & 1.42 & .234 & 0.00 & 0.01 & .909\\
\ \ \ After-slope of female controls vs. grandmothers 
($\hat{\gamma}_{21}$ + $\hat{\gamma}_{23}$) \textcolor{white}{H} & 0.02 & 2.40 & .122 & 0.01 & 0.65 & .419\\
\ \ \ Shift of female controls vs. grandmothers 
($\hat{\gamma}_{21}$ + $\hat{\gamma}_{31}$ + 
$\hat{\gamma}_{23}$ + $\hat{\gamma}_{33}$) \textcolor{white}{H} & -0.05 & 3.28 & .070 & -0.04 & 2.65 & .104\\
\ \ \ Shift of male vs. female controls 
($\hat{\gamma}_{22}$ + $\hat{\gamma}_{32}$) \textcolor{white}{H} & 0.02 & 1.88 & .171 & 0.02 & 2.10 & .147\\
\ \ \ Before-slope of grandfathers vs. grandmothers 
($\hat{\gamma}_{12}$ + $\hat{\gamma}_{13}$) \textcolor{white}{H} & 0.02 & 0.79 & .373 & 0.02 & 0.85 & .357\\
\ \ \ After-slope of grandfathers vs. grandmothers 
($\hat{\gamma}_{22}$ + $\hat{\gamma}_{23}$) \textcolor{white}{H} & -0.01 & 0.57 & .452 & -0.01 & 0.62 & .431\\
\ \ \ Shift of grandfathers vs. grandmothers 
($\hat{\gamma}_{22}$ + $\hat{\gamma}_{32}$ + 
$\hat{\gamma}_{23}$ + $\hat{\gamma}_{33}$) \textcolor{white}{H} & -0.02 & 0.44 & .508 & -0.02 & 0.47 & .495\\
\bottomrule
\addlinespace
\insertTableNotes
\end{longtable}

}

\end{lltable}








\begin{lltable}

\begin{TableNotes}[para]
\normalsize{\textit{Note.} Two models were computed (only HRS):
grandparents matched with parent controls and with nonparent controls.
CI = confidence interval. \(working=1\) indicates being employed in paid
work.}
\end{TableNotes}

\footnotesize{

\begin{longtable}{lrcrrrcrr}\noalign{\getlongtablewidth\global\LTcapwidth=\longtablewidth}
\caption{\label{tab:H1-extra-work-tab}Fixed Effects of Extraversion Over the
Transition to Grandparenthood Moderated by Performing Paid Work.}\\
\toprule
& \multicolumn{4}{c}{Parent controls} & \multicolumn{4}{c}{Nonparent controls} \\
\cmidrule(r){2-5} \cmidrule(r){6-9}
Parameter & $\hat{\gamma}$ & 95\% CI & $t$ & $p$ & $\hat{\gamma}$ & 95\% CI & $t$ & $p$\\
\midrule
\endfirsthead
\caption*{\normalfont{Table \ref{tab:H1-extra-work-tab} continued}}\\
\toprule
& \multicolumn{4}{c}{Parent controls} & \multicolumn{4}{c}{Nonparent controls} \\
\cmidrule(r){2-5} \cmidrule(r){6-9}
Parameter & $\hat{\gamma}$ & 95\% CI & $t$ & $p$ & $\hat{\gamma}$ & 95\% CI & $t$ & $p$\\
\midrule
\endhead
Intercept, $\hat{\gamma}_{00}$ & 3.18 & [3.13, 3.23] & 129.04 & < .001 & 3.12 & [3.07, 3.17] & 112.49 & < .001\\
Propensity score, $\hat{\gamma}_{02}$ & 0.01 & [-0.06, 0.08] & 0.31 & .757 & 0.03 & [-0.04, 0.10] & 0.77 & .439\\
Before-slope, $\hat{\gamma}_{20}$ & 0.02 & [0.00, 0.04] & 1.69 & .091 & 0.00 & [-0.02, 0.02] & 0.09 & .927\\
After-slope, $\hat{\gamma}_{40}$ & 0.00 & [-0.01, 0.01] & 0.12 & .901 & -0.01 & [-0.02, 0.00] & -1.24 & .213\\
Shift, $\hat{\gamma}_{60}$ & -0.04 & [-0.08, -0.01] & -2.48 & .013 & 0.02 & [-0.02, 0.05] & 0.91 & .364\\
Grandparent, $\hat{\gamma}_{01}$ & -0.06 & [-0.16, 0.04] & -1.23 & .217 & -0.01 & [-0.11, 0.09] & -0.18 & .853\\
Working, $\hat{\gamma}_{10}$ & 0.03 & [-0.02, 0.07] & 1.19 & .232 & 0.00 & [-0.05, 0.04] & -0.12 & .902\\
Before-slope * Grandparent, $\hat{\gamma}_{21}$ & 0.02 & [-0.03, 0.07] & 0.74 & .460 & 0.04 & [-0.02, 0.09] & 1.38 & .169\\
After-slope * Grandparent, $\hat{\gamma}_{41}$ & 0.02 & [0.00, 0.04] & 1.65 & .099 & 0.03 & [0.00, 0.05] & 2.32 & .021\\
Shift * Grandparent, $\hat{\gamma}_{61}$ & -0.02 & [-0.10, 0.06] & -0.46 & .643 & -0.08 & [-0.16, 0.00] & -2.02 & .044\\
Before-slope * Working, $\hat{\gamma}_{30}$ & -0.03 & [-0.05, -0.01] & -2.38 & .017 & 0.01 & [-0.02, 0.03] & 0.59 & .556\\
After-slope * Working, $\hat{\gamma}_{50}$ & 0.00 & [-0.02, 0.01] & -0.19 & .848 & 0.01 & [0.00, 0.03] & 1.79 & .074\\
Shift * Working, $\hat{\gamma}_{70}$ & 0.10 & [0.05, 0.14] & 4.18 & < .001 & -0.01 & [-0.06, 0.04] & -0.43 & .667\\
Grandparent * Working, $\hat{\gamma}_{11}$ & 0.08 & [-0.02, 0.18] & 1.53 & .126 & 0.11 & [0.01, 0.21] & 2.13 & .034\\
Before-slope * Grandparent * Working, $\hat{\gamma}_{31}$ & -0.01 & [-0.08, 0.05] & -0.46 & .646 & -0.05 & [-0.11, 0.01] & -1.69 & .092\\
After-slope * Grandparent * Working, $\hat{\gamma}_{51}$ & -0.01 & [-0.05, 0.02] & -0.80 & .425 & -0.03 & [-0.06, 0.00] & -1.69 & .090\\
Shift * Grandparent * Working, $\hat{\gamma}_{71}$ & -0.03 & [-0.13, 0.08] & -0.49 & .623 & 0.08 & [-0.02, 0.18] & 1.57 & .115\\
\bottomrule
\addlinespace
\insertTableNotes
\end{longtable}

}

\end{lltable}








\begin{lltable}

\begin{TableNotes}[para]
\normalsize{\textit{Note.} The linear contrasts are based on the
models from Table \ref{tab:H1-extra-work-tab}. \(\hat{\gamma}_{c}\) =
combined fixed-effects estimate.}
\end{TableNotes}

\footnotesize{

\begin{longtable}{lrrrrrr}\noalign{\getlongtablewidth\global\LTcapwidth=\longtablewidth}
\caption{\label{tab:H1-extra-work-contrasts}Linear Contrasts for Extraversion
(Moderated by Paid Work; only HRS).}\\
\toprule
& \multicolumn{3}{c}{Parent controls} & \multicolumn{3}{c}{Nonparent controls} \\
\cmidrule(r){2-4} \cmidrule(r){5-7}
Linear Contrast & $\hat{\gamma}_{c}$ & $\chi^2$ & $p$ & $\hat{\gamma}_{c}$ & $\chi^2$ & $p$\\
\midrule
\endfirsthead
\caption*{\normalfont{Table \ref{tab:H1-extra-work-contrasts} continued}}\\
\toprule
& \multicolumn{3}{c}{Parent controls} & \multicolumn{3}{c}{Nonparent controls} \\
\cmidrule(r){2-4} \cmidrule(r){5-7}
Linear Contrast & $\hat{\gamma}_{c}$ & $\chi^2$ & $p$ & $\hat{\gamma}_{c}$ & $\chi^2$ & $p$\\
\midrule
\endhead
Shift of not-working controls vs. 0 ($\hat{\gamma}_{40}$ + 
$\hat{\gamma}_{60}$) & -0.04 & 9.28 & .002 & 0.01 & 0.42 & .515\\
Shift of working controls vs. 0 ($\hat{\gamma}_{40}$ + 
$\hat{\gamma}_{60}$ + $\hat{\gamma}_{50}$ + 
$\hat{\gamma}_{70}$) & 0.05 & 22.76 & < .001 & 0.01 & 1.67 & .196\\
Shift of not-working grandparents vs. 0 ($\hat{\gamma}_{40}$ + 
$\hat{\gamma}_{60}$ + $\hat{\gamma}_{41}$ + 
$\hat{\gamma}_{61}$) & -0.04 & 2.05 & .152 & -0.04 & 2.20 & .138\\
Shift of working grandparents vs. 0 ($\hat{\gamma}_{40}$ + 
$\hat{\gamma}_{60}$ + $\hat{\gamma}_{41}$ + 
$\hat{\gamma}_{61}$ + $\hat{\gamma}_{50}$ + 
$\hat{\gamma}_{70}$ + $\hat{\gamma}_{51}$ +
$\hat{\gamma}_{71}$) & 0.01 & 0.40 & .526 & 0.01 & 0.42 & .517\\
Shift of not-working controls vs. not-working grandparents 
($\hat{\gamma}_{41}$ + $\hat{\gamma}_{61}$) & 0.00 & 0.00 & .957 & -0.05 & 2.60 & .107\\
Before-slope of working controls vs. working grandparents 
($\hat{\gamma}_{21}$ + $\hat{\gamma}_{31}$) & 0.01 & 0.12 & .729 & -0.02 & 1.06 & .303\\
After-slope of working controls vs. working grandparents 
($\hat{\gamma}_{41}$ + $\hat{\gamma}_{51}$) & 0.01 & 0.28 & .598 & 0.00 & 0.00 & .948\\
Shift of working controls vs. working grandparents 
($\hat{\gamma}_{41}$ + $\hat{\gamma}_{61}$ + 
$\hat{\gamma}_{51}$ + $\hat{\gamma}_{71}$) & -0.04 & 2.46 & .117 & 0.00 & 0.00 & .987\\
Shift of not-working controls vs. working controls 
($\hat{\gamma}_{50}$ + $\hat{\gamma}_{70}$) & 0.10 & 27.75 & < .001 & 0.00 & 0.04 & .852\\
Before-slope of not-working grandparents vs. working grandparents 
($\hat{\gamma}_{30}$ + $\hat{\gamma}_{31}$) & -0.04 & 2.34 & .126 & -0.04 & 2.52 & .113\\
After-slope of not-working grandparents vs. working grandparents 
($\hat{\gamma}_{50}$ + $\hat{\gamma}_{51}$) & -0.02 & 0.97 & .325 & -0.02 & 1.01 & .314\\
Shift of not-working grandparents vs. working grandparents 
($\hat{\gamma}_{50}$ + $\hat{\gamma}_{70}$ + 
$\hat{\gamma}_{51}$ + $\hat{\gamma}_{71}$) & 0.06 & 2.24 & .135 & 0.06 & 2.38 & .123\\
\bottomrule
\addlinespace
\insertTableNotes
\end{longtable}

}

\end{lltable}








\begin{lltable}

\begin{TableNotes}[para]
\normalsize{\textit{Note.} Two models were computed (only HRS):
grandparents matched with parent controls and with nonparent controls.
CI = confidence interval. \(caring=1\) indicates more than 100 hours of
grandchild care since the last assessment.}
\end{TableNotes}

\footnotesize{

\begin{longtable}{lrcrrrcrr}\noalign{\getlongtablewidth\global\LTcapwidth=\longtablewidth}
\caption{\label{tab:H1-extra-care-tab}Fixed Effects of Extraversion Over the
Transition to Grandparenthood Moderated by Grandchild Care.}\\
\toprule
& \multicolumn{4}{c}{Parent controls} & \multicolumn{4}{c}{Nonparent controls} \\
\cmidrule(r){2-5} \cmidrule(r){6-9}
Parameter & $\hat{\gamma}$ & 95\% CI & $t$ & $p$ & $\hat{\gamma}$ & 95\% CI & $t$ & $p$\\
\midrule
\endfirsthead
\caption*{\normalfont{Table \ref{tab:H1-extra-care-tab} continued}}\\
\toprule
& \multicolumn{4}{c}{Parent controls} & \multicolumn{4}{c}{Nonparent controls} \\
\cmidrule(r){2-5} \cmidrule(r){6-9}
Parameter & $\hat{\gamma}$ & 95\% CI & $t$ & $p$ & $\hat{\gamma}$ & 95\% CI & $t$ & $p$\\
\midrule
\endhead
Intercept, $\hat{\gamma}_{00}$ & 3.19 & [3.14, 3.24] & 128.26 & < .001 & 3.12 & [3.06, 3.18] & 102.87 & < .001\\
Propensity score, $\hat{\gamma}_{02}$ & 0.13 & [0.04, 0.22] & 2.98 & .003 & 0.08 & [-0.01, 0.17] & 1.67 & .096\\
After-slope, $\hat{\gamma}_{20}$ & -0.01 & [-0.03, 0.00] & -2.61 & .009 & 0.00 & [-0.01, 0.01] & -0.39 & .694\\
Grandparent, $\hat{\gamma}_{01}$ & -0.04 & [-0.11, 0.03] & -1.05 & .296 & 0.04 & [-0.04, 0.12] & 1.06 & .288\\
Caring, $\hat{\gamma}_{10}$ & 0.00 & [-0.03, 0.04] & 0.23 & .815 & 0.02 & [-0.02, 0.05] & 0.86 & .391\\
After-slope * Grandparent, $\hat{\gamma}_{21}$ & 0.02 & [-0.01, 0.04] & 1.32 & .186 & 0.00 & [-0.02, 0.02] & 0.30 & .767\\
After-slope * Caring, $\hat{\gamma}_{30}$ & 0.00 & [-0.02, 0.02] & -0.04 & .965 & 0.00 & [-0.02, 0.01] & -0.42 & .676\\
Grandparent * Caring, $\hat{\gamma}_{11}$ & -0.04 & [-0.13, 0.06] & -0.74 & .461 & -0.05 & [-0.14, 0.04] & -1.04 & .299\\
After-slope * Grandparent * Caring, $\hat{\gamma}_{31}$ & 0.03 & [-0.01, 0.06] & 1.56 & .119 & 0.03 & [0.00, 0.07] & 1.83 & .067\\
\bottomrule
\addlinespace
\insertTableNotes
\end{longtable}

}

\end{lltable}







\begin{lltable}

\begin{TableNotes}[para]
\normalsize{\textit{Note.} The linear contrasts are based on the
models from Table \ref{tab:H1-extra-care-tab}. \(\hat{\gamma}_{c}\) =
combined fixed-effects estimate.}
\end{TableNotes}

\footnotesize{

\begin{longtable}{lrrrrrr}\noalign{\getlongtablewidth\global\LTcapwidth=\longtablewidth}
\caption{\label{tab:H1-extra-care-contrasts}Linear Contrasts for Extraversion
(Moderated by Grandchild Care; only HRS).}\\
\toprule
& \multicolumn{3}{c}{Parent controls} & \multicolumn{3}{c}{Nonparent controls} \\
\cmidrule(r){2-4} \cmidrule(r){5-7}
Linear Contrast & $\hat{\gamma}_{c}$ & $\chi^2$ & $p$ & $\hat{\gamma}_{c}$ & $\chi^2$ & $p$\\
\midrule
\endfirsthead
\caption*{\normalfont{Table \ref{tab:H1-extra-care-contrasts} continued}}\\
\toprule
& \multicolumn{3}{c}{Parent controls} & \multicolumn{3}{c}{Nonparent controls} \\
\cmidrule(r){2-4} \cmidrule(r){5-7}
Linear Contrast & $\hat{\gamma}_{c}$ & $\chi^2$ & $p$ & $\hat{\gamma}_{c}$ & $\chi^2$ & $p$\\
\midrule
\endhead
After-slope of caring controls vs. caring grandparents 
($\hat{\gamma}_{21}$ + $\hat{\gamma}_{31}$) & 0.04 & 10.45 & .001 & 0.04 & 7.39 & .007\\
After-slope of not-caring grandparents vs. caring grandparents 
($\hat{\gamma}_{30}$ + $\hat{\gamma}_{31}$) & 0.03 & 2.98 & .084 & 0.03 & 3.37 & .066\\
\bottomrule
\addlinespace
\insertTableNotes
\end{longtable}

}

\end{lltable}








\begin{lltable}

\begin{TableNotes}[para]
\normalsize{\textit{Note.} Two models were computed for each of the two
samples (LISS, HRS): grandparents matched with parent controls and with
nonparent controls. CI = confidence interval.}
\end{TableNotes}

\footnotesize{

\begin{longtable}{lrcrrrcrr}\noalign{\getlongtablewidth\global\LTcapwidth=\longtablewidth}
\caption{\label{tab:H1-neur-tab}Fixed Effects of Neuroticism Over the Transition
to Grandparenthood.}\\
\toprule
& \multicolumn{4}{c}{Parent controls} & \multicolumn{4}{c}{Nonparent controls} \\
\cmidrule(r){2-5} \cmidrule(r){6-9}
Parameter & $\hat{\gamma}$ & 95\% CI & $t$ & $p$ & $\hat{\gamma}$ & 95\% CI & $t$ & $p$\\
\midrule
\endfirsthead
\caption*{\normalfont{Table \ref{tab:H1-neur-tab} continued}}\\
\toprule
& \multicolumn{4}{c}{Parent controls} & \multicolumn{4}{c}{Nonparent controls} \\
\cmidrule(r){2-5} \cmidrule(r){6-9}
Parameter & $\hat{\gamma}$ & 95\% CI & $t$ & $p$ & $\hat{\gamma}$ & 95\% CI & $t$ & $p$\\
\midrule
\endhead
LISS &  &  &  &  &  &  &  & \\
\ \ \ Intercept, $\hat{\gamma}_{00}$ \textcolor{white}{L} & 2.48 & [2.40, 2.56] & 63.09 & < .001 & 2.45 & [2.35, 2.54] & 51.88 & < .001\\
\ \ \ Propensity score, $\hat{\gamma}_{02}$ \textcolor{white}{L} & 0.01 & [-0.09, 0.11] & 0.19 & .852 & 0.00 & [-0.09, 0.09] & 0.04 & .967\\
\ \ \ Before-slope, $\hat{\gamma}_{10}$ \textcolor{white}{L} & 0.00 & [-0.01, 0.01] & -0.56 & .575 & -0.01 & [-0.02, -0.01] & -3.66 & < .001\\
\ \ \ After-slope, $\hat{\gamma}_{20}$ \textcolor{white}{L} & 0.00 & [0.00, 0.01] & 0.94 & .350 & 0.00 & [0.00, 0.01] & 1.31 & .190\\
\ \ \ Shift, $\hat{\gamma}_{30}$ \textcolor{white}{L} & -0.05 & [-0.08, -0.02] & -2.96 & .003 & -0.03 & [-0.06, 0.01] & -1.58 & .115\\
\ \ \ Grandparent, $\hat{\gamma}_{01}$ \textcolor{white}{L} & -0.08 & [-0.20, 0.03] & -1.37 & .170 & -0.04 & [-0.17, 0.08] & -0.67 & .500\\
\ \ \ Before-slope * Grandparent, $\hat{\gamma}_{11}$ \textcolor{white}{L} & 0.00 & [-0.01, 0.02] & 0.43 & .668 & 0.02 & [0.00, 0.03] & 1.83 & .067\\
\ \ \ After-slope * Grandparent, $\hat{\gamma}_{21}$ \textcolor{white}{L} & 0.00 & [-0.02, 0.01] & -0.33 & .744 & 0.00 & [-0.02, 0.01] & -0.48 & .635\\
\ \ \ Shift * Grandparent, $\hat{\gamma}_{31}$ \textcolor{white}{L} & -0.02 & [-0.09, 0.06] & -0.41 & .684 & -0.04 & [-0.12, 0.04] & -1.01 & .312\\
HRS &  &  &  &  &  &  &  & \\
\ \ \ Intercept, $\hat{\gamma}_{00}$ \textcolor{white}{H} & 2.07 & [2.03, 2.11] & 94.42 & < .001 & 2.07 & [2.02, 2.12] & 79.36 & < .001\\
\ \ \ Propensity score, $\hat{\gamma}_{02}$ \textcolor{white}{H} & 0.00 & [-0.07, 0.08] & 0.12 & .902 & 0.15 & [0.07, 0.23] & 3.70 & < .001\\
\ \ \ Before-slope, $\hat{\gamma}_{10}$ \textcolor{white}{H} & -0.01 & [-0.03, 0.00] & -1.90 & .057 & -0.03 & [-0.04, -0.02] & -4.70 & < .001\\
\ \ \ After-slope, $\hat{\gamma}_{20}$ \textcolor{white}{H} & -0.01 & [-0.01, 0.00] & -1.20 & .230 & -0.01 & [-0.02, -0.01] & -3.18 & .001\\
\ \ \ Shift, $\hat{\gamma}_{30}$ \textcolor{white}{H} & 0.01 & [-0.02, 0.03] & 0.42 & .675 & -0.03 & [-0.06, -0.01] & -2.36 & .018\\
\ \ \ Grandparent, $\hat{\gamma}_{01}$ \textcolor{white}{H} & -0.06 & [-0.13, 0.01] & -1.64 & .100 & -0.12 & [-0.19, -0.05] & -3.31 & .001\\
\ \ \ Before-slope * Grandparent, $\hat{\gamma}_{11}$ \textcolor{white}{H} & 0.02 & [-0.01, 0.05] & 1.28 & .201 & 0.04 & [0.01, 0.07] & 2.42 & .016\\
\ \ \ After-slope * Grandparent, $\hat{\gamma}_{21}$ \textcolor{white}{H} & -0.02 & [-0.04, 0.00] & -1.52 & .127 & -0.01 & [-0.03, 0.01] & -0.80 & .424\\
\ \ \ Shift * Grandparent, $\hat{\gamma}_{31}$ \textcolor{white}{H} & -0.06 & [-0.12, 0.00] & -2.12 & .034 & -0.03 & [-0.08, 0.03] & -0.88 & .381\\
\bottomrule
\addlinespace
\insertTableNotes
\end{longtable}

}

\end{lltable}









\begin{lltable}

\begin{TableNotes}[para]
\normalsize{\textit{Note.} The linear contrasts are needed in cases
where estimates of interest are represented by multiple fixed-effects
coefficients and are computed using the \emph{linearHypothesis} function
from the \emph{car} R package (Fox \& Weisberg, 2019) based on the
models from Table \ref{tab:H1-neur-tab}. \(\hat{\gamma}_{c}\) = combined
fixed-effects estimate.}
\end{TableNotes}

\footnotesize{

\begin{longtable}{lrrrrrr}\noalign{\getlongtablewidth\global\LTcapwidth=\longtablewidth}
\caption{\label{tab:H1-neur-contrasts}Linear Contrasts for Neuroticism.}\\
\toprule
& \multicolumn{3}{c}{Parent controls} & \multicolumn{3}{c}{Nonparent controls} \\
\cmidrule(r){2-4} \cmidrule(r){5-7}
Linear Contrast & $\hat{\gamma}_{c}$ & $\chi^2$ & $p$ & $\hat{\gamma}_{c}$ & $\chi^2$ & $p$\\
\midrule
\endfirsthead
\caption*{\normalfont{Table \ref{tab:H1-neur-contrasts} continued}}\\
\toprule
& \multicolumn{3}{c}{Parent controls} & \multicolumn{3}{c}{Nonparent controls} \\
\cmidrule(r){2-4} \cmidrule(r){5-7}
Linear Contrast & $\hat{\gamma}_{c}$ & $\chi^2$ & $p$ & $\hat{\gamma}_{c}$ & $\chi^2$ & $p$\\
\midrule
\endhead
LISS &  &  &  &  &  & \\
\ \ \ Shift of the controls vs. 0 ($\hat{\gamma}_{20}$ + 
$\hat{\gamma}_{30}$) \textcolor{white}{L} & -0.05 & 10.12 & .001 & -0.02 & 2.26 & .133\\
\ \ \ Shift of the grandparents vs. 0 ($\hat{\gamma}_{20}$ + 
$\hat{\gamma}_{30}$ + $\hat{\gamma}_{21}$ + 
$\hat{\gamma}_{31}$) \textcolor{white}{L} & -0.07 & 4.99 & .025 & -0.07 & 4.74 & .029\\
\ \ \ Shift of the controls vs. shift of the grandparents 
($\hat{\gamma}_{21}$ + $\hat{\gamma}_{31}$) \textcolor{white}{L} & -0.02 & 0.30 & .587 & -0.04 & 1.62 & .203\\
\ \ \ Before-slope of the grandparents vs. 0 ($\hat{\gamma}_{10}$ + 
$\hat{\gamma}_{11}$) \textcolor{white}{L} & 0.00 & 0.04 & .842 & 0.00 & 0.05 & .830\\
\ \ \ After-slope of the grandparents vs. 0 ($\hat{\gamma}_{20}$ + 
$\hat{\gamma}_{21}$) \textcolor{white}{L} & 0.00 & 0.01 & .914 & 0.00 & 0.02 & .900\\
HRS &  &  &  &  &  & \\
\ \ \ Shift of the controls vs. 0 ($\hat{\gamma}_{20}$ + 
$\hat{\gamma}_{30}$) \textcolor{white}{H} & 0.00 & 0.00 & .993 & -0.04 & 20.02 & < .001\\
\ \ \ Shift of the grandparents vs. 0 ($\hat{\gamma}_{20}$ + 
$\hat{\gamma}_{30}$ + $\hat{\gamma}_{21}$ + 
$\hat{\gamma}_{31}$) \textcolor{white}{H} & -0.08 & 15.10 & < .001 & -0.08 & 15.78 & < .001\\
\ \ \ Shift of the controls vs. shift of the grandparents 
($\hat{\gamma}_{21}$ + $\hat{\gamma}_{31}$) \textcolor{white}{H} & -0.08 & 12.06 & .001 & -0.03 & 2.29 & .130\\
\ \ \ Before-slope of the grandparents vs. 0 ($\hat{\gamma}_{10}$ + 
$\hat{\gamma}_{11}$) \textcolor{white}{H} & 0.01 & 0.25 & .618 & 0.01 & 0.19 & .666\\
\ \ \ After-slope of the grandparents vs. 0 ($\hat{\gamma}_{20}$ + 
$\hat{\gamma}_{21}$) \textcolor{white}{H} & -0.02 & 5.29 & .021 & -0.02 & 6.13 & .013\\
\bottomrule
\addlinespace
\insertTableNotes
\end{longtable}

}

\end{lltable}







\begin{lltable}

\begin{TableNotes}[para]
\normalsize{\textit{Note.} Two models were computed for each of the
two samples (LISS, HRS): grandparents matched with parent controls and
with nonparent controls. CI = confidence interval.}
\end{TableNotes}

\footnotesize{

\begin{longtable}{lrcrrrcrr}\noalign{\getlongtablewidth\global\LTcapwidth=\longtablewidth}
\caption{\label{tab:H1-neur-gender-tab}Fixed Effects of Neuroticism Over the
Transition to Grandparenthood Moderated by Gender.}\\
\toprule
& \multicolumn{4}{c}{Parent controls} & \multicolumn{4}{c}{Nonparent controls} \\
\cmidrule(r){2-5} \cmidrule(r){6-9}
Parameter & $\hat{\gamma}$ & 95\% CI & $t$ & $p$ & $\hat{\gamma}$ & 95\% CI & $t$ & $p$\\
\midrule
\endfirsthead
\caption*{\normalfont{Table \ref{tab:H1-neur-gender-tab} continued}}\\
\toprule
& \multicolumn{4}{c}{Parent controls} & \multicolumn{4}{c}{Nonparent controls} \\
\cmidrule(r){2-5} \cmidrule(r){6-9}
Parameter & $\hat{\gamma}$ & 95\% CI & $t$ & $p$ & $\hat{\gamma}$ & 95\% CI & $t$ & $p$\\
\midrule
\endhead
LISS &  &  &  &  &  &  &  & \\
\ \ \ Intercept, $\hat{\gamma}_{00}$ \textcolor{white}{L} & 2.45 & [2.34, 2.56] & 43.45 & < .001 & 2.32 & [2.19, 2.45] & 34.99 & < .001\\
\ \ \ Propensity score, $\hat{\gamma}_{04}$ \textcolor{white}{L} & 0.02 & [-0.09, 0.12] & 0.30 & .767 & 0.02 & [-0.08, 0.11] & 0.33 & .744\\
\ \ \ Before-slope, $\hat{\gamma}_{10}$ \textcolor{white}{L} & -0.01 & [-0.02, 0.00] & -1.89 & .059 & -0.01 & [-0.02, 0.00] & -1.12 & .263\\
\ \ \ After-slope, $\hat{\gamma}_{20}$ \textcolor{white}{L} & 0.01 & [0.00, 0.02] & 2.82 & .005 & 0.01 & [0.00, 0.02] & 2.43 & .015\\
\ \ \ Shift, $\hat{\gamma}_{30}$ \textcolor{white}{L} & -0.06 & [-0.11, -0.01] & -2.24 & .025 & -0.05 & [-0.10, 0.00] & -1.95 & .052\\
\ \ \ Grandparent, $\hat{\gamma}_{01}$ \textcolor{white}{L} & -0.18 & [-0.35, -0.01] & -2.11 & .035 & -0.05 & [-0.23, 0.13] & -0.56 & .574\\
\ \ \ Female, $\hat{\gamma}_{02}$ \textcolor{white}{L} & 0.05 & [-0.09, 0.20] & 0.72 & .474 & 0.22 & [0.05, 0.40] & 2.52 & .012\\
\ \ \ Before-slope * Grandparent, $\hat{\gamma}_{11}$ \textcolor{white}{L} & 0.01 & [-0.01, 0.04] & 0.82 & .413 & 0.01 & [-0.02, 0.03] & 0.46 & .643\\
\ \ \ After-slope * Grandparent, $\hat{\gamma}_{21}$ \textcolor{white}{L} & -0.02 & [-0.04, 0.01] & -1.36 & .173 & -0.01 & [-0.04, 0.01] & -1.15 & .250\\
\ \ \ Shift * Grandparent, $\hat{\gamma}_{31}$ \textcolor{white}{L} & -0.03 & [-0.14, 0.08] & -0.51 & .612 & -0.04 & [-0.15, 0.08] & -0.63 & .529\\
\ \ \ Before-slope * Female, $\hat{\gamma}_{12}$ \textcolor{white}{L} & 0.02 & [0.00, 0.03] & 2.03 & .043 & -0.01 & [-0.03, 0.00] & -1.83 & .067\\
\ \ \ After-slope * Female, $\hat{\gamma}_{22}$ \textcolor{white}{L} & -0.02 & [-0.03, -0.01] & -2.99 & .003 & -0.01 & [-0.03, 0.00] & -2.10 & .036\\
\ \ \ Shift * Female, $\hat{\gamma}_{32}$ \textcolor{white}{L} & 0.01 & [-0.05, 0.08] & 0.39 & .700 & 0.04 & [-0.03, 0.11] & 1.19 & .234\\
\ \ \ Grandparent * Female, $\hat{\gamma}_{03}$ \textcolor{white}{L} & 0.18 & [-0.05, 0.40] & 1.54 & .123 & 0.01 & [-0.24, 0.25] & 0.06 & .951\\
\ \ \ Before-slope * Grandparent * Female, $\hat{\gamma}_{13}$ \textcolor{white}{L} & -0.01 & [-0.05, 0.02] & -0.66 & .508 & 0.02 & [-0.02, 0.05] & 1.08 & .279\\
\ \ \ After-slope * Grandparent * Female, $\hat{\gamma}_{23}$ \textcolor{white}{L} & 0.02 & [-0.01, 0.05] & 1.48 & .138 & 0.02 & [-0.01, 0.05] & 1.08 & .282\\
\ \ \ Shift * Grandparent * Female, $\hat{\gamma}_{33}$ \textcolor{white}{L} & 0.03 & [-0.12, 0.18] & 0.35 & .730 & 0.00 & [-0.16, 0.15] & -0.03 & .975\\
HRS &  &  &  &  &  &  &  & \\
\ \ \ Intercept, $\hat{\gamma}_{00}$ \textcolor{white}{H} & 1.98 & [1.91, 2.04] & 62.73 & < .001 & 2.01 & [1.94, 2.08] & 56.33 & < .001\\
\ \ \ Propensity score, $\hat{\gamma}_{04}$ \textcolor{white}{H} & 0.01 & [-0.07, 0.09] & 0.26 & .798 & 0.15 & [0.07, 0.23] & 3.58 & < .001\\
\ \ \ Before-slope, $\hat{\gamma}_{10}$ \textcolor{white}{H} & -0.02 & [-0.04, 0.00] & -2.11 & .035 & -0.03 & [-0.05, -0.01] & -3.18 & .001\\
\ \ \ After-slope, $\hat{\gamma}_{20}$ \textcolor{white}{H} & -0.02 & [-0.03, 0.00] & -2.40 & .017 & -0.02 & [-0.03, -0.01] & -2.92 & .003\\
\ \ \ Shift, $\hat{\gamma}_{30}$ \textcolor{white}{H} & 0.08 & [0.04, 0.12] & 4.02 & < .001 & 0.00 & [-0.03, 0.04] & 0.21 & .834\\
\ \ \ Grandparent, $\hat{\gamma}_{01}$ \textcolor{white}{H} & -0.06 & [-0.16, 0.04] & -1.10 & .272 & -0.16 & [-0.26, -0.05] & -2.89 & .004\\
\ \ \ Female, $\hat{\gamma}_{02}$ \textcolor{white}{H} & 0.17 & [0.09, 0.25] & 4.19 & < .001 & 0.10 & [0.01, 0.19] & 2.23 & .026\\
\ \ \ Before-slope * Grandparent, $\hat{\gamma}_{11}$ \textcolor{white}{H} & 0.06 & [0.01, 0.10] & 2.26 & .024 & 0.06 & [0.02, 0.11] & 2.72 & .007\\
\ \ \ After-slope * Grandparent, $\hat{\gamma}_{21}$ \textcolor{white}{H} & 0.00 & [-0.03, 0.03] & 0.31 & .755 & 0.01 & [-0.02, 0.04] & 0.48 & .630\\
\ \ \ Shift * Grandparent, $\hat{\gamma}_{31}$ \textcolor{white}{H} & -0.16 & [-0.25, -0.07] & -3.60 & < .001 & -0.08 & [-0.17, 0.00] & -1.89 & .059\\
\ \ \ Before-slope * Female, $\hat{\gamma}_{12}$ \textcolor{white}{H} & 0.01 & [-0.01, 0.04] & 1.04 & .300 & 0.00 & [-0.03, 0.03] & 0.09 & .926\\
\ \ \ After-slope * Female, $\hat{\gamma}_{22}$ \textcolor{white}{H} & 0.02 & [0.00, 0.04] & 2.19 & .029 & 0.01 & [-0.01, 0.03] & 1.15 & .250\\
\ \ \ Shift * Female, $\hat{\gamma}_{32}$ \textcolor{white}{H} & -0.14 & [-0.19, -0.08] & -5.02 & < .001 & -0.06 & [-0.11, -0.01] & -2.33 & .020\\
\ \ \ Grandparent * Female, $\hat{\gamma}_{03}$ \textcolor{white}{H} & 0.00 & [-0.14, 0.13] & -0.01 & .993 & 0.06 & [-0.08, 0.20] & 0.82 & .410\\
\ \ \ Before-slope * Grandparent * Female, $\hat{\gamma}_{13}$ \textcolor{white}{H} & -0.06 & [-0.12, 0.00] & -1.85 & .065 & -0.05 & [-0.11, 0.01] & -1.49 & .138\\
\ \ \ After-slope * Grandparent * Female, $\hat{\gamma}_{23}$ \textcolor{white}{H} & -0.04 & [-0.08, 0.00] & -1.80 & .073 & -0.03 & [-0.07, 0.01] & -1.35 & .176\\
\ \ \ Shift * Grandparent * Female, $\hat{\gamma}_{33}$ \textcolor{white}{H} & 0.17 & [0.06, 0.29] & 2.90 & .004 & 0.10 & [-0.01, 0.21] & 1.71 & .087\\
\bottomrule
\addlinespace
\insertTableNotes
\end{longtable}

}

\end{lltable}








\begin{lltable}

\begin{TableNotes}[para]
\normalsize{\textit{Note.} The linear contrasts are based on
the models from Table \ref{tab:H1-neur-gender-tab}. \(\hat{\gamma}_{c}\)
= combined fixed-effects estimate.}
\end{TableNotes}

\footnotesize{

\begin{longtable}{lrrrrrr}\noalign{\getlongtablewidth\global\LTcapwidth=\longtablewidth}
\caption{\label{tab:H1-neur-gender-contrasts}Linear Contrasts for Neuroticism
(Moderated by Gender).}\\
\toprule
& \multicolumn{3}{c}{Parent controls} & \multicolumn{3}{c}{Nonparent controls} \\
\cmidrule(r){2-4} \cmidrule(r){5-7}
Linear Contrast & $\hat{\gamma}_{c}$ & $\chi^2$ & $p$ & $\hat{\gamma}_{c}$ & $\chi^2$ & $p$\\
\midrule
\endfirsthead
\caption*{\normalfont{Table \ref{tab:H1-neur-gender-contrasts} continued}}\\
\toprule
& \multicolumn{3}{c}{Parent controls} & \multicolumn{3}{c}{Nonparent controls} \\
\cmidrule(r){2-4} \cmidrule(r){5-7}
Linear Contrast & $\hat{\gamma}_{c}$ & $\chi^2$ & $p$ & $\hat{\gamma}_{c}$ & $\chi^2$ & $p$\\
\midrule
\endhead
LISS &  &  &  &  &  & \\
\ \ \ Shift of male controls vs. 0 ($\hat{\gamma}_{20}$ + 
$\hat{\gamma}_{30}$) \textcolor{white}{L} & -0.04 & 3.64 & .056 & -0.04 & 2.76 & .096\\
\ \ \ Shift of female controls vs. 0 ($\hat{\gamma}_{20}$ + 
$\hat{\gamma}_{30}$ + $\hat{\gamma}_{22}$ + 
$\hat{\gamma}_{32}$) \textcolor{white}{L} & -0.05 & 6.02 & .014 & -0.01 & 0.24 & .621\\
\ \ \ Shift of grandfathers vs. 0 ($\hat{\gamma}_{20}$ + 
$\hat{\gamma}_{30}$ + $\hat{\gamma}_{21}$ + 
$\hat{\gamma}_{31}$) \textcolor{white}{L} & -0.09 & 3.89 & .048 & -0.09 & 3.67 & .055\\
\ \ \ Shift of grandmothers vs. 0 ($\hat{\gamma}_{20}$ + 
$\hat{\gamma}_{30}$ + $\hat{\gamma}_{21}$ + 
$\hat{\gamma}_{31}$ + $\hat{\gamma}_{22}$ + 
$\hat{\gamma}_{32}$ + $\hat{\gamma}_{23}$ +
$\hat{\gamma}_{33}$) \textcolor{white}{L} & -0.04 & 1.25 & .263 & -0.05 & 1.20 & .273\\
\ \ \ Shift of male controls vs. grandfathers 
($\hat{\gamma}_{21}$ + $\hat{\gamma}_{31}$) \textcolor{white}{L} & -0.04 & 0.80 & .371 & -0.05 & 0.97 & .325\\
\ \ \ Before-slope of female controls vs. grandmothers 
($\hat{\gamma}_{11}$ + $\hat{\gamma}_{13}$) \textcolor{white}{L} & 0.00 & 0.01 & .935 & 0.03 & 4.48 & .034\\
\ \ \ After-slope of female controls vs. grandmothers 
($\hat{\gamma}_{21}$ + $\hat{\gamma}_{23}$) \textcolor{white}{L} & 0.01 & 0.51 & .476 & 0.00 & 0.12 & .730\\
\ \ \ Shift of female controls vs. grandmothers 
($\hat{\gamma}_{21}$ + $\hat{\gamma}_{31}$ + 
$\hat{\gamma}_{23}$ + $\hat{\gamma}_{33}$) \textcolor{white}{L} & 0.01 & 0.01 & .904 & -0.03 & 0.57 & .451\\
\ \ \ Shift of male vs. female controls 
($\hat{\gamma}_{22}$ + $\hat{\gamma}_{32}$) \textcolor{white}{L} & -0.01 & 0.06 & .799 & 0.03 & 0.76 & .382\\
\ \ \ Before-slope of grandfathers vs. grandmothers 
($\hat{\gamma}_{12}$ + $\hat{\gamma}_{13}$) \textcolor{white}{L} & 0.00 & 0.08 & .783 & 0.00 & 0.09 & .765\\
\ \ \ After-slope of grandfathers vs. grandmothers 
($\hat{\gamma}_{22}$ + $\hat{\gamma}_{23}$) \textcolor{white}{L} & 0.00 & 0.02 & .882 & 0.00 & 0.02 & .875\\
\ \ \ Shift of grandfathers vs. grandmothers 
($\hat{\gamma}_{22}$ + $\hat{\gamma}_{32}$ + 
$\hat{\gamma}_{23}$ + $\hat{\gamma}_{33}$) \textcolor{white}{L} & 0.04 & 0.50 & .481 & 0.04 & 0.46 & .498\\
HRS &  &  &  &  &  & \\
\ \ \ Shift of male controls vs. 0 ($\hat{\gamma}_{20}$ + 
$\hat{\gamma}_{30}$) \textcolor{white}{H} & 0.06 & 17.37 & < .001 & -0.02 & 1.08 & .299\\
\ \ \ Shift of female controls vs. 0 ($\hat{\gamma}_{20}$ + 
$\hat{\gamma}_{30}$ + $\hat{\gamma}_{22}$ + 
$\hat{\gamma}_{32}$) \textcolor{white}{H} & -0.05 & 13.66 & < .001 & -0.07 & 25.37 & < .001\\
\ \ \ Shift of grandfathers vs. 0 ($\hat{\gamma}_{20}$ + 
$\hat{\gamma}_{30}$ + $\hat{\gamma}_{21}$ + 
$\hat{\gamma}_{31}$) \textcolor{white}{H} & -0.09 & 9.12 & .003 & -0.09 & 9.50 & .002\\
\ \ \ Shift of grandmothers vs. 0 ($\hat{\gamma}_{20}$ + 
$\hat{\gamma}_{30}$ + $\hat{\gamma}_{21}$ + 
$\hat{\gamma}_{31}$ + $\hat{\gamma}_{22}$ + 
$\hat{\gamma}_{32}$ + $\hat{\gamma}_{23}$ +
$\hat{\gamma}_{33}$) \textcolor{white}{H} & -0.07 & 6.49 & .011 & -0.07 & 6.77 & .009\\
\ \ \ Shift of male controls vs. grandfathers 
($\hat{\gamma}_{21}$ + $\hat{\gamma}_{31}$) \textcolor{white}{H} & -0.16 & 20.99 & < .001 & -0.07 & 5.10 & .024\\
\ \ \ Before-slope of female controls vs. grandmothers 
($\hat{\gamma}_{11}$ + $\hat{\gamma}_{13}$) \textcolor{white}{H} & 0.00 & 0.05 & .821 & 0.02 & 0.73 & .392\\
\ \ \ After-slope of female controls vs. grandmothers 
($\hat{\gamma}_{21}$ + $\hat{\gamma}_{23}$) \textcolor{white}{H} & -0.03 & 5.41 & .020 & -0.02 & 2.20 & .138\\
\ \ \ Shift of female controls vs. grandmothers 
($\hat{\gamma}_{21}$ + $\hat{\gamma}_{31}$ + 
$\hat{\gamma}_{23}$ + $\hat{\gamma}_{33}$) \textcolor{white}{H} & -0.02 & 0.37 & .541 & 0.00 & 0.01 & .943\\
\ \ \ Shift of male vs. female controls 
($\hat{\gamma}_{22}$ + $\hat{\gamma}_{32}$) \textcolor{white}{H} & -0.12 & 31.04 & < .001 & -0.05 & 6.32 & .012\\
\ \ \ Before-slope of grandfathers vs. grandmothers 
($\hat{\gamma}_{12}$ + $\hat{\gamma}_{13}$) \textcolor{white}{H} & -0.05 & 2.41 & .120 & -0.05 & 2.56 & .109\\
\ \ \ After-slope of grandfathers vs. grandmothers 
($\hat{\gamma}_{22}$ + $\hat{\gamma}_{23}$) \textcolor{white}{H} & -0.02 & 0.84 & .360 & -0.02 & 0.88 & .349\\
\ \ \ Shift of grandfathers vs. grandmothers 
($\hat{\gamma}_{22}$ + $\hat{\gamma}_{32}$ + 
$\hat{\gamma}_{23}$ + $\hat{\gamma}_{33}$) \textcolor{white}{H} & 0.02 & 0.30 & .584 & 0.02 & 0.31 & .577\\
\bottomrule
\addlinespace
\insertTableNotes
\end{longtable}

}

\end{lltable}








\begin{lltable}

\begin{TableNotes}[para]
\normalsize{\textit{Note.} Two models were computed (only HRS):
grandparents matched with parent controls and with nonparent controls.
CI = confidence interval. \(working=1\) indicates being employed in paid
work.}
\end{TableNotes}

\footnotesize{

\begin{longtable}{lrcrrrcrr}\noalign{\getlongtablewidth\global\LTcapwidth=\longtablewidth}
\caption{\label{tab:H1-neur-work-tab}Fixed Effects of Neuroticism Over the
Transition to Grandparenthood Moderated by Performing Paid Work.}\\
\toprule
& \multicolumn{4}{c}{Parent controls} & \multicolumn{4}{c}{Nonparent controls} \\
\cmidrule(r){2-5} \cmidrule(r){6-9}
Parameter & $\hat{\gamma}$ & 95\% CI & $t$ & $p$ & $\hat{\gamma}$ & 95\% CI & $t$ & $p$\\
\midrule
\endfirsthead
\caption*{\normalfont{Table \ref{tab:H1-neur-work-tab} continued}}\\
\toprule
& \multicolumn{4}{c}{Parent controls} & \multicolumn{4}{c}{Nonparent controls} \\
\cmidrule(r){2-5} \cmidrule(r){6-9}
Parameter & $\hat{\gamma}$ & 95\% CI & $t$ & $p$ & $\hat{\gamma}$ & 95\% CI & $t$ & $p$\\
\midrule
\endhead
Intercept, $\hat{\gamma}_{00}$ & 2.02 & [1.96, 2.07] & 72.21 & < .001 & 2.02 & [1.96, 2.08] & 63.73 & < .001\\
Propensity score, $\hat{\gamma}_{02}$ & 0.00 & [-0.08, 0.08] & 0.01 & .993 & 0.15 & [0.06, 0.23] & 3.46 & .001\\
Before-slope, $\hat{\gamma}_{20}$ & 0.00 & [-0.02, 0.03] & 0.18 & .860 & -0.01 & [-0.04, 0.02] & -0.84 & .400\\
After-slope, $\hat{\gamma}_{40}$ & -0.01 & [-0.02, 0.01] & -0.79 & .429 & -0.01 & [-0.02, 0.00] & -1.41 & .159\\
Shift, $\hat{\gamma}_{60}$ & 0.04 & [0.00, 0.08] & 1.91 & .056 & -0.03 & [-0.07, 0.01] & -1.32 & .188\\
Grandparent, $\hat{\gamma}_{01}$ & 0.13 & [0.02, 0.25] & 2.28 & .022 & 0.07 & [-0.04, 0.19] & 1.27 & .203\\
Working, $\hat{\gamma}_{10}$ & 0.08 & [0.03, 0.13] & 2.94 & .003 & 0.07 & [0.02, 0.12] & 2.63 & .009\\
Before-slope * Grandparent, $\hat{\gamma}_{21}$ & -0.07 & [-0.13, 0.00] & -2.04 & .042 & -0.06 & [-0.12, 0.01] & -1.73 & .084\\
After-slope * Grandparent, $\hat{\gamma}_{41}$ & -0.02 & [-0.05, 0.01] & -1.55 & .122 & -0.02 & [-0.05, 0.01] & -1.37 & .170\\
Shift * Grandparent, $\hat{\gamma}_{61}$ & -0.05 & [-0.15, 0.05] & -1.03 & .303 & 0.02 & [-0.07, 0.11] & 0.45 & .655\\
Before-slope * Working, $\hat{\gamma}_{30}$ & -0.02 & [-0.05, 0.01] & -1.43 & .153 & -0.02 & [-0.05, 0.01] & -1.54 & .123\\
After-slope * Working, $\hat{\gamma}_{50}$ & 0.00 & [-0.02, 0.02] & -0.23 & .820 & -0.01 & [-0.02, 0.01] & -0.73 & .463\\
Shift * Working, $\hat{\gamma}_{70}$ & -0.05 & [-0.11, 0.00] & -1.90 & .058 & 0.00 & [-0.05, 0.06] & 0.13 & .893\\
Grandparent * Working, $\hat{\gamma}_{11}$ & -0.25 & [-0.38, -0.13] & -4.08 & < .001 & -0.25 & [-0.37, -0.13] & -4.20 & < .001\\
Before-slope * Grandparent * Working, $\hat{\gamma}_{31}$ & 0.11 & [0.04, 0.19] & 2.95 & .003 & 0.12 & [0.04, 0.19] & 3.13 & .002\\
After-slope * Grandparent * Working, $\hat{\gamma}_{51}$ & 0.01 & [-0.03, 0.05] & 0.51 & .613 & 0.02 & [-0.02, 0.06] & 0.75 & .451\\
Shift * Grandparent * Working, $\hat{\gamma}_{71}$ & -0.02 & [-0.15, 0.10] & -0.33 & .740 & -0.08 & [-0.20, 0.04] & -1.23 & .217\\
\bottomrule
\addlinespace
\insertTableNotes
\end{longtable}

}

\end{lltable}








\begin{lltable}

\begin{TableNotes}[para]
\normalsize{\textit{Note.} The linear contrasts are based on the
models from Table \ref{tab:H1-neur-work-tab}. \(\hat{\gamma}_{c}\) =
combined fixed-effects estimate.}
\end{TableNotes}

\footnotesize{

\begin{longtable}{lrrrrrr}\noalign{\getlongtablewidth\global\LTcapwidth=\longtablewidth}
\caption{\label{tab:H1-neur-work-contrasts}Linear Contrasts for Neuroticism
(Moderated by Paid Work; only HRS).}\\
\toprule
& \multicolumn{3}{c}{Parent controls} & \multicolumn{3}{c}{Nonparent controls} \\
\cmidrule(r){2-4} \cmidrule(r){5-7}
Linear Contrast & $\hat{\gamma}_{c}$ & $\chi^2$ & $p$ & $\hat{\gamma}_{c}$ & $\chi^2$ & $p$\\
\midrule
\endfirsthead
\caption*{\normalfont{Table \ref{tab:H1-neur-work-contrasts} continued}}\\
\toprule
& \multicolumn{3}{c}{Parent controls} & \multicolumn{3}{c}{Nonparent controls} \\
\cmidrule(r){2-4} \cmidrule(r){5-7}
Linear Contrast & $\hat{\gamma}_{c}$ & $\chi^2$ & $p$ & $\hat{\gamma}_{c}$ & $\chi^2$ & $p$\\
\midrule
\endhead
Shift of not-working controls vs. 0 ($\hat{\gamma}_{40}$ + 
$\hat{\gamma}_{60}$) & 0.04 & 4.30 & .038 & -0.04 & 4.61 & .032\\
Shift of working controls vs. 0 ($\hat{\gamma}_{40}$ + 
$\hat{\gamma}_{60}$ + $\hat{\gamma}_{50}$ + 
$\hat{\gamma}_{70}$) & -0.02 & 2.18 & .140 & -0.04 & 11.64 & .001\\
Shift of not-working grandparents vs. 0 ($\hat{\gamma}_{40}$ + 
$\hat{\gamma}_{60}$ + $\hat{\gamma}_{41}$ + 
$\hat{\gamma}_{61}$) & -0.04 & 1.12 & .290 & -0.04 & 1.24 & .266\\
Shift of working grandparents vs. 0 ($\hat{\gamma}_{40}$ + 
$\hat{\gamma}_{60}$ + $\hat{\gamma}_{41}$ + 
$\hat{\gamma}_{61}$ + $\hat{\gamma}_{50}$ + 
$\hat{\gamma}_{70}$ + $\hat{\gamma}_{51}$ +
$\hat{\gamma}_{71}$) & -0.10 & 15.38 & < .001 & -0.10 & 16.09 & < .001\\
Shift of not-working controls vs. not-working grandparents 
($\hat{\gamma}_{41}$ + $\hat{\gamma}_{61}$) & -0.07 & 3.47 & .063 & 0.00 & 0.00 & .974\\
Before-slope of working controls vs. working grandparents 
($\hat{\gamma}_{21}$ + $\hat{\gamma}_{31}$) & 0.05 & 5.89 & .015 & 0.06 & 11.29 & .001\\
After-slope of working controls vs. working grandparents 
($\hat{\gamma}_{41}$ + $\hat{\gamma}_{51}$) & -0.01 & 0.72 & .396 & 0.00 & 0.11 & .743\\
Shift of working controls vs. working grandparents 
($\hat{\gamma}_{41}$ + $\hat{\gamma}_{61}$ + 
$\hat{\gamma}_{51}$ + $\hat{\gamma}_{71}$) & -0.08 & 8.11 & .004 & -0.06 & 4.48 & .034\\
Shift of not-working controls vs. working controls 
($\hat{\gamma}_{50}$ + $\hat{\gamma}_{70}$) & -0.06 & 6.36 & .012 & 0.00 & 0.02 & .895\\
Before-slope of not-working grandparents vs. working grandparents 
($\hat{\gamma}_{30}$ + $\hat{\gamma}_{31}$) & 0.09 & 6.73 & .009 & 0.09 & 7.45 & .006\\
After-slope of not-working grandparents vs. working grandparents 
($\hat{\gamma}_{50}$ + $\hat{\gamma}_{51}$) & 0.01 & 0.20 & .651 & 0.01 & 0.23 & .634\\
Shift of not-working grandparents vs. working grandparents 
($\hat{\gamma}_{50}$ + $\hat{\gamma}_{70}$ + 
$\hat{\gamma}_{51}$ + $\hat{\gamma}_{71}$) & -0.07 & 2.14 & .143 & -0.06 & 2.17 & .141\\
\bottomrule
\addlinespace
\insertTableNotes
\end{longtable}

}

\end{lltable}








\begin{lltable}

\begin{TableNotes}[para]
\normalsize{\textit{Note.} Two models were computed (only HRS):
grandparents matched with parent controls and with nonparent controls.
CI = confidence interval. \(caring=1\) indicates more than 100 hours of
grandchild care since the last assessment.}
\end{TableNotes}

\footnotesize{

\begin{longtable}{lrcrrrcrr}\noalign{\getlongtablewidth\global\LTcapwidth=\longtablewidth}
\caption{\label{tab:H1-neur-care-tab}Fixed Effects of Neuroticism Over the
Transition to Grandparenthood Moderated by Grandchild Care.}\\
\toprule
& \multicolumn{4}{c}{Parent controls} & \multicolumn{4}{c}{Nonparent controls} \\
\cmidrule(r){2-5} \cmidrule(r){6-9}
Parameter & $\hat{\gamma}$ & 95\% CI & $t$ & $p$ & $\hat{\gamma}$ & 95\% CI & $t$ & $p$\\
\midrule
\endfirsthead
\caption*{\normalfont{Table \ref{tab:H1-neur-care-tab} continued}}\\
\toprule
& \multicolumn{4}{c}{Parent controls} & \multicolumn{4}{c}{Nonparent controls} \\
\cmidrule(r){2-5} \cmidrule(r){6-9}
Parameter & $\hat{\gamma}$ & 95\% CI & $t$ & $p$ & $\hat{\gamma}$ & 95\% CI & $t$ & $p$\\
\midrule
\endhead
Intercept, $\hat{\gamma}_{00}$ & 2.04 & [1.99, 2.09] & 75.41 & < .001 & 1.97 & [1.91, 2.04] & 59.05 & < .001\\
Propensity score, $\hat{\gamma}_{02}$ & -0.02 & [-0.12, 0.07] & -0.45 & .652 & 0.14 & [0.03, 0.24] & 2.59 & .010\\
After-slope, $\hat{\gamma}_{20}$ & 0.00 & [-0.01, 0.01] & -0.02 & .982 & -0.02 & [-0.03, 0.00] & -2.67 & .008\\
Grandparent, $\hat{\gamma}_{01}$ & -0.10 & [-0.18, -0.02] & -2.45 & .014 & -0.11 & [-0.20, -0.02] & -2.43 & .015\\
Caring, $\hat{\gamma}_{10}$ & 0.01 & [-0.04, 0.05] & 0.33 & .740 & 0.00 & [-0.04, 0.04] & -0.09 & .930\\
After-slope * Grandparent, $\hat{\gamma}_{21}$ & 0.00 & [-0.03, 0.02] & -0.17 & .865 & 0.01 & [-0.01, 0.04] & 1.06 & .291\\
After-slope * Caring, $\hat{\gamma}_{30}$ & -0.01 & [-0.03, 0.01] & -1.01 & .311 & 0.01 & [-0.01, 0.03] & 0.68 & .494\\
Grandparent * Caring, $\hat{\gamma}_{11}$ & 0.09 & [-0.02, 0.20] & 1.57 & .117 & 0.09 & [-0.02, 0.21] & 1.67 & .095\\
After-slope * Grandparent * Caring, $\hat{\gamma}_{31}$ & -0.03 & [-0.07, 0.01] & -1.34 & .182 & -0.04 & [-0.09, 0.00] & -2.07 & .038\\
\bottomrule
\addlinespace
\insertTableNotes
\end{longtable}

}

\end{lltable}







\begin{lltable}

\begin{TableNotes}[para]
\normalsize{\textit{Note.} The linear contrasts are based on the
models from Table \ref{tab:H1-neur-care-tab}. \(\hat{\gamma}_{c}\) =
combined fixed-effects estimate.}
\end{TableNotes}

\footnotesize{

\begin{longtable}{lrrrrrr}\noalign{\getlongtablewidth\global\LTcapwidth=\longtablewidth}
\caption{\label{tab:H1-neur-care-contrasts}Linear Contrasts for Neuroticism
(Moderated by Grandchild Care; only HRS).}\\
\toprule
& \multicolumn{3}{c}{Parent controls} & \multicolumn{3}{c}{Nonparent controls} \\
\cmidrule(r){2-4} \cmidrule(r){5-7}
Linear Contrast & $\hat{\gamma}_{c}$ & $\chi^2$ & $p$ & $\hat{\gamma}_{c}$ & $\chi^2$ & $p$\\
\midrule
\endfirsthead
\caption*{\normalfont{Table \ref{tab:H1-neur-care-contrasts} continued}}\\
\toprule
& \multicolumn{3}{c}{Parent controls} & \multicolumn{3}{c}{Nonparent controls} \\
\cmidrule(r){2-4} \cmidrule(r){5-7}
Linear Contrast & $\hat{\gamma}_{c}$ & $\chi^2$ & $p$ & $\hat{\gamma}_{c}$ & $\chi^2$ & $p$\\
\midrule
\endhead
After-slope of caring controls vs. caring grandparents 
($\hat{\gamma}_{21}$ + $\hat{\gamma}_{31}$) & -0.03 & 3.78 & .052 & -0.03 & 3.60 & .058\\
After-slope of not-caring grandparents vs. caring grandparents 
($\hat{\gamma}_{30}$ + $\hat{\gamma}_{31}$) & -0.04 & 4.06 & .044 & -0.04 & 3.90 & .048\\
\bottomrule
\addlinespace
\insertTableNotes
\end{longtable}

}

\end{lltable}








\begin{lltable}

\begin{TableNotes}[para]
\normalsize{\textit{Note.} Two models were computed for each of the two
samples (LISS, HRS): grandparents matched with parent controls and with
nonparent controls. CI = confidence interval.}
\end{TableNotes}

\footnotesize{

\begin{longtable}{lrcrrrcrr}\noalign{\getlongtablewidth\global\LTcapwidth=\longtablewidth}
\caption{\label{tab:H1-open-tab}Fixed Effects of Openness Over the Transition to
Grandparenthood.}\\
\toprule
& \multicolumn{4}{c}{Parent controls} & \multicolumn{4}{c}{Nonparent controls} \\
\cmidrule(r){2-5} \cmidrule(r){6-9}
Parameter & $\hat{\gamma}$ & 95\% CI & $t$ & $p$ & $\hat{\gamma}$ & 95\% CI & $t$ & $p$\\
\midrule
\endfirsthead
\caption*{\normalfont{Table \ref{tab:H1-open-tab} continued}}\\
\toprule
& \multicolumn{4}{c}{Parent controls} & \multicolumn{4}{c}{Nonparent controls} \\
\cmidrule(r){2-5} \cmidrule(r){6-9}
Parameter & $\hat{\gamma}$ & 95\% CI & $t$ & $p$ & $\hat{\gamma}$ & 95\% CI & $t$ & $p$\\
\midrule
\endhead
LISS &  &  &  &  &  &  &  & \\
\ \ \ Intercept, $\hat{\gamma}_{00}$ \textcolor{white}{L} & 3.48 & [3.42, 3.53] & 118.77 & < .001 & 3.52 & [3.45, 3.59] & 104.18 & < .001\\
\ \ \ Propensity score, $\hat{\gamma}_{02}$ \textcolor{white}{L} & 0.00 & [-0.08, 0.07] & -0.07 & .944 & 0.03 & [-0.03, 0.09] & 1.02 & .309\\
\ \ \ Before-slope, $\hat{\gamma}_{10}$ \textcolor{white}{L} & 0.00 & [-0.01, 0.00] & -1.58 & .114 & 0.00 & [-0.01, 0.00] & -0.68 & .494\\
\ \ \ After-slope, $\hat{\gamma}_{20}$ \textcolor{white}{L} & -0.01 & [-0.01, 0.00] & -2.36 & .018 & 0.00 & [0.00, 0.01] & 1.95 & .051\\
\ \ \ Shift, $\hat{\gamma}_{30}$ \textcolor{white}{L} & 0.02 & [0.00, 0.05] & 1.88 & .061 & 0.00 & [-0.02, 0.02] & 0.00 & .998\\
\ \ \ Grandparent, $\hat{\gamma}_{01}$ \textcolor{white}{L} & 0.01 & [-0.08, 0.09] & 0.16 & .872 & -0.05 & [-0.14, 0.04] & -1.06 & .290\\
\ \ \ Before-slope * Grandparent, $\hat{\gamma}_{11}$ \textcolor{white}{L} & 0.01 & [0.00, 0.02] & 1.23 & .220 & 0.01 & [-0.01, 0.02] & 0.87 & .384\\
\ \ \ After-slope * Grandparent, $\hat{\gamma}_{21}$ \textcolor{white}{L} & 0.00 & [-0.01, 0.01] & 0.11 & .910 & -0.01 & [-0.02, 0.00] & -1.92 & .055\\
\ \ \ Shift * Grandparent, $\hat{\gamma}_{31}$ \textcolor{white}{L} & -0.03 & [-0.08, 0.03] & -1.05 & .296 & -0.01 & [-0.06, 0.04] & -0.21 & .832\\
HRS &  &  &  &  &  &  &  & \\
\ \ \ Intercept, $\hat{\gamma}_{00}$ \textcolor{white}{H} & 3.04 & [3.00, 3.08] & 149.49 & < .001 & 3.01 & [2.96, 3.06] & 129.29 & < .001\\
\ \ \ Propensity score, $\hat{\gamma}_{02}$ \textcolor{white}{H} & 0.03 & [-0.04, 0.09] & 0.82 & .411 & 0.00 & [-0.06, 0.07] & 0.13 & .895\\
\ \ \ Before-slope, $\hat{\gamma}_{10}$ \textcolor{white}{H} & -0.02 & [-0.03, -0.01] & -3.29 & .001 & 0.00 & [-0.01, 0.01] & -0.68 & .495\\
\ \ \ After-slope, $\hat{\gamma}_{20}$ \textcolor{white}{H} & -0.02 & [-0.03, -0.01] & -5.28 & < .001 & -0.02 & [-0.02, -0.01] & -4.83 & < .001\\
\ \ \ Shift, $\hat{\gamma}_{30}$ \textcolor{white}{H} & 0.06 & [0.03, 0.08] & 4.92 & < .001 & 0.03 & [0.01, 0.05] & 3.26 & .001\\
\ \ \ Grandparent, $\hat{\gamma}_{01}$ \textcolor{white}{H} & -0.02 & [-0.08, 0.05] & -0.55 & .582 & 0.02 & [-0.04, 0.09] & 0.75 & .451\\
\ \ \ Before-slope * Grandparent, $\hat{\gamma}_{11}$ \textcolor{white}{H} & 0.02 & [-0.01, 0.04] & 1.36 & .172 & 0.00 & [-0.02, 0.03] & 0.19 & .850\\
\ \ \ After-slope * Grandparent, $\hat{\gamma}_{21}$ \textcolor{white}{H} & 0.02 & [0.00, 0.03] & 2.01 & .044 & 0.01 & [0.00, 0.03] & 1.74 & .083\\
\ \ \ Shift * Grandparent, $\hat{\gamma}_{31}$ \textcolor{white}{H} & -0.07 & [-0.12, -0.02] & -2.86 & .004 & -0.05 & [-0.09, 0.00] & -2.16 & .031\\
\bottomrule
\addlinespace
\insertTableNotes
\end{longtable}

}

\end{lltable}









\begin{lltable}

\begin{TableNotes}[para]
\normalsize{\textit{Note.} The linear contrasts are needed in cases
where estimates of interest are represented by multiple fixed-effects
coefficients and are computed using the \emph{linearHypothesis} function
from the \emph{car} R package (Fox \& Weisberg, 2019) based on the
models from Table \ref{tab:H1-open-tab}. \(\hat{\gamma}_{c}\) = combined
fixed-effects estimate.}
\end{TableNotes}

\footnotesize{

\begin{longtable}{lrrrrrr}\noalign{\getlongtablewidth\global\LTcapwidth=\longtablewidth}
\caption{\label{tab:H1-open-contrasts}Linear Contrasts for Openness.}\\
\toprule
& \multicolumn{3}{c}{Parent controls} & \multicolumn{3}{c}{Nonparent controls} \\
\cmidrule(r){2-4} \cmidrule(r){5-7}
Linear Contrast & $\hat{\gamma}_{c}$ & $\chi^2$ & $p$ & $\hat{\gamma}_{c}$ & $\chi^2$ & $p$\\
\midrule
\endfirsthead
\caption*{\normalfont{Table \ref{tab:H1-open-contrasts} continued}}\\
\toprule
& \multicolumn{3}{c}{Parent controls} & \multicolumn{3}{c}{Nonparent controls} \\
\cmidrule(r){2-4} \cmidrule(r){5-7}
Linear Contrast & $\hat{\gamma}_{c}$ & $\chi^2$ & $p$ & $\hat{\gamma}_{c}$ & $\chi^2$ & $p$\\
\midrule
\endhead
LISS &  &  &  &  &  & \\
\ \ \ Shift of the controls vs. 0 ($\hat{\gamma}_{20}$ + 
$\hat{\gamma}_{30}$) \textcolor{white}{L} & 0.02 & 2.57 & .109 & 0.00 & 0.21 & .650\\
\ \ \ Shift of the grandparents vs. 0 ($\hat{\gamma}_{20}$ + 
$\hat{\gamma}_{30}$ + $\hat{\gamma}_{21}$ + 
$\hat{\gamma}_{31}$) \textcolor{white}{L} & -0.01 & 0.25 & .618 & -0.01 & 0.30 & .585\\
\ \ \ Shift of the controls vs. shift of the grandparents 
($\hat{\gamma}_{21}$ + $\hat{\gamma}_{31}$) \textcolor{white}{L} & -0.03 & 1.38 & .241 & -0.02 & 0.48 & .489\\
\ \ \ Before-slope of the grandparents vs. 0 ($\hat{\gamma}_{10}$ + 
$\hat{\gamma}_{11}$) \textcolor{white}{L} & 0.00 & 0.34 & .561 & 0.00 & 0.40 & .528\\
\ \ \ After-slope of the grandparents vs. 0 ($\hat{\gamma}_{20}$ + 
$\hat{\gamma}_{21}$) \textcolor{white}{L} & -0.01 & 1.15 & .284 & -0.01 & 1.36 & .244\\
HRS &  &  &  &  &  & \\
\ \ \ Shift of the controls vs. 0 ($\hat{\gamma}_{20}$ + 
$\hat{\gamma}_{30}$) \textcolor{white}{H} & 0.03 & 16.48 & < .001 & 0.02 & 4.36 & .037\\
\ \ \ Shift of the grandparents vs. 0 ($\hat{\gamma}_{20}$ + 
$\hat{\gamma}_{30}$ + $\hat{\gamma}_{21}$ + 
$\hat{\gamma}_{31}$) \textcolor{white}{H} & -0.02 & 1.31 & .253 & -0.02 & 1.57 & .210\\
\ \ \ Shift of the controls vs. shift of the grandparents 
($\hat{\gamma}_{21}$ + $\hat{\gamma}_{31}$) \textcolor{white}{H} & -0.05 & 8.14 & .004 & -0.04 & 4.25 & .039\\
\ \ \ Before-slope of the grandparents vs. 0 ($\hat{\gamma}_{10}$ + 
$\hat{\gamma}_{11}$) \textcolor{white}{H} & 0.00 & 0.00 & .946 & 0.00 & 0.01 & .908\\
\ \ \ After-slope of the grandparents vs. 0 ($\hat{\gamma}_{20}$ + 
$\hat{\gamma}_{21}$) \textcolor{white}{H} & 0.00 & 0.14 & .709 & 0.00 & 0.20 & .658\\
\bottomrule
\addlinespace
\insertTableNotes
\end{longtable}

}

\end{lltable}







\begin{lltable}

\begin{TableNotes}[para]
\normalsize{\textit{Note.} Two models were computed for each of the
two samples (LISS, HRS): grandparents matched with parent controls and
with nonparent controls. CI = confidence interval.}
\end{TableNotes}

\footnotesize{

\begin{longtable}{lrcrrrcrr}\noalign{\getlongtablewidth\global\LTcapwidth=\longtablewidth}
\caption{\label{tab:H1-open-gender-tab}Fixed Effects of Openness Over the
Transition to Grandparenthood Moderated by Gender.}\\
\toprule
& \multicolumn{4}{c}{Parent controls} & \multicolumn{4}{c}{Nonparent controls} \\
\cmidrule(r){2-5} \cmidrule(r){6-9}
Parameter & $\hat{\gamma}$ & 95\% CI & $t$ & $p$ & $\hat{\gamma}$ & 95\% CI & $t$ & $p$\\
\midrule
\endfirsthead
\caption*{\normalfont{Table \ref{tab:H1-open-gender-tab} continued}}\\
\toprule
& \multicolumn{4}{c}{Parent controls} & \multicolumn{4}{c}{Nonparent controls} \\
\cmidrule(r){2-5} \cmidrule(r){6-9}
Parameter & $\hat{\gamma}$ & 95\% CI & $t$ & $p$ & $\hat{\gamma}$ & 95\% CI & $t$ & $p$\\
\midrule
\endhead
LISS &  &  &  &  &  &  &  & \\
\ \ \ Intercept, $\hat{\gamma}_{00}$ \textcolor{white}{L} & 3.47 & [3.39, 3.55] & 81.39 & < .001 & 3.54 & [3.45, 3.64] & 73.02 & < .001\\
\ \ \ Propensity score, $\hat{\gamma}_{04}$ \textcolor{white}{L} & 0.00 & [-0.08, 0.07] & -0.04 & .970 & 0.03 & [-0.03, 0.09] & 0.94 & .347\\
\ \ \ Before-slope, $\hat{\gamma}_{10}$ \textcolor{white}{L} & 0.00 & [-0.01, 0.01] & 0.17 & .864 & 0.01 & [0.00, 0.02] & 2.39 & .017\\
\ \ \ After-slope, $\hat{\gamma}_{20}$ \textcolor{white}{L} & 0.00 & [-0.01, 0.00] & -1.05 & .292 & 0.01 & [0.00, 0.01] & 1.53 & .126\\
\ \ \ Shift, $\hat{\gamma}_{30}$ \textcolor{white}{L} & -0.02 & [-0.05, 0.02] & -0.93 & .353 & -0.01 & [-0.04, 0.02] & -0.64 & .523\\
\ \ \ Grandparent, $\hat{\gamma}_{01}$ \textcolor{white}{L} & 0.11 & [-0.01, 0.24] & 1.78 & .076 & 0.03 & [-0.10, 0.16] & 0.44 & .661\\
\ \ \ Female, $\hat{\gamma}_{02}$ \textcolor{white}{L} & 0.01 & [-0.10, 0.12] & 0.16 & .871 & -0.05 & [-0.17, 0.08] & -0.69 & .488\\
\ \ \ Before-slope * Grandparent, $\hat{\gamma}_{11}$ \textcolor{white}{L} & 0.00 & [-0.02, 0.01] & -0.39 & .694 & -0.01 & [-0.03, 0.00] & -1.42 & .156\\
\ \ \ After-slope * Grandparent, $\hat{\gamma}_{21}$ \textcolor{white}{L} & -0.01 & [-0.02, 0.01] & -0.88 & .380 & -0.02 & [-0.03, 0.00] & -2.16 & .031\\
\ \ \ Shift * Grandparent, $\hat{\gamma}_{31}$ \textcolor{white}{L} & 0.03 & [-0.05, 0.12] & 0.84 & .400 & 0.03 & [-0.05, 0.10] & 0.75 & .452\\
\ \ \ Before-slope * Female, $\hat{\gamma}_{12}$ \textcolor{white}{L} & -0.01 & [-0.02, 0.00] & -1.64 & .102 & -0.02 & [-0.03, -0.01] & -3.89 & < .001\\
\ \ \ After-slope * Female, $\hat{\gamma}_{22}$ \textcolor{white}{L} & 0.00 & [-0.01, 0.01] & -0.79 & .431 & 0.00 & [-0.01, 0.01] & -0.24 & .812\\
\ \ \ Shift * Female, $\hat{\gamma}_{32}$ \textcolor{white}{L} & 0.08 & [0.03, 0.13] & 2.98 & .003 & 0.02 & [-0.03, 0.06] & 0.84 & .402\\
\ \ \ Grandparent * Female, $\hat{\gamma}_{03}$ \textcolor{white}{L} & -0.20 & [-0.37, -0.03] & -2.31 & .021 & -0.15 & [-0.33, 0.03] & -1.59 & .113\\
\ \ \ Before-slope * Grandparent * Female, $\hat{\gamma}_{13}$ \textcolor{white}{L} & 0.02 & [0.00, 0.05] & 1.70 & .090 & 0.03 & [0.01, 0.06] & 2.80 & .005\\
\ \ \ After-slope * Grandparent * Female, $\hat{\gamma}_{23}$ \textcolor{white}{L} & 0.01 & [-0.01, 0.04] & 1.29 & .197 & 0.01 & [-0.01, 0.03] & 1.14 & .255\\
\ \ \ Shift * Grandparent * Female, $\hat{\gamma}_{33}$ \textcolor{white}{L} & -0.12 & [-0.23, -0.01] & -2.11 & .035 & -0.06 & [-0.16, 0.04] & -1.21 & .225\\
HRS &  &  &  &  &  &  &  & \\
\ \ \ Intercept, $\hat{\gamma}_{00}$ \textcolor{white}{H} & 3.06 & [3.00, 3.12] & 108.70 & < .001 & 3.03 & [2.97, 3.09] & 97.90 & < .001\\
\ \ \ Propensity score, $\hat{\gamma}_{04}$ \textcolor{white}{H} & 0.03 & [-0.04, 0.09] & 0.86 & .391 & 0.00 & [-0.06, 0.07] & 0.03 & .976\\
\ \ \ Before-slope, $\hat{\gamma}_{10}$ \textcolor{white}{H} & -0.02 & [-0.04, 0.00] & -2.44 & .015 & -0.01 & [-0.03, 0.00] & -1.90 & .058\\
\ \ \ After-slope, $\hat{\gamma}_{20}$ \textcolor{white}{H} & -0.03 & [-0.04, -0.02] & -5.75 & < .001 & -0.01 & [-0.02, 0.00] & -2.04 & .042\\
\ \ \ Shift, $\hat{\gamma}_{30}$ \textcolor{white}{H} & 0.11 & [0.07, 0.14] & 6.34 & < .001 & 0.00 & [-0.03, 0.03] & -0.29 & .772\\
\ \ \ Grandparent, $\hat{\gamma}_{01}$ \textcolor{white}{H} & -0.03 & [-0.12, 0.06] & -0.62 & .535 & 0.01 & [-0.08, 0.10] & 0.24 & .813\\
\ \ \ Female, $\hat{\gamma}_{02}$ \textcolor{white}{H} & -0.03 & [-0.09, 0.04] & -0.80 & .423 & -0.04 & [-0.11, 0.04] & -0.98 & .328\\
\ \ \ Before-slope * Grandparent, $\hat{\gamma}_{11}$ \textcolor{white}{H} & 0.01 & [-0.03, 0.05] & 0.41 & .685 & 0.00 & [-0.03, 0.04] & 0.05 & .960\\
\ \ \ After-slope * Grandparent, $\hat{\gamma}_{21}$ \textcolor{white}{H} & 0.03 & [0.01, 0.06] & 2.66 & .008 & 0.01 & [-0.01, 0.03] & 0.94 & .346\\
\ \ \ Shift * Grandparent, $\hat{\gamma}_{31}$ \textcolor{white}{H} & -0.15 & [-0.22, -0.07] & -3.93 & < .001 & -0.03 & [-0.10, 0.03] & -1.00 & .316\\
\ \ \ Before-slope * Female, $\hat{\gamma}_{12}$ \textcolor{white}{H} & 0.00 & [-0.02, 0.03] & 0.28 & .781 & 0.02 & [0.00, 0.04] & 1.97 & .049\\
\ \ \ After-slope * Female, $\hat{\gamma}_{22}$ \textcolor{white}{H} & 0.02 & [0.01, 0.04] & 3.05 & .002 & -0.01 & [-0.02, 0.00] & -1.47 & .141\\
\ \ \ Shift * Female, $\hat{\gamma}_{32}$ \textcolor{white}{H} & -0.09 & [-0.14, -0.05] & -4.11 & < .001 & 0.06 & [0.03, 0.10] & 3.21 & .001\\
\ \ \ Grandparent * Female, $\hat{\gamma}_{03}$ \textcolor{white}{H} & 0.02 & [-0.10, 0.13] & 0.30 & .763 & 0.03 & [-0.09, 0.14] & 0.45 & .652\\
\ \ \ Before-slope * Grandparent * Female, $\hat{\gamma}_{13}$ \textcolor{white}{H} & 0.02 & [-0.04, 0.07] & 0.67 & .504 & 0.00 & [-0.05, 0.05] & 0.08 & .939\\
\ \ \ After-slope * Grandparent * Female, $\hat{\gamma}_{23}$ \textcolor{white}{H} & -0.03 & [-0.06, 0.00] & -1.75 & .079 & 0.00 & [-0.03, 0.03] & 0.27 & .790\\
\ \ \ Shift * Grandparent * Female, $\hat{\gamma}_{33}$ \textcolor{white}{H} & 0.14 & [0.04, 0.23] & 2.71 & .007 & -0.02 & [-0.11, 0.06] & -0.52 & .603\\
\bottomrule
\addlinespace
\insertTableNotes
\end{longtable}

}

\end{lltable}








\begin{lltable}

\begin{TableNotes}[para]
\normalsize{\textit{Note.} The linear contrasts are based on
the models from Table \ref{tab:H1-open-gender-tab}. \(\hat{\gamma}_{c}\)
= combined fixed-effects estimate.}
\end{TableNotes}

\footnotesize{

\begin{longtable}{lrrrrrr}\noalign{\getlongtablewidth\global\LTcapwidth=\longtablewidth}
\caption{\label{tab:H1-open-gender-contrasts}Linear Contrasts for Openness
(Moderated by Gender).}\\
\toprule
& \multicolumn{3}{c}{Parent controls} & \multicolumn{3}{c}{Nonparent controls} \\
\cmidrule(r){2-4} \cmidrule(r){5-7}
Linear Contrast & $\hat{\gamma}_{c}$ & $\chi^2$ & $p$ & $\hat{\gamma}_{c}$ & $\chi^2$ & $p$\\
\midrule
\endfirsthead
\caption*{\normalfont{Table \ref{tab:H1-open-gender-contrasts} continued}}\\
\toprule
& \multicolumn{3}{c}{Parent controls} & \multicolumn{3}{c}{Nonparent controls} \\
\cmidrule(r){2-4} \cmidrule(r){5-7}
Linear Contrast & $\hat{\gamma}_{c}$ & $\chi^2$ & $p$ & $\hat{\gamma}_{c}$ & $\chi^2$ & $p$\\
\midrule
\endhead
LISS &  &  &  &  &  & \\
\ \ \ Shift of male controls vs. 0 ($\hat{\gamma}_{20}$ + 
$\hat{\gamma}_{30}$) \textcolor{white}{L} & -0.02 & 1.70 & .192 & -0.01 & 0.14 & .706\\
\ \ \ Shift of female controls vs. 0 ($\hat{\gamma}_{20}$ + 
$\hat{\gamma}_{30}$ + $\hat{\gamma}_{22}$ + 
$\hat{\gamma}_{32}$) \textcolor{white}{L} & 0.05 & 11.29 & .001 & 0.01 & 0.84 & .359\\
\ \ \ Shift of grandfathers vs. 0 ($\hat{\gamma}_{20}$ + 
$\hat{\gamma}_{30}$ + $\hat{\gamma}_{21}$ + 
$\hat{\gamma}_{31}$) \textcolor{white}{L} & 0.01 & 0.03 & .853 & 0.01 & 0.04 & .833\\
\ \ \ Shift of grandmothers vs. 0 ($\hat{\gamma}_{20}$ + 
$\hat{\gamma}_{30}$ + $\hat{\gamma}_{21}$ + 
$\hat{\gamma}_{31}$ + $\hat{\gamma}_{22}$ + 
$\hat{\gamma}_{32}$ + $\hat{\gamma}_{23}$ +
$\hat{\gamma}_{33}$) \textcolor{white}{L} & -0.03 & 0.78 & .378 & -0.03 & 0.93 & .335\\
\ \ \ Shift of male controls vs. grandfathers 
($\hat{\gamma}_{21}$ + $\hat{\gamma}_{31}$) \textcolor{white}{L} & 0.03 & 0.57 & .450 & 0.01 & 0.13 & .721\\
\ \ \ Before-slope of female controls vs. grandmothers 
($\hat{\gamma}_{11}$ + $\hat{\gamma}_{13}$) \textcolor{white}{L} & 0.02 & 4.38 & .036 & 0.02 & 6.74 & .009\\
\ \ \ After-slope of female controls vs. grandmothers 
($\hat{\gamma}_{21}$ + $\hat{\gamma}_{23}$) \textcolor{white}{L} & 0.01 & 0.91 & .341 & 0.00 & 0.42 & .517\\
\ \ \ Shift of female controls vs. grandmothers 
($\hat{\gamma}_{21}$ + $\hat{\gamma}_{31}$ + 
$\hat{\gamma}_{23}$ + $\hat{\gamma}_{33}$) \textcolor{white}{L} & -0.08 & 5.37 & .020 & -0.04 & 1.63 & .202\\
\ \ \ Shift of male vs. female controls 
($\hat{\gamma}_{22}$ + $\hat{\gamma}_{32}$) \textcolor{white}{L} & 0.07 & 10.45 & .001 & 0.02 & 0.82 & .366\\
\ \ \ Before-slope of grandfathers vs. grandmothers 
($\hat{\gamma}_{12}$ + $\hat{\gamma}_{13}$) \textcolor{white}{L} & 0.01 & 1.16 & .282 & 0.01 & 1.41 & .236\\
\ \ \ After-slope of grandfathers vs. grandmothers 
($\hat{\gamma}_{22}$ + $\hat{\gamma}_{23}$) \textcolor{white}{L} & 0.01 & 1.10 & .294 & 0.01 & 1.33 & .249\\
\ \ \ Shift of grandfathers vs. grandmothers 
($\hat{\gamma}_{22}$ + $\hat{\gamma}_{32}$ + 
$\hat{\gamma}_{23}$ + $\hat{\gamma}_{33}$) \textcolor{white}{L} & -0.03 & 0.53 & .466 & -0.03 & 0.65 & .421\\
HRS &  &  &  &  &  & \\
\ \ \ Shift of male controls vs. 0 ($\hat{\gamma}_{20}$ + 
$\hat{\gamma}_{30}$) \textcolor{white}{H} & 0.07 & 32.25 & < .001 & -0.02 & 1.67 & .197\\
\ \ \ Shift of female controls vs. 0 ($\hat{\gamma}_{20}$ + 
$\hat{\gamma}_{30}$ + $\hat{\gamma}_{22}$ + 
$\hat{\gamma}_{32}$) \textcolor{white}{H} & 0.00 & 0.15 & .698 & 0.04 & 15.02 & < .001\\
\ \ \ Shift of grandfathers vs. 0 ($\hat{\gamma}_{20}$ + 
$\hat{\gamma}_{30}$ + $\hat{\gamma}_{21}$ + 
$\hat{\gamma}_{31}$) \textcolor{white}{H} & -0.04 & 2.39 & .122 & -0.04 & 2.82 & .093\\
\ \ \ Shift of grandmothers vs. 0 ($\hat{\gamma}_{20}$ + 
$\hat{\gamma}_{30}$ + $\hat{\gamma}_{21}$ + 
$\hat{\gamma}_{31}$ + $\hat{\gamma}_{22}$ + 
$\hat{\gamma}_{32}$ + $\hat{\gamma}_{23}$ +
$\hat{\gamma}_{33}$) \textcolor{white}{H} & 0.00 & 0.01 & .919 & 0.00 & 0.02 & .899\\
\ \ \ Shift of male controls vs. grandfathers 
($\hat{\gamma}_{21}$ + $\hat{\gamma}_{31}$) \textcolor{white}{H} & -0.11 & 15.71 & < .001 & -0.02 & 0.80 & .372\\
\ \ \ Before-slope of female controls vs. grandmothers 
($\hat{\gamma}_{11}$ + $\hat{\gamma}_{13}$) \textcolor{white}{H} & 0.03 & 2.17 & .141 & 0.00 & 0.03 & .863\\
\ \ \ After-slope of female controls vs. grandmothers 
($\hat{\gamma}_{21}$ + $\hat{\gamma}_{23}$) \textcolor{white}{H} & 0.00 & 0.10 & .747 & 0.01 & 2.08 & .150\\
\ \ \ Shift of female controls vs. grandmothers 
($\hat{\gamma}_{21}$ + $\hat{\gamma}_{31}$ + 
$\hat{\gamma}_{23}$ + $\hat{\gamma}_{33}$) \textcolor{white}{H} & -0.01 & 0.07 & .791 & -0.04 & 3.38 & .066\\
\ \ \ Shift of male vs. female controls 
($\hat{\gamma}_{22}$ + $\hat{\gamma}_{32}$) \textcolor{white}{H} & -0.07 & 15.92 & < .001 & 0.05 & 12.31 & < .001\\
\ \ \ Before-slope of grandfathers vs. grandmothers 
($\hat{\gamma}_{12}$ + $\hat{\gamma}_{13}$) \textcolor{white}{H} & 0.02 & 0.76 & .382 & 0.02 & 1.04 & .307\\
\ \ \ After-slope of grandfathers vs. grandmothers 
($\hat{\gamma}_{22}$ + $\hat{\gamma}_{23}$) \textcolor{white}{H} & -0.01 & 0.19 & .660 & -0.01 & 0.19 & .663\\
\ \ \ Shift of grandfathers vs. grandmothers 
($\hat{\gamma}_{22}$ + $\hat{\gamma}_{32}$ + 
$\hat{\gamma}_{23}$ + $\hat{\gamma}_{33}$) \textcolor{white}{H} & 0.04 & 1.17 & .280 & 0.04 & 1.35 & .245\\
\bottomrule
\addlinespace
\insertTableNotes
\end{longtable}

}

\end{lltable}








\begin{lltable}

\begin{TableNotes}[para]
\normalsize{\textit{Note.} Two models were computed (only HRS):
grandparents matched with parent controls and with nonparent controls.
CI = confidence interval. \(working=1\) indicates being employed in paid
work.}
\end{TableNotes}

\footnotesize{

\begin{longtable}{lrcrrrcrr}\noalign{\getlongtablewidth\global\LTcapwidth=\longtablewidth}
\caption{\label{tab:H1-open-work-tab}Fixed Effects of Openness Over the Transition
to Grandparenthood Moderated by Performing Paid Work.}\\
\toprule
& \multicolumn{4}{c}{Parent controls} & \multicolumn{4}{c}{Nonparent controls} \\
\cmidrule(r){2-5} \cmidrule(r){6-9}
Parameter & $\hat{\gamma}$ & 95\% CI & $t$ & $p$ & $\hat{\gamma}$ & 95\% CI & $t$ & $p$\\
\midrule
\endfirsthead
\caption*{\normalfont{Table \ref{tab:H1-open-work-tab} continued}}\\
\toprule
& \multicolumn{4}{c}{Parent controls} & \multicolumn{4}{c}{Nonparent controls} \\
\cmidrule(r){2-5} \cmidrule(r){6-9}
Parameter & $\hat{\gamma}$ & 95\% CI & $t$ & $p$ & $\hat{\gamma}$ & 95\% CI & $t$ & $p$\\
\midrule
\endhead
Intercept, $\hat{\gamma}_{00}$ & 3.02 & [2.97, 3.06] & 121.17 & < .001 & 3.03 & [2.97, 3.08] & 111.81 & < .001\\
Propensity score, $\hat{\gamma}_{02}$ & 0.01 & [-0.06, 0.07] & 0.25 & .800 & -0.01 & [-0.08, 0.05] & -0.39 & .693\\
Before-slope, $\hat{\gamma}_{20}$ & -0.01 & [-0.03, 0.01] & -1.03 & .303 & -0.01 & [-0.03, 0.01] & -0.96 & .339\\
After-slope, $\hat{\gamma}_{40}$ & -0.03 & [-0.04, -0.02] & -5.25 & < .001 & -0.02 & [-0.03, -0.01] & -4.51 & < .001\\
Shift, $\hat{\gamma}_{60}$ & 0.06 & [0.02, 0.09] & 3.20 & .001 & 0.04 & [0.00, 0.07] & 2.21 & .027\\
Grandparent, $\hat{\gamma}_{01}$ & -0.05 & [-0.15, 0.05] & -1.04 & .299 & -0.06 & [-0.15, 0.04] & -1.17 & .243\\
Working, $\hat{\gamma}_{10}$ & 0.05 & [0.01, 0.09] & 2.26 & .024 & -0.02 & [-0.06, 0.02] & -0.88 & .378\\
Before-slope * Grandparent, $\hat{\gamma}_{21}$ & 0.04 & [-0.02, 0.09] & 1.30 & .194 & 0.03 & [-0.01, 0.08] & 1.38 & .167\\
After-slope * Grandparent, $\hat{\gamma}_{41}$ & 0.05 & [0.02, 0.07] & 3.86 & < .001 & 0.04 & [0.02, 0.06] & 3.73 & < .001\\
Shift * Grandparent, $\hat{\gamma}_{61}$ & -0.14 & [-0.22, -0.06] & -3.37 & .001 & -0.12 & [-0.19, -0.04] & -3.14 & .002\\
Before-slope * Working, $\hat{\gamma}_{30}$ & -0.01 & [-0.04, 0.01] & -0.86 & .389 & 0.01 & [-0.01, 0.03] & 0.82 & .414\\
After-slope * Working, $\hat{\gamma}_{50}$ & 0.02 & [0.01, 0.04] & 2.94 & .003 & 0.02 & [0.00, 0.03] & 2.15 & .031\\
Shift * Working, $\hat{\gamma}_{70}$ & -0.01 & [-0.06, 0.04] & -0.44 & .661 & -0.01 & [-0.05, 0.03] & -0.52 & .606\\
Grandparent * Working, $\hat{\gamma}_{11}$ & 0.04 & [-0.06, 0.14] & 0.79 & .429 & 0.11 & [0.02, 0.20] & 2.33 & .020\\
Before-slope * Grandparent * Working, $\hat{\gamma}_{31}$ & -0.02 & [-0.08, 0.04] & -0.56 & .578 & -0.04 & [-0.10, 0.02] & -1.34 & .179\\
After-slope * Grandparent * Working, $\hat{\gamma}_{51}$ & -0.06 & [-0.10, -0.03] & -3.46 & .001 & -0.05 & [-0.08, -0.02] & -3.35 & .001\\
Shift * Grandparent * Working, $\hat{\gamma}_{71}$ & 0.13 & [0.02, 0.23] & 2.37 & .018 & 0.12 & [0.03, 0.22] & 2.62 & .009\\
\bottomrule
\addlinespace
\insertTableNotes
\end{longtable}

}

\end{lltable}








\begin{lltable}

\begin{TableNotes}[para]
\normalsize{\textit{Note.} The linear contrasts are based on the
models from Table \ref{tab:H1-open-work-tab}. \(\hat{\gamma}_{c}\) =
combined fixed-effects estimate.}
\end{TableNotes}

\footnotesize{

\begin{longtable}{lrrrrrr}\noalign{\getlongtablewidth\global\LTcapwidth=\longtablewidth}
\caption{\label{tab:H1-open-work-contrasts}Linear Contrasts for Openness
(Moderated by Paid Work; only HRS).}\\
\toprule
& \multicolumn{3}{c}{Parent controls} & \multicolumn{3}{c}{Nonparent controls} \\
\cmidrule(r){2-4} \cmidrule(r){5-7}
Linear Contrast & $\hat{\gamma}_{c}$ & $\chi^2$ & $p$ & $\hat{\gamma}_{c}$ & $\chi^2$ & $p$\\
\midrule
\endfirsthead
\caption*{\normalfont{Table \ref{tab:H1-open-work-contrasts} continued}}\\
\toprule
& \multicolumn{3}{c}{Parent controls} & \multicolumn{3}{c}{Nonparent controls} \\
\cmidrule(r){2-4} \cmidrule(r){5-7}
Linear Contrast & $\hat{\gamma}_{c}$ & $\chi^2$ & $p$ & $\hat{\gamma}_{c}$ & $\chi^2$ & $p$\\
\midrule
\endhead
Shift of not-working controls vs. 0 ($\hat{\gamma}_{40}$ + 
$\hat{\gamma}_{60}$) & 0.03 & 3.80 & .051 & 0.01 & 1.06 & .303\\
Shift of working controls vs. 0 ($\hat{\gamma}_{40}$ + 
$\hat{\gamma}_{60}$ + $\hat{\gamma}_{50}$ + 
$\hat{\gamma}_{70}$) & 0.04 & 13.84 & < .001 & 0.02 & 3.72 & .054\\
Shift of not-working grandparents vs. 0 ($\hat{\gamma}_{40}$ + 
$\hat{\gamma}_{60}$ + $\hat{\gamma}_{41}$ + 
$\hat{\gamma}_{61}$) & -0.06 & 4.22 & .040 & -0.06 & 5.04 & .025\\
Shift of working grandparents vs. 0 ($\hat{\gamma}_{40}$ + 
$\hat{\gamma}_{60}$ + $\hat{\gamma}_{41}$ + 
$\hat{\gamma}_{61}$ + $\hat{\gamma}_{50}$ + 
$\hat{\gamma}_{70}$ + $\hat{\gamma}_{51}$ +
$\hat{\gamma}_{71}$) & 0.02 & 0.61 & .433 & 0.02 & 0.75 & .385\\
Shift of not-working controls vs. not-working grandparents 
($\hat{\gamma}_{41}$ + $\hat{\gamma}_{61}$) & -0.09 & 7.30 & .007 & -0.07 & 6.07 & .014\\
Before-slope of working controls vs. working grandparents 
($\hat{\gamma}_{21}$ + $\hat{\gamma}_{31}$) & 0.02 & 1.23 & .267 & 0.00 & 0.10 & .751\\
After-slope of working controls vs. working grandparents 
($\hat{\gamma}_{41}$ + $\hat{\gamma}_{51}$) & -0.01 & 1.08 & .299 & -0.01 & 1.00 & .317\\
Shift of working controls vs. working grandparents 
($\hat{\gamma}_{41}$ + $\hat{\gamma}_{61}$ + 
$\hat{\gamma}_{51}$ + $\hat{\gamma}_{71}$) & -0.02 & 0.93 & .336 & 0.00 & 0.00 & .951\\
Shift of not-working controls vs. working controls 
($\hat{\gamma}_{50}$ + $\hat{\gamma}_{70}$) & 0.01 & 0.48 & .487 & 0.00 & 0.05 & .818\\
Before-slope of not-working grandparents vs. working grandparents 
($\hat{\gamma}_{30}$ + $\hat{\gamma}_{31}$) & -0.03 & 0.96 & .327 & -0.03 & 1.22 & .270\\
After-slope of not-working grandparents vs. working grandparents 
($\hat{\gamma}_{50}$ + $\hat{\gamma}_{51}$) & -0.04 & 5.78 & .016 & -0.04 & 7.17 & .007\\
Shift of not-working grandparents vs. working grandparents 
($\hat{\gamma}_{50}$ + $\hat{\gamma}_{70}$ + 
$\hat{\gamma}_{51}$ + $\hat{\gamma}_{71}$) & 0.08 & 4.30 & .038 & 0.08 & 5.16 & .023\\
\bottomrule
\addlinespace
\insertTableNotes
\end{longtable}

}

\end{lltable}








\begin{lltable}

\begin{TableNotes}[para]
\normalsize{\textit{Note.} Two models were computed (only HRS):
grandparents matched with parent controls and with nonparent controls.
CI = confidence interval. \(caring=1\) indicates more than 100 hours of
grandchild care since the last assessment.}
\end{TableNotes}

\footnotesize{

\begin{longtable}{lrcrrrcrr}\noalign{\getlongtablewidth\global\LTcapwidth=\longtablewidth}
\caption{\label{tab:H1-open-care-tab}Fixed Effects of Openness Over the Transition
to Grandparenthood Moderated by Grandchild Care.}\\
\toprule
& \multicolumn{4}{c}{Parent controls} & \multicolumn{4}{c}{Nonparent controls} \\
\cmidrule(r){2-5} \cmidrule(r){6-9}
Parameter & $\hat{\gamma}$ & 95\% CI & $t$ & $p$ & $\hat{\gamma}$ & 95\% CI & $t$ & $p$\\
\midrule
\endfirsthead
\caption*{\normalfont{Table \ref{tab:H1-open-care-tab} continued}}\\
\toprule
& \multicolumn{4}{c}{Parent controls} & \multicolumn{4}{c}{Nonparent controls} \\
\cmidrule(r){2-5} \cmidrule(r){6-9}
Parameter & $\hat{\gamma}$ & 95\% CI & $t$ & $p$ & $\hat{\gamma}$ & 95\% CI & $t$ & $p$\\
\midrule
\endhead
Intercept, $\hat{\gamma}_{00}$ & 3.06 & [3.01, 3.10] & 125.52 & < .001 & 3.00 & [2.95, 3.06] & 103.68 & < .001\\
Propensity score, $\hat{\gamma}_{02}$ & 0.08 & [-0.01, 0.16] & 1.81 & .070 & 0.22 & [0.13, 0.30] & 5.00 & < .001\\
After-slope, $\hat{\gamma}_{20}$ & -0.04 & [-0.05, -0.03] & -6.73 & < .001 & -0.02 & [-0.03, -0.01] & -4.90 & < .001\\
Grandparent, $\hat{\gamma}_{01}$ & -0.06 & [-0.14, 0.01] & -1.74 & .082 & -0.08 & [-0.16, -0.01] & -2.21 & .027\\
Caring, $\hat{\gamma}_{10}$ & -0.02 & [-0.06, 0.02] & -1.09 & .275 & 0.01 & [-0.02, 0.04] & 0.67 & .503\\
After-slope * Grandparent, $\hat{\gamma}_{21}$ & 0.02 & [0.00, 0.05] & 2.10 & .036 & 0.01 & [-0.01, 0.03] & 0.88 & .377\\
After-slope * Caring, $\hat{\gamma}_{30}$ & 0.01 & [0.00, 0.03] & 1.52 & .129 & 0.00 & [-0.02, 0.01] & -0.24 & .807\\
Grandparent * Caring, $\hat{\gamma}_{11}$ & 0.00 & [-0.10, 0.10] & 0.02 & .985 & -0.04 & [-0.12, 0.05] & -0.79 & .432\\
After-slope * Grandparent * Caring, $\hat{\gamma}_{31}$ & 0.01 & [-0.02, 0.05] & 0.74 & .457 & 0.03 & [0.00, 0.06] & 1.73 & .084\\
\bottomrule
\addlinespace
\insertTableNotes
\end{longtable}

}

\end{lltable}







\begin{lltable}

\begin{TableNotes}[para]
\normalsize{\textit{Note.} The linear contrasts are based on the
models from Table \ref{tab:H1-open-care-tab}. \(\hat{\gamma}_{c}\) =
combined fixed-effects estimate.}
\end{TableNotes}

\footnotesize{

\begin{longtable}{lrrrrrr}\noalign{\getlongtablewidth\global\LTcapwidth=\longtablewidth}
\caption{\label{tab:H1-open-care-contrasts}Linear Contrasts for Openness
(Moderated by Grandchild Care; only HRS).}\\
\toprule
& \multicolumn{3}{c}{Parent controls} & \multicolumn{3}{c}{Nonparent controls} \\
\cmidrule(r){2-4} \cmidrule(r){5-7}
Linear Contrast & $\hat{\gamma}_{c}$ & $\chi^2$ & $p$ & $\hat{\gamma}_{c}$ & $\chi^2$ & $p$\\
\midrule
\endfirsthead
\caption*{\normalfont{Table \ref{tab:H1-open-care-contrasts} continued}}\\
\toprule
& \multicolumn{3}{c}{Parent controls} & \multicolumn{3}{c}{Nonparent controls} \\
\cmidrule(r){2-4} \cmidrule(r){5-7}
Linear Contrast & $\hat{\gamma}_{c}$ & $\chi^2$ & $p$ & $\hat{\gamma}_{c}$ & $\chi^2$ & $p$\\
\midrule
\endhead
After-slope of caring controls vs. caring grandparents 
($\hat{\gamma}_{21}$ + $\hat{\gamma}_{31}$) & 0.04 & 7.78 & .005 & 0.04 & 9.46 & .002\\
After-slope of not-caring grandparents vs. caring grandparents 
($\hat{\gamma}_{30}$ + $\hat{\gamma}_{31}$) & 0.03 & 2.58 & .108 & 0.03 & 3.26 & .071\\
\bottomrule
\addlinespace
\insertTableNotes
\end{longtable}

}

\end{lltable}








\begin{lltable}

\begin{TableNotes}[para]
\normalsize{\textit{Note.} Two models were computed for each of the two
samples (LISS, HRS): grandparents matched with parent controls and with
nonparent controls. CI = confidence interval.}
\end{TableNotes}

\footnotesize{

\begin{longtable}{lrcrrrcrr}\noalign{\getlongtablewidth\global\LTcapwidth=\longtablewidth}
\caption{\label{tab:H1-swls-tab}Fixed Effects of Life Satisfaction Over the
Transition to Grandparenthood.}\\
\toprule
& \multicolumn{4}{c}{Parent controls} & \multicolumn{4}{c}{Nonparent controls} \\
\cmidrule(r){2-5} \cmidrule(r){6-9}
Parameter & $\hat{\gamma}$ & 95\% CI & $t$ & $p$ & $\hat{\gamma}$ & 95\% CI & $t$ & $p$\\
\midrule
\endfirsthead
\caption*{\normalfont{Table \ref{tab:H1-swls-tab} continued}}\\
\toprule
& \multicolumn{4}{c}{Parent controls} & \multicolumn{4}{c}{Nonparent controls} \\
\cmidrule(r){2-5} \cmidrule(r){6-9}
Parameter & $\hat{\gamma}$ & 95\% CI & $t$ & $p$ & $\hat{\gamma}$ & 95\% CI & $t$ & $p$\\
\midrule
\endhead
LISS &  &  &  &  &  &  &  & \\
\ \ \ Intercept, $\hat{\gamma}_{00}$ \textcolor{white}{L} & 5.11 & [4.99, 5.23] & 85.63 & < .001 & 5.13 & [4.99, 5.27] & 72.47 & < .001\\
\ \ \ Propensity score, $\hat{\gamma}_{02}$ \textcolor{white}{L} & 0.07 & [-0.10, 0.24] & 0.78 & .433 & 0.01 & [-0.15, 0.17] & 0.17 & .863\\
\ \ \ Before-slope, $\hat{\gamma}_{10}$ \textcolor{white}{L} & -0.01 & [-0.02, 0.01] & -1.06 & .288 & 0.02 & [0.00, 0.03] & 2.18 & .029\\
\ \ \ After-slope, $\hat{\gamma}_{20}$ \textcolor{white}{L} & 0.01 & [0.00, 0.02] & 2.13 & .033 & -0.01 & [-0.02, 0.01] & -0.93 & .351\\
\ \ \ Shift, $\hat{\gamma}_{30}$ \textcolor{white}{L} & 0.02 & [-0.04, 0.08] & 0.72 & .470 & -0.11 & [-0.17, -0.05] & -3.42 & .001\\
\ \ \ Grandparent, $\hat{\gamma}_{01}$ \textcolor{white}{L} & 0.07 & [-0.11, 0.25] & 0.73 & .464 & 0.07 & [-0.13, 0.26] & 0.66 & .510\\
\ \ \ Before-slope * Grandparent, $\hat{\gamma}_{11}$ \textcolor{white}{L} & 0.02 & [-0.01, 0.04] & 1.03 & .301 & -0.01 & [-0.04, 0.02] & -0.47 & .637\\
\ \ \ After-slope * Grandparent, $\hat{\gamma}_{21}$ \textcolor{white}{L} & -0.02 & [-0.05, 0.00] & -1.78 & .075 & 0.00 & [-0.03, 0.02] & -0.33 & .741\\
\ \ \ Shift * Grandparent, $\hat{\gamma}_{31}$ \textcolor{white}{L} & 0.05 & [-0.08, 0.18] & 0.79 & .428 & 0.18 & [0.04, 0.32] & 2.57 & .010\\
HRS &  &  &  &  &  &  &  & \\
\ \ \ Intercept, $\hat{\gamma}_{00}$ \textcolor{white}{H} & 4.81 & [4.69, 4.92] & 82.17 & < .001 & 4.58 & [4.45, 4.72] & 66.89 & < .001\\
\ \ \ Propensity score, $\hat{\gamma}_{02}$ \textcolor{white}{H} & 0.40 & [0.19, 0.61] & 3.78 & < .001 & 0.33 & [0.11, 0.54] & 3.01 & .003\\
\ \ \ Before-slope, $\hat{\gamma}_{10}$ \textcolor{white}{H} & -0.03 & [-0.07, 0.01] & -1.53 & .125 & 0.05 & [0.01, 0.08] & 2.50 & .013\\
\ \ \ After-slope, $\hat{\gamma}_{20}$ \textcolor{white}{H} & 0.01 & [-0.01, 0.04] & 0.83 & .405 & 0.04 & [0.01, 0.06] & 3.14 & .002\\
\ \ \ Shift, $\hat{\gamma}_{30}$ \textcolor{white}{H} & 0.02 & [-0.05, 0.10] & 0.58 & .564 & -0.05 & [-0.12, 0.02] & -1.50 & .135\\
\ \ \ Grandparent, $\hat{\gamma}_{01}$ \textcolor{white}{H} & -0.02 & [-0.21, 0.16] & -0.24 & .812 & 0.20 & [0.00, 0.39] & 1.98 & .048\\
\ \ \ Before-slope * Grandparent, $\hat{\gamma}_{11}$ \textcolor{white}{H} & 0.12 & [0.03, 0.21] & 2.58 & .010 & 0.05 & [-0.04, 0.13] & 1.06 & .290\\
\ \ \ After-slope * Grandparent, $\hat{\gamma}_{21}$ \textcolor{white}{H} & 0.03 & [-0.02, 0.09] & 1.17 & .241 & 0.01 & [-0.05, 0.06] & 0.31 & .753\\
\ \ \ Shift * Grandparent, $\hat{\gamma}_{31}$ \textcolor{white}{H} & -0.08 & [-0.24, 0.09] & -0.93 & .351 & -0.01 & [-0.17, 0.15] & -0.13 & .897\\
\bottomrule
\addlinespace
\insertTableNotes
\end{longtable}

}

\end{lltable}









\begin{lltable}

\begin{TableNotes}[para]
\normalsize{\textit{Note.} The linear contrasts are needed in cases
where estimates of interest are represented by multiple fixed-effects
coefficients and are computed using the \emph{linearHypothesis} function
from the \emph{car} R package (Fox \& Weisberg, 2019) based on the
models from Table \ref{tab:H1-swls-tab}. \(\hat{\gamma}_{c}\) = combined
fixed-effects estimate.}
\end{TableNotes}

\footnotesize{

\begin{longtable}{lrrrrrr}\noalign{\getlongtablewidth\global\LTcapwidth=\longtablewidth}
\caption{\label{tab:H1-swls-contrasts}Linear Contrasts for Life Satisfaction.}\\
\toprule
& \multicolumn{3}{c}{Parent controls} & \multicolumn{3}{c}{Nonparent controls} \\
\cmidrule(r){2-4} \cmidrule(r){5-7}
Linear Contrast & $\hat{\gamma}_{c}$ & $\chi^2$ & $p$ & $\hat{\gamma}_{c}$ & $\chi^2$ & $p$\\
\midrule
\endfirsthead
\caption*{\normalfont{Table \ref{tab:H1-swls-contrasts} continued}}\\
\toprule
& \multicolumn{3}{c}{Parent controls} & \multicolumn{3}{c}{Nonparent controls} \\
\cmidrule(r){2-4} \cmidrule(r){5-7}
Linear Contrast & $\hat{\gamma}_{c}$ & $\chi^2$ & $p$ & $\hat{\gamma}_{c}$ & $\chi^2$ & $p$\\
\midrule
\endhead
LISS &  &  &  &  &  & \\
\ \ \ Shift of the controls vs. 0 ($\hat{\gamma}_{20}$ + 
$\hat{\gamma}_{30}$) \textcolor{white}{L} & 0.03 & 1.76 & .185 & -0.12 & 17.14 & < .001\\
\ \ \ Shift of the grandparents vs. 0 ($\hat{\gamma}_{20}$ + 
$\hat{\gamma}_{30}$ + $\hat{\gamma}_{21}$ + 
$\hat{\gamma}_{31}$) \textcolor{white}{L} & 0.06 & 1.51 & .219 & 0.06 & 1.29 & .256\\
\ \ \ Shift of the controls vs. shift of the grandparents 
($\hat{\gamma}_{21}$ + $\hat{\gamma}_{31}$) \textcolor{white}{L} & 0.03 & 0.24 & .622 & 0.18 & 8.25 & .004\\
\ \ \ Before-slope of the grandparents vs. 0 ($\hat{\gamma}_{10}$ + 
$\hat{\gamma}_{11}$) \textcolor{white}{L} & 0.01 & 0.39 & .532 & 0.01 & 0.32 & .574\\
\ \ \ After-slope of the grandparents vs. 0 ($\hat{\gamma}_{20}$ + 
$\hat{\gamma}_{21}$) \textcolor{white}{L} & -0.01 & 0.84 & .358 & -0.01 & 0.70 & .403\\
HRS &  &  &  &  &  & \\
\ \ \ Shift of the controls vs. 0 ($\hat{\gamma}_{20}$ + 
$\hat{\gamma}_{30}$) \textcolor{white}{H} & 0.03 & 1.26 & .262 & -0.02 & 0.30 & .581\\
\ \ \ Shift of the grandparents vs. 0 ($\hat{\gamma}_{20}$ + 
$\hat{\gamma}_{30}$ + $\hat{\gamma}_{21}$ + 
$\hat{\gamma}_{31}$) \textcolor{white}{H} & -0.01 & 0.04 & .833 & -0.02 & 0.10 & .754\\
\ \ \ Shift of the controls vs. shift of the grandparents 
($\hat{\gamma}_{21}$ + $\hat{\gamma}_{31}$) \textcolor{white}{H} & -0.04 & 0.49 & .485 & 0.00 & 0.00 & .978\\
\ \ \ Before-slope of the grandparents vs. 0 ($\hat{\gamma}_{10}$ + 
$\hat{\gamma}_{11}$) \textcolor{white}{H} & 0.09 & 4.51 & .034 & 0.09 & 5.61 & .018\\
\ \ \ After-slope of the grandparents vs. 0 ($\hat{\gamma}_{20}$ + 
$\hat{\gamma}_{21}$) \textcolor{white}{H} & 0.04 & 2.98 & .084 & 0.05 & 3.67 & .055\\
\bottomrule
\addlinespace
\insertTableNotes
\end{longtable}

}

\end{lltable}







\begin{lltable}

\begin{TableNotes}[para]
\normalsize{\textit{Note.} Two models were computed for each of the
two samples (LISS, HRS): grandparents matched with parent controls and
with nonparent controls. CI = confidence interval.}
\end{TableNotes}

\footnotesize{

\begin{longtable}{lrcrrrcrr}\noalign{\getlongtablewidth\global\LTcapwidth=\longtablewidth}
\caption{\label{tab:H1-swls-gender-tab}Fixed Effects of Life Satisfaction Over the
Transition to Grandparenthood Moderated by Gender.}\\
\toprule
& \multicolumn{4}{c}{Parent controls} & \multicolumn{4}{c}{Nonparent controls} \\
\cmidrule(r){2-5} \cmidrule(r){6-9}
Parameter & $\hat{\gamma}$ & 95\% CI & $t$ & $p$ & $\hat{\gamma}$ & 95\% CI & $t$ & $p$\\
\midrule
\endfirsthead
\caption*{\normalfont{Table \ref{tab:H1-swls-gender-tab} continued}}\\
\toprule
& \multicolumn{4}{c}{Parent controls} & \multicolumn{4}{c}{Nonparent controls} \\
\cmidrule(r){2-5} \cmidrule(r){6-9}
Parameter & $\hat{\gamma}$ & 95\% CI & $t$ & $p$ & $\hat{\gamma}$ & 95\% CI & $t$ & $p$\\
\midrule
\endhead
LISS &  &  &  &  &  &  &  & \\
\ \ \ Intercept, $\hat{\gamma}_{00}$ \textcolor{white}{L} & 5.05 & [4.89, 5.21] & 61.49 & < .001 & 5.05 & [4.86, 5.24] & 51.98 & < .001\\
\ \ \ Propensity score, $\hat{\gamma}_{04}$ \textcolor{white}{L} & 0.06 & [-0.11, 0.23] & 0.70 & .485 & 0.01 & [-0.15, 0.17] & 0.17 & .866\\
\ \ \ Before-slope, $\hat{\gamma}_{10}$ \textcolor{white}{L} & -0.01 & [-0.03, 0.01] & -1.13 & .258 & 0.02 & [0.00, 0.05] & 2.28 & .023\\
\ \ \ After-slope, $\hat{\gamma}_{20}$ \textcolor{white}{L} & 0.01 & [0.00, 0.03] & 1.55 & .122 & -0.03 & [-0.04, -0.01] & -2.76 & .006\\
\ \ \ Shift, $\hat{\gamma}_{30}$ \textcolor{white}{L} & 0.10 & [0.01, 0.18] & 2.25 & .025 & 0.00 & [-0.09, 0.09] & -0.01 & .988\\
\ \ \ Grandparent, $\hat{\gamma}_{01}$ \textcolor{white}{L} & 0.21 & [-0.04, 0.46] & 1.67 & .096 & 0.23 & [-0.04, 0.50] & 1.65 & .099\\
\ \ \ Female, $\hat{\gamma}_{02}$ \textcolor{white}{L} & 0.12 & [-0.08, 0.32] & 1.18 & .239 & 0.16 & [-0.08, 0.40] & 1.28 & .203\\
\ \ \ Before-slope * Grandparent, $\hat{\gamma}_{11}$ \textcolor{white}{L} & 0.00 & [-0.04, 0.04] & 0.10 & .922 & -0.03 & [-0.08, 0.01] & -1.38 & .168\\
\ \ \ After-slope * Grandparent, $\hat{\gamma}_{21}$ \textcolor{white}{L} & -0.03 & [-0.07, 0.01] & -1.62 & .104 & 0.01 & [-0.03, 0.05] & 0.36 & .718\\
\ \ \ Shift * Grandparent, $\hat{\gamma}_{31}$ \textcolor{white}{L} & 0.01 & [-0.18, 0.20] & 0.10 & .919 & 0.11 & [-0.10, 0.31] & 1.03 & .303\\
\ \ \ Before-slope * Female, $\hat{\gamma}_{12}$ \textcolor{white}{L} & 0.01 & [-0.02, 0.03] & 0.55 & .581 & -0.02 & [-0.04, 0.01] & -1.10 & .273\\
\ \ \ After-slope * Female, $\hat{\gamma}_{22}$ \textcolor{white}{L} & 0.00 & [-0.02, 0.02] & -0.11 & .913 & 0.04 & [0.01, 0.06] & 2.95 & .003\\
\ \ \ Shift * Female, $\hat{\gamma}_{32}$ \textcolor{white}{L} & -0.14 & [-0.26, -0.02] & -2.37 & .018 & -0.21 & [-0.33, -0.08] & -3.28 & .001\\
\ \ \ Grandparent * Female, $\hat{\gamma}_{03}$ \textcolor{white}{L} & -0.27 & [-0.59, 0.05] & -1.67 & .097 & -0.31 & [-0.66, 0.05] & -1.71 & .088\\
\ \ \ Before-slope * Grandparent * Female, $\hat{\gamma}_{13}$ \textcolor{white}{L} & 0.03 & [-0.03, 0.08] & 0.87 & .385 & 0.05 & [-0.02, 0.11] & 1.48 & .138\\
\ \ \ After-slope * Grandparent * Female, $\hat{\gamma}_{23}$ \textcolor{white}{L} & 0.01 & [-0.04, 0.07] & 0.51 & .607 & -0.03 & [-0.08, 0.03] & -0.90 & .369\\
\ \ \ Shift * Grandparent * Female, $\hat{\gamma}_{33}$ \textcolor{white}{L} & 0.08 & [-0.17, 0.34] & 0.63 & .530 & 0.15 & [-0.13, 0.43] & 1.07 & .283\\
HRS &  &  &  &  &  &  &  & \\
\ \ \ Intercept, $\hat{\gamma}_{00}$ \textcolor{white}{H} & 4.67 & [4.52, 4.82] & 60.70 & < .001 & 4.54 & [4.37, 4.71] & 52.50 & < .001\\
\ \ \ Propensity score, $\hat{\gamma}_{04}$ \textcolor{white}{H} & 0.41 & [0.20, 0.62] & 3.84 & < .001 & 0.30 & [0.08, 0.51] & 2.71 & .007\\
\ \ \ Before-slope, $\hat{\gamma}_{10}$ \textcolor{white}{H} & 0.01 & [-0.04, 0.07] & 0.49 & .625 & 0.05 & [-0.01, 0.10] & 1.61 & .107\\
\ \ \ After-slope, $\hat{\gamma}_{20}$ \textcolor{white}{H} & 0.00 & [-0.04, 0.04] & 0.09 & .931 & 0.02 & [-0.01, 0.06] & 1.31 & .190\\
\ \ \ Shift, $\hat{\gamma}_{30}$ \textcolor{white}{H} & 0.07 & [-0.04, 0.18] & 1.23 & .220 & -0.16 & [-0.27, -0.05] & -2.91 & .004\\
\ \ \ Grandparent, $\hat{\gamma}_{01}$ \textcolor{white}{H} & 0.11 & [-0.15, 0.37] & 0.81 & .419 & 0.25 & [-0.02, 0.51] & 1.82 & .070\\
\ \ \ Female, $\hat{\gamma}_{02}$ \textcolor{white}{H} & 0.24 & [0.07, 0.41] & 2.75 & .006 & 0.10 & [-0.10, 0.29] & 0.98 & .329\\
\ \ \ Before-slope * Grandparent, $\hat{\gamma}_{11}$ \textcolor{white}{H} & 0.00 & [-0.13, 0.14] & 0.03 & .978 & -0.02 & [-0.15, 0.11] & -0.33 & .745\\
\ \ \ After-slope * Grandparent, $\hat{\gamma}_{21}$ \textcolor{white}{H} & 0.04 & [-0.04, 0.13] & 1.05 & .294 & 0.03 & [-0.05, 0.10] & 0.62 & .536\\
\ \ \ Shift * Grandparent, $\hat{\gamma}_{31}$ \textcolor{white}{H} & -0.08 & [-0.33, 0.16] & -0.65 & .514 & 0.14 & [-0.10, 0.37] & 1.16 & .246\\
\ \ \ Before-slope * Female, $\hat{\gamma}_{12}$ \textcolor{white}{H} & -0.08 & [-0.16, 0.00] & -2.08 & .037 & 0.01 & [-0.07, 0.08] & 0.14 & .887\\
\ \ \ After-slope * Female, $\hat{\gamma}_{22}$ \textcolor{white}{H} & 0.02 & [-0.03, 0.07] & 0.64 & .525 & 0.02 & [-0.03, 0.07] & 0.84 & .399\\
\ \ \ Shift * Female, $\hat{\gamma}_{32}$ \textcolor{white}{H} & -0.09 & [-0.24, 0.06] & -1.14 & .254 & 0.19 & [0.05, 0.33] & 2.59 & .010\\
\ \ \ Grandparent * Female, $\hat{\gamma}_{03}$ \textcolor{white}{H} & -0.23 & [-0.55, 0.09] & -1.42 & .156 & -0.08 & [-0.40, 0.25] & -0.47 & .637\\
\ \ \ Before-slope * Grandparent * Female, $\hat{\gamma}_{13}$ \textcolor{white}{H} & 0.21 & [0.03, 0.39] & 2.28 & .023 & 0.11 & [-0.05, 0.28] & 1.34 & .181\\
\ \ \ After-slope * Grandparent * Female, $\hat{\gamma}_{23}$ \textcolor{white}{H} & -0.02 & [-0.13, 0.09] & -0.37 & .714 & -0.03 & [-0.13, 0.08] & -0.50 & .615\\
\ \ \ Shift * Grandparent * Female, $\hat{\gamma}_{33}$ \textcolor{white}{H} & 0.01 & [-0.32, 0.34] & 0.06 & .954 & -0.26 & [-0.57, 0.05] & -1.63 & .103\\
\bottomrule
\addlinespace
\insertTableNotes
\end{longtable}

}

\end{lltable}








\begin{lltable}

\begin{TableNotes}[para]
\normalsize{\textit{Note.} The linear contrasts are based on
the models from Table \ref{tab:H1-swls-gender-tab}. \(\hat{\gamma}_{c}\)
= combined fixed-effects estimate.}
\end{TableNotes}

\footnotesize{

\begin{longtable}{lrrrrrr}\noalign{\getlongtablewidth\global\LTcapwidth=\longtablewidth}
\caption{\label{tab:H1-swls-gender-contrasts}Linear Contrasts for Life
Satisfaction (Moderated by Gender).}\\
\toprule
& \multicolumn{3}{c}{Parent controls} & \multicolumn{3}{c}{Nonparent controls} \\
\cmidrule(r){2-4} \cmidrule(r){5-7}
Linear Contrast & $\hat{\gamma}_{c}$ & $\chi^2$ & $p$ & $\hat{\gamma}_{c}$ & $\chi^2$ & $p$\\
\midrule
\endfirsthead
\caption*{\normalfont{Table \ref{tab:H1-swls-gender-contrasts} continued}}\\
\toprule
& \multicolumn{3}{c}{Parent controls} & \multicolumn{3}{c}{Nonparent controls} \\
\cmidrule(r){2-4} \cmidrule(r){5-7}
Linear Contrast & $\hat{\gamma}_{c}$ & $\chi^2$ & $p$ & $\hat{\gamma}_{c}$ & $\chi^2$ & $p$\\
\midrule
\endhead
LISS &  &  &  &  &  & \\
\ \ \ Shift of male controls vs. 0 ($\hat{\gamma}_{20}$ + 
$\hat{\gamma}_{30}$) \textcolor{white}{L} & 0.11 & 8.55 & .003 & -0.03 & 0.42 & .515\\
\ \ \ Shift of female controls vs. 0 ($\hat{\gamma}_{20}$ + 
$\hat{\gamma}_{30}$ + $\hat{\gamma}_{22}$ + 
$\hat{\gamma}_{32}$) \textcolor{white}{L} & -0.03 & 0.77 & .379 & -0.20 & 26.82 & < .001\\
\ \ \ Shift of grandfathers vs. 0 ($\hat{\gamma}_{20}$ + 
$\hat{\gamma}_{30}$ + $\hat{\gamma}_{21}$ + 
$\hat{\gamma}_{31}$) \textcolor{white}{L} & 0.09 & 1.42 & .233 & 0.09 & 1.17 & .279\\
\ \ \ Shift of grandmothers vs. 0 ($\hat{\gamma}_{20}$ + 
$\hat{\gamma}_{30}$ + $\hat{\gamma}_{21}$ + 
$\hat{\gamma}_{31}$ + $\hat{\gamma}_{22}$ + 
$\hat{\gamma}_{32}$ + $\hat{\gamma}_{23}$ +
$\hat{\gamma}_{33}$) \textcolor{white}{L} & 0.04 & 0.39 & .531 & 0.04 & 0.35 & .552\\
\ \ \ Shift of male controls vs. grandfathers 
($\hat{\gamma}_{21}$ + $\hat{\gamma}_{31}$) \textcolor{white}{L} & -0.02 & 0.07 & .794 & 0.12 & 1.58 & .208\\
\ \ \ Before-slope of female controls vs. grandmothers 
($\hat{\gamma}_{11}$ + $\hat{\gamma}_{13}$) \textcolor{white}{L} & 0.03 & 1.96 & .161 & 0.01 & 0.47 & .493\\
\ \ \ After-slope of female controls vs. grandmothers 
($\hat{\gamma}_{21}$ + $\hat{\gamma}_{23}$) \textcolor{white}{L} & -0.02 & 0.99 & .320 & -0.02 & 0.86 & .353\\
\ \ \ Shift of female controls vs. grandmothers 
($\hat{\gamma}_{21}$ + $\hat{\gamma}_{31}$ + 
$\hat{\gamma}_{23}$ + $\hat{\gamma}_{33}$) \textcolor{white}{L} & 0.07 & 0.92 & .338 & 0.24 & 8.27 & .004\\
\ \ \ Shift of male vs. female controls 
($\hat{\gamma}_{22}$ + $\hat{\gamma}_{32}$) \textcolor{white}{L} & -0.14 & 7.55 & .006 & -0.17 & 9.46 & .002\\
\ \ \ Before-slope of grandfathers vs. grandmothers 
($\hat{\gamma}_{12}$ + $\hat{\gamma}_{13}$) \textcolor{white}{L} & 0.03 & 1.56 & .211 & 0.03 & 1.23 & .267\\
\ \ \ After-slope of grandfathers vs. grandmothers 
($\hat{\gamma}_{22}$ + $\hat{\gamma}_{23}$) \textcolor{white}{L} & 0.01 & 0.27 & .602 & 0.01 & 0.22 & .638\\
\ \ \ Shift of grandfathers vs. grandmothers 
($\hat{\gamma}_{22}$ + $\hat{\gamma}_{32}$ + 
$\hat{\gamma}_{23}$ + $\hat{\gamma}_{33}$) \textcolor{white}{L} & -0.05 & 0.21 & .647 & -0.04 & 0.16 & .690\\
HRS &  &  &  &  &  & \\
\ \ \ Shift of male controls vs. 0 ($\hat{\gamma}_{20}$ + 
$\hat{\gamma}_{30}$) \textcolor{white}{H} & 0.07 & 2.68 & .101 & -0.14 & 10.20 & .001\\
\ \ \ Shift of female controls vs. 0 ($\hat{\gamma}_{20}$ + 
$\hat{\gamma}_{30}$ + $\hat{\gamma}_{22}$ + 
$\hat{\gamma}_{32}$) \textcolor{white}{H} & 0.00 & 0.00 & .973 & 0.07 & 4.01 & .045\\
\ \ \ Shift of grandfathers vs. 0 ($\hat{\gamma}_{20}$ + 
$\hat{\gamma}_{30}$ + $\hat{\gamma}_{21}$ + 
$\hat{\gamma}_{31}$) \textcolor{white}{H} & 0.04 & 0.17 & .680 & 0.03 & 0.12 & .732\\
\ \ \ Shift of grandmothers vs. 0 ($\hat{\gamma}_{20}$ + 
$\hat{\gamma}_{30}$ + $\hat{\gamma}_{21}$ + 
$\hat{\gamma}_{31}$ + $\hat{\gamma}_{22}$ + 
$\hat{\gamma}_{32}$ + $\hat{\gamma}_{23}$ +
$\hat{\gamma}_{33}$) \textcolor{white}{H} & -0.05 & 0.37 & .541 & -0.05 & 0.48 & .489\\
\ \ \ Shift of male controls vs. grandfathers 
($\hat{\gamma}_{21}$ + $\hat{\gamma}_{31}$) \textcolor{white}{H} & -0.04 & 0.15 & .700 & 0.16 & 3.22 & .073\\
\ \ \ Before-slope of female controls vs. grandmothers 
($\hat{\gamma}_{11}$ + $\hat{\gamma}_{13}$) \textcolor{white}{H} & 0.21 & 12.04 & .001 & 0.09 & 2.72 & .099\\
\ \ \ After-slope of female controls vs. grandmothers 
($\hat{\gamma}_{21}$ + $\hat{\gamma}_{23}$) \textcolor{white}{H} & 0.02 & 0.38 & .540 & 0.00 & 0.00 & .953\\
\ \ \ Shift of female controls vs. grandmothers 
($\hat{\gamma}_{21}$ + $\hat{\gamma}_{31}$ + 
$\hat{\gamma}_{23}$ + $\hat{\gamma}_{33}$) \textcolor{white}{H} & -0.05 & 0.31 & .575 & -0.12 & 2.31 & .129\\
\ \ \ Shift of male vs. female controls 
($\hat{\gamma}_{22}$ + $\hat{\gamma}_{32}$) \textcolor{white}{H} & -0.07 & 1.44 & .229 & 0.21 & 13.91 & < .001\\
\ \ \ Before-slope of grandfathers vs. grandmothers 
($\hat{\gamma}_{12}$ + $\hat{\gamma}_{13}$) \textcolor{white}{H} & 0.13 & 2.33 & .127 & 0.12 & 2.41 & .121\\
\ \ \ After-slope of grandfathers vs. grandmothers 
($\hat{\gamma}_{22}$ + $\hat{\gamma}_{23}$) \textcolor{white}{H} & 0.00 & 0.01 & .931 & -0.01 & 0.02 & .894\\
\ \ \ Shift of grandfathers vs. grandmothers 
($\hat{\gamma}_{22}$ + $\hat{\gamma}_{32}$ + 
$\hat{\gamma}_{23}$ + $\hat{\gamma}_{33}$) \textcolor{white}{H} & -0.08 & 0.52 & .471 & -0.08 & 0.52 & .470\\
\bottomrule
\addlinespace
\insertTableNotes
\end{longtable}

}

\end{lltable}








\begin{lltable}

\begin{TableNotes}[para]
\normalsize{\textit{Note.} Two models were computed (only HRS):
grandparents matched with parent controls and with nonparent controls.
CI = confidence interval. \(working=1\) indicates being employed in paid
work.}
\end{TableNotes}

\footnotesize{

\begin{longtable}{lrcrrrcrr}\noalign{\getlongtablewidth\global\LTcapwidth=\longtablewidth}
\caption{\label{tab:H1-swls-work-tab}Fixed Effects of Life Satisfaction Over the
Transition to Grandparenthood Moderated by Performing Paid Work.}\\
\toprule
& \multicolumn{4}{c}{Parent controls} & \multicolumn{4}{c}{Nonparent controls} \\
\cmidrule(r){2-5} \cmidrule(r){6-9}
Parameter & $\hat{\gamma}$ & 95\% CI & $t$ & $p$ & $\hat{\gamma}$ & 95\% CI & $t$ & $p$\\
\midrule
\endfirsthead
\caption*{\normalfont{Table \ref{tab:H1-swls-work-tab} continued}}\\
\toprule
& \multicolumn{4}{c}{Parent controls} & \multicolumn{4}{c}{Nonparent controls} \\
\cmidrule(r){2-5} \cmidrule(r){6-9}
Parameter & $\hat{\gamma}$ & 95\% CI & $t$ & $p$ & $\hat{\gamma}$ & 95\% CI & $t$ & $p$\\
\midrule
\endhead
Intercept, $\hat{\gamma}_{00}$ & 4.78 & [4.63, 4.93] & 62.86 & < .001 & 4.55 & [4.38, 4.71] & 53.96 & < .001\\
Propensity score, $\hat{\gamma}_{02}$ & 0.36 & [0.15, 0.57] & 3.33 & .001 & 0.28 & [0.06, 0.50] & 2.50 & .012\\
Before-slope, $\hat{\gamma}_{20}$ & -0.06 & [-0.13, 0.01] & -1.77 & .077 & -0.02 & [-0.09, 0.05] & -0.51 & .613\\
After-slope, $\hat{\gamma}_{40}$ & -0.03 & [-0.07, 0.00] & -1.73 & .083 & 0.08 & [0.04, 0.12] & 4.32 & < .001\\
Shift, $\hat{\gamma}_{60}$ & 0.13 & [0.01, 0.25] & 2.11 & .034 & 0.07 & [-0.05, 0.19] & 1.17 & .243\\
Grandparent, $\hat{\gamma}_{01}$ & -0.02 & [-0.33, 0.30] & -0.09 & .925 & 0.22 & [-0.09, 0.53] & 1.37 & .169\\
Working, $\hat{\gamma}_{10}$ & 0.07 & [-0.07, 0.22] & 0.99 & .324 & 0.12 & [-0.02, 0.25] & 1.64 & .102\\
Before-slope * Grandparent, $\hat{\gamma}_{21}$ & 0.14 & [-0.04, 0.32] & 1.50 & .134 & 0.10 & [-0.07, 0.27] & 1.12 & .264\\
After-slope * Grandparent, $\hat{\gamma}_{41}$ & 0.07 & [-0.02, 0.15] & 1.57 & .116 & -0.05 & [-0.12, 0.03] & -1.20 & .231\\
Shift * Grandparent, $\hat{\gamma}_{61}$ & -0.04 & [-0.31, 0.22] & -0.31 & .755 & 0.01 & [-0.24, 0.27] & 0.10 & .917\\
Before-slope * Working, $\hat{\gamma}_{30}$ & 0.05 & [-0.03, 0.14] & 1.21 & .225 & 0.09 & [0.00, 0.17] & 1.99 & .047\\
After-slope * Working, $\hat{\gamma}_{50}$ & 0.10 & [0.05, 0.15] & 3.83 & < .001 & -0.08 & [-0.13, -0.03] & -3.16 & .002\\
Shift * Working, $\hat{\gamma}_{70}$ & -0.20 & [-0.35, -0.04] & -2.50 & .012 & -0.15 & [-0.30, 0.00] & -1.94 & .052\\
Grandparent * Working, $\hat{\gamma}_{11}$ & -0.02 & [-0.36, 0.32] & -0.11 & .912 & -0.07 & [-0.39, 0.25] & -0.42 & .676\\
Before-slope * Grandparent * Working, $\hat{\gamma}_{31}$ & -0.03 & [-0.24, 0.18] & -0.28 & .777 & -0.06 & [-0.26, 0.13] & -0.63 & .527\\
After-slope * Grandparent * Working, $\hat{\gamma}_{51}$ & -0.08 & [-0.20, 0.03] & -1.40 & .161 & 0.10 & [-0.01, 0.21] & 1.79 & .073\\
Shift * Grandparent * Working, $\hat{\gamma}_{71}$ & -0.03 & [-0.38, 0.32] & -0.18 & .859 & -0.09 & [-0.42, 0.24] & -0.54 & .590\\
\bottomrule
\addlinespace
\insertTableNotes
\end{longtable}

}

\end{lltable}








\begin{lltable}

\begin{TableNotes}[para]
\normalsize{\textit{Note.} The linear contrasts are based on the
models from Table \ref{tab:H1-swls-work-tab}. \(\hat{\gamma}_{c}\) =
combined fixed-effects estimate.}
\end{TableNotes}

\footnotesize{

\begin{longtable}{lrrrrrr}\noalign{\getlongtablewidth\global\LTcapwidth=\longtablewidth}
\caption{\label{tab:H1-swls-work-contrasts}Linear Contrasts for Life Satisfaction
(Moderated by Paid Work; only HRS).}\\
\toprule
& \multicolumn{3}{c}{Parent controls} & \multicolumn{3}{c}{Nonparent controls} \\
\cmidrule(r){2-4} \cmidrule(r){5-7}
Linear Contrast & $\hat{\gamma}_{c}$ & $\chi^2$ & $p$ & $\hat{\gamma}_{c}$ & $\chi^2$ & $p$\\
\midrule
\endfirsthead
\caption*{\normalfont{Table \ref{tab:H1-swls-work-contrasts} continued}}\\
\toprule
& \multicolumn{3}{c}{Parent controls} & \multicolumn{3}{c}{Nonparent controls} \\
\cmidrule(r){2-4} \cmidrule(r){5-7}
Linear Contrast & $\hat{\gamma}_{c}$ & $\chi^2$ & $p$ & $\hat{\gamma}_{c}$ & $\chi^2$ & $p$\\
\midrule
\endhead
Shift of not-working controls vs. 0 ($\hat{\gamma}_{40}$ + 
$\hat{\gamma}_{60}$) & 0.10 & 3.85 & .050 & 0.15 & 9.24 & .002\\
Shift of working controls vs. 0 ($\hat{\gamma}_{40}$ + 
$\hat{\gamma}_{60}$ + $\hat{\gamma}_{50}$ + 
$\hat{\gamma}_{70}$) & 0.00 & 0.00 & .969 & -0.08 & 5.03 & .025\\
Shift of not-working grandparents vs. 0 ($\hat{\gamma}_{40}$ + 
$\hat{\gamma}_{60}$ + $\hat{\gamma}_{41}$ + 
$\hat{\gamma}_{61}$) & 0.12 & 1.47 & .226 & 0.12 & 1.63 & .201\\
Shift of working grandparents vs. 0 ($\hat{\gamma}_{40}$ + 
$\hat{\gamma}_{60}$ + $\hat{\gamma}_{41}$ + 
$\hat{\gamma}_{61}$ + $\hat{\gamma}_{50}$ + 
$\hat{\gamma}_{70}$ + $\hat{\gamma}_{51}$ +
$\hat{\gamma}_{71}$) & -0.09 & 1.57 & .210 & -0.10 & 2.13 & .144\\
Shift of not-working controls vs. not-working grandparents 
($\hat{\gamma}_{41}$ + $\hat{\gamma}_{61}$) & 0.02 & 0.04 & .834 & -0.03 & 0.10 & .746\\
Before-slope of working controls vs. working grandparents 
($\hat{\gamma}_{21}$ + $\hat{\gamma}_{31}$) & 0.11 & 3.95 & .047 & 0.03 & 0.44 & .505\\
After-slope of working controls vs. working grandparents 
($\hat{\gamma}_{41}$ + $\hat{\gamma}_{51}$) & -0.02 & 0.17 & .676 & 0.05 & 1.82 & .178\\
Shift of working controls vs. working grandparents 
($\hat{\gamma}_{41}$ + $\hat{\gamma}_{61}$ + 
$\hat{\gamma}_{51}$ + $\hat{\gamma}_{71}$) & -0.09 & 1.21 & .270 & -0.03 & 0.11 & .746\\
Shift of not-working controls vs. working controls 
($\hat{\gamma}_{50}$ + $\hat{\gamma}_{70}$) & -0.10 & 2.47 & .116 & -0.23 & 13.96 & < .001\\
Before-slope of not-working grandparents vs. working grandparents 
($\hat{\gamma}_{30}$ + $\hat{\gamma}_{31}$) & 0.02 & 0.05 & .823 & 0.02 & 0.05 & .818\\
After-slope of not-working grandparents vs. working grandparents 
($\hat{\gamma}_{50}$ + $\hat{\gamma}_{51}$) & 0.02 & 0.12 & .727 & 0.02 & 0.17 & .678\\
Shift of not-working grandparents vs. working grandparents 
($\hat{\gamma}_{50}$ + $\hat{\gamma}_{70}$ + 
$\hat{\gamma}_{51}$ + $\hat{\gamma}_{71}$) & -0.21 & 2.87 & .090 & -0.22 & 3.48 & .062\\
\bottomrule
\addlinespace
\insertTableNotes
\end{longtable}

}

\end{lltable}








\begin{lltable}

\begin{TableNotes}[para]
\normalsize{\textit{Note.} Two models were computed (only HRS):
grandparents matched with parent controls and with nonparent controls.
CI = confidence interval. \(caring=1\) indicates more than 100 hours of
grandchild care since the last assessment.}
\end{TableNotes}

\footnotesize{

\begin{longtable}{lrcrrrcrr}\noalign{\getlongtablewidth\global\LTcapwidth=\longtablewidth}
\caption{\label{tab:H1-swls-care-tab}Fixed Effects of Life Satisfaction Over the
Transition to Grandparenthood Moderated by Grandchild Care.}\\
\toprule
& \multicolumn{4}{c}{Parent controls} & \multicolumn{4}{c}{Nonparent controls} \\
\cmidrule(r){2-5} \cmidrule(r){6-9}
Parameter & $\hat{\gamma}$ & 95\% CI & $t$ & $p$ & $\hat{\gamma}$ & 95\% CI & $t$ & $p$\\
\midrule
\endfirsthead
\caption*{\normalfont{Table \ref{tab:H1-swls-care-tab} continued}}\\
\toprule
& \multicolumn{4}{c}{Parent controls} & \multicolumn{4}{c}{Nonparent controls} \\
\cmidrule(r){2-5} \cmidrule(r){6-9}
Parameter & $\hat{\gamma}$ & 95\% CI & $t$ & $p$ & $\hat{\gamma}$ & 95\% CI & $t$ & $p$\\
\midrule
\endhead
Intercept, $\hat{\gamma}_{00}$ & 4.86 & [4.72, 5.00] & 67.71 & < .001 & 4.75 & [4.58, 4.92] & 55.25 & < .001\\
Propensity score, $\hat{\gamma}_{02}$ & 0.27 & [0.01, 0.53] & 2.05 & .040 & 0.05 & [-0.21, 0.31] & 0.35 & .728\\
After-slope, $\hat{\gamma}_{20}$ & 0.00 & [-0.04, 0.03] & -0.02 & .986 & 0.03 & [0.00, 0.06] & 1.99 & .047\\
Grandparent, $\hat{\gamma}_{01}$ & 0.00 & [-0.22, 0.21] & -0.04 & .967 & 0.17 & [-0.06, 0.40] & 1.45 & .148\\
Caring, $\hat{\gamma}_{10}$ & -0.10 & [-0.22, 0.02] & -1.67 & .094 & 0.02 & [-0.09, 0.12] & 0.34 & .738\\
After-slope * Grandparent, $\hat{\gamma}_{21}$ & 0.07 & [0.00, 0.14] & 1.85 & .065 & 0.04 & [-0.02, 0.11] & 1.24 & .216\\
After-slope * Caring, $\hat{\gamma}_{30}$ & 0.04 & [-0.01, 0.10] & 1.70 & .089 & -0.01 & [-0.06, 0.03] & -0.59 & .557\\
Grandparent * Caring, $\hat{\gamma}_{11}$ & 0.32 & [0.02, 0.62] & 2.08 & .038 & 0.21 & [-0.07, 0.48] & 1.45 & .147\\
After-slope * Grandparent * Caring, $\hat{\gamma}_{31}$ & -0.08 & [-0.19, 0.03] & -1.40 & .162 & -0.03 & [-0.13, 0.08] & -0.51 & .613\\
\bottomrule
\addlinespace
\insertTableNotes
\end{longtable}

}

\end{lltable}







\begin{lltable}

\begin{TableNotes}[para]
\normalsize{\textit{Note.} The linear contrasts are based on the
models from Table \ref{tab:H1-swls-care-tab}. \(\hat{\gamma}_{c}\) =
combined fixed-effects estimate.}
\end{TableNotes}

\footnotesize{

\begin{longtable}{lrrrrrr}\noalign{\getlongtablewidth\global\LTcapwidth=\longtablewidth}
\caption{\label{tab:H1-swls-care-contrasts}Linear Contrasts for Life Satisfaction
(Moderated by Grandchild Care; only HRS).}\\
\toprule
& \multicolumn{3}{c}{Parent controls} & \multicolumn{3}{c}{Nonparent controls} \\
\cmidrule(r){2-4} \cmidrule(r){5-7}
Linear Contrast & $\hat{\gamma}_{c}$ & $\chi^2$ & $p$ & $\hat{\gamma}_{c}$ & $\chi^2$ & $p$\\
\midrule
\endfirsthead
\caption*{\normalfont{Table \ref{tab:H1-swls-care-contrasts} continued}}\\
\toprule
& \multicolumn{3}{c}{Parent controls} & \multicolumn{3}{c}{Nonparent controls} \\
\cmidrule(r){2-4} \cmidrule(r){5-7}
Linear Contrast & $\hat{\gamma}_{c}$ & $\chi^2$ & $p$ & $\hat{\gamma}_{c}$ & $\chi^2$ & $p$\\
\midrule
\endhead
After-slope of caring controls vs. caring grandparents 
($\hat{\gamma}_{21}$ + $\hat{\gamma}_{31}$) & -0.01 & 0.10 & .751 & 0.01 & 0.13 & .722\\
After-slope of not-caring grandparents vs. caring grandparents 
($\hat{\gamma}_{30}$ + $\hat{\gamma}_{31}$) & -0.04 & 0.49 & .486 & -0.04 & 0.73 & .392\\
\bottomrule
\addlinespace
\insertTableNotes
\end{longtable}

}

\end{lltable}
















\begin{lltable}

\begin{TableNotes}[para]
\normalsize{\textit{Note.} The heterogeneous variance models (\emph{df} =
16) differ only in the random effects from the comparison models
(\emph{df} = 13). In addition to two random slope variances (instead of
one), the heterogeneous variance models estimate two additional random
intercept/slope covariances. Both models estimate heterogeneous random
intercept variances for the grandparent and control groups. \emph{Var.}
= random slope variance; \emph{SD} = standard deviation; \emph{LR} =
likelihood ratio; \emph{p} = \emph{p}-value (of the LR test); \emph{GP
greater} = indicating if the random slope variance of the grandparents
is larger than that of either control group.}
\end{TableNotes}

\scriptsize{

\begin{longtable}{lrrrrcrrrrc}\noalign{\getlongtablewidth\global\LTcapwidth=\longtablewidth}
\caption{\label{tab:H2-hetvar-tab-agree}Tests of Heterogeneous Random Slope
Variance Models for Agreeableness Against Comparison Models With a
Uniform Random Slope Variance.}\\
\toprule
& \multicolumn{5}{c}{Parent controls} & \multicolumn{5}{c}{Nonparent controls} \\
\cmidrule(r){2-6} \cmidrule(r){7-11}
& Var. & SD & LR & p & GP greater & Var. & SD & LR & p & GP greater\\
\midrule
\endfirsthead
\caption*{\normalfont{Table \ref{tab:H2-hetvar-tab-agree} continued}}\\
\toprule
& \multicolumn{5}{c}{Parent controls} & \multicolumn{5}{c}{Nonparent controls} \\
\cmidrule(r){2-6} \cmidrule(r){7-11}
& Var. & SD & LR & p & GP greater & Var. & SD & LR & p & GP greater\\
\midrule
\endhead
LISS &  &  &  &  &  &  &  &  &  & \\
\ \ \ Before-slope: uniform \textcolor{white}{L} & 0.00 & 0.04 &  &  &  & 0.00 & 0.04 &  &  & \\
\ \ \ Before-slope: heterogeneous (controls) \textcolor{white}{L} & 0.00 & 0.05 &  &  &  & 0.00 & 0.05 &  &  & \\
\ \ \ Before-slope: heterogeneous (grandparents) \textcolor{white}{L} & 0.00 & 0.04 & 9.72 & .021 & no & 0.00 & 0.03 & 17.01 & < .001 & no\\
\ \ \ After-slope: uniform \textcolor{white}{L} & 0.00 & 0.04 &  &  &  & 0.00 & 0.04 &  &  & \\
\ \ \ After-slope: heterogeneous (controls) \textcolor{white}{L} & 0.00 & 0.04 &  &  &  & 0.00 & 0.04 &  &  & \\
\ \ \ After-slope: heterogeneous (grandparents) \textcolor{white}{L} & 0.00 & 0.03 & 3.34 & .343 & no & 0.00 & 0.03 & 9.22 & .026 & no\\
\ \ \ Shift: uniform \textcolor{white}{L} & 0.03 & 0.16 &  &  &  & 0.02 & 0.15 &  &  & \\
\ \ \ Shift: heterogeneous (controls) \textcolor{white}{L} & 0.03 & 0.17 &  &  &  & 0.03 & 0.16 &  &  & \\
\ \ \ Shift: heterogeneous (grandparents) \textcolor{white}{L} & 0.02 & 0.13 & 3.79 & .285 & no & 0.01 & 0.12 & 7.32 & .062 & no\\
HRS &  &  &  &  &  &  &  &  &  & \\
\ \ \ Before-slope: uniform \textcolor{white}{H} & 0.01 & 0.12 &  &  &  & 0.01 & 0.12 &  &  & \\
\ \ \ Before-slope: heterogeneous (controls) \textcolor{white}{H} & 0.02 & 0.15 &  &  &  & 0.02 & 0.15 &  &  & \\
\ \ \ Before-slope: heterogeneous (grandparents) \textcolor{white}{H} & 0.01 & 0.12 & 75.87 & < .001 & no & 0.02 & 0.14 & 82.20 & < .001 & no\\
\ \ \ After-slope: uniform \textcolor{white}{H} & 0.01 & 0.10 &  &  &  & 0.01 & 0.11 &  &  & \\
\ \ \ After-slope: heterogeneous (controls) \textcolor{white}{H} & 0.01 & 0.11 &  &  &  & 0.02 & 0.13 &  &  & \\
\ \ \ After-slope: heterogeneous (grandparents) \textcolor{white}{H} & 0.01 & 0.08 & 37.85 & < .001 & no & 0.01 & 0.09 & 69.06 & < .001 & no\\
\ \ \ Shift: uniform \textcolor{white}{H} & 0.06 & 0.25 &  &  &  & 0.07 & 0.26 &  &  & \\
\ \ \ Shift: heterogeneous (controls) \textcolor{white}{H} & 0.08 & 0.28 &  &  &  & 0.09 & 0.29 &  &  & \\
\ \ \ Shift: heterogeneous (grandparents) \textcolor{white}{H} & 0.05 & 0.22 & 68.99 & < .001 & no & 0.06 & 0.24 & 91.90 & < .001 & no\\
\bottomrule
\addlinespace
\insertTableNotes
\end{longtable}

}

\end{lltable}





\begin{lltable}

\begin{TableNotes}[para]
\normalsize{\textit{Note.} The heterogeneous variance models (\emph{df} =
16) differ only in the random effects from the comparison models
(\emph{df} = 13). In addition to two random slope variances (instead of
one), the heterogeneous variance models estimate two additional random
intercept/slope covariances. Both models estimate heterogeneous random
intercept variances for the grandparent and control groups. \emph{Var.}
= random slope variance; \emph{SD} = standard deviation; \emph{LR} =
likelihood ratio; \emph{p} = \emph{p}-value (of the LR test); \emph{GP
greater} = indicating if the random slope variance of the grandparents
is larger than that of either control group.}
\end{TableNotes}

\scriptsize{

\begin{longtable}{lrrrrcrrrrc}\noalign{\getlongtablewidth\global\LTcapwidth=\longtablewidth}
\caption{\label{tab:H2-hetvar-tab-con}Tests of Heterogeneous Random Slope Variance
Models for Conscientiousness Against Comparison Models With a Uniform
Random Slope Variance.}\\
\toprule
& \multicolumn{5}{c}{Parent controls} & \multicolumn{5}{c}{Nonparent controls} \\
\cmidrule(r){2-6} \cmidrule(r){7-11}
& Var. & SD & LR & p & GP greater & Var. & SD & LR & p & GP greater\\
\midrule
\endfirsthead
\caption*{\normalfont{Table \ref{tab:H2-hetvar-tab-con} continued}}\\
\toprule
& \multicolumn{5}{c}{Parent controls} & \multicolumn{5}{c}{Nonparent controls} \\
\cmidrule(r){2-6} \cmidrule(r){7-11}
& Var. & SD & LR & p & GP greater & Var. & SD & LR & p & GP greater\\
\midrule
\endhead
LISS &  &  &  &  &  &  &  &  &  & \\
\ \ \ Before-slope: uniform \textcolor{white}{L} & 0.00 & 0.04 &  &  &  & 0.00 & 0.04 &  &  & \\
\ \ \ Before-slope: heterogeneous (controls) \textcolor{white}{L} & 0.00 & 0.05 &  &  &  & 0.00 & 0.05 &  &  & \\
\ \ \ Before-slope: heterogeneous (grandparents) \textcolor{white}{L} & 0.00 & 0.02 & 45.09 & < .001 & no & 0.00 & 0.02 & 26.46 & < .001 & no\\
\ \ \ After-slope: uniform \textcolor{white}{L} & 0.00 & 0.04 &  &  &  & 0.00 & 0.04 &  &  & \\
\ \ \ After-slope: heterogeneous (controls) \textcolor{white}{L} & 0.00 & 0.05 &  &  &  & 0.00 & 0.04 &  &  & \\
\ \ \ After-slope: heterogeneous (grandparents) \textcolor{white}{L} & 0.00 & 0.03 & 18.06 & < .001 & no & 0.00 & 0.03 & 8.69 & .034 & no\\
\ \ \ Shift: uniform \textcolor{white}{L} & 0.03 & 0.16 &  &  &  & 0.02 & 0.14 &  &  & \\
\ \ \ Shift: heterogeneous (controls) \textcolor{white}{L} & 0.04 & 0.19 &  &  &  & 0.02 & 0.16 &  &  & \\
\ \ \ Shift: heterogeneous (grandparents) \textcolor{white}{L} & 0.02 & 0.12 & 21.47 & < .001 & no & 0.01 & 0.11 & 8.86 & .031 & no\\
HRS &  &  &  &  &  &  &  &  &  & \\
\ \ \ Before-slope: uniform \textcolor{white}{H} & 0.01 & 0.11 &  &  &  & 0.01 & 0.11 &  &  & \\
\ \ \ Before-slope: heterogeneous (controls) \textcolor{white}{H} & 0.02 & 0.14 &  &  &  & 0.02 & 0.14 &  &  & \\
\ \ \ Before-slope: heterogeneous (grandparents) \textcolor{white}{H} & 0.01 & 0.11 & 92.92 & < .001 & no & 0.02 & 0.13 & 103.88 & < .001 & no\\
\ \ \ After-slope: uniform \textcolor{white}{H} & 0.01 & 0.10 &  &  &  & 0.01 & 0.10 &  &  & \\
\ \ \ After-slope: heterogeneous (controls) \textcolor{white}{H} & 0.01 & 0.11 &  &  &  & 0.01 & 0.12 &  &  & \\
\ \ \ After-slope: heterogeneous (grandparents) \textcolor{white}{H} & 0.01 & 0.09 & 61.33 & < .001 & no & 0.01 & 0.09 & 77.41 & < .001 & no\\
\ \ \ Shift: uniform \textcolor{white}{H} & 0.06 & 0.24 &  &  &  & 0.06 & 0.25 &  &  & \\
\ \ \ Shift: heterogeneous (controls) \textcolor{white}{H} & 0.07 & 0.27 &  &  &  & 0.08 & 0.28 &  &  & \\
\ \ \ Shift: heterogeneous (grandparents) \textcolor{white}{H} & 0.06 & 0.23 & 83.05 & < .001 & no & 0.06 & 0.25 & 97.85 & < .001 & no\\
\bottomrule
\addlinespace
\insertTableNotes
\end{longtable}

}

\end{lltable}





\begin{lltable}

\begin{TableNotes}[para]
\normalsize{\textit{Note.} The heterogeneous variance models (\emph{df} =
16) differ only in the random effects from the comparison models
(\emph{df} = 13). In addition to two random slope variances (instead of
one), the heterogeneous variance models estimate two additional random
intercept/slope covariances. Both models estimate heterogeneous random
intercept variances for the grandparent and control groups. \emph{Var.}
= random slope variance; \emph{SD} = standard deviation; \emph{LR} =
likelihood ratio; \emph{p} = \emph{p}-value (of the LR test); \emph{GP
greater} = indicating if the random slope variance of the grandparents
is larger than that of either control group.}
\end{TableNotes}

\scriptsize{

\begin{longtable}{lrrrrcrrrrc}\noalign{\getlongtablewidth\global\LTcapwidth=\longtablewidth}
\caption{\label{tab:H2-hetvar-tab-extra}Tests of Heterogeneous Random Slope
Variance Models for Extraversion Against Comparison Models With a
Uniform Random Slope Variance.}\\
\toprule
& \multicolumn{5}{c}{Parent controls} & \multicolumn{5}{c}{Nonparent controls} \\
\cmidrule(r){2-6} \cmidrule(r){7-11}
& Var. & SD & LR & p & GP greater & Var. & SD & LR & p & GP greater\\
\midrule
\endfirsthead
\caption*{\normalfont{Table \ref{tab:H2-hetvar-tab-extra} continued}}\\
\toprule
& \multicolumn{5}{c}{Parent controls} & \multicolumn{5}{c}{Nonparent controls} \\
\cmidrule(r){2-6} \cmidrule(r){7-11}
& Var. & SD & LR & p & GP greater & Var. & SD & LR & p & GP greater\\
\midrule
\endhead
LISS &  &  &  &  &  &  &  &  &  & \\
\ \ \ Before-slope: uniform \textcolor{white}{L} & 0.00 & 0.05 &  &  &  & 0.00 & 0.05 &  &  & \\
\ \ \ Before-slope: heterogeneous (controls) \textcolor{white}{L} & 0.00 & 0.06 &  &  &  & 0.00 & 0.06 &  &  & \\
\ \ \ Before-slope: heterogeneous (grandparents) \textcolor{white}{L} & 0.00 & 0.04 & 14.67 & .002 & no & 0.00 & 0.04 & 25.96 & < .001 & no\\
\ \ \ After-slope: uniform \textcolor{white}{L} & 0.00 & 0.04 &  &  &  & 0.00 & 0.05 &  &  & \\
\ \ \ After-slope: heterogeneous (controls) \textcolor{white}{L} & 0.00 & 0.04 &  &  &  & 0.00 & 0.05 &  &  & \\
\ \ \ After-slope: heterogeneous (grandparents) \textcolor{white}{L} & 0.00 & 0.03 & 7.37 & .061 & no & 0.00 & 0.03 & 13.50 & .004 & no\\
\ \ \ Shift: uniform \textcolor{white}{L} & 0.03 & 0.17 &  &  &  & 0.03 & 0.18 &  &  & \\
\ \ \ Shift: heterogeneous (controls) \textcolor{white}{L} & 0.04 & 0.19 &  &  &  & 0.04 & 0.21 &  &  & \\
\ \ \ Shift: heterogeneous (grandparents) \textcolor{white}{L} & 0.01 & 0.12 & 11.13 & .011 & no & 0.02 & 0.13 & 13.00 & .005 & no\\
HRS &  &  &  &  &  &  &  &  &  & \\
\ \ \ Before-slope: uniform \textcolor{white}{H} & 0.02 & 0.12 &  &  &  & 0.01 & 0.12 &  &  & \\
\ \ \ Before-slope: heterogeneous (controls) \textcolor{white}{H} & 0.02 & 0.15 &  &  &  & 0.02 & 0.14 &  &  & \\
\ \ \ Before-slope: heterogeneous (grandparents) \textcolor{white}{H} & 0.01 & 0.12 & 59.59 & < .001 & no & 0.02 & 0.13 & 61.85 & < .001 & no\\
\ \ \ After-slope: uniform \textcolor{white}{H} & 0.01 & 0.10 &  &  &  & 0.01 & 0.12 &  &  & \\
\ \ \ After-slope: heterogeneous (controls) \textcolor{white}{H} & 0.01 & 0.11 &  &  &  & 0.02 & 0.14 &  &  & \\
\ \ \ After-slope: heterogeneous (grandparents) \textcolor{white}{H} & 0.01 & 0.09 & 27.05 & < .001 & no & 0.01 & 0.10 & 61.55 & < .001 & no\\
\ \ \ Shift: uniform \textcolor{white}{H} & 0.07 & 0.26 &  &  &  & 0.08 & 0.29 &  &  & \\
\ \ \ Shift: heterogeneous (controls) \textcolor{white}{H} & 0.08 & 0.29 &  &  &  & 0.10 & 0.32 &  &  & \\
\ \ \ Shift: heterogeneous (grandparents) \textcolor{white}{H} & 0.06 & 0.25 & 44.54 & < .001 & no & 0.07 & 0.26 & 70.11 & < .001 & no\\
\bottomrule
\addlinespace
\insertTableNotes
\end{longtable}

}

\end{lltable}





\begin{lltable}

\begin{TableNotes}[para]
\normalsize{\textit{Note.} The heterogeneous variance models (\emph{df} =
16) differ only in the random effects from the comparison models
(\emph{df} = 13). In addition to two random slope variances (instead of
one), the heterogeneous variance models estimate two additional random
intercept/slope covariances. Both models estimate heterogeneous random
intercept variances for the grandparent and control groups. \emph{Var.}
= random slope variance; \emph{SD} = standard deviation; \emph{LR} =
likelihood ratio; \emph{p} = \emph{p}-value (of the LR test); \emph{GP
greater} = indicating if the random slope variance of the grandparents
is larger than that of either control group.}
\end{TableNotes}

\scriptsize{

\begin{longtable}{lrrrrcrrrrc}\noalign{\getlongtablewidth\global\LTcapwidth=\longtablewidth}
\caption{\label{tab:H2-hetvar-tab-neur}Tests of Heterogeneous Random Slope
Variance Models for Neuroticism Against Comparison Models With a Uniform
Random Slope Variance.}\\
\toprule
& \multicolumn{5}{c}{Parent controls} & \multicolumn{5}{c}{Nonparent controls} \\
\cmidrule(r){2-6} \cmidrule(r){7-11}
& Var. & SD & LR & p & GP greater & Var. & SD & LR & p & GP greater\\
\midrule
\endfirsthead
\caption*{\normalfont{Table \ref{tab:H2-hetvar-tab-neur} continued}}\\
\toprule
& \multicolumn{5}{c}{Parent controls} & \multicolumn{5}{c}{Nonparent controls} \\
\cmidrule(r){2-6} \cmidrule(r){7-11}
& Var. & SD & LR & p & GP greater & Var. & SD & LR & p & GP greater\\
\midrule
\endhead
LISS &  &  &  &  &  &  &  &  &  & \\
\ \ \ Before-slope: uniform \textcolor{white}{L} & 0.00 & 0.06 &  &  &  & 0.00 & 0.07 &  &  & \\
\ \ \ Before-slope: heterogeneous (controls) \textcolor{white}{L} & 0.00 & 0.06 &  &  &  & 0.01 & 0.08 &  &  & \\
\ \ \ Before-slope: heterogeneous (grandparents) \textcolor{white}{L} & 0.00 & 0.06 & 3.74 & .291 & yes & 0.00 & 0.06 & 19.38 & < .001 & no\\
\ \ \ After-slope: uniform \textcolor{white}{L} & 0.00 & 0.05 &  &  &  & 0.00 & 0.06 &  &  & \\
\ \ \ After-slope: heterogeneous (controls) \textcolor{white}{L} & 0.00 & 0.05 &  &  &  & 0.00 & 0.07 &  &  & \\
\ \ \ After-slope: heterogeneous (grandparents) \textcolor{white}{L} & 0.00 & 0.05 & 1.09 & .781 & no & 0.00 & 0.05 & 6.22 & .101 & no\\
\ \ \ Shift: uniform \textcolor{white}{L} & 0.04 & 0.20 &  &  &  & 0.06 & 0.24 &  &  & \\
\ \ \ Shift: heterogeneous (controls) \textcolor{white}{L} & 0.04 & 0.20 &  &  &  & 0.07 & 0.26 &  &  & \\
\ \ \ Shift: heterogeneous (grandparents) \textcolor{white}{L} & 0.04 & 0.21 & 3.32 & .344 & yes & 0.05 & 0.21 & 3.27 & .352 & no\\
HRS &  &  &  &  &  &  &  &  &  & \\
\ \ \ Before-slope: uniform \textcolor{white}{H} & 0.02 & 0.15 &  &  &  & 0.02 & 0.15 &  &  & \\
\ \ \ Before-slope: heterogeneous (controls) \textcolor{white}{H} & 0.03 & 0.19 &  &  &  & 0.03 & 0.18 &  &  & \\
\ \ \ Before-slope: heterogeneous (grandparents) \textcolor{white}{H} & 0.03 & 0.17 & 95.90 & < .001 & no & 0.03 & 0.18 & 73.45 & < .001 & yes\\
\ \ \ After-slope: uniform \textcolor{white}{H} & 0.01 & 0.12 &  &  &  & 0.02 & 0.12 &  &  & \\
\ \ \ After-slope: heterogeneous (controls) \textcolor{white}{H} & 0.02 & 0.13 &  &  &  & 0.02 & 0.15 &  &  & \\
\ \ \ After-slope: heterogeneous (grandparents) \textcolor{white}{H} & 0.01 & 0.10 & 79.78 & < .001 & no & 0.01 & 0.11 & 101.07 & < .001 & no\\
\ \ \ Shift: uniform \textcolor{white}{H} & 0.10 & 0.31 &  &  &  & 0.10 & 0.32 &  &  & \\
\ \ \ Shift: heterogeneous (controls) \textcolor{white}{H} & 0.13 & 0.35 &  &  &  & 0.13 & 0.36 &  &  & \\
\ \ \ Shift: heterogeneous (grandparents) \textcolor{white}{H} & 0.09 & 0.29 & 116.36 & < .001 & no & 0.09 & 0.30 & 116.43 & < .001 & no\\
\bottomrule
\addlinespace
\insertTableNotes
\end{longtable}

}

\end{lltable}





\begin{lltable}

\begin{TableNotes}[para]
\normalsize{\textit{Note.} The heterogeneous variance models (\emph{df} =
16) differ only in the random effects from the comparison models
(\emph{df} = 13). In addition to two random slope variances (instead of
one), the heterogeneous variance models estimate two additional random
intercept/slope covariances. Both models estimate heterogeneous random
intercept variances for the grandparent and control groups. \emph{Var.}
= random slope variance; \emph{SD} = standard deviation; \emph{LR} =
likelihood ratio; \emph{p} = \emph{p}-value (of the LR test); \emph{GP
greater} = indicating if the random slope variance of the grandparents
is larger than that of either control group.}
\end{TableNotes}

\scriptsize{

\begin{longtable}{lrrrrcrrrrc}\noalign{\getlongtablewidth\global\LTcapwidth=\longtablewidth}
\caption{\label{tab:H2-hetvar-tab-open}Tests of Heterogeneous Random Slope
Variance Models for Openness Against Comparison Models With a Uniform
Random Slope Variance.}\\
\toprule
& \multicolumn{5}{c}{Parent controls} & \multicolumn{5}{c}{Nonparent controls} \\
\cmidrule(r){2-6} \cmidrule(r){7-11}
& Var. & SD & LR & p & GP greater & Var. & SD & LR & p & GP greater\\
\midrule
\endfirsthead
\caption*{\normalfont{Table \ref{tab:H2-hetvar-tab-open} continued}}\\
\toprule
& \multicolumn{5}{c}{Parent controls} & \multicolumn{5}{c}{Nonparent controls} \\
\cmidrule(r){2-6} \cmidrule(r){7-11}
& Var. & SD & LR & p & GP greater & Var. & SD & LR & p & GP greater\\
\midrule
\endhead
LISS &  &  &  &  &  &  &  &  &  & \\
\ \ \ Before-slope: uniform \textcolor{white}{L} & 0.00 & 0.04 &  &  &  & 0.00 & 0.03 &  &  & \\
\ \ \ Before-slope: heterogeneous (controls) \textcolor{white}{L} & 0.00 & 0.05 &  &  &  & 0.00 & 0.04 &  &  & \\
\ \ \ Before-slope: heterogeneous (grandparents) \textcolor{white}{L} & 0.00 & 0.04 & 19.82 & < .001 & no & 0.00 & 0.04 & 25.90 & < .001 & yes\\
\ \ \ After-slope: uniform \textcolor{white}{L} & 0.00 & 0.04 &  &  &  & 0.00 & 0.03 &  &  & \\
\ \ \ After-slope: heterogeneous (controls) \textcolor{white}{L} & 0.00 & 0.05 &  &  &  & 0.00 & 0.03 &  &  & \\
\ \ \ After-slope: heterogeneous (grandparents) \textcolor{white}{L} & 0.00 & 0.02 & 26.80 & < .001 & no & 0.00 & 0.02 & 9.20 & .027 & no\\
\ \ \ Shift: uniform \textcolor{white}{L} & 0.03 & 0.16 &  &  &  & 0.02 & 0.13 &  &  & \\
\ \ \ Shift: heterogeneous (controls) \textcolor{white}{L} & 0.03 & 0.18 &  &  &  & 0.02 & 0.14 &  &  & \\
\ \ \ Shift: heterogeneous (grandparents) \textcolor{white}{L} & 0.01 & 0.10 & 17.96 & < .001 & no & 0.02 & 0.12 & 10.36 & .016 & no\\
HRS &  &  &  &  &  &  &  &  &  & \\
\ \ \ Before-slope: uniform \textcolor{white}{H} & 0.01 & 0.11 &  &  &  & 0.01 & 0.12 &  &  & \\
\ \ \ Before-slope: heterogeneous (controls) \textcolor{white}{H} & 0.02 & 0.14 &  &  &  & 0.02 & 0.14 &  &  & \\
\ \ \ Before-slope: heterogeneous (grandparents) \textcolor{white}{H} & 0.01 & 0.09 & 55.99 & < .001 & no & 0.02 & 0.14 & 50.54 & < .001 & no\\
\ \ \ After-slope: uniform \textcolor{white}{H} & 0.01 & 0.10 &  &  &  & 0.01 & 0.11 &  &  & \\
\ \ \ After-slope: heterogeneous (controls) \textcolor{white}{H} & 0.01 & 0.11 &  &  &  & 0.02 & 0.13 &  &  & \\
\ \ \ After-slope: heterogeneous (grandparents) \textcolor{white}{H} & 0.01 & 0.09 & 37.59 & < .001 & no & 0.01 & 0.10 & 50.64 & < .001 & no\\
\ \ \ Shift: uniform \textcolor{white}{H} & 0.07 & 0.26 &  &  &  & 0.07 & 0.27 &  &  & \\
\ \ \ Shift: heterogeneous (controls) \textcolor{white}{H} & 0.08 & 0.28 &  &  &  & 0.09 & 0.30 &  &  & \\
\ \ \ Shift: heterogeneous (grandparents) \textcolor{white}{H} & 0.06 & 0.24 & 58.39 & < .001 & no & 0.07 & 0.26 & 67.21 & < .001 & no\\
\bottomrule
\addlinespace
\insertTableNotes
\end{longtable}

}

\end{lltable}





\begin{lltable}

\begin{TableNotes}[para]
\normalsize{\textit{Note.} The heterogeneous variance models (\emph{df} =
16) differ only in the random effects from the comparison models
(\emph{df} = 13). In addition to two random slope variances (instead of
one), the heterogeneous variance models estimate two additional random
intercept/slope covariances. Both models estimate heterogeneous random
intercept variances for the grandparent and control groups. \emph{Var.}
= random slope variance; \emph{SD} = standard deviation; \emph{LR} =
likelihood ratio; \emph{p} = \emph{p}-value (of the LR test); \emph{GP
greater} = indicating if the random slope variance of the grandparents
is larger than that of either control group.}
\end{TableNotes}

\scriptsize{

\begin{longtable}{lrrrrcrrrrc}\noalign{\getlongtablewidth\global\LTcapwidth=\longtablewidth}
\caption{\label{tab:H2-hetvar-tab-swls}Tests of Heterogeneous Random Slope
Variance Models for Life Satisfaction Against Comparison Models With a
Uniform Random Slope Variance.}\\
\toprule
& \multicolumn{5}{c}{Parent controls} & \multicolumn{5}{c}{Nonparent controls} \\
\cmidrule(r){2-6} \cmidrule(r){7-11}
& Var. & SD & LR & p & GP greater & Var. & SD & LR & p & GP greater\\
\midrule
\endfirsthead
\caption*{\normalfont{Table \ref{tab:H2-hetvar-tab-swls} continued}}\\
\toprule
& \multicolumn{5}{c}{Parent controls} & \multicolumn{5}{c}{Nonparent controls} \\
\cmidrule(r){2-6} \cmidrule(r){7-11}
& Var. & SD & LR & p & GP greater & Var. & SD & LR & p & GP greater\\
\midrule
\endhead
LISS &  &  &  &  &  &  &  &  &  & \\
\ \ \ Before-slope: uniform \textcolor{white}{L} & 0.01 & 0.11 &  &  &  & 0.01 & 0.10 &  &  & \\
\ \ \ Before-slope: heterogeneous (controls) \textcolor{white}{L} & 0.02 & 0.13 &  &  &  & 0.01 & 0.12 &  &  & \\
\ \ \ Before-slope: heterogeneous (grandparents) \textcolor{white}{L} & 0.02 & 0.14 & 41.47 & < .001 & yes & 0.01 & 0.12 & 21.10 & < .001 & no\\
\ \ \ After-slope: uniform \textcolor{white}{L} & 0.01 & 0.11 &  &  &  & 0.01 & 0.12 &  &  & \\
\ \ \ After-slope: heterogeneous (controls) \textcolor{white}{L} & 0.01 & 0.10 &  &  &  & 0.01 & 0.12 &  &  & \\
\ \ \ After-slope: heterogeneous (grandparents) \textcolor{white}{L} & 0.02 & 0.13 & 11.74 & .008 & yes & 0.02 & 0.12 & 5.26 & .154 & yes\\
\ \ \ Shift: uniform \textcolor{white}{L} & 0.20 & 0.45 &  &  &  & 0.18 & 0.42 &  &  & \\
\ \ \ Shift: heterogeneous (controls) \textcolor{white}{L} & 0.19 & 0.44 &  &  &  & 0.17 & 0.41 &  &  & \\
\ \ \ Shift: heterogeneous (grandparents) \textcolor{white}{L} & 0.25 & 0.50 & 10.00 & .019 & yes & 0.21 & 0.46 & 4.50 & .212 & yes\\
HRS &  &  &  &  &  &  &  &  &  & \\
\ \ \ Before-slope: uniform \textcolor{white}{H} & 0.14 & 0.37 &  &  &  & 0.14 & 0.37 &  &  & \\
\ \ \ Before-slope: heterogeneous (controls) \textcolor{white}{H} & 0.28 & 0.53 &  &  &  & 0.22 & 0.47 &  &  & \\
\ \ \ Before-slope: heterogeneous (grandparents) \textcolor{white}{H} & 0.26 & 0.50 & 140.31 & < .001 & no & 0.34 & 0.58 & 111.97 & < .001 & yes\\
\ \ \ After-slope: uniform \textcolor{white}{H} & 0.10 & 0.32 &  &  &  & 0.14 & 0.37 &  &  & \\
\ \ \ After-slope: heterogeneous (controls) \textcolor{white}{H} & 0.13 & 0.36 &  &  &  & 0.21 & 0.46 &  &  & \\
\ \ \ After-slope: heterogeneous (grandparents) \textcolor{white}{H} & 0.08 & 0.28 & 93.14 & < .001 & no & 0.10 & 0.32 & 108.41 & < .001 & no\\
\ \ \ Shift: uniform \textcolor{white}{H} & 0.83 & 0.91 &  &  &  & 0.93 & 0.96 &  &  & \\
\ \ \ Shift: heterogeneous (controls) \textcolor{white}{H} & 1.07 & 1.04 &  &  &  & 1.24 & 1.11 &  &  & \\
\ \ \ Shift: heterogeneous (grandparents) \textcolor{white}{H} & 0.80 & 0.89 & 172.53 & < .001 & no & 0.91 & 0.96 & 153.16 & < .001 & no\\
\bottomrule
\addlinespace
\insertTableNotes
\end{longtable}

}

\end{lltable}












\begin{lltable}

\begin{TableNotes}[para]
\normalsize{\textit{Note.} Test-retest correlations as indicators of
rank-order stability, and p-values indicating significant group
differences therein between grandparents and each control group. The
average retest intervals in years are 3.06 (\(SD\) = 0.91) for the LISS
parent sample, 3.06 (\(SD\) = 0.89) for the LISS nonparent sample, 4.15
(\(SD\) = 0.77) for the HRS parent sample, and 4.11 (\(SD\) = 0.67) for
the HRS nonparent sample. \(Cor\) = correlation; \(GP\) = grandparents;
\(con\) = controls.}
\end{TableNotes}

\small{

\begin{longtable}{lrrrrrrrr}\noalign{\getlongtablewidth\global\LTcapwidth=\longtablewidth}
\caption{\label{tab:H3-rankorder-tab}Rank-Order Stability.}\\
\toprule
& \multicolumn{4}{c}{Parent controls} & \multicolumn{4}{c}{Nonparent controls} \\
\cmidrule(r){2-5} \cmidrule(r){6-9}
Outcome & $Cor_{all}$ & $Cor_{GP}$ & $Cor_{con}$ & $p$ & $Cor_{all}$ & $Cor_{GP}$ & $Cor_{con}$ & $p$\\
\midrule
\endfirsthead
\caption*{\normalfont{Table \ref{tab:H3-rankorder-tab} continued}}\\
\toprule
& \multicolumn{4}{c}{Parent controls} & \multicolumn{4}{c}{Nonparent controls} \\
\cmidrule(r){2-5} \cmidrule(r){6-9}
Outcome & $Cor_{all}$ & $Cor_{GP}$ & $Cor_{con}$ & $p$ & $Cor_{all}$ & $Cor_{GP}$ & $Cor_{con}$ & $p$\\
\midrule
\endhead
LISS &  &  &  &  &  &  &  & \\
\ \ \ Agreeableness \textcolor{white}{L} & 0.79 & 0.81 & 0.78 & .619 & 0.76 & 0.81 & 0.75 & .009\\
\ \ \ Conscientiousness \textcolor{white}{L} & 0.76 & 0.80 & 0.75 & .102 & 0.79 & 0.80 & 0.78 & .480\\
\ \ \ Extraversion \textcolor{white}{L} & 0.81 & 0.86 & 0.80 & .768 & 0.86 & 0.86 & 0.85 & .284\\
\ \ \ Neuroticism \textcolor{white}{L} & 0.71 & 0.77 & 0.68 & .060 & 0.76 & 0.77 & 0.76 & .262\\
\ \ \ Openness \textcolor{white}{L} & 0.75 & 0.79 & 0.74 & .126 & 0.79 & 0.79 & 0.79 & .531\\
\ \ \ Life Satisfaction \textcolor{white}{L} & 0.69 & 0.66 & 0.70 & .647 & 0.63 & 0.66 & 0.62 & .674\\
HRS &  &  &  &  &  &  &  & \\
\ \ \ Agreeableness \textcolor{white}{H} & 0.68 & 0.70 & 0.67 & .506 & 0.73 & 0.70 & 0.74 & .304\\
\ \ \ Conscientiousness \textcolor{white}{H} & 0.71 & 0.69 & 0.72 & .201 & 0.70 & 0.69 & 0.70 & .467\\
\ \ \ Extraversion \textcolor{white}{H} & 0.72 & 0.75 & 0.71 & .007 & 0.74 & 0.75 & 0.74 & .029\\
\ \ \ Neuroticism \textcolor{white}{H} & 0.66 & 0.71 & 0.65 & .654 & 0.68 & 0.71 & 0.67 & .709\\
\ \ \ Openness \textcolor{white}{H} & 0.69 & 0.73 & 0.67 & .015 & 0.76 & 0.73 & 0.76 & .241\\
\ \ \ Life Satisfaction \textcolor{white}{H} & 0.51 & 0.55 & 0.50 & .090 & 0.55 & 0.55 & 0.55 & .439\\
\bottomrule
\addlinespace
\insertTableNotes
\end{longtable}

}

\end{lltable}













\begin{lltable}

\begin{TableNotes}[para]
\normalsize{\textit{Note.} Test-retest correlations as indicators of
rank-order stability, and p-values indicating significant group
differences therein between grandparents and each control group. The
average retest intervals in years are 8.08 (\(SD\) = 2.06) for the LISS
parent sample, 8.13 (\(SD\) = 1.95) for the LISS nonparent sample, 6.83
(\(SD\) = 2.23) for the HRS parent sample, and 6.92 (\(SD\) = 2.26) for
the HRS nonparent sample. \(Cor\) = correlation; \(GP\) = grandparents;
\(con\) = controls.}
\end{TableNotes}

\small{

\begin{longtable}{lrrrrrrrr}\noalign{\getlongtablewidth\global\LTcapwidth=\longtablewidth}
\caption{\label{tab:H3-rankordermax-tab}Rank-Order Stability With Maximal Retest
Interval.}\\
\toprule
& \multicolumn{4}{c}{Parent controls} & \multicolumn{4}{c}{Nonparent controls} \\
\cmidrule(r){2-5} \cmidrule(r){6-9}
Outcome & $Cor_{all}$ & $Cor_{GP}$ & $Cor_{con}$ & $p$ & $Cor_{all}$ & $Cor_{GP}$ & $Cor_{con}$ & $p$\\
\midrule
\endfirsthead
\caption*{\normalfont{Table \ref{tab:H3-rankordermax-tab} continued}}\\
\toprule
& \multicolumn{4}{c}{Parent controls} & \multicolumn{4}{c}{Nonparent controls} \\
\cmidrule(r){2-5} \cmidrule(r){6-9}
Outcome & $Cor_{all}$ & $Cor_{GP}$ & $Cor_{con}$ & $p$ & $Cor_{all}$ & $Cor_{GP}$ & $Cor_{con}$ & $p$\\
\midrule
\endhead
LISS &  &  &  &  &  &  &  & \\
\ \ \ Agreeableness \textcolor{white}{L} & 0.73 & 0.73 & 0.73 & .754 & 0.60 & 0.73 & 0.57 & < .001\\
\ \ \ Conscientiousness \textcolor{white}{L} & 0.68 & 0.77 & 0.66 & .004 & 0.73 & 0.77 & 0.73 & .091\\
\ \ \ Extraversion \textcolor{white}{L} & 0.76 & 0.82 & 0.74 & .021 & 0.82 & 0.82 & 0.82 & .568\\
\ \ \ Neuroticism \textcolor{white}{L} & 0.68 & 0.76 & 0.65 & .001 & 0.72 & 0.76 & 0.71 & .534\\
\ \ \ Openness \textcolor{white}{L} & 0.72 & 0.77 & 0.71 & .290 & 0.81 & 0.77 & 0.82 & .316\\
\ \ \ Life Satisfaction \textcolor{white}{L} & 0.65 & 0.53 & 0.68 & .086 & 0.48 & 0.53 & 0.48 & .309\\
HRS &  &  &  &  &  &  &  & \\
\ \ \ Agreeableness \textcolor{white}{H} & 0.67 & 0.68 & 0.67 & .641 & 0.70 & 0.68 & 0.71 & .498\\
\ \ \ Conscientiousness \textcolor{white}{H} & 0.65 & 0.68 & 0.65 & .289 & 0.64 & 0.68 & 0.63 & .819\\
\ \ \ Extraversion \textcolor{white}{H} & 0.70 & 0.73 & 0.70 & .093 & 0.71 & 0.73 & 0.70 & .038\\
\ \ \ Neuroticism \textcolor{white}{H} & 0.64 & 0.67 & 0.63 & .704 & 0.64 & 0.67 & 0.63 & .265\\
\ \ \ Openness \textcolor{white}{H} & 0.69 & 0.71 & 0.69 & .894 & 0.75 & 0.71 & 0.76 & .001\\
\ \ \ Life Satisfaction \textcolor{white}{H} & 0.53 & 0.54 & 0.53 & .675 & 0.48 & 0.54 & 0.47 & .166\\
\bottomrule
\addlinespace
\insertTableNotes
\end{longtable}

}

\end{lltable}













\begin{lltable}

\begin{TableNotes}[para]
\normalsize{\textit{Note.} Test-retest correlations as indicators of
rank-order stability, and p-values indicating significant group
differences therein between grandparents and each control group. The
average retest intervals in years are 2.94 (\(SD\) = 0.94) for the LISS
parent sample, 2.95 (\(SD\) = 0.92) for the LISS nonparent sample, 3.88
(\(SD\) = 1.01) for the HRS parent sample, and 3.87 (\(SD\) = 0.96) for
the HRS nonparent sample. \(Cor\) = correlation; \(GP\) = grandparents;
\(con\) = controls.}
\end{TableNotes}

\small{

\begin{longtable}{lrrrrrrrr}\noalign{\getlongtablewidth\global\LTcapwidth=\longtablewidth}
\caption{\label{tab:H3-rankorderuni-tab}Rank-Order Stability Excluding Duplicate
Control Observations.}\\
\toprule
& \multicolumn{4}{c}{Parent controls} & \multicolumn{4}{c}{Nonparent controls} \\
\cmidrule(r){2-5} \cmidrule(r){6-9}
Outcome & $Cor_{all}$ & $Cor_{GP}$ & $Cor_{con}$ & $p$ & $Cor_{all}$ & $Cor_{GP}$ & $Cor_{con}$ & $p$\\
\midrule
\endfirsthead
\caption*{\normalfont{Table \ref{tab:H3-rankorderuni-tab} continued}}\\
\toprule
& \multicolumn{4}{c}{Parent controls} & \multicolumn{4}{c}{Nonparent controls} \\
\cmidrule(r){2-5} \cmidrule(r){6-9}
Outcome & $Cor_{all}$ & $Cor_{GP}$ & $Cor_{con}$ & $p$ & $Cor_{all}$ & $Cor_{GP}$ & $Cor_{con}$ & $p$\\
\midrule
\endhead
LISS &  &  &  &  &  &  &  & \\
\ \ \ Agreeableness \textcolor{white}{L} & 0.80 & 0.81 & 0.79 & .760 & 0.80 & 0.81 & 0.80 & .641\\
\ \ \ Conscientiousness \textcolor{white}{L} & 0.78 & 0.80 & 0.77 & .315 & 0.80 & 0.80 & 0.80 & .493\\
\ \ \ Extraversion \textcolor{white}{L} & 0.84 & 0.86 & 0.82 & .832 & 0.87 & 0.86 & 0.88 & .444\\
\ \ \ Neuroticism \textcolor{white}{L} & 0.78 & 0.77 & 0.78 & .522 & 0.80 & 0.77 & 0.84 & .914\\
\ \ \ Openness \textcolor{white}{L} & 0.79 & 0.79 & 0.79 & .547 & 0.79 & 0.79 & 0.80 & .467\\
\ \ \ Life Satisfaction \textcolor{white}{L} & 0.67 & 0.66 & 0.68 & .708 & 0.69 & 0.66 & 0.72 & .269\\
HRS &  &  &  &  &  &  &  & \\
\ \ \ Agreeableness \textcolor{white}{H} & 0.69 & 0.70 & 0.69 & .504 & 0.71 & 0.70 & 0.74 & .445\\
\ \ \ Conscientiousness \textcolor{white}{H} & 0.71 & 0.69 & 0.72 & .208 & 0.70 & 0.69 & 0.72 & .297\\
\ \ \ Extraversion \textcolor{white}{H} & 0.75 & 0.75 & 0.75 & .315 & 0.74 & 0.75 & 0.73 & .122\\
\ \ \ Neuroticism \textcolor{white}{H} & 0.69 & 0.71 & 0.67 & .543 & 0.70 & 0.71 & 0.70 & .367\\
\ \ \ Openness \textcolor{white}{H} & 0.75 & 0.73 & 0.76 & .396 & 0.74 & 0.73 & 0.75 & .855\\
\ \ \ Life Satisfaction \textcolor{white}{H} & 0.58 & 0.55 & 0.59 & .317 & 0.58 & 0.55 & 0.61 & .015\\
\bottomrule
\addlinespace
\insertTableNotes
\end{longtable}

}

\end{lltable}

\noindent \textbf{Supplemental Figures}





\begin{figure}

{\centering \includegraphics[width=1\linewidth]{Figs/flowchart-participant-1} 

}

\caption{Participant flowchart demonstrating the
composition of the four analysis samples via matching (1:4 matching
ratio with replacement). obs. = longitudinal observations.}\label{fig:flowchart-participant}
\end{figure}




\begin{figure}
\centering
\includegraphics{Figs/pscore-overlap-1.pdf}
\caption{\label{fig:pscore-overlap}Distributional Overlap of the Propensity Score
in the Four Analysis Samples at the Time of Matching.}
\end{figure}




\begin{figure}
\centering
\includegraphics{Figs/loess-agree-1.pdf}
\caption{\label{fig:loess-agree}Violin Plots for Agreeableness Including Means
Over Time and LOESS Line.}
\end{figure}




\begin{figure}
\centering
\includegraphics{Figs/loess-con-1.pdf}
\caption{\label{fig:loess-con}Violin Plots for Conscientiousness Including Means
Over Time and LOESS Line.}
\end{figure}




\begin{figure}
\centering
\includegraphics{Figs/loess-extra-1.pdf}
\caption{\label{fig:loess-extra}Violin Plots for Extraversion Including Means Over
Time and LOESS Line.}
\end{figure}




\begin{figure}
\centering
\includegraphics{Figs/loess-neur-1.pdf}
\caption{\label{fig:loess-neur}Violin Plots for Neuroticism Including Means Over
Time and LOESS Line.}
\end{figure}




\begin{figure}
\centering
\includegraphics{Figs/loess-open-1.pdf}
\caption{\label{fig:loess-open}Violin Plots for Openness Including Means Over Time
and LOESS Line.}
\end{figure}




\begin{figure}
\centering
\includegraphics{Figs/loess-swls-1.pdf}
\caption{\label{fig:loess-swls}Violin Plots for Life Satisfaction Including Means
Over Time and LOESS Line.}
\end{figure}








\begin{figure}
\centering
\includegraphics{Figs/H1-agree-fig-1.pdf}
\caption{\label{fig:H1-agree-fig}Change trajectories of agreeableness based on the
basic models (left column) and the models including the gender
interaction (right column). The error bars are 95\% confidence intervals
of the predicted values which only account for the fixed-effects portion
of the model. The vertical line indicates the approximate time of the
transition to grandparenthood.}
\end{figure}










\begin{figure}
\centering
\includegraphics{Figs/H1-agree-work-fig-1.pdf}
\caption{\label{fig:H1-agree-work-fig}Change trajectories of agreeableness based
on the models of moderation by paid work (see Table
\ref{tab:H1-agree-work-tab}). The error bars are 95\% confidence
intervals of the predicted values which only account for the
fixed-effects portion of the model. The vertical line indicates the
approximate time of the transition to grandparenthood. The plots in the
left column are the same as in Figure \ref{fig:H1-agree-fig} (basic
models) and added here for better comparability.}
\end{figure}









\begin{figure}
\centering
\includegraphics{Figs/H1-agree-care-fig-1.pdf}
\caption{\label{fig:H1-agree-care-fig}Change trajectories of agreeableness based
on the models of moderation by grandchild care (see Table
\ref{tab:H1-agree-care-tab}). The error bars are 95\% confidence
intervals of the predicted values which only account for the
fixed-effects portion of the model. The plots in the left column are the
same as in Figure \ref{fig:H1-agree-fig} (basic models) but restricted
to the post-transition period for better comparability.}
\end{figure}








\begin{figure}
\centering
\includegraphics{Figs/H1-con-fig-1.pdf}
\caption{\label{fig:H1-con-fig}Change trajectories of conscientiousness based on
the basic models (left column) and the models including the gender
interaction (right column). The error bars are 95\% confidence intervals
of the predicted values which only account for the fixed-effects portion
of the model. The vertical line indicates the approximate time of the
transition to grandparenthood.}
\end{figure}










\begin{figure}
\centering
\includegraphics{Figs/H1-con-work-fig-1.pdf}
\caption{\label{fig:H1-con-work-fig}Change trajectories of conscientiousness based
on the models of moderation by paid work (see Table
\ref{tab:H1-con-work-tab}). The error bars are 95\% confidence intervals
of the predicted values which only account for the fixed-effects portion
of the model. The vertical line indicates the approximate time of the
transition to grandparenthood. The plots in the left column are the same
as in Figure \ref{fig:H1-con-fig} (basic models) and added here for
better comparability.}
\end{figure}









\begin{figure}
\centering
\includegraphics{Figs/H1-con-care-fig-1.pdf}
\caption{\label{fig:H1-con-care-fig}Change trajectories of conscientiousness based
on the models of moderation by grandchild care (see Table
\ref{tab:H1-con-care-tab}). The error bars are 95\% confidence intervals
of the predicted values which only account for the fixed-effects portion
of the model. The plots in the left column are the same as in Figure
\ref{fig:H1-con-fig} (basic models) but restricted to the
post-transition period for better comparability.}
\end{figure}








\begin{figure}
\centering
\includegraphics{Figs/H1-extra-fig-1.pdf}
\caption{\label{fig:H1-extra-fig}Change trajectories of extraversion based on the
basic models (left column) and the models including the gender
interaction (right column). The error bars are 95\% confidence intervals
of the predicted values which only account for the fixed-effects portion
of the model. The vertical line indicates the approximate time of the
transition to grandparenthood.}
\end{figure}










\begin{figure}
\centering
\includegraphics{Figs/H1-extra-work-fig-1.pdf}
\caption{\label{fig:H1-extra-work-fig}Change trajectories of extraversion based on
the models of moderation by paid work (see Table
\ref{tab:H1-extra-work-tab}). The error bars are 95\% confidence
intervals of the predicted values which only account for the
fixed-effects portion of the model. The vertical line indicates the
approximate time of the transition to grandparenthood. The plots in the
left column are the same as in Figure \ref{fig:H1-extra-fig} (basic
models) and added here for better comparability.}
\end{figure}









\begin{figure}
\centering
\includegraphics{Figs/H1-extra-care-fig-1.pdf}
\caption{\label{fig:H1-extra-care-fig}Change trajectories of extraversion based on
the models of moderation by grandchild care (see Table
\ref{tab:H1-extra-care-tab}). The error bars are 95\% confidence
intervals of the predicted values which only account for the
fixed-effects portion of the model. The plots in the left column are the
same as in Figure \ref{fig:H1-extra-fig} (basic models) but restricted
to the post-transition period for better comparability.}
\end{figure}








\begin{figure}
\centering
\includegraphics{Figs/H1-neur-fig-1.pdf}
\caption{\label{fig:H1-neur-fig}Change trajectories of neuroticism based on the
basic models (left column) and the models including the gender
interaction (right column). The error bars are 95\% confidence intervals
of the predicted values which only account for the fixed-effects portion
of the model. The vertical line indicates the approximate time of the
transition to grandparenthood.}
\end{figure}










\begin{figure}
\centering
\includegraphics{Figs/H1-neur-work-fig-1.pdf}
\caption{\label{fig:H1-neur-work-fig}Change trajectories of neuroticism based on
the models of moderation by paid work (see Table
\ref{tab:H1-neur-work-tab}). The error bars are 95\% confidence
intervals of the predicted values which only account for the
fixed-effects portion of the model. The vertical line indicates the
approximate time of the transition to grandparenthood. The plots in the
left column are the same as in Figure \ref{fig:H1-neur-fig} (basic
models) and added here for better comparability.}
\end{figure}









\begin{figure}
\centering
\includegraphics{Figs/H1-neur-care-fig-1.pdf}
\caption{\label{fig:H1-neur-care-fig}Change trajectories of neuroticism based on
the models of moderation by grandchild care (see Table
\ref{tab:H1-neur-care-tab}). The error bars are 95\% confidence
intervals of the predicted values which only account for the
fixed-effects portion of the model. The plots in the left column are the
same as in Figure \ref{fig:H1-neur-fig} (basic models) but restricted to
the post-transition period for better comparability.}
\end{figure}








\begin{figure}
\centering
\includegraphics{Figs/H1-open-fig-1.pdf}
\caption{\label{fig:H1-open-fig}Change trajectories of openness based on the basic
models (left column) and the models including the gender interaction
(right column). The error bars are 95\% confidence intervals of the
predicted values which only account for the fixed-effects portion of the
model. The vertical line indicates the approximate time of the
transition to grandparenthood.}
\end{figure}










\begin{figure}
\centering
\includegraphics{Figs/H1-open-work-fig-1.pdf}
\caption{\label{fig:H1-open-work-fig}Change trajectories of openness based on the
models of moderation by paid work (see Table
\ref{tab:H1-open-work-tab}). The error bars are 95\% confidence
intervals of the predicted values which only account for the
fixed-effects portion of the model. The vertical line indicates the
approximate time of the transition to grandparenthood. The plots in the
left column are the same as in Figure \ref{fig:H1-open-fig} (basic
models) and added here for better comparability.}
\end{figure}









\begin{figure}
\centering
\includegraphics{Figs/H1-open-care-fig-1.pdf}
\caption{\label{fig:H1-open-care-fig}Change trajectories of openness based on the
models of moderation by grandchild care (see Table
\ref{tab:H1-open-care-tab}). The error bars are 95\% confidence
intervals of the predicted values which only account for the
fixed-effects portion of the model. The plots in the left column are the
same as in Figure \ref{fig:H1-open-fig} (basic models) but restricted to
the post-transition period for better comparability.}
\end{figure}








\begin{figure}
\centering
\includegraphics{Figs/H1-swls-fig-1.pdf}
\caption{\label{fig:H1-swls-fig}Change trajectories of life satisfaction based on
the basic models (left column) and the models including the gender
interaction (right column). The error bars are 95\% confidence intervals
of the predicted values which only account for the fixed-effects portion
of the model. The vertical line indicates the approximate time of the
transition to grandparenthood.}
\end{figure}










\begin{figure}
\centering
\includegraphics{Figs/H1-swls-work-fig-1.pdf}
\caption{\label{fig:H1-swls-work-fig}Change trajectories of life satisfaction
based on the models of moderation by paid work (see Table
\ref{tab:H1-swls-work-tab}). The error bars are 95\% confidence
intervals of the predicted values which only account for the
fixed-effects portion of the model. The vertical line indicates the
approximate time of the transition to grandparenthood. The plots in the
left column are the same as in Figure \ref{fig:H1-swls-fig} (basic
models) and added here for better comparability.}
\end{figure}









\begin{figure}
\centering
\includegraphics{Figs/H1-swls-care-fig-1.pdf}
\caption{\label{fig:H1-swls-care-fig}Change trajectories of life satisfaction
based on the models of moderation by grandchild care (see Table
\ref{tab:H1-swls-care-tab}). The error bars are 95\% confidence
intervals of the predicted values which only account for the
fixed-effects portion of the model. The plots in the left column are the
same as in Figure \ref{fig:H1-swls-fig} (basic models) but restricted to
the post-transition period for better comparability.}
\end{figure}












\begin{figure}
\centering
\includegraphics{Figs/effects-basic-plot-1.pdf}
\caption{\label{fig:effects-basic-plot}Unstandardized Effect Sizes of the Basic
Models Across Analysis Samples (Regression Coefficients \(\hat{\gamma}\)
or Linear Contrasts \(\hat{\gamma}_{c}\) From Multilevel Models, see
Tables \ref{tab:H1-agree-tab}, \ref{tab:H1-agree-contrasts},
\ref{tab:H1-con-tab}, \ref{tab:H1-con-contrasts},
\ref{tab:H1-extra-tab}, \ref{tab:H1-extra-contrasts},
\ref{tab:H1-neur-tab}, \ref{tab:H1-neur-contrasts},
\ref{tab:H1-open-tab}, \ref{tab:H1-open-contrasts},
\ref{tab:H1-swls-tab}, \ref{tab:H1-swls-contrasts}). Error Bars
Represent 95\% Confidence Intervals.}
\end{figure}













\begin{figure}
\centering
\includegraphics{Figs/effects-gender-plot-1.pdf}
\caption{\label{fig:effects-gender-plot}Unstandardized Effect Sizes of the Models
Including the Gender Interaction Across Analysis Samples (Regression
Coefficients \(\hat{\gamma}\) or Linear Contrasts \(\hat{\gamma}_{c}\)
From Multilevel Models, see Tables \ref{tab:H1-agree-gender-tab},
\ref{tab:H1-agree-gender-contrasts}, \ref{tab:H1-con-gender-tab},
\ref{tab:H1-con-gender-contrasts}, \ref{tab:H1-extra-gender-tab},
\ref{tab:H1-extra-gender-contrasts}, \ref{tab:H1-neur-gender-tab},
\ref{tab:H1-neur-gender-contrasts}, \ref{tab:H1-open-gender-tab},
\ref{tab:H1-open-gender-contrasts}, \ref{tab:H1-swls-gender-tab},
\ref{tab:H1-swls-gender-contrasts}). Error Bars Represent 95\%
Confidence Intervals.}
\end{figure}

\newpage

\noindent  \textbf{Complete Software and Session Information}

We used R (Version 4.0.4; R Core Team, 2021) and the R-packages
\emph{car} (Version 3.0.10; Fox et al., 2020a, 2020b; Yentes \& Wilhelm,
2018), \emph{carData} (Version 3.0.4; Fox et al., 2020b),
\emph{careless} (Version 1.1.3; Yentes \& Wilhelm, 2018), \emph{citr}
(Version 0.3.2; Aust, 2019), \emph{corrplot2017} (Wei \& Simko, 2017),
\emph{cowplot} (Version 1.1.0; Wilke, 2020), \emph{dplyr} (Version
1.0.2; Wickham, François, et al., 2020), \emph{effects} (Version 4.2.0;
Fox \& Weisberg, 2018; Fox, 2003; Fox \& Hong, 2009), \emph{forcats}
(Version 0.5.0; Wickham, 2020a), \emph{foreign} (Version 0.8.81; R Core
Team, 2020), \emph{Formula} (Version 1.2.4; Zeileis \& Croissant, 2010),
\emph{ggplot2} (Version 3.3.5; Wickham, 2016), \emph{ggplotify} (Version
0.0.7; Yu, 2021), \emph{GPArotation} (Version 2014.11.1; Bernaards \&
I.Jennrich, 2005), \emph{Hmisc} (Version 4.4.2; Harrell Jr et al.,
2020), \emph{interactions} (Version 1.1.3; Long, 2019), \emph{jtools}
(Version 2.1.1; Long, 2020), \emph{knitr} (Version 1.30; Xie, 2015),
\emph{lattice} (Version 0.20.41; Sarkar, 2008), \emph{lme4} (Version
1.1.26; Bates et al., 2015), \emph{lmerTest} (Version 3.1.3; Kuznetsova
et al., 2017), \emph{magick} (Version 2.6.0; Ooms, 2021), \emph{MASS}
(Version 7.3.53; Venables \& Ripley, 2002), \emph{MatchIt} (Version
4.1.0; Ho et al., 2020), \emph{Matrix} (Version 1.3.2; Bates \&
Maechler, 2021), \emph{multcomp} (Version 1.4.17; Hothorn et al., 2008),
\emph{mvtnorm} (Version 1.1.1; Genz \& Bretz, 2009), \emph{papaja}
(Version 0.1.0.9997; Aust \& Barth, 2020), \emph{patchwork} (Version
1.1.0.9000; Pedersen, 2020), \emph{png} (Version 0.1.7; Urbanek, 2013),
\emph{psych} (Version 2.0.9; Revelle, 2020), \emph{purrr} (Version
0.3.4; Henry \& Wickham, 2020), \emph{readr} (Version 1.4.0; Wickham \&
Hester, 2020), \emph{readxl} (Version 1.3.1; Wickham \& Bryan, 2019),
\emph{robustlmm} (Version 2.3; Koller, 2016), \emph{scales} (Version
1.1.1; Wickham \& Seidel, 2020), \emph{shiny} (Version 1.5.0; Chang et
al., 2020), \emph{stringr} (Version 1.4.0; Wickham, 2019),
\emph{survival} (Version 3.2.7; Terry M. Therneau \& Patricia M.
Grambsch, 2000), \emph{TH.data} (Version 1.0.10; Hothorn, 2019),
\emph{tibble} (Version 3.1.2; Müller \& Wickham, 2020), \emph{tidyr}
(Version 1.1.2; Wickham, 2020b), \emph{tidyverse} (Version 1.3.0;
Wickham, Averick, et al., 2019), and \emph{tinylabels} (Version 0.1.0;
Barth, 2020) for data wrangling, analyses, and plots.\\
The following is the output of R's \emph{sessionInfo()} command, which
shows information to aid analytic reproducibility of the analyses.

R version 4.0.4 (2021-02-15) Platform: x86\_64-apple-darwin17.0 (64-bit)
Running under: macOS Big Sur 10.16

Matrix products: default BLAS:
/Library/Frameworks/R.framework/Versions/4.0/Resources/lib/libRblas.dylib
LAPACK:
/Library/Frameworks/R.framework/Versions/4.0/Resources/lib/libRlapack.dylib

locale: {[}1{]}
en\_US.UTF-8/en\_US.UTF-8/en\_US.UTF-8/C/en\_US.UTF-8/en\_US.UTF-8

attached base packages: {[}1{]} grid stats graphics grDevices utils
datasets methods\\
{[}8{]} base

other attached packages: {[}1{]} png\_0.1-7 car\_3.0-10 carData\_3.0-4\\
{[}4{]} scales\_1.1.1 cowplot\_1.1.0 lmerTest\_3.1-3\\
{[}7{]} lme4\_1.1-26 Matrix\_1.3-2 GPArotation\_2014.11-1 {[}10{]}
psych\_2.0.9 forcats\_0.5.0 stringr\_1.4.0\\
{[}13{]} dplyr\_1.0.2 purrr\_0.3.4 readr\_1.4.0\\
{[}16{]} tidyr\_1.1.2 tibble\_3.1.2 tidyverse\_1.3.0\\
{[}19{]} Hmisc\_4.4-2 ggplot2\_3.3.5 Formula\_1.2-4\\
{[}22{]} lattice\_0.20-41 multcomp\_1.4-17 TH.data\_1.0-10\\
{[}25{]} MASS\_7.3-53 survival\_3.2-7 mvtnorm\_1.1-1\\
{[}28{]} citr\_0.3.2 papaja\_0.1.0.9997 tinylabels\_0.1.0

loaded via a namespace (and not attached): {[}1{]} minqa\_1.2.4
colorspace\_2.0-1 rio\_0.5.16\\
{[}4{]} ellipsis\_0.3.2 htmlTable\_2.1.0 base64enc\_0.1-3\\
{[}7{]} fs\_1.5.0 rstudioapi\_0.13 fansi\_0.5.0\\
{[}10{]} lubridate\_1.7.9.2 xml2\_1.3.2 codetools\_0.2-18\\
{[}13{]} splines\_4.0.4 mnormt\_2.0.2 knitr\_1.30\\
{[}16{]} jsonlite\_1.7.2 nloptr\_1.2.2.2 broom\_0.7.6\\
{[}19{]} cluster\_2.1.0 dbplyr\_1.4.4 shiny\_1.5.0\\
{[}22{]} compiler\_4.0.4 httr\_1.4.2 backports\_1.2.1\\
{[}25{]} assertthat\_0.2.1 fastmap\_1.0.1 cli\_2.5.0\\
{[}28{]} later\_1.1.0.1 htmltools\_0.5.0 tools\_4.0.4\\
{[}31{]} gtable\_0.3.0 glue\_1.4.2 Rcpp\_1.0.6\\
{[}34{]} cellranger\_1.1.0 vctrs\_0.3.8 nlme\_3.1-152\\
{[}37{]} xfun\_0.19 openxlsx\_4.2.3 rvest\_0.3.6\\
{[}40{]} mime\_0.9 miniUI\_0.1.1.1 lifecycle\_1.0.0\\
{[}43{]} statmod\_1.4.35 zoo\_1.8-8 hms\_0.5.3\\
{[}46{]} promises\_1.1.1 parallel\_4.0.4 sandwich\_3.0-0\\
{[}49{]} RColorBrewer\_1.1-2 curl\_4.3.1 yaml\_2.2.1\\
{[}52{]} gridExtra\_2.3 rpart\_4.1-15 latticeExtra\_0.6-29 {[}55{]}
stringi\_1.5.3 checkmate\_2.0.0 zip\_2.1.1\\
{[}58{]} boot\_1.3-26 rlang\_0.4.11 pkgconfig\_2.0.3\\
{[}61{]} evaluate\_0.14 htmlwidgets\_1.5.2 tidyselect\_1.1.0\\
{[}64{]} magrittr\_2.0.1 bookdown\_0.21 R6\_2.5.0\\
{[}67{]} generics\_0.1.0 DBI\_1.1.0 pillar\_1.6.1\\
{[}70{]} haven\_2.3.1 foreign\_0.8-81 withr\_2.4.2\\
{[}73{]} abind\_1.4-5 nnet\_7.3-15 modelr\_0.1.8\\
{[}76{]} crayon\_1.4.1 utf8\_1.2.1 tmvnsim\_1.0-2\\
{[}79{]} rmarkdown\_2.5 jpeg\_0.1-8.1 readxl\_1.3.1\\
{[}82{]} data.table\_1.13.2 blob\_1.2.1 reprex\_0.3.0\\
{[}85{]} digest\_0.6.27 xtable\_1.8-4 httpuv\_1.5.4\\
{[}88{]} numDeriv\_2016.8-1.1 munsell\_0.5.0

\newpage

\noindent  \textbf{References}

\begingroup
\setlength{\parindent}{-0.5in}
\setlength{\leftskip}{0.5in}

\hypertarget{refs}{}
\leavevmode\hypertarget{ref-R-citr}{}%
Aust, F. (2019). \emph{Citr: 'RStudio' add-in to insert markdown
citations}. \url{https://github.com/crsh/citr}

\leavevmode\hypertarget{ref-R-papaja}{}%
Aust, F., \& Barth, M. (2020). \emph{papaja: Prepare reproducible APA
journal articles with R Markdown}. \url{https://github.com/crsh/papaja}

\leavevmode\hypertarget{ref-austinIntroductionPropensityScore2011}{}%
Austin, P. C. (2011). An introduction to propensity score methods for
reducing the effects of confounding in observational studies.
\emph{Multivariate Behavioral Research}, \emph{46}(3), 399--424.
\url{https://doi.org/10.1080/00273171.2011.568786}

\leavevmode\hypertarget{ref-R-tinylabels}{}%
Barth, M. (2020). \emph{Tinylabels: Lightweight variable labels}.
\url{https://CRAN.R-project.org/package=tinylabels}

\leavevmode\hypertarget{ref-R-Matrix}{}%
Bates, D., \& Maechler, M. (2021). \emph{Matrix: Sparse and dense matrix
classes and methods}. \url{https://CRAN.R-project.org/package=Matrix}

\leavevmode\hypertarget{ref-R-lme4}{}%
Bates, D., Mächler, M., Bolker, B., \& Walker, S. (2015). Fitting linear
mixed-effects models using lme4. \emph{Journal of Statistical Software},
\emph{67}(1), 1--48. \url{https://doi.org/10.18637/jss.v067.i01}

\leavevmode\hypertarget{ref-R-GPArotation}{}%
Bernaards, C. A., \& I.Jennrich, R. (2005). Gradient projection
algorithms and software for arbitrary rotation criteria in factor
analysis. \emph{Educational and Psychological Measurement}, \emph{65},
676--696.

\leavevmode\hypertarget{ref-R-shiny}{}%
Chang, W., Cheng, J., Allaire, J., Xie, Y., \& McPherson, J. (2020).
\emph{Shiny: Web application framework for r}.
\url{https://CRAN.R-project.org/package=shiny}

\leavevmode\hypertarget{ref-R-effects_b}{}%
Fox, J. (2003). Effect displays in R for generalised linear models.
\emph{Journal of Statistical Software}, \emph{8}(15), 1--27.
\url{https://www.jstatsoft.org/article/view/v008i15}

\leavevmode\hypertarget{ref-R-effects_c}{}%
Fox, J., \& Hong, J. (2009). Effect displays in R for multinomial and
proportional-odds logit models: Extensions to the effects package.
\emph{Journal of Statistical Software}, \emph{32}(1), 1--24.
\url{https://www.jstatsoft.org/article/view/v032i01}

\leavevmode\hypertarget{ref-R-effects_a}{}%
Fox, J., \& Weisberg, S. (2018). Visualizing fit and lack of fit in
complex regression models with predictor effect plots and partial
residuals. \emph{Journal of Statistical Software}, \emph{87}(9), 1--27.
\url{https://doi.org/10.18637/jss.v087.i09}

\leavevmode\hypertarget{ref-car2019}{}%
Fox, J., \& Weisberg, S. (2019). \emph{An R companion to applied
regression} (Third). Sage.

\leavevmode\hypertarget{ref-R-car}{}%
Fox, J., Weisberg, S., \& Price, B. (2020a). \emph{Car: Companion to
applied regression} {[}Manual{]}.

\leavevmode\hypertarget{ref-R-carData}{}%
Fox, J., Weisberg, S., \& Price, B. (2020b). \emph{CarData: Companion to
applied regression data sets}.
\url{https://CRAN.R-project.org/package=carData}

\leavevmode\hypertarget{ref-R-mvtnorm}{}%
Genz, A., \& Bretz, F. (2009). \emph{Computation of multivariate normal
and t probabilities}. Springer-Verlag.

\leavevmode\hypertarget{ref-R-Hmisc}{}%
Harrell Jr, F. E., Charles Dupont, \& others. (2020). \emph{Hmisc:
Harrell miscellaneous}. \url{https://CRAN.R-project.org/package=Hmisc}

\leavevmode\hypertarget{ref-R-purrr}{}%
Henry, L., \& Wickham, H. (2020). \emph{Purrr: Functional programming
tools}. \url{https://CRAN.R-project.org/package=purrr}

\leavevmode\hypertarget{ref-R-MatchIt}{}%
Ho, D., Imai, K., King, G., Stuart, E., \& Greifer, N. (2020).
\emph{MatchIt: Nonparametric preprocessing for parametric causal
inference} {[}Manual{]}.

\leavevmode\hypertarget{ref-R-TH.data}{}%
Hothorn, T. (2019). \emph{TH.data: TH's data archive}.
\url{https://CRAN.R-project.org/package=TH.data}

\leavevmode\hypertarget{ref-R-multcomp}{}%
Hothorn, T., Bretz, F., \& Westfall, P. (2008). Simultaneous inference
in general parametric models. \emph{Biometrical Journal}, \emph{50}(3),
346--363.

\leavevmode\hypertarget{ref-R-robustlmm}{}%
Koller, M. (2016). robustlmm: An R package for robust estimation of
linear mixed-effects models. \emph{Journal of Statistical Software},
\emph{75}(6), 1--24. \url{https://doi.org/10.18637/jss.v075.i06}

\leavevmode\hypertarget{ref-R-lmerTest}{}%
Kuznetsova, A., Brockhoff, P. B., \& Christensen, R. H. B. (2017).
lmerTest package: Tests in linear mixed effects models. \emph{Journal of
Statistical Software}, \emph{82}(13), 1--26.
\url{https://doi.org/10.18637/jss.v082.i13}

\leavevmode\hypertarget{ref-R-interactions}{}%
Long, J. A. (2019). \emph{Interactions: Comprehensive, user-friendly
toolkit for probing interactions}.
\url{https://cran.r-project.org/package=interactions}

\leavevmode\hypertarget{ref-R-jtools}{}%
Long, J. A. (2020). \emph{Jtools: Analysis and presentation of social
scientific data}. \url{https://cran.r-project.org/package=jtools}

\leavevmode\hypertarget{ref-R-tibble}{}%
Müller, K., \& Wickham, H. (2020). \emph{Tibble: Simple data frames}.
\url{https://CRAN.R-project.org/package=tibble}

\leavevmode\hypertarget{ref-R-magick}{}%
Ooms, J. (2021). \emph{Magick: Advanced graphics and image-processing in
r}. \url{https://CRAN.R-project.org/package=magick}

\leavevmode\hypertarget{ref-R-patchwork}{}%
Pedersen, T. L. (2020). \emph{Patchwork: The composer of plots}.

\leavevmode\hypertarget{ref-R-foreign}{}%
R Core Team. (2020). \emph{Foreign: Read data stored by 'minitab', 's',
'sas', 'spss', 'stata', 'systat', 'weka', 'dBase', ...}
\url{https://CRAN.R-project.org/package=foreign}

\leavevmode\hypertarget{ref-R-base}{}%
R Core Team. (2021). \emph{R: A language and environment for statistical
computing}. R Foundation for Statistical Computing.
\url{https://www.R-project.org/}

\leavevmode\hypertarget{ref-R-psych}{}%
Revelle, W. (2020). \emph{Psych: Procedures for psychological,
psychometric, and personality research}. Northwestern University.
\url{https://CRAN.R-project.org/package=psych}

\leavevmode\hypertarget{ref-R-lattice}{}%
Sarkar, D. (2008). \emph{Lattice: Multivariate data visualization with
r}. Springer. \url{http://lmdvr.r-forge.r-project.org}

\leavevmode\hypertarget{ref-stuartMatchingMethodsCausal2010}{}%
Stuart, E. A. (2010). Matching methods for causal inference: A review
and a look forward. \emph{Statistical Science: A Review Journal of the
Institute of Mathematical Statistics}, \emph{25}(1), 1--21.
\url{https://doi.org/10.1214/09-STS313}

\leavevmode\hypertarget{ref-R-survival-book}{}%
Terry M. Therneau, \& Patricia M. Grambsch. (2000). \emph{Modeling
survival data: Extending the Cox model}. Springer.

\leavevmode\hypertarget{ref-R-png}{}%
Urbanek, S. (2013). \emph{Png: Read and write png images}.
\url{https://CRAN.R-project.org/package=png}

\leavevmode\hypertarget{ref-R-MASS}{}%
Venables, W. N., \& Ripley, B. D. (2002). \emph{Modern applied
statistics with s} (Fourth). Springer.
\url{http://www.stats.ox.ac.uk/pub/MASS4/}

\leavevmode\hypertarget{ref-R-corrplot2017}{}%
Wei, T., \& Simko, V. (2017). \emph{R package "corrplot": Visualization
of a correlation matrix}. \url{https://github.com/taiyun/corrplot}

\leavevmode\hypertarget{ref-R-ggplot2}{}%
Wickham, H. (2016). \emph{Ggplot2: Elegant graphics for data analysis}.
Springer-Verlag New York. \url{https://ggplot2.tidyverse.org}

\leavevmode\hypertarget{ref-R-stringr}{}%
Wickham, H. (2019). \emph{Stringr: Simple, consistent wrappers for
common string operations}.
\url{https://CRAN.R-project.org/package=stringr}

\leavevmode\hypertarget{ref-R-forcats}{}%
Wickham, H. (2020a). \emph{Forcats: Tools for working with categorical
variables (factors)}. \url{https://CRAN.R-project.org/package=forcats}

\leavevmode\hypertarget{ref-R-tidyr}{}%
Wickham, H. (2020b). \emph{Tidyr: Tidy messy data}.
\url{https://CRAN.R-project.org/package=tidyr}

\leavevmode\hypertarget{ref-R-tidyverse}{}%
Wickham, H., Averick, M., Bryan, J., Chang, W., McGowan, L. D.,
François, R., Grolemund, G., Hayes, A., Henry, L., Hester, J., Kuhn, M.,
Pedersen, T. L., Miller, E., Bache, S. M., Müller, K., Ooms, J.,
Robinson, D., Seidel, D. P., Spinu, V., \ldots{} Yutani, H. (2019).
Welcome to the tidyverse. \emph{Journal of Open Source Software},
\emph{4}(43), 1686. \url{https://doi.org/10.21105/joss.01686}

\leavevmode\hypertarget{ref-R-readxl}{}%
Wickham, H., \& Bryan, J. (2019). \emph{Readxl: Read excel files}.
\url{https://CRAN.R-project.org/package=readxl}

\leavevmode\hypertarget{ref-R-dplyr}{}%
Wickham, H., François, R., Henry, L., \& Müller, K. (2020). \emph{Dplyr:
A grammar of data manipulation}.
\url{https://CRAN.R-project.org/package=dplyr}

\leavevmode\hypertarget{ref-R-readr}{}%
Wickham, H., \& Hester, J. (2020). \emph{Readr: Read rectangular text
data}. \url{https://CRAN.R-project.org/package=readr}

\leavevmode\hypertarget{ref-R-scales}{}%
Wickham, H., \& Seidel, D. (2020). \emph{Scales: Scale functions for
visualization}. \url{https://CRAN.R-project.org/package=scales}

\leavevmode\hypertarget{ref-R-cowplot}{}%
Wilke, C. O. (2020). \emph{Cowplot: Streamlined plot theme and plot
annotations for 'ggplot2'}.
\url{https://CRAN.R-project.org/package=cowplot}

\leavevmode\hypertarget{ref-R-knitr}{}%
Xie, Y. (2015). \emph{Dynamic documents with R and knitr} (2nd ed.).
Chapman; Hall/CRC. \url{https://yihui.org/knitr/}

\leavevmode\hypertarget{ref-R-careless}{}%
Yentes, R. D., \& Wilhelm, F. (2018). \emph{Careless: Procedures for
computing indices of careless responding}.

\leavevmode\hypertarget{ref-R-ggplotify}{}%
Yu, G. (2021). \emph{Ggplotify: Convert plot to 'grob' or 'ggplot'
object}. \url{https://CRAN.R-project.org/package=ggplotify}

\leavevmode\hypertarget{ref-R-Formula}{}%
Zeileis, A., \& Croissant, Y. (2010). Extended model formulas in R:
Multiple parts and multiple responses. \emph{Journal of Statistical
Software}, \emph{34}(1), 1--13.
\url{https://doi.org/10.18637/jss.v034.i01}

\endgroup
\end{appendix}

\end{document}
